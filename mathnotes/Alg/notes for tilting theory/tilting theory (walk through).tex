\documentclass[11pt]{ctexart}

\usepackage[margin=2cm, right = 3cm ,a4paper]{geometry}
\usepackage{amsthm, amssymb, mathrsfs, newclude, tikz-cd, tikz, ctex, stmaryrd, datetime} % 基本配置. 

\usepackage{cite} 
\usepackage[english]{babel}
%Includes "References" in the table of contents
\usepackage[nottoc]{tocbibind}
% 使用 BiBTeX. 

\usepackage{fontspec}
\usepackage{unicode-math}

\usepackage{quiver}

\usepackage{changepage} % 局部縮進. 


% Include the x-color package for color support
\usepackage{xcolor}

\usepackage{csvsimple}
% 使用 csv 文件

%\setmainfont{Caladea}

%% 也可以选用其它字库:
% \setCJKmainfont[%
%   ItalicFont=AR PL KaitiM GB,
%   BoldFont=Noto Sans CJK SC,
% ]{Noto Serif CJK SC}
% \setCJKsansfont{Noto Sans CJK SC}
% \renewcommand{\kaishu}{\CJKfontspec{AR PL KaitiM GB}}

\usepackage{parnotes}
% Define the color for the margin notes
\definecolor{mynotecolor}{rgb}{1, 0, 0} % Red color

% Redefine \parnote to include the color
\renewcommand{\parnote}[1]{\marginpar{\textcolor{mynotecolor}{\footnotesize #1}}}


\usepackage[colorlinks = true,
linkcolor = blue,
urlcolor  = blue,
citecolor = blue,
anchorcolor = blue]{hyperref}

% Define a new environment for red comments
\usepackage{verbatim} % Required for the comment environment
\usepackage{environ}

\usepackage{mdframed} % Include mdframed for creating framed environments

\definecolor{pinked}{RGB}{255,231,229} % Define a base color 
% Define a new environment with a background color
\newmdenv[
  backgroundcolor=pinked, % Set the desired background color
  linecolor=white, % Optional: Set the border line color
  linewidth=1pt, % Optional: Set the border line width
  roundcorner=5pt, % Optional: Set rounded corners
  nobreak=true % Optional: Prevent page breaks within the environment
]{pinked}

\theoremstyle{definition}
\newtheorem{qqq}{问题}[section]

\newcommand{\ExternalLink}{%
    \tikz[x=1.2ex, y=1.2ex, baseline=-0.05ex]{% 
        \begin{scope}[x=1ex, y=1ex]
            \clip (-0.1,-0.1) 
                --++ (-0, 1.2) 
                --++ (0.6, 0) 
                --++ (0, -0.6) 
                --++ (0.6, 0) 
                --++ (0, -1);
            \path[draw, 
                line width = 0.5, 
                rounded corners=0.5] 
                (0,0) rectangle (1,1);
        \end{scope}
        \path[draw, line width = 0.5] (0.5, 0.5) 
            -- (1, 1);
        \path[draw, line width = 0.5] (0.6, 1) 
            -- (1, 1) -- (1, 0.6);
        }
    }

\NewEnviron{aaa}{~\\
    \noindent {\textcolor{teal}{\textbf{解答}} \BODY }
}

\NewEnviron{llll}{
    \noindent {~\\$\ExternalLink$ 外部链接 $\,\,\,$ \color{blue}\url{\BODY} }
}

% \renewcommand{\proofname}{证明}
% \renewcommand\qedsymbol{${\boxed{\substack{\textit{完证}\\\textit{毕明}}}}$}
% \let\oldproof\proof
% \renewcommand{\proof}{\color{blue}\oldproof}




\theoremstyle{definition}
\newtheorem*{definition}{定义}
\newtheorem*{proposition}{命题}
\newtheorem*{theorem}{定理}
\newtheorem*{notation}{记号}
\newtheorem*{example}{例子}
\newtheorem{exercise}{\textcolor{blue}{习题}}
\theoremstyle{remark}
\newtheorem*{remark}{备注}
\newtheorem*{lemma}{引理}
\newtheorem*{solution}{解答}
\newtheorem*{corollary}{推论}

\title{Tilting Theory Walk through}
\author{ZCC}
\date{\today}

\setcounter{page}{0}

\setlength\parindent{0pt}

\begin{document}

\maketitle

\vspace{5cm} 

\begin{abstract}
    本文走馬觀花式地介紹 tilting 理論. 先從藍寶書 \cite{e1} 之以下幾節開始. 
    \begin{center}
        \begin{tabular}{|| c | c | l ||}%
            \bfseries Sec & \bfseries SubSec & Title % specify table head
            \csvreader[head to column names, separator=semicolon]{TOC-e1.csv}{}% use head of csv as column names
            {\\\hline\ \Sec & \SubSec & \Title}% specify your columns here
            \end{tabular}
    \end{center}
    
\end{abstract}


\newpage

\tableofcontents

\newpage


\section{代數基礎}

\subsection{Artin 代數的簡易對象: 單模, Rad, Top, 冪等分解}
\begin{notation}[Artin 代數上的有限表現模]\label{NotofArtinAlg}
    除非單獨强調, 否則行文遵照以下約定. \parnote{Artin $↓$}
    \begin{enumerate}
        \item 默認所有域是代數閉域, 即 $k = \overline k$; 但特徵 $\mathrm{char}(k)$ 未必是零. \parnote{代數閉}
        \item 默認所有代數是有限維的, 但不必交換, 記作範疇 $𝐚𝐥𝐠_k$. \parnote{f.d.}
        \item 默認所有模都是有限生成右模, 記作 $𝐦𝐨𝐝_A$ ($A ∈ 𝐚𝐥𝐠_k$), 左模記作右 $A^{\mathrm{op}}$-模. \parnote{f.g. 右模}
        \item 對於單一的模, 將之視作固定的集合. 此時的\textbf{子模與商模是直接通過集合定義的}, 子模與商模亦可直接談論大小, 並無``同構下唯一''之說. \parnote{子集商集}
        \item 出於習慣, 將商模的子模表述做子模的商模, 稱作子商.
    \end{enumerate}
\end{notation}

\begin{definition}[Jacobson 根]\label{Rad} 
    暫置 $k$ 爲一般交換環. $\mathrm{Rad}(A)$ 等價定義 如下. \parnote{Rad ↓}
\begin{enumerate}
    \item $\mathrm{Rad}(A)$ 是一切極大左理想之交, 亦是一切極大右理想之交. \parnote{理想視角}
    \begin{pinked}
        $\mathrm{Rad}(A)$ 是\textbf{單邊定義的雙邊理想}; 但 $\mathrm{Rad}$ 未必是極大雙邊理的交 (見此 MSE 回答 \cite{775}).
    \end{pinked}
    \item $\mathrm{Rad}(A)$ 由符合以下等價性質的元素 $r$ 組成, \parnote{逆元視角}
    \begin{enumerate}
        \item 對任意 $a ∈ A$, 總有 $1-ar$ 可逆;
        \item 對任意 $a ∈ A$, 總有 $1-ra$ 可逆;
        \item 對任意 $a ∈ A$, 總有 $1-ar$ 存在右逆;
        \item 對任意 $a ∈ A$, 總有 $1-ra$ 存在左逆.
    \end{enumerate}
    \begin{pinked}
        $\mathrm{Rad}(A)$ 由\textbf{所謂的小量}構成. 正如 $x⋅f(x)$ 之於 $ℝ[x]$.
    \end{pinked}
    \item $\mathrm{Rad}(A)$ 是使得 $A / \mathrm{Rad}(A)$ 半單的最小模. \parnote{半單視角}
\end{enumerate}

\end{definition}

\begin{definition}[頂]\label{Top}
    藉由以上第三點, 定義 $\mathrm{Top}(A):=A / \mathrm{Rad}(A)$. 等價地, \parnote{Top ↓}
    \begin{enumerate}
        \item $\mathrm{Top}$ 是 $A$ 的極大半單商環,
        \item $\mathrm{Top}$ 亦是 $A$ 極大半單商模.
        \item \textbf{對交換代數而言}, $\mathrm{Rad}(A)$ 恰是所有冪零對象 (同埋 $0$) 組成的理想. \parnote{reduced}
    \end{enumerate}
    商去 $\mathrm{Rad}$ 所得的半單代數常記作 $A_\mathrm{red}$ ($A$-reduced). 
\end{definition}

\begin{theorem}
    $A ↠ \mathrm{Top}(A)$ 是範疇 $𝐚𝐥𝐠_k$ 的可裂滿. 換言之, 
    \begin{pinked}
        $\mathrm{Top}$ 是半單的截面. (定理 1.6, \cite{e1})
    \end{pinked}
\end{theorem}

\begin{proposition}[環的半單性: Wedderburn-Artin]
    以下是 $A$ 半單的等價定義 (以分號記). \parnote{WA 定理}
\begin{enumerate}
    \item 所有 $𝐦𝐨𝐝_A$ 是半單 $A$-模; 所有 $𝐦𝐨𝐝_{A^{\mathrm{op}}}$ 是半單 $A^{\mathrm{op}}$-模;
    \item $A$ 是半單左 $A$-模; $A$ 是半單左 $A^{\mathrm{op}}$-模;
    \item (作爲 $A$-模, 下同) $\mathrm{Top}(A) = A$; $\mathrm{Rad}(A) = 0$; $\mathrm{Soc}(A) = A$;
    \item $A$ 同構於矩陣乘法環 $∏ 𝕄_{m_i}(k)$.
\end{enumerate}
\begin{pinked}
    半單無關左右之選取, 在允許足夠不變子空間時 (如代數閉域), 一切都是矩陣除環.
\end{pinked}
\end{proposition}

\begin{proposition}[回顧模半單性]\label{semisimple}
    總結以下重要而基本的定理. \parnote{半單定理}
    \begin{enumerate}
        \item 單對象的 Schur 引理: $(S_i , S_j ) ≃ k ⋅ \mathrm{id}⋅ δ _{i,j}$. \parnote{Schur}
        \item Krull-Schmidt 定理: $𝐦𝐨𝐝_A$ 的任何對象唯一分解做不可分解對象的直和. \parnote{KS 範疇}
        \item Jordan-Holder 定理: 合成列良定義. 合成列在同構在允許重數, 相差一個置換的意義下唯一. \parnote{$G_0$-群}
    \end{enumerate} 
\end{proposition}

\begin{example}[冪等分解]
    考慮 $1 ∈ A$ 的兩類冪等分解: 
    \begin{enumerate}
        \item (冪等分解) 存在極大的 $\{e_i\}_{i=1}^n$ 使得 $e_i^2 = e_i$ 恆成立 (同構下唯一). \parnote{頂點 $Q_0$}
        \item (正交冪等分解) 存在極大的 $\{e_i\}_{i=1}^n$ 使得 $e_i^2 = e_i$ 恆成立, 且諸 $e_i$ 乘法交換 (同構下唯一). \parnote{連通分支}
    \end{enumerate}
    \begin{pinked}
        說白了, 就是環的積與模的積之別. 
    \end{pinked}
\end{example}

\begin{remark}
    選定冪零理想 $I ⊆ A$, 則 $A/ I$ 的(正交)冪等元通過商映射 $A ↠ A / I$ 提升.  \parnote{冪等提升}

    Formally smooth algebra (see \cite{BerestMehrle2017}), lifting along irreducible polynomials, lifting of $\triangle$ morphisms (See personal notes $\triangle$ 1.1.1.). 
\end{remark}

\begin{definition}
    給定不可分解對象, 有 $M ≫ \mathrm{Top}(M) ≫ \mathrm{End}(S)$ 與 $M ≫ \mathrm{End}(M) ≫ k(\mathrm{End}(M))$ 兩條路可選. 最後指向是同一剩餘域 (自同態對主體部分貢獻的數乘). 
    稱 $R$ 是局部環, 當且僅當以下等價定義成立. \parnote{局部環 ↓}
    \begin{enumerate}
        \item 存在最大左理想; 存在最大右理想; 
        \item 所有非單位元恰好構成雙邊理想; \parnote{理想視角}
        \item 對任意 $x ∈ R$, 有且僅有 $x$ 可逆或者 $(1-x)$ 可逆 (alternative); \parnote{逆元二擇}
        \item 冪等元只有 $0$ 和 $1$; 
        \item 剩餘域 $A/\mathrm{Rad}(A) ≃ k$. \parnote{視作單點}
    \end{enumerate}
\end{definition}

\begin{example}[從單模到不可分解模]
    局部環幫助檢視不可分解對象 $Ae$ 的自同態環, 因爲這等價於談論 $eAe$ 是局部環. 以下描述等價.
\begin{enumerate}
    \item $M$ 是不可分解模; 
    \item $\mathrm{End}(M)$ 是局部環; 
    \item 所有 $f : M → M$ 是同構或是冪零的. \parnote{Fitting}
\end{enumerate}
\end{example}

\begin{remark}
    不可分解對象與單對象一一對應 (相差一個 reduction). 本質上, $\mathrm{End}(S)$ 形如一維的 Jordan 塊, 但 $\mathrm{End}(M)$ 可以是高維的 Jordan 塊.
\end{remark}








\subsection{Artin 代數的基本對偶: 投射蓋, 內射包}
\begin{definition}[模之 Radical]\label{RadMod}
    給定子模 $N ≤ M$, 以下是 $N = \mathrm{Rad}(M)$ 的等價定義. \parnote{Rad 模}
    \begin{enumerate}
        \item $N = M ⋅ \mathrm{Rad}(A)$; 
        \item $N$ 是 $M$ 的極大子模的交; \parnote{min-max}
        \item $M / N$ 是 $M$ 的極大半單商模. 
    \end{enumerate}
    類似地定義 $\mathrm{Top}(M) := M / \mathrm{Rad}(M)$. 
\end{definition}

\begin{remark}
    $\mathrm{Rad}$ 是 $\mathrm{id}$ 的加法子函子, $\mathrm{Top}$ 是 $\mathrm{id}$ 的加法商函子. 
\end{remark}

\begin{definition}[盈餘]\label{superfluous}  
    稱 $L ⊆ M$ 是盈餘的, 當且僅當對一切 $N ⊆ M$, $(L + N = M) ⟺ (N = M)$. \parnote{中山引理}

    更範疇化的解釋: $L ⊆ M$ 可以左向延拓成子模的 PBPO 方塊, 當且僅當 $N ⊆ M$ 取等號, 即, 
\begin{equation}
    % https://q.uiver.app/#q=WzAsOCxbMSwxLCJNIl0sWzEsMCwiTCJdLFswLDEsIk4iXSxbMCwwLCI/Il0sWzMsMCwiTCJdLFs0LDEsIk0iXSxbMywxLCJNIl0sWzQsMCwiTCJdLFszLDEsIlxcc3Vic2V0ZXEgIiwzLHsic3R5bGUiOnsiYm9keSI6eyJuYW1lIjoibm9uZSJ9LCJoZWFkIjp7Im5hbWUiOiJub25lIn19fV0sWzEsMCwiXFxzdWJzZXRlcSAiLDMseyJzdHlsZSI6eyJib2R5Ijp7Im5hbWUiOiJub25lIn0sImhlYWQiOnsibmFtZSI6Im5vbmUifX19XSxbMywyLCJcXHN1YnNldGVxICIsMyx7InN0eWxlIjp7ImJvZHkiOnsibmFtZSI6Im5vbmUifSwiaGVhZCI6eyJuYW1lIjoibm9uZSJ9fX1dLFsyLDAsIlxcc3Vic2V0ZXEgIiwzLHsic3R5bGUiOnsiYm9keSI6eyJuYW1lIjoibm9uZSJ9LCJoZWFkIjp7Im5hbWUiOiJub25lIn19fV0sWzMsMCwiXFxzdWJzdGFja3tcXHRleHR7UEJ9XFxcXFxcdGV4dHtQT319IiwxLHsic3R5bGUiOnsiYm9keSI6eyJuYW1lIjoibm9uZSJ9LCJoZWFkIjp7Im5hbWUiOiJub25lIn19fV0sWzcsNSwiXFxzdWJzZXRlcSIsMyx7InN0eWxlIjp7ImJvZHkiOnsibmFtZSI6Im5vbmUifSwiaGVhZCI6eyJuYW1lIjoibm9uZSJ9fX1dLFs0LDYsIlxcc3Vic2V0ZXEiLDMseyJzdHlsZSI6eyJib2R5Ijp7Im5hbWUiOiJub25lIn0sImhlYWQiOnsibmFtZSI6Im5vbmUifX19XSxbNCw3LCIiLDMseyJsZXZlbCI6Miwic3R5bGUiOnsiaGVhZCI6eyJuYW1lIjoibm9uZSJ9fX1dLFs2LDUsIiIsMyx7ImxldmVsIjoyLCJzdHlsZSI6eyJoZWFkIjp7Im5hbWUiOiJub25lIn19fV0sWzksMTQsIiIsMyx7InNob3J0ZW4iOnsic291cmNlIjozMCwidGFyZ2V0IjozMH0sInN0eWxlIjp7InRhaWwiOnsibmFtZSI6ImFycm93aGVhZCJ9fX1dXQ==
\begin{tikzcd}[ampersand replacement=\&]
	{?} \& L \&\& L \& L \\
	N \& M \&\& M \& M
	\arrow["{\subseteq }"{marking, allow upside down}, draw=none, from=1-1, to=1-2]
	\arrow["{\subseteq }"{marking, allow upside down}, draw=none, from=1-1, to=2-1]
	\arrow["\begin{array}{c} \substack{\text{PB}\\\text{PO}} \end{array}"{description}, draw=none, from=1-1, to=2-2]
	\arrow[""{name=0, anchor=center, inner sep=0}, "{\subseteq }"{marking, allow upside down}, draw=none, from=1-2, to=2-2]
	\arrow[equals, from=1-4, to=1-5]
	\arrow[""{name=1, anchor=center, inner sep=0}, "\subseteq"{marking, allow upside down}, draw=none, from=1-4, to=2-4]
	\arrow["\subseteq"{marking, allow upside down}, draw=none, from=1-5, to=2-5]
	\arrow["{\subseteq }"{marking, allow upside down}, draw=none, from=2-1, to=2-2]
	\arrow[equals, from=2-4, to=2-5]
	\arrow[shorten <=19pt, shorten >=19pt, Rightarrow, 2tail reversed, from=0, to=1]
\end{tikzcd}.
\end{equation}
\end{definition}

\begin{remark}
    $\mathrm{Rad}(M)$ 是最大的盈餘模, 類似``皇次子'' (組合的第二比較量, 第一處 syzygy, 等等). 
\end{remark}

\begin{example}[更精細的滿-單分解]\label{Topggid}
    模的表現由生成元與生成關係決定. 在本文約定下, 任意模都是某一 $A^n$ 的商. 容易發現
    \begin{pinked}
        $f : M → N$ 是滿射, 當且僅當誘導的 $\mathrm{Top}(f) : \mathrm{Top}(M) → \mathrm{Top}(N)$ 是滿射. 
    \end{pinked}\parnote{Top:\\保, 返滿}
    滿-單分解之``滿'', 可以分解作生成元的滿射(先)與生成關係的滿射(後). 如極小投射表現, 談論這一分解的函子性暫無意義. 
\end{example}

\begin{definition}[極小滿態射]
    稱滿態射 $p : X ↠ Y$ 是極小满态射, 當且僅當以下等價條件成立. \parnote{極小滿}
    \begin{enumerate}
        \item 對任意 $f : A → X$, 複合 $p ∘ f$ 滿當且僅當 $f$ 滿. \parnote{$p ∘(-)$ 保, 返滿}
        \item $\ker p ⊆ X$ 是 superfluous 的子模.
        \item 上述``滿-滿-單''分解必然是``同構-滿-同構''. \parnote{僅商去生成關係}
    \end{enumerate}
\end{definition}

\begin{definition}[投射蓋] 
    稱 $P(M) ↠ M$ 是投射蓋, 當且僅當這是投射模出發的極小滿態射.
    \begin{enumerate}
        \item 投射蓋 $=$ 極少生成元自由張成的模. 此處的極小生成元可以選取 $\mathrm{Top}(M)$ 的提升.
        \item 若 $Q ↠ M$ 是投射模出發的滿態射, 則 $P(M)$ 是 $Q$ 的直和项.
        \item 將 $M$ 視作 $P(M)$ 的商集, 則 $\mathrm{Top}(M) = \mathrm{Top}(P(M))$.  
        \item $P = P(M)$, 當且僅當 $M$ 同構於 $P ≥ M ≥ \mathrm{Top}(P)$. \parnote{極大被商模}
    \end{enumerate}

\begin{pinked}
    投射蓋即最小的待商的投射模, 通過 $\mathrm{Top}$ 定位. 商去的子對象均是組合意義下的小量, 含於 $\mathrm{Rad}$.
\end{pinked}
\end{definition}

\begin{definition}[兩大基本對偶]\label{Ddual}\label{tdual}
    $(-, A)_A, (-, k)_k : 𝐦𝐨𝐝_A → 𝐦𝐨𝐝_{A^{\mathrm{op}}}$ 是反變函子. 
\end{definition}

\begin{remark}
    簡單地記 $D := (-, k)_k$. 不區分 $D$ 與 $D^{\mathrm{op}}$. \parnote{$D$}
\end{remark}

\begin{definition}[餘盈餘]\label{cosuperfluous}
    稱商模 $M \overset{-/∼} ↠$ 是餘盈餘的, 當且僅當商集的 PBPO 方塊蘊含等號: 
    \begin{equation}
        % https://q.uiver.app/#q=WzAsOCxbMSwxLCI/Il0sWzEsMCwiTiJdLFswLDEsIkwiXSxbMCwwLCJNIl0sWzMsMCwiTSJdLFs0LDEsIkwiXSxbMywxLCJMIl0sWzQsMCwiTSJdLFszLDEsIlxcb3ZlcnNldCB7LS9cXHNpbX0gXFx0d29oZWFkcmlnaHRhcnJvdyIsMyx7InN0eWxlIjp7ImJvZHkiOnsibmFtZSI6Im5vbmUifSwiaGVhZCI6eyJuYW1lIjoibm9uZSJ9fX1dLFsxLDAsIlxcb3ZlcnNldCB7LS9cXHNpbX0gXFx0d29oZWFkcmlnaHRhcnJvdyIsMyx7InN0eWxlIjp7ImJvZHkiOnsibmFtZSI6Im5vbmUifSwiaGVhZCI6eyJuYW1lIjoibm9uZSJ9fX1dLFszLDIsIlxcb3ZlcnNldCB7LS9cXHNpbX0gXFx0d29oZWFkcmlnaHRhcnJvdyIsMyx7InN0eWxlIjp7ImJvZHkiOnsibmFtZSI6Im5vbmUifSwiaGVhZCI6eyJuYW1lIjoibm9uZSJ9fX1dLFsyLDAsIlxcb3ZlcnNldCB7LS9cXHNpbX0gXFx0d29oZWFkcmlnaHRhcnJvdyIsMyx7InN0eWxlIjp7ImJvZHkiOnsibmFtZSI6Im5vbmUifSwiaGVhZCI6eyJuYW1lIjoibm9uZSJ9fX1dLFszLDAsIlxcc3Vic3RhY2t7XFx0ZXh0e1BCfVxcXFxcXHRleHR7UE99fSIsMSx7InN0eWxlIjp7ImJvZHkiOnsibmFtZSI6Im5vbmUifSwiaGVhZCI6eyJuYW1lIjoibm9uZSJ9fX1dLFs3LDUsIlxcb3ZlcnNldCB7LS9cXHNpbX0gXFx0d29oZWFkcmlnaHRhcnJvdyIsMyx7InN0eWxlIjp7ImJvZHkiOnsibmFtZSI6Im5vbmUifSwiaGVhZCI6eyJuYW1lIjoibm9uZSJ9fX1dLFs0LDYsIlxcb3ZlcnNldCB7LS9cXHNpbX0gXFx0d29oZWFkcmlnaHRhcnJvdyIsMyx7InN0eWxlIjp7ImJvZHkiOnsibmFtZSI6Im5vbmUifSwiaGVhZCI6eyJuYW1lIjoibm9uZSJ9fX1dLFs0LDcsIiIsMyx7ImxldmVsIjoyLCJzdHlsZSI6eyJoZWFkIjp7Im5hbWUiOiJub25lIn19fV0sWzYsNSwiIiwzLHsibGV2ZWwiOjIsInN0eWxlIjp7ImhlYWQiOnsibmFtZSI6Im5vbmUifX19XSxbOSwxNCwiIiwzLHsic2hvcnRlbiI6eyJzb3VyY2UiOjMwLCJ0YXJnZXQiOjMwfSwic3R5bGUiOnsidGFpbCI6eyJuYW1lIjoiYXJyb3doZWFkIn19fV1d
\begin{tikzcd}[ampersand replacement=\&]
	M \& N \&\& M \& M \\
	L \& {?} \&\& L \& L
	\arrow["{\overset {-/\sim} \twoheadrightarrow}"{marking, allow upside down}, draw=none, from=1-1, to=1-2]
	\arrow["{\overset {-/\sim} \twoheadrightarrow}"{marking, allow upside down}, draw=none, from=1-1, to=2-1]
	\arrow["\begin{array}{c} \substack{\text{PB}\\\text{PO}} \end{array}"{description}, draw=none, from=1-1, to=2-2]
	\arrow[""{name=0, anchor=center, inner sep=0}, "{\overset {-/\sim} \twoheadrightarrow}"{marking, allow upside down}, draw=none, from=1-2, to=2-2]
	\arrow[equals, from=1-4, to=1-5]
	\arrow[""{name=1, anchor=center, inner sep=0}, "{\overset {-/\sim} \twoheadrightarrow}"{marking, allow upside down}, draw=none, from=1-4, to=2-4]
	\arrow["{\overset {-/\sim} \twoheadrightarrow}"{marking, allow upside down}, draw=none, from=1-5, to=2-5]
	\arrow["{\overset {-/\sim} \twoheadrightarrow}"{marking, allow upside down}, draw=none, from=2-1, to=2-2]
	\arrow[equals, from=2-4, to=2-5]
	\arrow[shorten <=19pt, shorten >=19pt, Rightarrow, 2tail reversed, from=0, to=1]
\end{tikzcd}.
    \end{equation}
    換言之, 稱商模 $M ↠ N$ 餘盈餘, 當且僅當 $(\mathrm{lcm}(N,L) = M) ⟺ (N = M)$.  
\end{definition}

\begin{pinked}
    盈餘子模: 雞肋的子對象 (生成元); 餘盈餘商模: 雞肋的商對象 (等價關係). 
\end{pinked}

\begin{definition}[本性擴張]
    若將餘盈餘模視作``內部滿射'', 對應的``內部單射 ($\ker$)''稱作本性擴張. 
\end{definition}

\begin{theorem}
    $K ⊆ M$ 是本性擴張, 當且僅當 $M$ 的任意非零子模與 $K$ 恆有非零交. \parnote{對等表述: 本性擴張, 餘盈餘}
    \begin{proof}
        $⇒$ 向: 假定 $K ⊆ M$ 本性擴張, 且存在非零子模 $M_0 ⊆ M$ 使得 $K ∩ M_0 = 0$, 則 $% https://q.uiver.app/#q=WzAsNCxbMCwwLCJNIl0sWzEsMCwiXFxmcmFjIE1LIl0sWzAsMSwiXFxmcmFjIE0ge01fMH0iXSxbMSwxLCJcXGZyYWMgTSB7SyArIE1fMH0iXSxbMCwxXSxbMCwyXSxbMSwzXSxbMiwzXV0=
        \begin{tikzcd}[ampersand replacement=\&, sep = small]
            M \& {\frac MK} \\
            {\frac M {M_0}} \& {\frac M {K + M_0}}
            \arrow[from=1-1, to=1-2]
            \arrow[from=1-1, to=2-1]
            \arrow[from=1-2, to=2-2]
            \arrow[from=2-1, to=2-2]
        \end{tikzcd}$ 是推出拉回, 此時 $M_0 = 0$ 導出矛盾. $⇐$ 向: 假定 $M$ 的任意非零子模與 $K$ 恆有非零交, 但存在推出拉回方塊 $% https://q.uiver.app/#q=WzAsNCxbMCwwLCJNIl0sWzEsMCwiXFxmcmFjIE1LIl0sWzAsMSwiXFxmcmFjIE0gQSJdLFsxLDEsIlxcZnJhYyBNIHtCfSJdLFswLDFdLFswLDIsIlxcbmVxICIsMl0sWzEsM10sWzIsM11d
        \begin{tikzcd}[ampersand replacement=\&, sep = small]
            M \& {\frac MK} \\
            {\frac M A} \& {\frac M {B}}
            \arrow[from=1-1, to=1-2]
            \arrow["{\neq }"', from=1-1, to=2-1]
            \arrow[from=1-2, to=2-2]
            \arrow[from=2-1, to=2-2]
        \end{tikzcd}$, 依照商集結構知 $K ∩ A = 0$, 此時 $K$ 與非零子模 $A$ 有零交, 矛盾. 
    \end{proof}
    \begin{pinked}
        簡單地說, 餘盈餘恰是本性擴張的商!
    \end{pinked}
\end{theorem}

\begin{remark}
    常用例子: $ℤ ⊆ ℚ$ 是 $ℤ$-模範疇的本性擴張. 作爲收尾, \textbf{內射模的本性擴張是平凡的}. 
\end{remark}

\begin{definition}[模之 $\mathrm{Soc}$]
    $\mathrm{Soc}$ 是與 $\mathrm{Top}$ 對標的概念: 極大半單子模, 即所有不可分解對象的極大半單子的直和. 
\end{definition}

\begin{pinked}
    \begin{remark}
        改用對等表述: 極大半單子模, 就是極小本性子模. 
    \end{remark}
\end{pinked}

\begin{definition}[極小單態射, 內射包等]
    仿照 \ref{Topggid}, $\mathrm{Soc}(f)$ 單當且僅當 $f$ 單. 極小滿態射 $i$ 的等價定義: 
    \begin{enumerate}
        \item $\xrightarrow i \ \xrightarrow f$ 單, 當且僅當 $f$ 單. 
        \item $i$ 視作子模包含, 是本性擴張 (等價地, $\mathrm{coker}$ 餘盈餘).
        \item 類似地, ``滿-單-單''分解必然是``同構-單-同構''. 
    \end{enumerate}
    稱 $M ↪ I(M)$ 是內射包, 當且僅當這是指向內射模的極小單態射. 跨度 $\mathrm{Soc}(M) ⊆ ? ⊆ I(M)$. \parnote{極大本性擴張}
\end{definition}

\begin{theorem}\label{Nakayama}
    不可分解投射對象, 不可分解内射對象, 單對象一一對應. 
    \begin{equation}
        % https://q.uiver.app/#q=WzAsMyxbMCwwLCJcXG1hdGhybXtpbmp9Il0sWzQsMCwiXFxtYXRocm17cHJvan0iXSxbMiwwLCJcXG1hdGhybXtzaW19Il0sWzAsMiwiXFxtYXRocm17U29jfSIsMCx7Im9mZnNldCI6LTV9XSxbMSwyLCJcXG1hdGhybXtUb3B9IiwyLHsib2Zmc2V0Ijo1fV0sWzIsMSwiUCgtKSIsMix7Im9mZnNldCI6NX1dLFsyLDAsIkkoLSkiLDAseyJvZmZzZXQiOi01fV1d
\begin{tikzcd}[ampersand replacement=\&,sep=small]
	{\mathrm{inj}} \&\& {\mathrm{sim}} \&\& {\mathrm{proj}}
	\arrow["{\mathrm{Soc}}", shift left=5, from=1-1, to=1-3]
	\arrow["{I(-)}", shift left=5, from=1-3, to=1-1]
	\arrow["{P(-)}"', shift right=5, from=1-3, to=1-5]
	\arrow["{\mathrm{Top}}"', shift right=5, from=1-5, to=1-3]
\end{tikzcd}.
    \end{equation}
    中山函子 $ν = D(-,A) ≃ - ⊗ DA : \mathrm{End}(𝐦𝐨𝐝 _A)$ 直接對換不可分解投射模與內射模. \parnote{$ν$ 正合}  
\end{theorem}

\begin{definition}[基礎代數]
    將 $A$ 寫作不可分解投射對象的直和, 當且僅當直和項彼此不同構, 稱 $A$ 爲基礎代數. 
\end{definition}

\begin{remark}
    思想: 扔掉重數. 構造: $A$ 變成 $A^e := ∑ e_i A e_i$. 函子視角: Morita 等價, 模都一樣: \parnote{方便畫 quiver}
    \begin{equation}
        (- ⊗ Ae_r) : 𝐦𝐨𝐝_A ≃  𝐦𝐨𝐝 _{A^e} : (- ⊗ e_r A). 
    \end{equation}
\end{remark}

\subsection{雜七雜八: Quiver 表示, 維度公式, [待歸類至別處]}
\begin{definition}[箭圖]
    略. 有限型, 有限, 以及 $𝐫𝐞𝐩(Q, I) ≃ 𝐦𝐨𝐝_{kQ/I}$ \parnote{日後補充}
\end{definition}

\begin{definition}[容許理想]\label{admissibleideal}
    容許理想 (Admissible ideal), 即增速可控的理想, 形如 $\mathrm{Rad}^2 ⊆ I ⊆ \mathrm{Rad}^k$ ($∃ n$). 
\end{definition}

\begin{remark}
    換言之, 生成關係是由``長 $≥2$ 的邊''決定. 
\end{remark}

\begin{theorem}[Artin 代數實現作 $(Q,I)$ 的方式]
    給定 $A$. 不妨設 $A$ 是基礎代數, 同時聯通. \parnote{造 $(kQ, I)$}
    \begin{enumerate}
        \item (點) 取正交冪等分解 $A = ⨁ A e_i$, 
        \item (邊) 找到 $\mathrm{Rad}(A) / \mathrm{Rad}^2(A)$ 的一組基, 
        \item (rel) 依照需求, 對前兩步決定的 quiver without rel 進行商. 
    \end{enumerate}
    特別地, 
    \begin{enumerate}
        \item 不可分解投射模 $P(i)$ 是 $i$ 出發的路代數;
        \item 不可分解內射模 $I(i)$ 是 $i$ 收尾的路代數 (投射的商);
        \item 單模 $S(i)$ 是 $i$ 單點所示的路代數; 
        \item $\mathrm{Rad}$ 是路理想 $kQ_1$-模.
    \end{enumerate}
\end{theorem}

\begin{theorem}[$𝐫𝐞𝐩(Q, I)$ 模結構]
    給定 $M ∈ 𝐫𝐞𝐩(Q, I)$, 則
    \begin{enumerate}
        \item 當且僅當 $M$ 半單, 則 $φ_α=0$ 對一切 $α ∈ Q_1$ 成立. \parnote{半單 $⟺$ 無邊}
        \item $\mathrm{Soc}(M)$ 在第 $a ∈ Q_0$ 位的分量是 $⋂_{f : a → ?} \ker (f)$. 特別地, $⋂_∅ = 0$. 
        \begin{pinked}
            極大單子對象, 即向外箭頭之公共核.
        \end{pinked}
        \item $\mathrm{Top}(M)$ 在第 $a ∈ Q_0$ 位的分量是 $⋀_{g : ? → a} \mathrm{coker}(g)$. 特別地, $⋀_∅ = \text{全}$. 此處商模之 $∧$ 即推出 (等價關係之并). $φ$ 由對象的取法誘導, 自然是 $0$.
        \begin{pinked}
            極大單商對象, 即向內箭頭之公共等價類.
        \end{pinked}
        \item $\mathrm{Rad}(M)$ 在第 $a ∈ Q_0$ 位的分量是 $∑_{h : ? → a} \mathrm{im}(h)$. 特別地, $∑_∅ = 0$.
        \begin{pinked}
            根即入勢之公共像, 亦可視爲 $kQ_1$ 誘導的東西.
        \end{pinked}
    \end{enumerate}
\end{theorem}

\begin{proposition}[擴張]
    $\dim \mathrm{Ext}^1 (S(a), S(b))$ 是 $a$ 至 $b$ 的邊數. 實現方式: 
    \begin{equation}
        \mathrm{Ext}{^1}(S(a), S(b)) ≃ \left([0 → S(b) → E → S(a) → 0] / ∼\right) ≃ e_a ⋅ \frac{\mathrm{Rad}(A)}{\mathrm{Rad}^2(A)} ⋅ e_b.
    \end{equation}
\end{proposition}













\subsection{簡單的 AR 理論}
\begin{abstract}
    使用不可約態射, 左右極小幾乎可裂態射, AR 平移三種表述以刻畫\textbf{幾乎可裂短正合列}. 最後使用函子視角, 對已知的語言重述某些``不明所以然''的概念. 所有未給出的證明都能在 \cite{e1}, \cite{auslander1997representation} 中找到, 具體頁碼暫時從略. 
\end{abstract}

\subsubsection{範疇的 Rad, 不可約態射}

\begin{proposition}
    同環態射, 範疇之 $k$-函子保持 $\mathrm{Rad}$. 特別地, 有以下特例. \parnote{等同 \ref{Rad}}
    \begin{enumerate}
        \item $\mathrm{Rad}(X,X)$ 就是 $\mathrm{Rad}(\mathrm{End}(X))$. 
        \item 若 $\mathrm{End}(X)$ 與 $\mathrm{End}(Y)$ 是局部環, 則 $\mathrm{Rad}(X, Y)$ 恰是 $X → Y$ 的非同構. 
        \item 若 $\mathrm{End}(X)$ 與 $\mathrm{End}(Y)$ 是局部環, 且 $X ≇ Y$, 則 $\mathrm{Rad}(X, Y) = \mathrm{Hom}(X,Y)$.
    \end{enumerate}
簡單地說, $\mathrm{Rad}(X, Y) = \mathrm{Rad}(Y,Y) ∘ \mathrm{Hom}(X,Y) ∘ \mathrm{Rad}(X,X)$. 
\end{proposition}

\begin{remark}
    $\mathrm{End}(X)$ 局部 $⇒$ $X$ 不可分解; 反之, 通常需要 $𝒞$ Abel 以及 $\dim_k \mathrm{End}(X) < ∞$ 等條件. \parnote{$𝐦𝐨𝐝 _A$ 好}
\end{remark}

\begin{definition}[不可約態射]
    約定 $\mathrm{Rad} := \mathrm{Rad}_A := \mathrm{Rad}_{𝐦𝐨𝐝_A}$, 歸納地定義 
    \begin{enumerate}
        \item $\mathrm{Rad}^0 = \mathrm{Hom}$, $\mathrm{Rad}^1 = \mathrm{Rad}$, 以及
        \item $\mathrm{Rad}^{n+1} = \mathrm{Rad} ∘ \mathrm{Rad}^n$. \parnote{路}
    \end{enumerate}
\end{definition}

\begin{example}[$\mathrm{Rad}$ 的結構]
    定 $X$ 與 $Y$ 不可分解. 
    \begin{enumerate}
        \item 若 $X = Y$, 則 $\mathrm{Rad}(X, Y)$ 是 $\mathrm{Rad}(\mathrm{End}(X))$; 
        \item 若 $X ≇ Y$, 則 $\mathrm{Rad}(X, Y) = \mathrm{Hom}(X, Y)$; 
        \item 假定 $X$ 不可分解, 則 $X \xrightarrow f Y$ 非可裂單, 當且僅當任意 $X \xrightarrow f Y\xrightarrow g X$ 屬於 $\mathrm{Rad}(\mathrm{End}(X))$; \parnote{$\mathrm{Rad}$ 吸收非可裂單, 滿的特定復合}
        \item 假定 $Y$ 不可分解, 則 $X \xrightarrow f Y$ 非可裂滿, 當且僅當任意 $Y \xrightarrow g X\xrightarrow f Y$ 屬於 $\mathrm{Rad}(\mathrm{End}(Y))$; 
        \item $f ∈ (\mathrm{Rad}(X, Y) \backslash \mathrm{Rad}^2(X, Y))$ (稱作``不可約態射'') 具有以下性質: \parnote{不可約態射}
        \begin{enumerate}
            \item $f$ 既不是可裂單, 又不是可裂滿; 
            \item 對任何分解 $X → Z → Y$, 或前箭頭是可裂單, 或後箭頭是可裂滿.
        \end{enumerate}
        總之, 若 $X → Z$ 可裂單, 則 $X → Z → X$ 相當于回頭路. 不可約態射 = 不能分作兩段單向路. 
    \end{enumerate}
\end{example}

\begin{theorem}[不可約單 (滿) 態射的結構]\label{irrmonoepi}
    給定不可裂的正合列 $0 → L \overset f → M \overset g → N → 0$. 
    \begin{enumerate}
        \item $f$ 是不可約的, 當且僅當任意 $? → N$ 分解 $g$, 或被 $g$ 分解. 推論: $N$ 不可分解. \parnote{$g$ 幾乎可裂滿}
        \item $g$ 是不可約的, 當且僅當任意 $L → ?$ 分解 $f$, 或被 $f$ 分解. 推論: $L$ 不可分解. \parnote{$f$ 幾乎可裂單}
    \end{enumerate}
\end{theorem}

\begin{pinked}
    \begin{remark}
        幾乎可裂單 (滿): 僅次於可裂單 (滿). $a$ 的不可約, 對應 $b$ 的幾乎可裂性. 
    \end{remark}
\end{pinked}\parnote{交叉表述} 

\subsubsection{極小態射}

\begin{definition}[左 (右) 極小態射]
    大概會有
    \begin{equation}
        0 → L \xrightarrow{\text{左極小, 左幾乎可裂}} M \xrightarrow{\text{右極小, 右幾乎可裂}} N → 0.
    \end{equation}
    . 此 ses 僅供輔助記憶, 實際上未必有單, 滿的假定. 
    \begin{enumerate}
        \item 稱 $f : L → M$ 是左極小的, 若 $M$ 的自同構 $α$ 方使 $α ∘ f = f$. \parnote{若吸收了``大地方''某自同態, 則該自同態必爲同構}
        \item 稱 $g : M → N$ 是右極小的, 若 $M$ 的自同構 $α$ 方使 $g ∘ α = g$. 
        \item 稱 $f : L → M$ 是左幾乎可裂的, 若 $f$ 非可裂滿, 且任意非可裂滿 $L → ?$ 必經 $f$ 分解; \parnote{如幾乎可裂單}
        \item 稱 $g : M → N$ 是右幾乎可裂的, 若 $g$ 非可裂單, 且任意非可裂單 $? → N$ 必經 $g$ 分解;  \parnote{如幾乎可裂滿}
    \end{enumerate}
\end{definition}

\begin{proposition}
    對 Abel 範疇, 以下是左 (右) 幾乎可裂態射單 (滿) 的充要條件:
    \begin{enumerate}
        \item 若 $g : M → N$ 是右幾乎可裂態射, 則 $g$ 滿當且僅當 $N$ 非投射對象; 
        \item 若 $f : L → M$ 是左幾乎可裂態射, 則 $f$ 單當且僅當 $K$ 非投射對象. 
    \end{enumerate}
    \begin{proof}
        僅看第一者, 假定 $g$ 是右幾乎可裂態射. 若 $g$ 滿, 則 $g$ 非可裂滿, 從而 $N$ 非投射. 若 $g$ 非滿, 下證 $N$ 投射. 
        \begin{adjustwidth}{10pt}{}
            (使用反證法) 假定 $N$ 非投射, 則存在滿態射 $ψ : A ↠ B$ 使得 $N$ 無法提升之. 提升性等價於 $ψ$ 的拉回 (記作 $α$) 可裂. 由極小幾乎可裂的定義知, 存在 $φ$ 使得下圖交換: 
            \begin{equation}
% https://q.uiver.app/#q=WzAsNyxbMSwyLCJBIl0sWzIsMiwiQiJdLFsyLDEsIk4iXSxbMCwwLCJNIl0sWzEsMSwiXFxidWxsZXQiXSxbMywxLCJcXG1hdGhybXtjb2tlcn0oZikiXSxbNCwxLCIwIl0sWzAsMSwiXFxwc2kgIiwyLHsic3R5bGUiOnsiaGVhZCI6eyJuYW1lIjoiZXBpIn19fV0sWzMsMiwiZiIsMCx7ImN1cnZlIjotMn1dLFsyLDFdLFs0LDIsIlxcYWxwaGEgIiwwLHsic3R5bGUiOnsiYm9keSI6eyJuYW1lIjoiZGFzaGVkIn19fV0sWzQsMCwiIiwxLHsic3R5bGUiOnsiYm9keSI6eyJuYW1lIjoiZGFzaGVkIn19fV0sWzQsMSwiXFx0ZXh0e1BCfSIsMSx7InN0eWxlIjp7ImJvZHkiOnsibmFtZSI6Im5vbmUifSwiaGVhZCI6eyJuYW1lIjoibm9uZSJ9fX1dLFs0LDMsIlxcdmFycGhpICJdLFsyLDVdLFs1LDYsIlxcbmVxICIsMSx7InN0eWxlIjp7ImJvZHkiOnsibmFtZSI6Im5vbmUifSwiaGVhZCI6eyJuYW1lIjoibm9uZSJ9fX1dXQ==
\begin{tikzcd}[ampersand replacement=\&,sep=small]
	M \\
	\& \bullet \& N \& {\mathrm{coker}(f)} \& 0 \\
	\& A \& B
	\arrow["f", curve={height=-12pt}, from=1-1, to=2-3]
	\arrow["{\varphi }", from=2-2, to=1-1]
	\arrow["{\alpha }", dashed, from=2-2, to=2-3]
	\arrow[dashed, from=2-2, to=3-2]
	\arrow["{\text{PB}}"{description}, draw=none, from=2-2, to=3-3]
	\arrow[from=2-3, to=2-4]
	\arrow[from=2-3, to=3-3]
	\arrow["{\neq }"{description}, draw=none, from=2-4, to=2-5]
	\arrow["{\psi }"', two heads, from=3-2, to=3-3]
\end{tikzcd}.
            \end{equation}
            由 $α$ 是滿態射, $∙ → N → \mathrm{coker}(f)$ 亦滿. 這一復合是零態射. 從而 $\mathrm{coker}(f)=0$, 矛盾. 
        \end{adjustwidth}
    \end{proof}
\end{proposition}

\begin{example}
    對 Artin 代數的設定, 右極小幾乎可裂態射要麼是非投射模的投射蓋, 要麼是投射模 $P$ 的單態射 $\mathrm{Rad}(P) ↪ P$. 另一方向同理. \parnote{極小幾乎可裂, 二擇}
\end{example}

\begin{theorem}[不可約 v.s. 極小幾乎可裂]
    左 (右) 極小幾乎可裂態射的來源 (去向) 不可分解. 特別地, 
    \begin{enumerate}
        \item 若 $f : L → M$ 左極小幾乎可裂, 則 $f$ 不可約; 
        \item 若 $f : L → M$ 不可約 ($M ≠ 0$), 當且僅當 $f$ 可以補全作左極小幾乎可裂態射 $\binom{f}{g} : L → M ⊕ \overline M$; 
        \item 若 $g : M → N$ 不可約 ($N ≠ 0$), 當且僅當 $g$ 可以補全作右極小幾乎可裂態射 $(g,h) : M ⊕ \widetilde M → N$. 
    \end{enumerate}
    \begin{pinked}
        極小幾乎可裂 $⊆$ 不可約; 不可約可反向補全作極小幾乎可裂態射. 此處沒有單射滿射的限定!
    \end{pinked}
\end{theorem}

\subsubsection{AR 平移, \texorpdfstring{$τ$}{PDFstring}}

\begin{definition}[AR 轉置 $\mathrm{Tr}(-)$]\label{Tr}
    考慮極小投射表現 $M = \mathrm{coker}(P_1 \xrightarrow f P_0)$, 定義 $\mathrm{Tr}(M) := \mathrm{coker}(f^t)$. 
    \begin{equation}
        % https://q.uiver.app/#q=WzAsMTAsWzIsMCwiUF8xICJdLFszLDAsIlBfMCJdLFs0LDAsIk0iXSxbNSwwLCIwIl0sWzUsMSwiMCJdLFs0LDEsIk1edCAiXSxbMywxLCJQXzEgXnQiXSxbMiwxLCJQXzEgXnQiXSxbMSwxLCJcXG1hdGhybXtUcn0oTSkiXSxbMCwxLCIwIl0sWzAsMSwiZiJdLFsxLDJdLFsyLDNdLFs0LDVdLFs1LDZdLFs2LDcsImZedCIsMl0sWzcsOF0sWzgsOV1d
\begin{tikzcd}[ampersand replacement=\&,sep=small]
	\&\& {P_1 } \& {P_0} \& M \& 0 \\
	0 \& {\mathrm{Tr}(M)} \& {P_1 ^t} \& {P_0 ^t} \& {M^t } \& 0
	\arrow["f", from=1-3, to=1-4]
	\arrow[from=1-4, to=1-5]
	\arrow[from=1-5, to=1-6]
	\arrow[from=2-2, to=2-1]
	\arrow[from=2-3, to=2-2]
	\arrow["{f^t}"', from=2-4, to=2-3]
	\arrow[from=2-5, to=2-4]
	\arrow[from=2-6, to=2-5]
\end{tikzcd}.
    \end{equation}\parnote{$\mathrm{Tr}(\mathrm{cok}(f))$ → $\mathrm{cok}(f^t)$}
\end{definition}

\begin{proposition}[$\mathrm{Tr}$ 基本性質]
    $\mathrm{Tr}$ 與直和交換. 以下假定 $M$ 不可分解. 
    \begin{enumerate}
        \item 當且僅當 $M$ 投射, $\mathrm{Tr}(M) = 0$; 
        \item 若 $M$ 非投射, 則 $\mathrm{Tr}(\mathrm{Tr}(M)) ≃ M$ 是典範同構. 
    \end{enumerate}
\end{proposition}

\begin{pinked}
    \begin{remark}
        $\mathrm{Tr}(-)$ 暫時沒有函子性 (畢竟 $\mathrm{Tr}(P)=0$), 需要藉助穩定範疇. 
    \end{remark}
\end{pinked}

\begin{definition}[AR 平移]
    給定極小投射表現 $θ : P_1 → P_0 → M → 0$, 考慮 $D(θ^t)$, 得, 
    \begin{equation}
        0 → τ(M) → ν (P_1) → ν (P_0) → ν (M) → 0. 
    \end{equation}\parnote{$τ = D ∘ \mathrm{Tr}$}
    類似地, 對極小內射表現 $η : 0 → M → E_1 → E_0$, 考慮 $τ⁻¹ ; \mathrm{Tr} ∘ D$, 得 
    \begin{equation}
        0 → ν^{-1}(M) → ν ^{-1}(E_1) → ν ^{-1}(E_0) → τ ^{-1}(M) → 0 . 
    \end{equation}
\end{definition}

\begin{remark}
    中山函子對偶投射模與內射模; AR 平移對偶極小投射表現與極小內射表現? \parnote{待解釋...}
\end{remark}

\subsubsection{幾乎可裂短正合列}

\begin{theorem}
    給定不可裂的短正合列 
\begin{equation}
    0 → L \xrightarrow f M \xrightarrow g N → 0. 
\end{equation}
以下是幾乎可裂的等價命題. 
\begin{enumerate}
    \item $f$ 左極小幾乎可裂, 且 $g$ 右極小幾乎可裂; 
    \item $f$ 左極小幾乎可裂; $g$ 右極小幾乎可裂; \parnote{單邊}
    \item $f$ 左幾乎可裂, 且 $N$ 不可分解; $L$ 不可分解, 且 $g$ 右幾乎可裂. 
    \item $L$ 与 $N$ 不可分解, $f$ 与 $g$ 不可约. 
\end{enumerate}
\end{theorem}

\begin{proposition}\label{PreAR}
    函子性定理: 若 $0 → L₀ → M₀ → N₀ → 0$ 是另一幾乎可列短正合列, 则
\begin{equation}
    \text{正合列同构} ⟺ (L ≃ L₀) ⟺ (N ≃ N₀).
\end{equation}
實際上, 幾乎可裂短正合列必形如以下: \parnote{$τ (\mathrm{Top}(M)) = \mathrm{Rad}(M)$}
\begin{equation}
    0 → N → ? → τ ⁻¹ N → 0\quad (\mathrm{Ext}^1(τ ⁻¹ N, N) ≃ k).
\end{equation} 
\end{proposition}

\begin{theorem}[中項結構]
    假定 $L$ 不可約, 則 $L → ⨁ M_i^{⊕ n_i}$ 左極小幾乎可裂等價於以下同時成立.
    \begin{enumerate}
        \item 所有分量 $L →  M_i$ 屬於 $\mathrm{Rad}$. 
        \item 若有不可分解 $M′$ 使得 $\mathrm{Rad}(L, M′) ≠ 0$, 則 $M′$ 是直和項. \parnote{恰 Rad}
        \item 對任意 $i$, 以上 $n_i$ 個態射恰是 $\frac{\mathrm{Rad}(L, M_i)}{\mathrm{Rad}^2(L, M_i)}$ 的一組基. 
    \end{enumerate}
\end{theorem}

\subsection{AR 大定理}

\begin{definition}[穩定範疇,穩定 $\mathrm{Hom}$]
    定義 $\underline{𝐦𝐨𝐝_A}$ 爲 $𝐦𝐨𝐝_A$ 的投射穩定範疇 (加法商範疇, 局部化範疇). 特別地, 
    \begin{equation}
        \underline{(X , Y)} = (X, Y) / \{\text{通過投射對象分解者}\}.
    \end{equation}
\end{definition}

\begin{theorem}[穩定等價大定理 I]
    $\mathrm{Tr}: \underline{𝐦𝐨𝐝 _A} → \underline{𝐦𝐨𝐝 _{A^{\mathrm{op}}}}$ 是穩定範疇的等價. \parnote{非投射單}
\end{theorem}

\begin{example}
    穩定 $\mathrm{Hom}$ 是賦值的餘核, 即以下正合 
    \begin{equation}
        Y ⊗ X^t \xrightarrow α  (X, Y) → \underline{(X, Y)} → 0;\quad α : [y ⊗ f] ↦ [x ↦ y⋅ f(x)]. 
    \end{equation}
\end{example}

\begin{proposition}[穩定等價大定理 II]
    對不可分解模 $M$ 與 $N$. 有穩定等價 
    \begin{equation}
        τ : \underline{𝐦𝐨𝐝_A} → \overline{𝐦𝐨𝐝_A} : τ ⁻¹ .
    \end{equation}
    \begin{enumerate}
        \item $τ M = 0$ 當且僅當 $M$ 投射; 
        \item 若 $M$ 不可約且非投射, 則 $τ M$ 不可約且非内射, 此時 $τ ⁻¹ τ M ≃ M$; 
        \item 作爲上一條的推論, $M ≃ M₀$ 當且僅當 $τ M ≃ τ M₀$;
        \item $τ⁻¹ N = 0$ 當且僅當 $N$ 内射; 
        \item 若 $N$ 不可約且非内射, 則 $τ⁻¹ M$ 不可約且非投射, 此時 $τ τ ⁻¹ N ≃ N$; 
        \item 作爲上一條的推論, $N ≃ N₀$ 當且僅當 $τ ⁻¹ N ≃ τ ⁻¹ N₀$. 
    \end{enumerate}
\end{proposition}

\begin{remark}
    $τ$ 將模向投射方向推移, $τ ⁻¹$ 將模向内射方向推移. 示意圖如下: 
    \begin{equation}
        % https://q.uiver.app/#q=WzAsNSxbMSwwLCJcXHRleHR75oqV5bCEfSJdLFszLDAsIlxcdGV4dHvlhaflsIR9Il0sWzIsMCwiXFxidWxsZXQiXSxbMCwwLCIwIl0sWzQsMCwiMCJdLFswLDIsIlxcdGF1XnstMX0iLDAseyJvZmZzZXQiOi01fV0sWzIsMSwiXFx0YXVeey0xfSIsMCx7Im9mZnNldCI6LTV9XSxbMSw0LCJcXHRhdV57LTF9IiwwLHsib2Zmc2V0IjotNX1dLFsxLDIsIlxcdGF1IiwwLHsib2Zmc2V0IjotNX1dLFsyLDAsIlxcdGF1IiwwLHsib2Zmc2V0IjotNX1dLFswLDMsIlxcdGF1IiwwLHsib2Zmc2V0IjotNX1dXQ==
\begin{tikzcd}[ampersand replacement=\&,sep=small]
	0 \& {\text{投射}} \& \bullet \& {\text{內射}} \& 0
	\arrow["\tau", shift left=5, from=1-2, to=1-1]
	\arrow["{\tau^{-1}}", shift left=5, from=1-2, to=1-3]
	\arrow["\tau", shift left=5, from=1-3, to=1-2]
	\arrow["{\tau^{-1}}", shift left=5, from=1-3, to=1-4]
	\arrow["\tau", shift left=5, from=1-4, to=1-3]
	\arrow["{\tau^{-1}}", shift left=5, from=1-4, to=1-5]
\end{tikzcd}.
    \end{equation}
\end{remark}

\begin{theorem}[AR 引理]
    有函子同構 $\mathrm{Ext}^1 (M, N) ≃ D\underline{(τ ⁻¹ N, M)} ≃ D\overline {(N , τ M)}$. 特別地, \parnote{摘線?}
    \begin{enumerate}
        \item 若 $p.\dim M ≤ 1$, 則可以摘掉上劃綫, 得 $\mathrm{Ext}^1 (M, N) ≃ D {(N , τ M)}$; 
        \item 若 $i.\dim N ≤ 1$, 則可以摘掉下劃綫, 得 $\mathrm{Ext}^1 (M, N) ≃ D{(τ ⁻¹ N, M)}$; 
        \item 若 $p.\dim M ≤ 1$ 且 $i.\dim N ≤ 1$, 則 ${(τ ⁻¹ N, M)} ≃ {(N , τ M)}$; 
        \item 若 $p.\dim M ≤ 1$ 且 $i.\dim τ N ≤ 1$, 則 ${(N, M)} ≃ {(τ N , τ M)}$; 
        \item 若 $p.\dim τ ⁻¹ M ≤ 1$ 且 $i.\dim N ≤ 1$, 則 ${(τ ⁻¹ N, τ ⁻¹ M)} ≃ {(N , M)}$. 
    \end{enumerate}
\end{theorem}

\begin{theorem}[AR 大定理]
    回顧 \ref{PreAR} 中的正合列. 
    \begin{enumerate}
        \item 若 $M$ 是非投射的不可分解模, 則存在幾乎可列短正合列 $0 → τM → ? → M → 0$. 參考 $\mathrm{Ext}^1 (M, τ M ) ≃ D\underline{\mathrm{End}(M)}$.
        \item 若 $N$ 是非内射的不可分解模, 則存在幾乎可裂短正合列 $0 → N → ? → τ ⁻¹ N → 0$. 參考 $\mathrm{Ext}^1 (τ ⁻¹ N , N) ≃ D\overline {\mathrm{End}(N)}$. 
    \end{enumerate}
\end{theorem}

\subsubsection{函子角度}

\begin{definition}
    $A  ∈ 𝐚𝐥𝐠 _k$, 默認 
\begin{enumerate}
    \item $\mathrm{Fun}^{\mathrm{op}}(A) :=\mathrm{Funct}_k (𝐦𝐨𝐝 _A^{\mathrm{op}}, 𝐦𝐨𝐝 _k)$ 是預層; \parnote{$(-, M)$}
    \item $\mathrm{Fun}(A) :=\mathrm{Funct}_k (𝐦𝐨𝐝 _A, 𝐦𝐨𝐝 _k)$ 是餘預層; \parnote{$(M, -)$}
\end{enumerate}
這些均是 $k$-範疇 ($k$-充實的 Abel 範疇).
\end{definition}

\begin{proposition}[單函子結構]
選定不可分解模 $M$, 相應的可表函子是函子範疇的不可分解投射對象. 
\begin{enumerate}
    \item $\mathrm{Fun}^{\mathrm{op}}(A)$ 的單函子恰形如 $S^M := (-, M) / \mathrm{Rad}(-, M)$; \parnote{均是投射蓋}
    \item $\mathrm{Fun}(A)$ 的單函子恰形如 $S_M := (M, -) / \mathrm{Rad}(M, -)$;
    \item $S^M(M) = S_M(M)$ 是 $\mathrm{End}(M)$ 的剩餘域; \parnote{示性函子}
    \item 若 $X ≇ M$ 是不可分解的, 則 $S^M(X) = 0$ 且 $S_M(X) = 0$. 
\end{enumerate}
\end{proposition}

\begin{proposition}[米田嵌入的伴隨]
    米田引理表明, 函子範疇的``(不可分解) 有限表現投射對象''恰好是``(不可分解) 可表對象''; 米田嵌入的右伴隨 (一向 $\mathrm{coker}(-, f) ↦ \mathrm{coker}(f)$) 有何性質? \parnote{單 → 不可分解}
    \begin{enumerate}
        \item $f$ (極小) 左幾乎可裂, 當且僅當 $(-,f)$ 誘導了 (極小) 投射表現. 此時,
        \begin{enumerate}
            \item $s(f) \xrightarrow f t(f) → \mathrm{coker}(f) → 0$ 正合, 且 $\mathrm{coker}(f)$ 不可分解;
            \item $(-,s(f)) \xrightarrow{(-,f)} (-, t(f)) → S^{t(f)} → 0$ 正合, 且 $\mathrm{coker} ((-,f))$ 是單對象. 
        \end{enumerate}
        \item $f$ (極小) 右幾乎可裂, 當且僅當 $(f,-)$ 誘導了 (極小) 投射表現. 此時,
        \begin{enumerate}
            \item $0 → \ker (f) → s(f) \xrightarrow f t(f)$ 正合, 且 $\ker (f)$ 不可分解; 
            \item $(t(f),-) \xrightarrow{(f,-)} (s(f),-) → S_{s(f)} → 0$ 正合, 且 $\ker ((f,-))$ 是單對象. 
        \end{enumerate}
    \end{enumerate}
\end{proposition}


\subsection{一些 quiver 計算; Gabriel 定理}
\begin{abstract}
    介紹基本 Quiver 表示, 並從一些代數幾何視角 (將``一個表示''對應至代數群作用下的``一個軌道''), 並證明 Gabriel 定理. 
    
    La preuve du théorème de Gabriel nécessite trois ingrédients: 
    \begin{enumerate}
        \item Classical geometric representation theory (representation varieties)
        \item Noncommutative, homological algebra (Ringel Lemma, ``brick'')
        \item Classification of graphs (due to Tits)
    \end{enumerate}
    主要參考 \cite{Brion2009RepresentationsOQ} 與 \cite{BerestMehrle2017} 等筆記. 
\end{abstract}

\subsubsection{箭圖表示的基本定理, Euler 二次型等}

\begin{definition}[箭圖 (quiver)]\label{quiver}
    要件 $(Q_0, Q_1, s, t)$, 記路代數 $A := kQ$, \textbf{默認左模}. 特別地, 有兩種``有限'': 
    \begin{enumerate}
        \item (有限型). $|Q_0| + |Q_1| < ∞$; \parnote{圖性質}
        \item (有限維). $\dim < ∞$, (等價地, 有限型且沒有環). \parnote{表示性質}
    \end{enumerate}
\end{definition}

\begin{remark}
    \textbf{默認箭圖是有限型的}: 對有限型箭圖, 冪等分解 $∑ e_i$ 方有意義. \parnote{type finie}
\end{remark}

\begin{theorem}[標準消解]
    有函子的短正合列
    \begin{equation}
        0 → ∑_{α ∈ Q_1} \underbracket{A e_{t(α)}}\limits_{P(t(α))} ⊗_k e_{s(α)} X → ∑_{i ∈ Q_0} \underbracket{Ae_i}\limits_{P(i)} ⊗_k e_i X → X → 0. 
    \end{equation} \parnote{$⊗_k$ 實際是關於``某單代數''的張量}
    ``從邊到點''的態射: 對 $r ⊗ x ∈ P(t(α)) ⊗ e_{s(α)}A$, 定義
    \begin{equation}
        r ⊗ x ↦ (r ⋅ α) ⊗ x − r ⊗ (α ⋅ x).   
    \end{equation}
\end{theorem}

\begin{example}[分次代數視角]
    $kQ$ 無非單代數 $S:= kQ_0$ 與雙模 $V:= kQ_1$ 張成的分次代數. 以上 
    \begin{equation}
        0 → A ⊗_S V ⊗_S X \xrightarrow{[1∣ 2∣ 3] \ ↦ \ ([12∣ 3]-[1∣ 23])} A ⊗_S X \xrightarrow{[1∣ 2] \ ↦ \ [12]} X → 0
    \end{equation}
\end{example}

\begin{definition}[維度向量]
    維度向量衡量了 Krull-Schmidt 範疇中不可分解對象的維度, 也就是 $K_0$ 群中對應的元素. 例如 $𝐝𝐢𝐦 M = (\dim (e_a M))_{a ∈ Q_0}$. \parnote{合成列單模數}
\end{definition}

\begin{definition}[Euler 型 $(-,-)_Q$, 或 Tits 形式]
    受 $0 → (X, Y) → (A⊗ _S X, Y) → (A⊗ _S V⊗ _S X, Y) → \mathrm{Ext}^1(X,Y) → 0$ 啟發, 定義 $𝐝𝐢𝐦 X$ 與 $𝐝𝐢𝐦 Y$ 的雙線性運算 ($C$ 是邊的鄰接矩陣): 
    \begin{equation}
        \dim \mathrm{Hom}(X,Y) - \dim \mathrm{Ext}^1(X,Y) = \underbracket{∑ _{i ∈ Q_0} (e_iX, e_i Y)}\limits_{𝐝𝐢𝐦 X⋅𝐝𝐢𝐦 Y} - \underbracket{∑ _{α ∈ Q_1}(e_{t(α)}X, e_{s(α)}Y)}\limits_{𝐝𝐢𝐦 X ⋅ C ⋅ 𝐝𝐢𝐦 Y}. 
    \end{equation}
    提煉出 $(-,-)_Q$ 定義式
    \begin{equation}
        (𝐝𝐢𝐦 X, 𝐝𝐢𝐦 Y) := (𝐝𝐢𝐦 X)^T ⋅ (I - C) ⋅ 𝐝𝐢𝐦 Y
    \end{equation}
    從不可分解模的角度, $\mathrm{Hom}(-,-)$ 貢獻點; $\mathrm{Ext}^1(-,-)$ 貢獻邊. 
\end{definition}

\begin{remark}
    此處\textbf{箭圖不帶關係}, Euler 型與 Tits 型姑且可以混同. 
\end{remark}

\begin{definition}[Euler 二次型]
    定義 $⟨ -,-⟩$ 是 $(-,-)_Q$ 的對稱化, 形如``無向圖的 Cartan 矩陣''. 記 $q(X) = ⟨ X, X⟩$
\end{definition}

\subsubsection{幾何視角}

\begin{example}[表示空間: 群作用, 軌道]
    表示論的目標: 描述所有的不可約表示. 今給定路代數 $A:= kQ$, 描述等價類的方式是
    \begin{itemize}
        \item 先確定維數向量, 再確定模結構; 等價地, \parnote{用 $𝐝𝐢𝐦$ 確定更粗的等價類}
        \item 先給定 $S:= kQ_0$ 模, 再擴張作 $A=kQ$ 模. 
    \end{itemize}
    對維數向量 $v$, 表示的等價類形如 $∏_{α ∈ Q_1} (k^{v_{t(α)}},k^{t_{s(α)}}) / ∼$, 即, ``一堆矩陣組''的等價類. 
    
    等價類即群 $∏_i \mathrm{Gl}_{v_i}(k)$-共軛作用的軌道, 例如 $(m)_{i → j}$ 被作用爲 $(g)_{v_i × v_i} ⋅ (m)_{i → j} ⋅ (g)⁻¹_{v_j × v_j}$. \parnote{廣群``共軛''}

    爲將``線性空間維度''和``概型維度''搭配, 另需引入 Krull 維度, 其精髓在於軌道公式. 
\end{example}

\begin{definition}[Krull 維度]
    選用 $𝔸 ^d := 𝔸_k ^d$ 上的 Zariski 拓撲. 稱 $X ⊆ 𝔸^d$ 局部閉, 當且僅當 $X$ 是 $\overline X$ 的開子集, 或等價地, 一個開集與一個閉集的交. \parnote{locally closed}

稱 $X$ 不可約, 當且僅當非空開子集必稠密; 等價地, 不存在 $\boxed{∘ \ ∘ }$. \parnote{irreducible}

最後定義局部閉集 $X ⊆ 𝔸^d$ 的 Krull 維度: 內部不可約集組成的閉鏈的最大長度. 
\end{definition}

\begin{example}
    例如 $𝔸^2 = \mathrm{Spec}(k[x,y])$ 中直線 $𝔸^1 ≃ \mathrm{Spec}(k[x])$ 的 Krull 維度是 $1$: $k ⊊ k[x]$ 是長度爲 $1$ 的極大鏈. 
\end{example}

\begin{proposition}
    依照 Proposition I.7.1, \cite{hartshorne1999algebraic} 等結論, 
    \begin{enumerate}
        \item $\dim 𝔸^d = d$, (約定) $\dim ∅ = -∞$; 
        \item $\dim (U ∪ V) = \max(\dim U , \dim V)$; 
        \item $\dim (X ∩ Y) ≥ \dim X+ \dim Y - d$; 
        \item $\dim U = \dim \overline U$. \parnote{$\dim \mathrm{Aut}$ = $\dim \mathrm{End}$}
    \end{enumerate}
    也可以先從拓撲空間的 Krull 維度入手. 概型的 Krull 維度就是拓撲空間的 Krull 維度. 
\end{proposition}

\begin{theorem}[軌道公式]
    假定 $G$ 是連通的代數群概型, $X$ 是 $G$-代數簇, 則 
    \begin{enumerate}
        \item 所有 $G$-軌道 $O_x$ 都是不可約的局部閉集; \parnote{軌道性質}
        \item 對任意 $x$, 穩定子群 $\mathrm{Stab}_G(x)$ 是閉子群; \parnote{穩定子群}
        \item 對任意 $x$, 有 $\dim G = \dim O_x + \dim \mathrm{Stab}_G(x)$; \parnote{維數公式}
        \item 對任意 $x$, $(\overline{O_x} \backslash O_x) = ⋃ _{\dim O_y < \dim O_x} O_y$. \parnote{主元}
    \end{enumerate}
\end{theorem}

\begin{proposition}
    假定 $\dim$ 是 Krull 維度, $\dim _k$ 是線性空間維度, 則有連接維數的關鍵公式: 對任意 $𝐝𝐢𝐦 X =v$, 總有 
    \begin{equation}
        \underbracket{\dim \mathrm{Rep}(v)}\limits_{v ^T ⋅ Q_1 ⋅ v} = \underbracket{\dim O_X}\limits_{v ^T ⋅ Q_1 ⋅ v - \dim \mathrm{End}(X)} + \dim _k\mathrm{Ext}^1 (X, X)
    \end{equation}
\end{proposition}

\begin{example}[自垂直對象]
    對給定的 $v = 𝐝𝐢𝐦 X$, 是否存在特殊的對象: $\mathrm{Ext}^1 (X , X) = 0$? \parnote{結合 Kac 定理}
    \begin{enumerate}
        \item $\mathrm{Ext}^1 (X, X) = 0$ 當且僅當 $\overline{O_X} = \mathrm{Rep}(v)$. \parnote{極大軌道}
        \item 對給定的 $v$, 作用 $\mathrm{Gl}(v) → \mathrm{Rep}(v)$ 至多有一條極大軌道; 
        \item 對不可裂短正合列 $0 → L → M → N → 0$, 總有 $O_{L ⊕ N} ⊆ O_M$; 
        \item 若極大軌道的對象 $X ≃ L ⊕ N$, 則 $\mathrm{Ext}^1(L, N) =0$. 
        \item 當且僅當 $X$ 半單, $O_X = \overline{O_X}$. 
    \end{enumerate}
\end{example}

\subsubsection{圖的結論: 一些線性代數}

\begin{definition}[圖的二次型]
    給定有限無向圖 (允許自環, 重邊) $Γ$, 記 $q_Γ$ 是鄰接矩陣對應的二次型. 此時 $q_Q = q_{|Q|}$ 與 Euler 二次型吻合. \parnote{與 Quiver 定向無關} 照常定義 $q_Γ$ 正定, 半正定, 以及零空間 $N(q_Γ)$.  
\end{definition}

\begin{theorem}[有限圖分類定理]
    給定連通圖 $Γ$, 以下分類 $q_Γ$ (有限型, 仿射型, 其他). 
    \begin{enumerate}
        \item (有限型, Dynkin 型, 即 $q_Γ$ 正定) 也就是熟知的 $A_{≥ 1}$, $D_{≥ 3}$, $E_{6,7,8}$, $n$ 是頂點數. 
        \item (仿射型, Euclidean 型, 即 $q_Γ$ 半定但 $\dim \ker N(q_Γ) =1$) 也就是熟知的 $\widetilde A_{≥ 0}$, $\widetilde D_{≥ 4}$, $\widetilde E_{6,7,8}$, 頂點數 $n +1$. \parnote{extending vertex}將張成 $\ker N(q_Γ)$ 的所有格點悉列諸下圖: 
        \begin{equation}
            % https://q.uiver.app/#q=WzAsNDQsWzAsMiwiMSJdLFsyLDIsIjEiXSxbMCwxLCJcXHZkb3RzICJdLFsyLDEsIlxcdmRvdHMgIl0sWzAsMCwiMSJdLFsyLDAsIjEiXSxbMSwxLCJcXGJveGVke1xcd2lkZXRpbGRlIEFfbn0iXSxbMywwLCIxIl0sWzMsMiwiMSJdLFszLDEsIjIiXSxbNywwLCIxIl0sWzcsMSwiMiJdLFs3LDIsIjEiXSxbNSwyLCJcXGJveGVke1xcd2lkZXRpbGRlIERfbiB9Il0sWzExLDEsIlxcYm94ZWR7XFx3aWRldGlsZGUgRV82fSJdLFs0LDEsIjIiXSxbNiwxLCIyIl0sWzUsMSwiXFxjZG90cyAiXSxbMTIsMCwiMSJdLFsxMSwwLCIyIl0sWzEwLDAsIjMiXSxbOSwwLCIyIl0sWzgsMCwiMSJdLFsxMCwxLCIyIl0sWzEwLDIsIjEiXSxbNCw1LCJcXGJveGVke1xcd2lkZXRpbGRlIEVfN30iXSxbMyw0LCI0Il0sWzAsNCwiMSJdLFsxLDQsIjIiXSxbMiw0LCIzIl0sWzQsNCwiMyJdLFs1LDQsIjIiXSxbNiw0LCIxIl0sWzMsNSwiMiJdLFs4LDUsIlxcYm94ZWR7XFx3aWRldGlsZGUgRV84fSJdLFs3LDQsIjYiXSxbNyw2LCIyIl0sWzcsNSwiNCJdLFs3LDMsIjMiXSxbOCw0LCI1Il0sWzksNCwiNCJdLFsxMCw0LCIyIl0sWzExLDQsIjIiXSxbMTIsNCwiMSJdLFswLDIsIiIsMCx7InN0eWxlIjp7ImhlYWQiOnsibmFtZSI6Im5vbmUifX19XSxbMiw0LCIiLDAseyJzdHlsZSI6eyJoZWFkIjp7Im5hbWUiOiJub25lIn19fV0sWzQsNSwiIiwwLHsic3R5bGUiOnsiaGVhZCI6eyJuYW1lIjoibm9uZSJ9fX1dLFsxLDAsIiIsMCx7InN0eWxlIjp7ImhlYWQiOnsibmFtZSI6Im5vbmUifX19XSxbNyw5LCIiLDAseyJzdHlsZSI6eyJoZWFkIjp7Im5hbWUiOiJub25lIn19fV0sWzksOCwiIiwwLHsic3R5bGUiOnsiaGVhZCI6eyJuYW1lIjoibm9uZSJ9fX1dLFsxMSwxMiwiIiwwLHsic3R5bGUiOnsiaGVhZCI6eyJuYW1lIjoibm9uZSJ9fX1dLFsxMCwxMSwiIiwwLHsic3R5bGUiOnsiaGVhZCI6eyJuYW1lIjoibm9uZSJ9fX1dLFs5LDE1LCIiLDAseyJzdHlsZSI6eyJoZWFkIjp7Im5hbWUiOiJub25lIn19fV0sWzE1LDE3LCIiLDAseyJzdHlsZSI6eyJoZWFkIjp7Im5hbWUiOiJub25lIn19fV0sWzE3LDE2LCIiLDAseyJzdHlsZSI6eyJoZWFkIjp7Im5hbWUiOiJub25lIn19fV0sWzE2LDExLCIiLDAseyJzdHlsZSI6eyJoZWFkIjp7Im5hbWUiOiJub25lIn19fV0sWzIwLDE5LCIiLDAseyJzdHlsZSI6eyJoZWFkIjp7Im5hbWUiOiJub25lIn19fV0sWzE5LDE4LCIiLDAseyJzdHlsZSI6eyJoZWFkIjp7Im5hbWUiOiJub25lIn19fV0sWzIwLDIzLCIiLDIseyJzdHlsZSI6eyJoZWFkIjp7Im5hbWUiOiJub25lIn19fV0sWzIzLDI0LCIiLDIseyJzdHlsZSI6eyJoZWFkIjp7Im5hbWUiOiJub25lIn19fV0sWzIwLDIxLCIiLDIseyJzdHlsZSI6eyJoZWFkIjp7Im5hbWUiOiJub25lIn19fV0sWzIxLDIyLCIiLDIseyJzdHlsZSI6eyJoZWFkIjp7Im5hbWUiOiJub25lIn19fV0sWzI2LDI5LCIiLDIseyJzdHlsZSI6eyJoZWFkIjp7Im5hbWUiOiJub25lIn19fV0sWzI5LDI4LCIiLDIseyJzdHlsZSI6eyJoZWFkIjp7Im5hbWUiOiJub25lIn19fV0sWzI4LDI3LCIiLDIseyJzdHlsZSI6eyJoZWFkIjp7Im5hbWUiOiJub25lIn19fV0sWzI2LDMwLCIiLDAseyJzdHlsZSI6eyJoZWFkIjp7Im5hbWUiOiJub25lIn19fV0sWzMwLDMxLCIiLDAseyJzdHlsZSI6eyJoZWFkIjp7Im5hbWUiOiJub25lIn19fV0sWzMxLDMyLCIiLDAseyJzdHlsZSI6eyJoZWFkIjp7Im5hbWUiOiJub25lIn19fV0sWzMzLDI2LCIiLDAseyJzdHlsZSI6eyJoZWFkIjp7Im5hbWUiOiJub25lIn19fV0sWzM2LDM3LCIiLDAseyJzdHlsZSI6eyJoZWFkIjp7Im5hbWUiOiJub25lIn19fV0sWzM3LDM1LCIiLDEseyJzdHlsZSI6eyJoZWFkIjp7Im5hbWUiOiJub25lIn19fV0sWzM1LDM5LCIiLDEseyJzdHlsZSI6eyJoZWFkIjp7Im5hbWUiOiJub25lIn19fV0sWzM5LDQwLCIiLDEseyJzdHlsZSI6eyJoZWFkIjp7Im5hbWUiOiJub25lIn19fV0sWzQwLDQxLCIiLDEseyJzdHlsZSI6eyJoZWFkIjp7Im5hbWUiOiJub25lIn19fV0sWzQyLDQzLCIiLDEseyJzdHlsZSI6eyJoZWFkIjp7Im5hbWUiOiJub25lIn19fV0sWzUsMywiIiwwLHsic3R5bGUiOnsiaGVhZCI6eyJuYW1lIjoibm9uZSJ9fX1dLFszLDEsIiIsMCx7InN0eWxlIjp7ImhlYWQiOnsibmFtZSI6Im5vbmUifX19XSxbMzgsMzUsIiIsMCx7InN0eWxlIjp7ImhlYWQiOnsibmFtZSI6Im5vbmUifX19XSxbNDEsNDIsIiIsMSx7InN0eWxlIjp7ImhlYWQiOnsibmFtZSI6Im5vbmUifX19XV0=
\begin{tikzcd}[ampersand replacement=\&,sep=tiny]
	1 \&\& 1 \& 1 \&\&\&\& 1 \& 1 \& 2 \& 3 \& 2 \& 1 \\
	{\vdots } \& {\boxed{\widetilde A_n}} \& {\vdots } \& 2 \& 2 \& {\cdots } \& 2 \& 2 \&\&\& 2 \& {\boxed{\widetilde E_6}} \\
	1 \&\& 1 \& 1 \&\& {\boxed{\widetilde D_n }} \&\& 1 \&\&\& 1 \\
	\&\&\&\&\&\&\& 3 \\
	1 \& 2 \& 3 \& 4 \& 3 \& 2 \& 1 \& 6 \& 5 \& 4 \& 3 \& 2 \& 1 \\
	\&\&\& 2 \& {\boxed{\widetilde E_7}} \&\&\& 4 \& {\boxed{\widetilde E_8}} \\
	\&\&\&\&\&\&\& 2
	\arrow[no head, from=1-1, to=1-3]
	\arrow[no head, from=1-3, to=2-3]
	\arrow[no head, from=1-4, to=2-4]
	\arrow[no head, from=1-8, to=2-8]
	\arrow[no head, from=1-10, to=1-9]
	\arrow[no head, from=1-11, to=1-10]
	\arrow[no head, from=1-11, to=1-12]
	\arrow[no head, from=1-11, to=2-11]
	\arrow[no head, from=1-12, to=1-13]
	\arrow[no head, from=2-1, to=1-1]
	\arrow[no head, from=2-3, to=3-3]
	\arrow[no head, from=2-4, to=2-5]
	\arrow[no head, from=2-4, to=3-4]
	\arrow[no head, from=2-5, to=2-6]
	\arrow[no head, from=2-6, to=2-7]
	\arrow[no head, from=2-7, to=2-8]
	\arrow[no head, from=2-8, to=3-8]
	\arrow[no head, from=2-11, to=3-11]
	\arrow[no head, from=3-1, to=2-1]
	\arrow[no head, from=3-3, to=3-1]
	\arrow[no head, from=4-8, to=5-8]
	\arrow[no head, from=5-2, to=5-1]
	\arrow[no head, from=5-3, to=5-2]
	\arrow[no head, from=5-4, to=5-3]
	\arrow[no head, from=5-4, to=5-5]
	\arrow[no head, from=5-5, to=5-6]
	\arrow[no head, from=5-6, to=5-7]
	\arrow[no head, from=5-8, to=5-9]
	\arrow[no head, from=5-9, to=5-10]
	\arrow[no head, from=5-10, to=5-11]
	\arrow[no head, from=5-11, to=5-12]
	\arrow[no head, from=5-12, to=5-13]
	\arrow[no head, from=6-4, to=5-4]
	\arrow[no head, from=6-8, to=5-8]
	\arrow[no head, from=7-8, to=6-8]
\end{tikzcd}.
        \end{equation}
        \item (無限型) 其他類型. 
    \end{enumerate}
\end{theorem}

\begin{definition}[根系]
    給定有限型或仿射型 $Γ$, 定義根系 $Δ := \{v ∈ ℤ ^n ∣ q_Γ (v) ≤ 1\}$. 約定 $0 ∉ Δ$. 定義
    \begin{enumerate}
        \item (實根) 使得 $q_Γ(v) = 1$ 的根, 記作 $v ∈ Δ^r$; 
        \item (虛根) 使得 $q_Γ(v) = 1$ 的根, 記作 $v ∈ Δ^i$; 
        \item (正根) 各項爲正整數的根, 記作 $v ∈ Δ_+$. 
    \end{enumerate}
\end{definition}

\begin{proposition}[根系結構定理]
    仍選用有限型或仿射型 $Γ$, 
    \begin{enumerate}
        \item $Δ$ 關於對稱, Weyl 反射封閉; 
        \item $Δ = Δ_+ ⊔ Δ _+$, 換言之, 沒有既正又負的根; 
        \item $Γ$ 是有限型 $⟺$ $Δ$ 是有限集 $⟺$ $Δ^i = ∅$; 
        \item $Γ$ 是仿射型 $⟺$ $Δ$ 是無限集 $⟺$ $Δ^i ≃ ℤ$. 
    \end{enumerate}
\end{proposition}

\begin{theorem}[Gabriel]
    給定有限型 $Γ$, 有限維不可約表示一一對應 $Δ_+$. 
    \begin{enumerate}
        \item 從不可約表示到 $Δ_+$, 對應方式 $X ↦ 𝐝𝐢𝐦 X$; 
        \item 從 $Δ_+$ 到不可約表示, 對應方式 $v ↦ \mathrm{Rep}(v)$ 中極大軌道. 
    \end{enumerate} 
    $kQ$ 的不可約表示有限, 當且僅當 $Γ$ 是有限型圖 (及其有限無交並). 
\end{theorem}

\begin{remark}
    對一般的有限箭圖, 可以類似地定義根系. Kac 定理解答了如下問題: 以 $v$ 爲維度向量的不可約表示有多少? 
\end{remark}

\begin{theorem}[Kac]
    當且僅當 $v ∈ Δ _+$, 存在以 $v$ 爲維度向量的不可約表示. 
    \begin{enumerate}
        \item $v ∈ Δ _+$ $⟺$ $\mathrm{Rep}(v)$ 存在極大軌道 $⟺$ 存在 $𝐝𝐢𝐦 X = v$ 使得 $\mathrm{Ext}^1(X,X) = 0$. 
        \item 當且僅當 $v ∈ Δ _+^r$, 以上不可約表示唯一; 
        \item 當且僅當 $v ∈ Δ _+^i$, 以上不可約表示無窮. 
    \end{enumerate}
\end{theorem}

\begin{remark}
    Ringel Hall 代數等? 
\end{remark}

\subsection{總結: 有限生成模, 有限表現模, 投射模, 內射模, 平坦模等的結構}
\begin{abstract}
    總結 F.P. 模, 內射模, 平坦模, 投射模的等價定義(判准). 
    \begin{enumerate}
        \item (F.P.) 有限表現模的等價判據. $\mathrm{coker}(R^{n × m})$; $(M, -)$ 保持濾過餘極限; $M ⊗ -$ 保持積.
        \item (Plat.) 平坦模的等價判據. 
        \begin{enumerate}
            \item[a] $M ⊗ -$ 保單; $M ⊗ -$ 保 F.G. 模之單; $M ⊗ -$ 正合; $\mathrm{Tor}(M, - )=0$; 
            \item[b] $I ⊗ M ≃ IM$; $I ⊗ M ≃ IM$ 保 F.G. 理想之單; 
            \item[c] $Rm=0$ 蘊含 $R A =0$ 且 $m = A m′$; $0 → (A, M) → M^n → M^m$ 可補全作 ses $M^l → M^n → M^m$; 
            \item[d] $\text{F.P.} → M$ 定被 $R^n$ 分解; 自由模的濾過餘極限
        \end{enumerate}
        \item (特征模函子) $(-)^+$ 正合忠實; $X$ 的平坦維度等於 $X^+$ 的內射維度.
        \item (Inj.) 內射模的等價判據. 
        \begin{enumerate}
            \item[a] $(-, M)$ 正合, $\mathrm{Ext}^1(-, M)=0$; $\mathrm{Ext}^1(\text{商環},M)=0$;
            \item[b] 提升單射; 僅提升單射 $I ↪ R$; 本性擴張平凡; $M ↪ ?$ 可裂; 
            \item[c] $(R^{⊕})^+$ 直和項; $∏ R^+$ 的直和項. 
        \end{enumerate}
        \item (Proj.) 投射模的等價判據. 
        \begin{enumerate}
            \item[a] $(M, -)$ 正合, $\mathrm{Ext}^1(M, -)=0$; 
            \item[b] 提升滿射; $? ↠ M$ 可裂; 
            \item[c] $(R^{⊕})^+$ 直和項; 存在``投射基''; 
            \item[d] 投射模是可數生成投射模的直和. 
        \end{enumerate} 
    \end{enumerate}
\end{abstract}

\subsubsection{有限表現模的結構}

\begin{theorem}
    記 $I$ 濾過, $J$ 有限, $F: I × J → 𝐒𝐞𝐭𝐬$ 爲函子, 則典範態射 \parnote{濾過餘極限和極限交換}
    \begin{equation}\label{exchangelimcolim}
        \varinjlim_I \varprojlim_J F → \varprojlim_J \varinjlim_I F 
    \end{equation}
    是同構. 
    \begin{proof}
        §IX.2, \cite{lane1998categories} (GTM5).

        極限 (積的子) 和濾過餘極限 (有限歸納之等價類) 是兩大邏輯要件. 
    \end{proof}
\end{theorem}

\begin{remark}\label{LeX}
    若存在函子 $U : 𝒞 → 𝐒𝐞𝐭𝐬$, 其忠實, 保持濾過餘極限和小極限, 且返一切同構, 則上述交換定理對 $F: I × J → 𝒞$ 成立. 特例: 帶點集合, 環, 模等代數結構 (按 \cite{algstructure}). 
\end{remark}

\begin{definition}[緊對象]
    稱 $K$ 是範疇 $𝒞$ 之緊對象, 當且僅當 $(K, -)$ 與濾過餘極限交換. \parnote{緊 = 兼容歸納}
\end{definition}

\begin{remark}
    對 \ref{LeX} 中的 $(U, 𝒞)$, 應用式 \ref{exchangelimcolim} 知緊對象對有限極限封閉. 對於緊對象組成的範疇, 見 \cite{nlab:lex} 及其關聯內容.
\end{remark}

\begin{proposition}
    任意模是有限表現模的濾過餘極限. 
    \begin{proof}
        任意 $X$ 適用 ses: $0→ K→ R^{⊕ X}→ X→ 0$. 記濾過系統爲二元組 $\{(S, f)\}$, 其中 $S ⊆ X$ 是有限子集, $f : K_f → R^{⊕ S}$, $K_f$ 有限生成. 如此一來, 
        \begin{equation}
            \varinjlim_{(S, f)} (\mathrm{coker} (f)) = X
        \end{equation}
    \end{proof}
\end{proposition}

\begin{theorem}[A criterion for compact modules]
    以下條件等價. \parnote{f.p. 模刻畫}
    \begin{enumerate}
        \item $X$ 是有限表現模, 即生成元與生成關係有限的模; \parnote{f.p. 定義}
        \item $(X, -)$ 與濾過餘極限交換; \parnote{Hom 刻畫}
        \item $X ⊗ -$ 與任意積交換.  \parnote{$⊗$ 刻畫}
    \end{enumerate}
\begin{proof}
    順序 $1 ⟺ 2$, $1 ⟺ 3$. 
    \begin{enumerate}
        \item ($1 ⟹ 2$). 給定表現 $R^m → R^n → X → 0$, 由濾過餘極限和 Hom 的左正合性得
        \begin{equation}
            % https://q.uiver.app/#q=WzAsOCxbMCwwLCIwIl0sWzEsMCwiKFgsIFxcdmFyaW5qbGltICgtKSkiXSxbMiwwLCIoUl5uLCBcXHZhcmluamxpbSAoLSkpIl0sWzMsMCwiKFJebSwgXFx2YXJpbmpsaW0gKC0pKSJdLFswLDEsIjAiXSxbMSwxLCJcXHZhcmluamxpbSAoWCwgLSkiXSxbMiwxLCJcXHZhcmluamxpbSAoUl5uLCAtKSJdLFszLDEsIlxcdmFyaW5qbGltIChSXm0sIC0pIl0sWzAsMV0sWzEsMl0sWzIsM10sWzQsNV0sWzUsNl0sWzYsN10sWzMsNywiXFxzaW1lcSAiXSxbMiw2LCJcXHNpbWVxICJdLFsxLDVdXQ==
\begin{tikzcd}[ampersand replacement=\&,sep=small]
	0 \& {(X, \varinjlim (-))} \& {(R^n, \varinjlim (-))} \& {(R^m, \varinjlim (-))} \\
	0 \& {\varinjlim (X, -)} \& {\varinjlim (R^n, -)} \& {\varinjlim (R^m, -)}
	\arrow[from=1-1, to=1-2]
	\arrow[from=1-2, to=1-3]
	\arrow[from=1-2, to=2-2]
	\arrow[from=1-3, to=1-4]
	\arrow["{\simeq }", from=1-3, to=2-3]
	\arrow["{\simeq }", from=1-4, to=2-4]
	\arrow[from=2-1, to=2-2]
	\arrow[from=2-2, to=2-3]
	\arrow[from=2-3, to=2-4]
\end{tikzcd}.
        \end{equation}
        依照五引理, 得同構. 
        \item ($2 ⟹ 1$). 將緊模 $X$ 寫作有限表現模的濾過餘極限, 則
        \begin{equation}
            \mathrm{id}_X ∈ (X, X) = (X, \varinjlim K) ≃ \varinjlim (X, K). 
        \end{equation}
        因此, $\mathrm{id}_X$ 通過某一有限表現模分解. 這說明 $X$ 是有限表現模的直和項.
        \item ($1 ⟹ 3$) 給定表現 $R^m → R^n → X → 0$, 由 AB4* 和 $⊗$ 的右正合性得 
        \begin{equation}
            % https://q.uiver.app/#q=WzAsOCxbMCwxLCIoXFxwcm9kX2kgUl5tIFxcb3RpbWVzIE1faSkiXSxbMSwxLCIoXFxwcm9kX2kgUl5uIFxcb3RpbWVzIE1faSkiXSxbMiwxLCIoXFxwcm9kX2kgWCBcXG90aW1lcyBNX2kpIl0sWzMsMSwiMCJdLFswLDAsIlJebSBcXG90aW1lcyAoXFxwcm9kX2kgTV9pKSJdLFsxLDAsIlJebiBcXG90aW1lcyAoXFxwcm9kX2kgTV9pKSJdLFsyLDAsIlggXFxvdGltZXMgKFxccHJvZF9pIE1faSkiXSxbMywwLCIwIl0sWzAsMV0sWzEsMl0sWzIsM10sWzQsNV0sWzUsNl0sWzYsN10sWzQsMCwiXFxzaW1lcSAiXSxbNSwxLCJcXHNpbWVxICJdLFs2LDJdXQ==
\begin{tikzcd}[ampersand replacement=\&,sep=small]
	{R^m \otimes (\prod_i M_i)} \& {R^n \otimes (\prod_i M_i)} \& {X \otimes (\prod_i M_i)} \& 0 \\
	{(\prod_i R^m \otimes M_i)} \& {(\prod_i R^n \otimes M_i)} \& {(\prod_i X \otimes M_i)} \& 0
	\arrow[from=1-1, to=1-2]
	\arrow["{\simeq }", from=1-1, to=2-1]
	\arrow[from=1-2, to=1-3]
	\arrow["{\simeq }", from=1-2, to=2-2]
	\arrow[from=1-3, to=1-4]
	\arrow[from=1-3, to=2-3]
	\arrow[from=2-1, to=2-2]
	\arrow[from=2-2, to=2-3]
	\arrow[from=2-3, to=2-4]
\end{tikzcd}.
        \end{equation}
        依照五引理, 得同構. 
        \item ($3 ⟹ 1$) 類似以上對 $\mathrm{id}_X$ 的操作, 考慮 
        \begin{equation}
            \mathrm{id}_X ∈ X^X ≃ ∏ _X (X ⊗ R) ≃ X ⊗ ∏ _XR ∋ ∑ x_i ⊗ f_i. 
        \end{equation}
        追從同構, 恆有 $x = ∑ x_i ⋅ f_i(x)$, 故 $X$ 有限生成. 考慮 ses $0 → K → R^l → X → 0$, 得 
        \begin{equation}
            % https://q.uiver.app/#q=WzAsMTAsWzAsMSwiMCJdLFsxLDEsIlxccHJvZCBfSyBLIl0sWzIsMSwiXFxwcm9kIF9LIFJebCAiXSxbMywxLCJcXHByb2QgX0sgWCJdLFs0LDEsIjAiXSxbMywwLCJYIFxcb3RpbWVzIFxccHJvZCBfSyBSIl0sWzIsMCwiUl5sICBcXG90aW1lcyBcXHByb2QgX0sgUiJdLFsxLDAsIksgXFxvdGltZXMgXFxwcm9kIF9LIFIiXSxbNCwwLCIwIl0sWzAsMCwiPyJdLFswLDFdLFsxLDJdLFsyLDNdLFszLDRdLFs3LDZdLFs2LDVdLFs1LDhdLFs5LDddLFs5LDAsIiIsMSx7InN0eWxlIjp7ImhlYWQiOnsibmFtZSI6ImVwaSJ9fX1dLFs2LDIsIlxcc2ltZXEgIl0sWzcsMSwiIiwxLHsic3R5bGUiOnsiYm9keSI6eyJuYW1lIjoiZGFzaGVkIn19fV0sWzUsMywiXFxzaW1lcSAiXV0=
\begin{tikzcd}[ampersand replacement=\&,sep=small]
	{?} \& {K \otimes \prod _K R} \& {R^l  \otimes \prod _K R} \& {X \otimes \prod _K R} \& 0 \\
	0 \& {\prod _K K} \& {\prod _K R^l } \& {\prod _K X} \& 0
	\arrow[from=1-1, to=1-2]
	\arrow[two heads, from=1-1, to=2-1]
	\arrow[from=1-2, to=1-3]
	\arrow[dashed, from=1-2, to=2-2]
	\arrow[from=1-3, to=1-4]
	\arrow["{\simeq }", from=1-3, to=2-3]
	\arrow[from=1-4, to=1-5]
	\arrow["{\simeq }", from=1-4, to=2-4]
	\arrow[from=2-1, to=2-2]
	\arrow[from=2-2, to=2-3]
	\arrow[from=2-3, to=2-4]
	\arrow[from=2-4, to=2-5]
\end{tikzcd}.
        \end{equation}
        由五引理, 虛線處爲滿, 從而 $K$ 有限生成. 
    \end{enumerate}
\end{proof}
\end{theorem}

\subsubsection{有限生成模的結構}


\subsubsection{內射模的結構}

\begin{theorem}[內射模的 Baer 判別]
    $M$ 是內射模, 當且僅當對任意理想的包含 $i : I ⊆ R$, 態射 \parnote{Baer, 內射模}
    \begin{equation}
        (i, M) : (R, M) → (I, M) ,\quad f ↦ f ∘ i
    \end{equation}
    是滿的. 換言之, 任意模同態 $I → M$ 通過 $i$ 分解. 
    \begin{proof}
        僅證 $⇐$. 茲檢驗 $M$ 對任意單態射 $L ↪ N$ 之提升性, 不妨視 $L$ 爲 $N$ 的子模. 記部分 $N$-子模構成的集合\parnote{$𝒮$: 能提升哪些擴張?}
\begin{equation}
    𝒮 := \{K ∣ \text{$L ⊆ K ⊆ N$ 爲子模的包含鏈, 且任意 $L → M$ 通過 $K$ 分解}\}. 
\end{equation}
顯然 $L ∈ 𝒮$ 非空. 任取極大元 $Q ∈ 𝒮$ (Zorn 引理), 只需證明 \parnote{後繼歸納}
\begin{itemize}
    \item 若 $Q ≠ N$, 則對存在 $(n ∈ N) ∧ (n ∉ Q)$ 使得 $(⟨ n ⟩ + Q) ∈ 𝒮$, 得矛盾.  
\end{itemize}
實際上, 任意選定 $n$, 記理想 $I := \{r ∣ n ⋅ r  ∈  Q\}$. 對任意 $φ: Q → M$, 定義 
\begin{equation}
    (⟨ n ⟩ + Q) → M , \quad (n ⋅ r + q) ↦ α (r) + φ (q); \quad \text{其中,} \ \begin{tikzcd}[ampersand replacement=\&, sep = small]
        I \& R \\
        Q \& M 
        \arrow[hook, from=1-1, to=1-2]
        \arrow["{n ⋅ }"', from=1-1, to=2-1]
        \arrow["{α }", dashed, from=1-2, to=2-2]
        \arrow["{φ }"', from=2-1, to=2-2]
    \end{tikzcd}.
\end{equation}
這說明提升至 $Q$ 的態射可進一步地提升至 $⟨ n ⟩ + Q$. 
    \end{proof}
    以上證明使用最簡單的超限歸納, 對後繼序數的證明是關鍵, 對極限序數的證明由 Zorn 引理一筆帶過. 
\end{theorem}

\begin{remark}
    Baer's criterion 的意圖非常簡單: 驗證內射模本需檢驗對``所有單射''之提升性, 現將``所有單射''簡化至``所有形如 $I ⊆ R$''的包含. \parnote{所有單 $⟹$ 特殊單}
\end{remark}

\begin{example}
    將``全體理想''換做``全體有限生成的理想'', 命題不正確. 提示: $R := ℝ[x_{≥ 0}]$, 構造略. 
\end{example}

\subsubsection{平坦模的結構}

\begin{proposition}
    一般記號 $A$ 環, $I$ 理想, $\{M,N,L\}$ 左模, $\{X,Y,Z\}$ 右模. 以下關於 $M$ 的表述等價. \parnote{Baer, 平坦模}
    \begin{enumerate}
        \item 對任意單射 $f : X ↪ Y$, 態射 $f ⊗ \mathrm{id}_M : X ⊗ M → Y ⊗ M$ 單. 
        \item 對任意有限生成模的單射 $f : X ↪ Y$, 態射 $f ⊗ \mathrm{id}_M : X ⊗ M → Y ⊗ M$ 單. \parnote{僅需看有限生成模}
        \item 對任意理想 $I ⊆ A$, 自然態射 $I ⊗ M → IM$ 是同構. 
        \item 對任意有限生成的理想 $I ⊆ A$, 自然態射 $I ⊗ M → IM$ 是同構. 
    \end{enumerate}
    \begin{proof}
        過程: $1 ⟺ 2$ 且 $3 ⟺ 4$, $3 ⟹ 2$, $1 ⟹ 4$. \parnote{強 $⟹$ 弱}
\begin{enumerate}
    \item ($1 ⟺ 2$). 顯然 $1 ⇒ 2$, 下證明 $2 ⇒ 1$: 檢驗 $∑ x_i ⊗ m_i$ 與 $∑ x_j ⊗ m_j$ 在 $f ⊗ \mathrm{id}_M$ 下的像相同與否, 僅需考慮 $f$ 在有限生成模 $⟨\{x_i\} ∪ \{x_j\}⟩$ 上的限制, 遂得證. 同理, $3 ⟺ 4$. 
    \item ($3 ⟹ 2$). 不妨設子模 $X ⊆ Y$ 通過添加一個生成元得到, 則存在理想 $I$ 使得 $Y / X ≅ A / I$. 任取提升 $\begin{tikzcd}[ampersand replacement=\&, sep = small]
        A \& {A/ I} \\
        Y \& {Y / X}
        \arrow[two heads, from=1-1, to=1-2]
        \arrow["{\varphi }"', dashed, from=1-1, to=2-1]
        \arrow["{\simeq }", from=1-2, to=2-2]
        \arrow[two heads, from=2-1, to=2-2]
    \end{tikzcd}$, 得滿射 $(φ , ⊆ ) : A ⊕ X → Y$, 進而補全滿態射的推出拉回方塊\parnote{理想 $⟹$ 模}
\begin{equation}
    % https://q.uiver.app/#q=WzAsMTYsWzIsMiwiQSBcXG9wbHVzIFgiXSxbMiwzLCJZIl0sWzIsNCwiMCJdLFsxLDMsIlgiXSxbMywzLCJBIC8gSSJdLFsxLDIsIlgiXSxbMywyLCJBIl0sWzMsMSwiSSJdLFszLDQsIjAiXSxbNCwzLCIwIl0sWzAsMywiMCJdLFszLDAsIjAiXSxbNCwyLCIwIl0sWzAsMiwiMCJdLFsyLDEsIkkiXSxbMiwwLCIwIl0sWzAsMSwiKFxcdmFycGhpICwgXFxzdWJzZXRlcSApIiwyXSxbMSwyXSxbNSwwLCJcXGJpbm9tIDAxIl0sWzAsNiwiKDEsMCkiXSxbNSwzLCIiLDIseyJsZXZlbCI6Miwic3R5bGUiOnsiaGVhZCI6eyJuYW1lIjoibm9uZSJ9fX1dLFszLDEsIlxcc3Vic2V0ZXEgIiwyXSxbMSw0XSxbMTAsM10sWzQsOV0sWzExLDddLFs3LDYsIiIsMix7InN0eWxlIjp7ImJvZHkiOnsibmFtZSI6ImRhc2hlZCJ9fX1dLFs2LDRdLFs0LDhdLFsxMyw1XSxbNiwxMl0sWzE1LDE0XSxbMTQsMCwiIiwxLHsic3R5bGUiOnsiYm9keSI6eyJuYW1lIjoiZGFzaGVkIn19fV0sWzE0LDcsIiIsMSx7ImxldmVsIjoyLCJzdHlsZSI6eyJoZWFkIjp7Im5hbWUiOiJub25lIn19fV1d
\begin{tikzcd}[ampersand replacement=\&, sep = small]
	\&\& 0 \& 0 \\
	\&\& I \& I \\
	0 \& X \& {A \oplus X} \& A \& 0 \\
	0 \& X \& Y \& {A / I} \& 0 \\
	\&\& 0 \& 0
	\arrow[from=1-3, to=2-3]
	\arrow[from=1-4, to=2-4]
	\arrow[equals, from=2-3, to=2-4]
	\arrow[dashed, from=2-3, to=3-3]
	\arrow[dashed, from=2-4, to=3-4]
	\arrow[from=3-1, to=3-2]
	\arrow["{\binom 01}", from=3-2, to=3-3]
	\arrow[equals, from=3-2, to=4-2]
	\arrow["{(1,0)}", from=3-3, to=3-4]
	\arrow["{(\varphi , \subseteq )}"', from=3-3, to=4-3]
	\arrow[from=3-4, to=3-5]
	\arrow[from=3-4, to=4-4]
	\arrow[from=4-1, to=4-2]
	\arrow["{\subseteq }"', from=4-2, to=4-3]
	\arrow[from=4-3, to=4-4]
	\arrow[from=4-3, to=5-3]
	\arrow[from=4-4, to=4-5]
	\arrow[from=4-4, to=5-4]
\end{tikzcd}.
\end{equation}
作用 $M ⊗ -$, 可裂短正合列 R3 ses, 依歸納假設 C4 ses. 遂有蛇引理模型
\begin{equation}
    % https://q.uiver.app/#q=WzAsMTcsWzIsMiwiTVxcb3BsdXMgKE0gXFxvdGltZXMgWCkiXSxbMiwzLCJNIFxcb3RpbWVzIFkiXSxbMiw0LCIwIl0sWzEsMywiTSBcXG90aW1lcyBYIl0sWzMsMywiTSBcXG90aW1lcyAoQSAvIEkpIl0sWzEsMiwiTSBcXG90aW1lcyBYIl0sWzMsMiwiTSJdLFszLDEsIk0gXFxvdGltZXMgSSJdLFszLDQsIjAiXSxbNCwzLCIwIl0sWzQsMiwiMCJdLFswLDIsIjAiXSxbMiwxLCJNIFxcb3RpbWVzIEkiXSxbMywwLCIwIl0sWzEsMSwiMCJdLFsyLDAsIj8iXSxbNCwxLCIwIl0sWzEsMl0sWzUsMywiIiwyLHsibGV2ZWwiOjIsInN0eWxlIjp7ImhlYWQiOnsibmFtZSI6Im5vbmUifX19XSxbMSw0XSxbNCw5XSxbNiw0XSxbNCw4XSxbMTEsNV0sWzYsMTBdLFsxMiwwLCIiLDEseyJzdHlsZSI6eyJib2R5Ijp7Im5hbWUiOiJkYXNoZWQifX19XSxbMTIsNywiIiwxLHsibGV2ZWwiOjIsInN0eWxlIjp7ImhlYWQiOnsibmFtZSI6Im5vbmUifX19XSxbMCwxXSxbNSwwXSxbMywxXSxbMCw2XSxbMTMsN10sWzcsNl0sWzE0LDEyXSxbMTQsNV0sWzE1LDEyXSxbMTUsMTMsIiIsMSx7InN0eWxlIjp7ImhlYWQiOnsibmFtZSI6ImVwaSJ9fX1dLFsxMywzLCIiLDEseyJzdHlsZSI6eyJib2R5Ijp7Im5hbWUiOiJkb3R0ZWQifX19XSxbNywxNl1d
\begin{tikzcd}[ampersand replacement=\&,sep=small]
	\&\& {?} \& 0 \\
	\& 0 \& {M \otimes I} \& {M \otimes I} \& 0 \\
	0 \& {M \otimes X} \& {M\oplus (M \otimes X)} \& M \& 0 \\
	\& {M \otimes X} \& {M \otimes Y} \& {M \otimes (A / I)} \& 0 \\
	\&\& 0 \& 0
	\arrow[two heads, from=1-3, to=1-4]
	\arrow[from=1-3, to=2-3]
	\arrow[from=1-4, to=2-4]
	\arrow[dotted, from=1-4, to=4-2]
	\arrow[from=2-2, to=2-3]
	\arrow[from=2-2, to=3-2]
	\arrow[equals, from=2-3, to=2-4]
	\arrow[dashed, from=2-3, to=3-3]
	\arrow[from=2-4, to=2-5]
	\arrow[from=2-4, to=3-4]
	\arrow[from=3-1, to=3-2]
	\arrow[from=3-2, to=3-3]
	\arrow[equals, from=3-2, to=4-2]
	\arrow[from=3-3, to=3-4]
	\arrow[from=3-3, to=4-3]
	\arrow[from=3-4, to=3-5]
	\arrow[from=3-4, to=4-4]
	\arrow[from=4-2, to=4-3]
	\arrow[from=4-3, to=4-4]
	\arrow[from=4-3, to=5-3]
	\arrow[from=4-4, to=4-5]
	\arrow[from=4-4, to=5-4]
\end{tikzcd}.
\end{equation}
因此 $M ⊗ X → M ⊗ Y$ 是單態射. 
\item ($1 ⟹ 4$). 顯然 $R ⊗ M ≃ M$, $IM ⊆ M$. 依假設, $I ⊗ M$ 視同 $R ⊗ M$ 的子模. 鑒於\parnote{模 $⟹$ 理想}
\begin{enumerate}
    \item $R ⊗ M → M, ∑ r_i ⊗ m_i ↦ ∑ r_im_i$ 映 $I ⊗ M$ 至 $IM$, 以及
    \item $M → R ⊗ M, m ↦ 1 ⊗ m$ 映 $IM$ 至 $I ⊗ M$, 
\end{enumerate}
因此 $R ⊗ M ≃ M$ 誘導了子對象同構 $IM ≃ I ⊗ M$. 
\end{enumerate}
    \end{proof}
\end{proposition}

\begin{proposition}[平坦模的計算性質 I]
    $M$ 是平坦模, 當且僅當對任意矩陣式 $𝐑 ⋅ \overrightarrow{𝐦} = \overrightarrow{\mathbf 0}$, 存在 $𝐀$ 使得 \parnote{零化式平凡}
    \begin{equation}
        𝐑 ⋅ \overrightarrow{𝐦} = 𝐑 ⋅ \underbracket{(𝐀  ⋅ \overrightarrow{𝐦 ′ })}\limits_{ = 𝐦 } =  \underbracket{(𝐑 ⋅𝐀)}\limits_{ = 𝐎}⋅ \overrightarrow{𝐦′} = \overrightarrow{\mathbf 0}
    \end{equation}
    \begin{proof}
        若 $M$ 平坦, 由 ses $0 → \ker (𝐑 ⋅ ) → R^m → R^n → 0$ 得 ses
        \begin{equation}
            0 → \ker (𝐑 ⋅ ) ⊗ M → M^m \xrightarrow{𝐑 ⋅ } M^n → 0. 
        \end{equation}
        由正合性, 任意 $\overrightarrow{𝐦 } ∈ \ker (R ⋅ )$ 總是形如 $𝐀 ⊗ \overrightarrow{𝐦 ′}$; 其中 $𝐀 ∈ \ker (𝐑 ⋅ )$, 即 $𝐑⋅ 𝐀 = 𝐎$. 

        反之, 若任意矩陣式 $𝐑 ⋅ \overrightarrow{𝐦} = \overrightarrow{\mathbf 0}$ 允許上述媒介 $𝐀$, 往證對有限生成理想 $I$ 總有 $IM ≃ I ⊗ M$ (同構繼承自 $M ≃ R ⊗ M$). 顯然 $I ⊗ M ↠  IM$, 下證難點 $I ⊗ M ↪ IM$: 若 $\overrightarrow 𝐚^T ⋅ \overrightarrow 𝐦 = 0$, 則存在 $𝐀$ 與 $\overrightarrow{𝐦 ′ }$ 使得 $\overrightarrow 𝐚^T ⋅ 𝐀 =\overrightarrow{\mathbf 0}^T$ 與 $\overrightarrow{𝐦} = 𝐀 ⋅ \overrightarrow{𝐦 ′ }$. 這意味著 
        \begin{equation}
            \text{原輸入} =  ∑ a ⊗ m = ∑ (aA) ⊗ m′ = 0.  
        \end{equation} 
        從而單. 
    \end{proof}
\begin{pinked}
    $𝐑 ⋅ \overrightarrow{𝐦}$ 是 $IM$ 的元素, 而二元組所在的等價類 $(𝐑, \overrightarrow{𝐦}) / ∼ _𝐀$ 是 $I ⊗ M$ 的元素. 平坦模的內蘊性質即``$∼ _A$ 與 $⋅$ 是相同的等價關係''. 
\end{pinked}
\end{proposition}

\begin{proposition}[平坦模的計算性質 II]
    $M$ 是平坦模, 當且僅當一切有限表現模出發的態射 $X → M$ 通過某一有限自由模分解. \parnote{Plat → (Libre) → F.P.}
    \begin{proof}
        假定 $M$ 平坦, $R^m \xrightarrow {B ⋅ } R^n → X → 0$ 正合. 態射 $φ : X → M$ 滿足以下性質
        \begin{enumerate}
            \item $φ$ 由 $\{x_i ↦ m_i\}$ 決定, 其中 $x_i$ 是生成元, 即, $e_i ∈ R^n$ 在 $X$ 中的像.  
            \item 正合列 $0 → (X, M) → M^n → M^m$ 說明向量 $φ ∼ (m_i) ∈ M^n$ 恰使 $B^T ⋅ (m_i)=0$. 
        \end{enumerate}
        取媒介 $A$ 使得 $B^T ⋅ A^T =O$ 且 $A^T⋅ (m′ _j) = m_i$. 此時 $(A⋅ ): R^n → R^l$ 即爲所求. 
        
    反之, 證明過程類似. 
    \end{proof}
\begin{pinked}
    以上給出``séquence plat''中 representable syzygy 的性質: 若 $0 → (X, M )→ M^n → M^m$, 則必然能填充作 
    \begin{equation}
        % https://q.uiver.app/#q=WzAsNCxbMCwwLCJGXmwgIl0sWzIsMCwiRl5uICJdLFs0LDAsIkZebSJdLFsxLDEsIihYLCBNKSJdLFswLDFdLFsxLDJdLFszLDEsIiIsMix7InN0eWxlIjp7InRhaWwiOnsibmFtZSI6Imhvb2siLCJzaWRlIjoidG9wIn19fV0sWzAsMywiIiwxLHsic3R5bGUiOnsiaGVhZCI6eyJuYW1lIjoiZXBpIn19fV1d
\begin{tikzcd}[ampersand replacement=\&]
	{F^l } \&\& {F^n } \&\& {F^m} \\
	\& {(X, M)}
	\arrow[from=1-1, to=1-3]
	\arrow[two heads, from=1-1, to=2-2]
	\arrow[from=1-3, to=1-5]
	\arrow[hook, from=2-2, to=1-3]
\end{tikzcd}.
    \end{equation}
\end{pinked} \parnote{函子語言表述}
\end{proposition}

\begin{theorem}[Lazard 定理, 見 \cite{BSMF_1969__97__81_0}]
    平坦模恰是有限自由模的濾過餘極限. 
    \begin{proof}
        顯然 $R^n ⊗ -$ 和濾過餘極限正合, 從而有限自由模的濾過餘極限是平坦模. 反之, 將平坦模寫作有限表現模的濾過餘極限, 將每一態射 $X → M$ 通過有限自由模分解 (濾過子系統), 仔細驗證子系統共尾 (cofinal) 即可. 
    \end{proof}
\end{theorem}

\begin{remark}
    同理, 平坦模恰是自由模 (resp. 投射模, 有限生成投射模) 的濾過餘極限. 
\end{remark}

\subsection{特征模理論}

\begin{definition}[特征模] 
    函子 $(-)^+ := (-, ℚ / ℤ )_ℤ: 𝐌𝐨𝐝_R → 𝐌𝐨𝐝_{R^{\mathrm{op}}}$ 映 $M$ 至特征模. 
\end{definition}

\begin{proposition}[特征模函子的性質]\label{ChaMod}
    $ℚ / ℤ$ 是內射餘生成子; 相應地, $(-) ^+$ 是正合的忠實函子. 作爲推論: 
    \begin{enumerate}
        \item $A = 0$ 當且僅當 $A^+ = 0$; \parnote{特征模更像線性泛函!}
        \item $f$ 單射 (resp. 滿射) 當且僅當 $f^+$ 滿射 (resp. 單射); 
        \item $θ$ 正合當且僅當 $θ ^+$ 正合; 
        \item $X$ 平坦當且僅當 $X^+$ 內射; \parnote{Lambek}
        \item $(X ⊗ ^L Y)^+ ≃ R\mathrm{Hom}(X, Y^+)$; 
        \item $X$ 的平坦維度等於 $X^+$ 的內射維度. 
    \end{enumerate}
\end{proposition}

\begin{example}[應用: 對偶商空間]
    子模 $N ⊆ M$ 誘導了單射 $(M / N)^+ → M^+$. 泛函 $f ∈ M^+$ 屬於該像, 當且僅當 $f|_N =0$;
\end{example}

\begin{theorem}[應用: 模範疇有足夠的多內射對象]
    即任意對象存是某內射模的子模. \parnote{Bare 定理}
    \begin{proof}
        考慮 $R^{⊕ M^+} ↠ M^+$, 得 $M ↪ M^{++} ↪ (R^{⊕ M^+})^+$.
    \end{proof}
\end{theorem}

\begin{theorem}[應用: 內射模結構定理]
    內射 $R$-模形如``自由 $R^{\mathrm{op}}$-模之特征模''的直和項. 
    \begin{proof}
        $⟹$ 由以上 $M ↪ (R^{⊕ M^+})^+$ 給出. 另一方向顯然. 
    \end{proof}
\end{theorem}

\subsection{投射模的結構定理 (簡單的 transfinite dévissage)}

\begin{definition}[$α$-濾過與分次化]
    給定序數 $α$, 稱 $M_α$ 是模 $M$ 關於 $α$ 的濾過, 當且僅當 
    \begin{enumerate}
        \item $M_0 = 0$; \parnote{初始}
        \item $M_γ ⊆ M_{γ+1}$ 是子模; \parnote{後繼}
        \item 對極限序數 $γ$, 有 $M_γ := ⋃_{δ < γ }M_δ$. \parnote{極限}
    \end{enumerate}
\end{definition}

\begin{example}[超限歸納的一些例子]
    \begin{adjustwidth}{20pt}{}
        \begin{proposition}[簡答例子]
            定義 $M$ 關於濾過 $M_α$ 的分次模爲 $\mathrm{Gr}(M, α) := ⨁_{γ < α } M_{γ +1 }/ M_{γ}$. 若對任意序數 $γ < α$, $M_γ$ 總是 $M_{γ+1}$ 的直和項, 則 $\mathrm{Gr}(M, α) ≃ M$. 
            \begin{proof}
                爲證明上述對任意序數成立, 只需考慮初始, 後繼, 極限三類序數. 
                \begin{enumerate}
                    \item $M_0 = 0 = \mathrm{Gr}(M_0, 0)$, 顯然. 
                    \item 若 $\mathrm{Gr}(M_β, β) ≃ M_β$, 則 
                    \begin{equation}
                        \mathrm{Gr}(M_{β +1}, β+1) ≃  \mathrm{Gr}(M_{β}, β)  ⊕ \frac{M_{β +1}}{M_β }  ≃ M_β ⊕ \frac{M_{β +1}}{M_β } ≃ M_{β +1 }. 
                    \end{equation}
                    \item 若命題對一切 $γ < β$ 成立, 則 
                    \begin{equation}
                        \mathrm{Gr}(M_β , β ) ≃ ⋃ _{γ < β} \mathrm{Gr}(M_γ  , γ) ≃⋃ _{γ < β} M_γ  ≃ M_β. 
                    \end{equation}
                \end{enumerate}
            \end{proof}
        \end{proposition}      
        \end{adjustwidth}

        \begin{adjustwidth}{20pt}{}
\begin{proposition}[可數生成模的結構]\label{cgMod}
    假定 $M = ⨁_{i ∈ I} M_i$ 是可數生成模的任意直和, $\{f_n\}_{n ∈ ℕ}$ 是可數個自同態. 則存在濾過 $\{M_γ\}_{γ ≤ β}$ 使得 
    \begin{enumerate}
        \item $M_{γ +1} = M_γ ⊕ ⨁ (\text{countably many $M_i$'s})$. \parnote{新添可數}
        \item 任意 $(γ , n)$, $f_n$ 總是限制在 $M_γ$ 上的自同態. \parnote{子-自同態}
    \end{enumerate}
    \begin{proof}
        依次構造 $0$, 後繼, 極限. 
        \begin{enumerate}
            \item $M_0 = 0$. 略. 
            \item ($β → β +1 $) 若 $M_{≤ β}$ 構造成功, 且 $M_{β} ≠ M$. 下構造 $M_{β +1 }$. 任取 $M_β$ 之外的可數生成直和項 $M^0$, 則 $∑ _{n ∈ ℕ}\mathrm{im}(f_n(M^0))$ 可數生成, 因此是某一可數生成直和項 $M^1$ 的子模. 類似地構造 $M^k$, 記 $M_{β +1} = ⨁_{k≥ 0} M^k$ 即可. 
            \item ($(< β) → β$) 無需構造, 僅需驗證自同態的歸納極限是自同態, 非常顯然. 
        \end{enumerate}
    \end{proof}
\end{proposition} 
\begin{remark}
    約定``可數生成''表示至多可數生成, 不必是無限的.  
\end{remark}
            \end{adjustwidth}

        \begin{adjustwidth}{20pt}{}
            \begin{proposition}[有趣的例子: 雙點集問題; 問題 7.12, \cite{PBMSet}; \cite{Bienias2013}]
                是否存在 $ℝ²$ 的子集 $A$, 其與任意直線恰交於兩點? 
                \begin{proof}
                    用序數 $α$ 標記 $ℝ²$ 的全體直線 $\{ℓ _β \}_{β < α}$, 並對應地構造 $\{A_β\}$, 滿足 
                    \begin{enumerate}
                        \item 對任意 $β < α$, $|A_β| ≤ 2$; 
                        \item 對任意 $β < α$, $⋃_{γ ≤  β} A_γ$ 不存在共線的三點; 
                        \item 對任意 $β < α$, $⋃_{γ ≤  β} A_γ$ 與 $ℓ _β$ 有且僅有兩個交點. 
                    \end{enumerate}
                    若能暢通無阻地進行後繼歸納和極限歸納, 則原命題爲真. 
                    \begin{enumerate}
                        \item ($β =0$). 任取 $ℓ _0$ 與 $A_0$ 即可, 無較話頭. 
                        \item ($β → β +1$). $ℓ _{β +1}$ (新線) 與 $⋃_{γ ≤  β} A_γ$ (舊集) 的交點數量只能爲 $\{0,1,2\}$. 
                        \begin{enumerate}
                            \item 若交點數爲 $2$, 則直接取 $A_{β  +1 } = ∅$.\parnote{基數定義爲序數的 $|-|$-等價類的極小元; 對無窮基數: \\ $λ + κ = \max{}$, \\ $λ ⋅ κ = {\max{}}$. } 
                            \item 若交點數爲 $1$, 則需有 $|A_{β  +1 }| = 1$. 新的點既要落在落在 $ℓ _{β +1}$ (available) 之中, 又要落在 $⋃_{γ ≤  β} A_γ$ 的``兩點成線集'' (non-available) 之外. 由於 $|\text{non-avali}|$ 不超過 $|β|$ 的加法乘法式, 從而不超過 $\max \{ℵ_0, |β|\}$. 不妨設 $|α|$ 是基數, 則 $|β| < |α|$. 這說明新點有處可落.  
                            \item 若交點數爲 $0$, 同理以上. 
                        \end{enumerate}
                        \item ($\{< β\} → β$, $β$ 是極限序數) 同上, $ℓ _{β}$ (新線) 與 $⋃_{γ  <  β} A_γ$ (全體舊集之並) 的交點數量只能爲 $\{0,1,2\}$. 
                        \begin{enumerate}
                            \item 若交點數爲 $2$, 則 $A_β = ∅$. 
                            \item 若交點數爲 $0, 1$, 參考以上對``新點有處可落''之證明. 
                        \end{enumerate}
                    \end{enumerate}
                \end{proof}
            \end{proposition}
        \end{adjustwidth}
\end{example}

\begin{proposition}[投射模的結構 I] 
    投射模是可數生成投射模的直和. 約定: 可數 = 至多可數. \parnote{I. Kaplansky}
    \begin{proof}
        投射模 $P$ 是自由模 $M$ 的直和項, 即某一冪等自同態 $e$ 的像. 對資料 $(M, \{\mathrm{id}, e\})$ 使用 \ref{cgMod}, 得 $P$ 的可數直和濾過. 
    \end{proof}
\end{proposition}

\begin{proposition}[投射模的結構 II]
    $P$ 是投射模, 當且僅當存在一組 $\{(e_i, f_i)\}_{i ∈ I} ⊆ P × (P, R)$, 使得\parnote{對偶基}
    \begin{equation}
        ∀ x ∈ R \  (x = ∑ _{i ∈ I} e_i ⋅ f_i(x)\quad \text{爲有限和}). 
    \end{equation}
\begin{proof}
    若有上述``基'', 則有態射
    \begin{enumerate}
        \item $R^{⊕ I} → P, \quad r_i ↦ e_i ⋅ r_i$; 
        \item $P → R^{⊕ I}, \quad p ↦ (f_i(p))_{i ∈ I}$. 
    \end{enumerate}
    由於 $P → R^{⊕ I} → P,\quad p ↦ e_i ⋅ f_i (p)$ 是恆等, 得 $P$ 是截面. \parnote{本質直和項}

    反之, 若 $P$ 是投射模, 依照直和項 $P → R^{⊕ I} → P$ 直接構造即可. 
\end{proof}
\end{proposition}

\begin{remark}
    對投射模, 似乎沒有很好的 Baer 判別法. 見 \cite{Trlifaj_2018}. 
\end{remark}


\subsubsection{\texorpdfstring{$\mathrm{Ext}$}{PDFstring} 與 \texorpdfstring{$\mathrm{Tor}$}{PDFstring} 的 Baer 和刻畫}
\begin{abstract}
    如何將群 $\mathrm{Tor}$ 中元素具體地取出來? 是否存在類似米田 $\mathrm{Ext}^n$-群的構造? 

先給出 Abel 群中 $\mathrm{Tor}(A, X)$ 的構造. 
\end{abstract}

\begin{example}[Abel 群範疇中的簡單構造]
    $\mathrm{Tor}_A(A, B)$ 就是扭元構成的群, 其中生成元形如三元組
    \begin{equation}
        \left\{(a, n, b) ∈ A × ℤ × B \quad ∣ \quad (an = 0) ∧ (nb = 0)\right\}. 
    \end{equation} \parnote{左: 左群元素 \\ 中: 零化數乘 \\ 右: 右群元素}
    對以上元素張成的自由 Abel 群, 考慮如下生成關係:  
    \begin{enumerate}
        \item (生成關係 I) $(-, n, b)$ 與 $(a, n, -)$ 都是 $ℤ$-線性映射, 即保持加法同態; 
        \item (生成關係 II-l) 若 $(a, mn, b)$ 與 $(am, n, b)$ 都是生成元, 則兩者等同; 
        \item (生成關係 II-r) 若 $(a, mn, b)$ 與 $(a, m, nb)$ 都是生成元, 則兩者等同.
    \end{enumerate}
\end{example}

\begin{example}[$\mathrm{Tor}_ℤ(-,-)$ 的雙函子性]
    加法結構, 零元等都是直接的. 下證明 $f : X → Y$ 給出良定義的
    \begin{equation}
        f_∗ : \mathrm{Tor}(A, X) → \mathrm{Tor}(A, Y),\quad (a, n, x) ↦ (a, n, f(x)). 
    \end{equation}
    按步就班地檢驗泛性質: 映射 $G(A × ℤ × X) → \mathrm{Tor}(A, Y)$ 將三種生成關係零化, 則來源下降至商空間 $\mathrm{Tor}(A, X)$. 這由群同態的定義直接保證. \parnote{驗證函子性} 另一側同理. 
\end{example}

\begin{example}
    \parnote{第一處的核} 給定短正合列 $0 → A \overset f→ B \overset g→ C → 0$ 與 $-⊗X$, 則
    \begin{equation}
        \ker [f ⊗ X : A ⊗ X → B ⊗ Y] ⊆ A ⊗ X. 
    \end{equation}
    基於張量秩的一些考量, 這一核由秩 $1$ 的張量 $a ⊗ x$ 生成. 
    \begin{itemize}
        \item $a ⊗ x$ 應滿足 $f(a) ⊗ x = 0$. 由於 $f$ 是單射, 故存在 $f(a) = b ⋅ n$ 使得 $n ⋅ x = 0$. 同時 $c := g(b)$ 被 $n$ 零化. 
        \item 三元組 $(c,n,x)$ ($cn = 0$ 且 $nx = 0$) 的某一等價類決定了 $a ⊗ x$. 形式地看, $a$ 可取作任意 $f⁻¹ (g⁻¹ (c) ⋅ n)$ 中元素, 實際上 $a ⊗ x$ 與原像的選取無關. \parnote{驗證!}
    \end{itemize}
\end{example}

\begin{proposition}[第一處連接態射]
    給定 $0 → A \xrightarrow f B \xrightarrow g C → 0$, 對任意 $X$ 總有長正合列 
    \begin{equation}
        0 → \mathrm{Tor}(A, X) → \mathrm{Tor}(B, X) → \mathrm{Tor}(C, X) \xrightarrow δ A ⊗ X → B ⊗ X → C ⊗ X → 0. 
    \end{equation}
    特別地, $δ : \{(c, n ,x) \} ↠ \{f⁻¹ (g⁻¹ (c) ⋅ n) ⊗ x\} ↪ A ⊗ X$. 
\end{proposition}\parnote{證明!}

\begin{definition}[生成對]
    假定 $U$ 是範疇 $𝒜$ 的生成元. \parnote{$\mathrm{Hom}(U,-)$ 即 $∙ ∈ (-)$} 若存在反變函子 $K: 𝒜 → ℬ$ 與 $L: ℬ → 𝒜$ 使得存在生成元 $D = K(U)$ 與 $LK(U) L(D) = U$. 
\end{definition}

\begin{remark}
    特例: $L, K = (-, R)$ 與 $U,D = R$.  
\end{remark}

\begin{definition}[生成對的 $\mathrm{Tor}$-群]
    \begin{pinked}
    一般的 Abel 範疇顯然沒有 Tor, 定義所謂 Tor 需要一些附加條件 (需要往模範疇處靠, 但不必像模範疇這般好).
    \end{pinked}\parnote{待論證}
    假定 $(U = L(D) ∈ 𝒜, D = K(U) ∈ ℬ)$ 是預張量範疇對. 此時的 $\mathrm{Tor}_n : 𝒜 × ℬ → 𝐁𝐢𝐠𝐀𝐛$ 定義如下. 選定大 Abel 群 $\mathrm{Tor}_n (G, C)$. 
    \begin{enumerate}
        \item 生成元選作三元組 $(μ, L_∙,ν) = (G\xleftarrow μ L_∙ ∣ DL_∙ \xrightarrow ν C)$. 其中, 
        \begin{enumerate}
            \item $L^∙ = [L_n → \cdots → L_0]$ 是有限生成的 $U$-復形, $μ:L → G$ 是鏈映射; 
            \item $D(L)^∙ := [K(L_0) → \cdots → K(L_n)]$ 是對偶復形, $ν: D(L) → C$ 是鏈映射. 
        \end{enumerate}
        \item 等價關係: 對任意 $α : P_∙ → L_∙$, 總有
        \begin{equation}
            (G\xleftarrow{μ} L_∙ ∣ DL_∙ \xrightarrow {ν ∘ D(α)} C) ∼ (G\xleftarrow {μ ∘ α} P_∙ ∣ DP_∙ \xrightarrow {ν} C). 
        \end{equation}
        \item 驗證這是雙加法函子. 加法結構由如下 Baer 和給出
        \begin{equation}
            (μ, L^∙,ν) + (λ, K^∙,ι) := ((μ, λ), L^∙ ⊕ K^∙, (ν, ι)^T).  
        \end{equation}
    \end{enumerate} 
    \begin{proof}
        \parnote{待補充}
    \end{proof}
\end{definition}

\begin{proposition}
    證明存在長正合列. 
    \begin{proof}
        \parnote{待補充}
    \end{proof}
\end{proposition}

\begin{example}
    對模範疇, $\mathrm{Tor}_0 = ⊗$. 
\end{example}

\begin{proposition}[對稱定理]
    $\mathrm{Tor}^{𝒜 ∣ ℬ}_∙(−, ∣) ≃ \mathrm{Tor}^{ℬ ∣ 𝒜}_∙(∣, −)$.
\end{proposition}

\begin{proposition}[對消公式]
    存在如下``自然同態''. \parnote{待仔細刻畫}
    \begin{enumerate}
        \item $\mathrm{Ext}_{ℬ}^n(-, X) ⊗ \mathrm{Tor}^{𝒜 ∣ ℬ}_{n+k}(Y,-) → \mathrm{Tor}^{𝒜 ∣ ℬ}_{k}(Y,X)$. 
        \item $\mathrm{Ext}_{ℬ}^n(X, -) ⊗ \mathrm{Tor}^{ℬ ∣ 𝒜}_{n+k}(-, Y) → \mathrm{Tor}^{ℬ ∣ 𝒜}_{k}(X, Y)$.
        \item 嵌套公式? 
    \end{enumerate}
\parnote{廣義伴隨?}
\end{proposition}












\newpage 
\begin{abstract}
    給出 $\mathrm{Ext}$ 與 $\mathrm{Tor}$ 的 Baer 和構造. 前者的系統性描述見 \cite{mitchell1965theory} 之 Chapter VII, 後者見 \cite{maclane2012homology} 之 Chapter five. 

    加入 satellite $\mathrm{Nat}(\mathrm{Ext}^n_A(X, - ), Y ⊗_A - ) ≃ \mathrm{Tor}_n^A(Y, X)$ 以串通之? 

    $\mathrm{Ext}$-部分照抄個人筆記. 茲將提綱暨主要結論悉列如下: 
    \begin{enumerate}
        \item 形式化定義雙函子 $\mathrm{Hom}(-,-)$ 的``加群''結構: $f + g := ∇ ∘ (f ⊕ g) ∘ Δ$; \parnote{真類亦可}
        \item 形式化定義雙函子 $\mathrm{Ext}^1(-,-)$ 的``加群''結構: $θ + τ = ∇ ∘ (θ ⊕ τ) ∘ Δ$; \parnote{擴張同構類} 
        \item $\mathrm{Ext}^n$ 的雙函子性, $\mathrm{Ext}$-乘法結構;  
        \item 連接態射 $δ^0$, $δ^{≥ 1}$ 保持長正合列 (附: $\mathrm{Ext}^1=0$, $\mathrm{Ext}^2=0$ 的充要條件); \parnote{$≥ 1$ 與 $1$ 證明相同, 歸納}
        \item 小技巧: $(3 × 3)$-模型的一對逆元 (推論: $\mathrm{Ext}^2$ 中零元恰補全作推出拉回); 
        \item $\mathrm{Ext}$-群的 zigzag 等價方式: 最多等價三次. 
        \item 假定局部小範疇, 足夠投射對象, 則以上與導出範疇相同. 
    \end{enumerate}
\end{abstract}

\begin{example}[$\mathrm{Ext}^0=\mathrm{Hom}$ 的双函子性]
    以下验证 $\mathrm{Hom}(-,-)$ 的双函子性 (从 Abel 范畴 $\mathcal A$ 至加群 (包括类) 范畴), 其包括两点: 一切 $\mathrm{Hom}(-,-)$ 是加群, 加法结构与来源, 去向的变换相容.
    \begin{enumerate}
        \item (定义 $(B,A)$ 的加法结构) 定义对角变换 $\Delta_X:X\xrightarrow{\binom 11} X\oplus X$ \parnote{$Δ$: 分開} 与余对角变换 $\nabla_X:X\oplus X\xrightarrow{(1\,\,1)} X$ \parnote{$∇$ 合上}. 既将其视作自然变换, 往后省略 $\Delta$ 与 $\nabla$ 的角标. 态射 $f,g\in (B,A)$ 的加法定义如下
              \begin{equation}
                  % https://q.uiver.app/#q=WzAsNyxbMCwyLCJCXFxvcGx1cyBCIl0sWzIsMiwiQVxcb3BsdXMgQSJdLFsyLDEsIkEiXSxbMCwxLCJCXFxvcGx1cyBCIl0sWzAsMCwiQiJdLFsyLDAsIkEiXSxbMywyLCJcXG1hdGhybXt9Il0sWzAsMSwiXFxiaW5vbXtmXFxxdWFkIH17XFxxdWFkIGd9Il0sWzIsMSwiXFxEZWx0YSAiXSxbMCwzLCIiLDAseyJsZXZlbCI6Miwic3R5bGUiOnsiaGVhZCI6eyJuYW1lIjoibm9uZSJ9fX1dLFszLDIsIihmXFxxdWFkIGcpIiwwLHsic3R5bGUiOnsiYm9keSI6eyJuYW1lIjoiZGFzaGVkIn19fV0sWzMsNCwiXFxuYWJsYSAiXSxbMiw1LCIiLDIseyJsZXZlbCI6Miwic3R5bGUiOnsiaGVhZCI6eyJuYW1lIjoibm9uZSJ9fX1dLFs0LDUsImZcXG9wbHVzIGciXV0=
                  \begin{tikzcd}[ampersand replacement=\&]
                      B \&\& A \\
                      {B\oplus B} \&\& A \\
                      {B\oplus B} \&\& {A\oplus A} \& {\mathrm{}}
                      \arrow["{f\oplus g}", from=1-1, to=1-3]
                      \arrow["{\nabla }", from=2-1, to=1-1]
                      \arrow["{(f\quad g)}", dashed, from=2-1, to=2-3]
                      \arrow[Rightarrow, no head, from=2-3, to=1-3]
                      \arrow["{\Delta }", from=2-3, to=3-3]
                      \arrow[Rightarrow, no head, from=3-1, to=2-1]
                      \arrow["{\binom{f\quad }{\quad g}}", from=3-1, to=3-3]
                  \end{tikzcd}.
              \end{equation}
              以算式记之, $(f+g)=\nabla (f\oplus g)\Delta$. 此处, $(f\oplus g):=\binom{f\quad }{\quad g}$.
        \item 显然 $0$ 是零元, $f$ 的逆元是 $-f$.
        \item 来源的变换 (拉回) $\beta^\ast : \mathrm{Hom}(B,A)\to \mathrm{Hom}(B',A),\quad f\mapsto f\circ \beta$ 是良定义的群同态, 去向的变换 (推出) 亦然.
    \end{enumerate}
\end{example}

\begin{remark}
    对 $\mathrm{Ext}^0=\mathrm{Hom}$ 而言, ``推出''与``拉回''是前向复合与后项复合; 对 $\mathrm{Ext}^{\geq 1}$ 而言, ``推出''与``拉回''由熟悉的 $2\times 2$ 方块构造. 这两个名词可以统一解释作 $\mathrm{Ext}$-群关于来源与去向的变换.
\end{remark}

\begin{definition}[短正合列的等价关系]
    考虑拆解短正合列的同态为如下四行正合列:
    \begin{equation}
        % https://q.uiver.app/#q=WzAsMjIsWzIsMCwiQSJdLFszLDAsIkIiXSxbNCwwLCJDIl0sWzIsMywiQSciXSxbMywzLCJCJyJdLFs0LDMsIkMnIl0sWzIsMiwiQSciXSxbMywyLCJFJyJdLFs0LDIsIkMiXSxbMiwxLCJBJyJdLFs0LDEsIkMiXSxbMywxLCJFIl0sWzEsMCwiMCJdLFsxLDEsIjAiXSxbMSwyLCIwIl0sWzEsMywiMCJdLFs1LDAsIjAiXSxbNSwxLCIwIl0sWzUsMiwiMCJdLFs1LDMsIjAiXSxbMCwwLCJbRV0iXSxbMCwzLCJbRSddIl0sWzgsNSwiXFxnYW1tYSJdLFs3LDRdLFs3LDhdLFs0LDVdLFswLDksIlxcYWxwaGEiLDJdLFswLDFdLFsxLDJdLFsyLDEwLCIiLDEseyJsZXZlbCI6Miwic3R5bGUiOnsiaGVhZCI6eyJuYW1lIjoibm9uZSJ9fX1dLFs5LDExXSxbMTEsMTBdLFsxLDExXSxbNiwzLCIiLDEseyJsZXZlbCI6Miwic3R5bGUiOnsiaGVhZCI6eyJuYW1lIjoibm9uZSJ9fX1dLFs5LDYsIiIsMSx7ImxldmVsIjoyLCJzdHlsZSI6eyJoZWFkIjp7Im5hbWUiOiJub25lIn19fV0sWzEwLDgsIiIsMSx7ImxldmVsIjoyLCJzdHlsZSI6eyJoZWFkIjp7Im5hbWUiOiJub25lIn19fV0sWzExLDddLFs2LDddLFszLDRdLFsxMiwwXSxbMTMsOV0sWzE0LDZdLFsxNSwzXSxbMiwxNl0sWzEwLDE3XSxbOCwxOF0sWzUsMTldLFsyMCwyMSwiKFxcYWxwaGEsXFxiZXRhLFxcZ2FtbWEpIiwxXSxbMjMsMjIsIlxcdGV4dHtQQn0iLDEseyJzaG9ydGVuIjp7InNvdXJjZSI6MjAsInRhcmdldCI6MjB9LCJzdHlsZSI6eyJib2R5Ijp7Im5hbWUiOiJub25lIn0sImhlYWQiOnsibmFtZSI6Im5vbmUifX19XSxbMjYsMzIsIlxcdGV4dHtQT30iLDEseyJzaG9ydGVuIjp7InNvdXJjZSI6MjAsInRhcmdldCI6MjB9LCJzdHlsZSI6eyJib2R5Ijp7Im5hbWUiOiJub25lIn0sImhlYWQiOnsibmFtZSI6Im5vbmUifX19XV0=
        \begin{tikzcd}[ampersand replacement=\&]
            {[E]} \& 0 \& A \& B \& C \& 0 \\
            \& 0 \& {A'} \& E \& C \& 0 \\
            \& 0 \& {A'} \& {E'} \& C \& 0 \\
            {[E']} \& 0 \& {A'} \& {B'} \& {C'} \& 0
            \arrow["{(\alpha,\beta,\gamma)}"{description}, from=1-1, to=4-1]
            \arrow[from=1-2, to=1-3]
            \arrow[from=1-3, to=1-4]
            \arrow[""{name=0, anchor=center, inner sep=0}, "\alpha"', from=1-3, to=2-3]
            \arrow[from=1-4, to=1-5]
            \arrow[""{name=1, anchor=center, inner sep=0}, from=1-4, to=2-4]
            \arrow[from=1-5, to=1-6]
            \arrow[Rightarrow, no head, from=1-5, to=2-5]
            \arrow[from=2-2, to=2-3]
            \arrow[from=2-3, to=2-4]
            \arrow[Rightarrow, no head, from=2-3, to=3-3]
            \arrow[from=2-4, to=2-5]
            \arrow[from=2-4, to=3-4]
            \arrow[from=2-5, to=2-6]
            \arrow[Rightarrow, no head, from=2-5, to=3-5]
            \arrow[from=3-2, to=3-3]
            \arrow[from=3-3, to=3-4]
            \arrow[Rightarrow, no head, from=3-3, to=4-3]
            \arrow[from=3-4, to=3-5]
            \arrow[""{name=2, anchor=center, inner sep=0}, from=3-4, to=4-4]
            \arrow[from=3-5, to=3-6]
            \arrow[""{name=3, anchor=center, inner sep=0}, "\gamma", from=3-5, to=4-5]
            \arrow[from=4-2, to=4-3]
            \arrow[from=4-3, to=4-4]
            \arrow[from=4-4, to=4-5]
            \arrow[from=4-5, to=4-6]
            \arrow["{\text{PO}}"{description}, draw=none, from=0, to=1]
            \arrow["{\text{PB}}"{description}, draw=none, from=2, to=3]
        \end{tikzcd}
    \end{equation}
    依照五引理, 中间两行正合列同构. 这表明 $(\alpha,\beta,\gamma)$ 实际由 $(\alpha,\gamma)$ 决定 (同构的意义下). 从短正合列的同构类的等价关系下, 定义 $\alpha_\ast [E]=\gamma^\ast[E']$. 简单地, 记作 $\alpha E=E'\gamma$.
\end{definition}

\begin{proposition}[$\mathrm{Ext}^1$ 作为双函子]
    需要依次定义 $\mathrm{Ext}^1(B,A)$ 的加法结构, 并验证加法结构与来源, 去向的变换相容.
    \begin{enumerate}
        \item 给定短正合列 $E_i:[0\to B\to X_i\to A\to 0]$, 定义加法 $E_1+E_2=\nabla (E_1\oplus E_2)\Delta$ 如下:
              \begin{equation}
                  % https://q.uiver.app/#q=WzAsMTcsWzIsMCwiQlxcb3BsdXMgQiJdLFs2LDAsIkFcXG9wbHVzIEEiXSxbNCwwLCJYXzFcXG9wbHVzIFhfMiJdLFsxLDAsIjAiXSxbNywwLCIwIl0sWzIsMSwiQiJdLFs0LDEsIlkiXSxbNiwxLCJBXFxvcGx1cyBBIl0sWzcsMSwiMCJdLFsxLDEsIjAiXSxbNiwyLCJBIl0sWzcsMiwiMCJdLFsyLDIsIkIiXSxbMSwyLCIwIl0sWzQsMiwiWiJdLFswLDAsIkVfMVxcb3BsdXMgRV8yIl0sWzAsMiwiRV8xK0VfMiJdLFswLDJdLFsyLDFdLFszLDBdLFsxLDRdLFswLDUsIlxcbmFibGEiLDJdLFs1LDZdLFs2LDddLFs3LDhdLFs5LDVdLFsxLDcsIiIsMSx7ImxldmVsIjoyLCJzdHlsZSI6eyJoZWFkIjp7Im5hbWUiOiJub25lIn19fV0sWzEwLDcsIlxcRGVsdGEgIiwyXSxbNSwxMiwiIiwyLHsibGV2ZWwiOjIsInN0eWxlIjp7ImhlYWQiOnsibmFtZSI6Im5vbmUifX19XSxbMTMsMTJdLFsxMiwxNF0sWzE0LDEwXSxbMTAsMTFdLFsyLDZdLFs2LDE0XSxbMjEsMzMsIlxcdGV4dHvmjqjlh7p9IiwxLHsic2hvcnRlbiI6eyJzb3VyY2UiOjIwLCJ0YXJnZXQiOjIwfSwic3R5bGUiOnsiYm9keSI6eyJuYW1lIjoibm9uZSJ9LCJoZWFkIjp7Im5hbWUiOiJub25lIn19fV0sWzM0LDI3LCJcXHRleHR75ouJ5ZuefSIsMSx7InNob3J0ZW4iOnsic291cmNlIjoyMCwidGFyZ2V0IjoyMH0sInN0eWxlIjp7ImJvZHkiOnsibmFtZSI6Im5vbmUifSwiaGVhZCI6eyJuYW1lIjoibm9uZSJ9fX1dXQ==
                  \begin{tikzcd}[ampersand replacement=\&]
                      {E_1\oplus E_2} \& 0 \& {B\oplus B} \&\& {X_1\oplus X_2} \&\& {A\oplus A} \& 0 \\
                      \& 0 \& B \&\& Y \&\& {A\oplus A} \& 0 \\
                      {E_1+E_2} \& 0 \& B \&\& Z \&\& A \& 0
                      \arrow[from=1-2, to=1-3]
                      \arrow[from=1-3, to=1-5]
                      \arrow[""{name=0, anchor=center, inner sep=0}, "\nabla"', from=1-3, to=2-3]
                      \arrow[from=1-5, to=1-7]
                      \arrow[""{name=1, anchor=center, inner sep=0}, from=1-5, to=2-5]
                      \arrow[from=1-7, to=1-8]
                      \arrow[Rightarrow, no head, from=1-7, to=2-7]
                      \arrow[from=2-2, to=2-3]
                      \arrow[from=2-3, to=2-5]
                      \arrow[Rightarrow, no head, from=2-3, to=3-3]
                      \arrow[from=2-5, to=2-7]
                      \arrow[""{name=2, anchor=center, inner sep=0}, from=2-5, to=3-5]
                      \arrow[from=2-7, to=2-8]
                      \arrow[from=3-2, to=3-3]
                      \arrow[from=3-3, to=3-5]
                      \arrow[from=3-5, to=3-7]
                      \arrow[""{name=3, anchor=center, inner sep=0}, "{\Delta }"', from=3-7, to=2-7]
                      \arrow[from=3-7, to=3-8]
                      \arrow["{\text{推出}}"{description}, draw=none, from=0, to=1]
                      \arrow["{\text{拉回}}"{description}, draw=none, from=2, to=3]
                  \end{tikzcd}
              \end{equation}
        \item 零元即可裂短正合列, 置上图 $E_1:[0\to B\xrightarrow i C\xrightarrow c A\to 0]$ 与 $E_2:[0\to B\to B\oplus A\to A \to 0]$, 其和为
              \begin{equation}
                  % https://q.uiver.app/#q=WzAsMTcsWzIsMCwiQlxcb3BsdXMgQiJdLFs2LDAsIkFcXG9wbHVzIEEiXSxbNCwwLCJDXFxvcGx1cyBCXFxvcGx1cyBBIl0sWzEsMCwiMCJdLFs3LDAsIjAiXSxbMiwxLCJCIl0sWzQsMSwiQ1xcb3BsdXMgQSJdLFs2LDEsIkFcXG9wbHVzIEEiXSxbNywxLCIwIl0sWzEsMSwiMCJdLFs2LDIsIkEiXSxbNywyLCIwIl0sWzIsMiwiQiJdLFsxLDIsIjAiXSxbNCwyLCJDIl0sWzAsMCwiRV8xXFxvcGx1cyBFXzIiXSxbMCwyLCJFXzErRV8yIl0sWzAsMiwiXFxsZWZ0KFxcc3Vic3RhY2t7aVxcLFxcLDBcXFxcMFxcLFxcLDFcXFxcMFxcLFxcLDB9XFxyaWdodCkiXSxbMiwxLCJcXGJpbm9te2NcXCxcXCwwXFwsXFwsMH17MFxcLFxcLDBcXCxcXCwxfSJdLFszLDBdLFsxLDRdLFswLDUsIigxXFwsXFwsMSkiLDJdLFs1LDYsIlxcYmlub20gaTAiLDJdLFs2LDcsIlxcYmlub217Y1xcLFxcLDB9ezBcXCxcXCwxfSJdLFs3LDhdLFs5LDVdLFsxLDcsIiIsMSx7ImxldmVsIjoyLCJzdHlsZSI6eyJoZWFkIjp7Im5hbWUiOiJub25lIn19fV0sWzEwLDcsIlxcYmlub217MX17MX0iLDJdLFs1LDEyLCIiLDIseyJsZXZlbCI6Miwic3R5bGUiOnsiaGVhZCI6eyJuYW1lIjoibm9uZSJ9fX1dLFsxMywxMl0sWzEyLDE0LCJpIl0sWzE0LDEwLCJjIiwyXSxbMTAsMTFdLFsyLDYsIlxcYmlub217MVxcLFxcLGlcXCxcXCwwfXswXFwsXFwsMFxcLFxcLDF9Il0sWzYsMTQsIlxcYmlub20gMWMiLDJdLFsyMywzMSwiXFx0ZXh0e+aLieWbnn0iLDEseyJzaG9ydGVuIjp7InNvdXJjZSI6MjAsInRhcmdldCI6MjB9LCJzdHlsZSI6eyJib2R5Ijp7Im5hbWUiOiJub25lIn0sImhlYWQiOnsibmFtZSI6Im5vbmUifX19XSxbMTcsMjIsIlxcdGV4dHvmjqjlh7p9IiwxLHsic2hvcnRlbiI6eyJzb3VyY2UiOjIwLCJ0YXJnZXQiOjIwfSwic3R5bGUiOnsiYm9keSI6eyJuYW1lIjoibm9uZSJ9LCJoZWFkIjp7Im5hbWUiOiJub25lIn19fV1d
                  \begin{tikzcd}[ampersand replacement=\&]
                      {E_1\oplus E_2} \& 0 \& {B\oplus B} \&\& {C\oplus B\oplus A} \&\& {A\oplus A} \& 0 \\
                      \& 0 \& B \&\& {C\oplus A} \&\& {A\oplus A} \& 0 \\
                      {E_1+E_2} \& 0 \& B \&\& C \&\& A \& 0
                      \arrow[from=1-2, to=1-3]
                      \arrow[""{name=0, anchor=center, inner sep=0}, "\begin{array}{c} \left(\substack{i\,\,0\\0\,\,1\\0\,\,0}\right) \end{array}", from=1-3, to=1-5]
                      \arrow["{(1\,\,1)}"', from=1-3, to=2-3]
                      \arrow["{\binom{c\,\,0\,\,0}{0\,\,0\,\,1}}", from=1-5, to=1-7]
                      \arrow["{\binom{1\,\,i\,\,0}{0\,\,0\,\,1}}", from=1-5, to=2-5]
                      \arrow[from=1-7, to=1-8]
                      \arrow[Rightarrow, no head, from=1-7, to=2-7]
                      \arrow[from=2-2, to=2-3]
                      \arrow[""{name=1, anchor=center, inner sep=0}, "{\binom i0}"', from=2-3, to=2-5]
                      \arrow[Rightarrow, no head, from=2-3, to=3-3]
                      \arrow[""{name=2, anchor=center, inner sep=0}, "{\binom{c\,\,0}{0\,\,1}}", from=2-5, to=2-7]
                      \arrow["{\binom 1c}"', from=2-5, to=3-5]
                      \arrow[from=2-7, to=2-8]
                      \arrow[from=3-2, to=3-3]
                      \arrow["i", from=3-3, to=3-5]
                      \arrow[""{name=3, anchor=center, inner sep=0}, "c"', from=3-5, to=3-7]
                      \arrow["{\binom{1}{1}}"', from=3-7, to=2-7]
                      \arrow[from=3-7, to=3-8]
                      \arrow["{\text{推出}}"{description}, draw=none, from=0, to=1]
                      \arrow["{\text{拉回}}"{description}, draw=none, from=2, to=3]
                  \end{tikzcd}.
              \end{equation}
        \item (加法的函子性) 只验证一侧, 另一侧同理.
              \begin{enumerate}
                  \item $(\alpha_1\oplus \alpha_2)(E_1\oplus E_2)=\alpha_1 E_1\oplus \alpha_2 E_2$. 这由矩阵乘法直接给出.
                  \item $(\alpha_1+\alpha_2)E=\nabla (\alpha_1\oplus \alpha_2)\Delta E=\nabla (\alpha_1\oplus \alpha_2)(E\oplus E)\Delta =\nabla (\alpha_1 E\oplus \alpha_2 E)\Delta =\alpha_1E+\alpha_2E$.
                  \item $\alpha (E_1+E_2)=\alpha \nabla (E_1\oplus E_2)\Delta = \nabla (\alpha\oplus \alpha)(E_1\oplus E_2)\Delta =  \nabla (\alpha E_1\oplus \alpha E_2)\Delta = \alpha E_1+\alpha E_2$.
              \end{enumerate}
              另需验证 $(\alpha E)\beta = \alpha(E\beta)$, 即, 推出与拉回交换. 实际上, 视 $(E\to E_\beta)\to \alpha (E\to E_\beta)$ 为态射范畴的推出即可. 图略.
        \item (加法结合律) 依照定义,
              \begin{equation}
                  (E_1+E_2)+E_3 = \nabla (E_1\oplus E_2)\Delta + E_3 = \nabla (\nabla (E_1\oplus E_2)\Delta \oplus E_3 ) \Delta = (\nabla (\nabla\oplus 1))(E_1\oplus E_2\oplus E_3) ((\Delta \oplus 1)\Delta).
              \end{equation}
              此处 $\nabla (\nabla\oplus 1)=\nabla (1\oplus \nabla)$, $\Delta$ 同理.
              \item (加法交换律) 依照 $E_1\oplus E_2\sim E_2\oplus E_1$ 即可. 记对换 $\tau=\binom{0\,\,1}{1\,\,0}$, 上式即 
              \begin{equation}
                \nabla (E_1\oplus E_2)\Delta = \nabla (\tau (E_1\oplus E_2))\Delta = \nabla ((E_2\oplus E_1)\tau )\Delta = \nabla (E_2\oplus E_1)\Delta. 
              \end{equation}
    \end{enumerate}
\end{proposition}

\begin{remark}
    容易验证, $0\to A\xrightarrow i B\xrightarrow c X\to 0$ 的加法逆元是 $0\to A\xrightarrow i B\xrightarrow {-c} X\to 0$. 改变任意一处符号即可. 
\end{remark}

\begin{definition}[长正合列的连接态射]
    给定 $0\to A\to B\to C\to 0$, 定义连接态射 $\delta:(X,C)\to \mathrm{Ext}^1(X,A),\quad f\mapsto Ef$ 为拉回
    \begin{equation}
        % https://q.uiver.app/#q=WzAsMTIsWzEsMSwiMCJdLFsyLDEsIkEiXSxbMywxLCJCIl0sWzQsMSwiQyJdLFs1LDEsIjAiXSxbNCwwLCJYIl0sWzMsMCwiXFxidWxsZXQiXSxbNSwwLCIwIl0sWzIsMCwiQSJdLFsxLDAsIjAiXSxbMCwwLCJFZiJdLFswLDEsIkUiXSxbMCwxXSxbMSwyXSxbMiwzXSxbMyw0XSxbNSwzLCJmIl0sWzksOF0sWzgsNiwiIiwxLHsic3R5bGUiOnsiYm9keSI6eyJuYW1lIjoiZGFzaGVkIn19fV0sWzYsNSwiIiwxLHsic3R5bGUiOnsiYm9keSI6eyJuYW1lIjoiZGFzaGVkIn19fV0sWzUsN10sWzgsMSwiIiwxLHsibGV2ZWwiOjIsInN0eWxlIjp7ImhlYWQiOnsibmFtZSI6Im5vbmUifX19XSxbNiwyLCIiLDEseyJzdHlsZSI6eyJib2R5Ijp7Im5hbWUiOiJkYXNoZWQifX19XSxbMTEsMTAsIiIsMCx7InN0eWxlIjp7ImJvZHkiOnsibmFtZSI6ImRhc2hlZCJ9fX1dLFsxOSwxNCwiXFx0ZXh0e+aLieWbnn0iLDEseyJzaG9ydGVuIjp7InNvdXJjZSI6MjAsInRhcmdldCI6MjB9LCJzdHlsZSI6eyJib2R5Ijp7Im5hbWUiOiJub25lIn0sImhlYWQiOnsibmFtZSI6Im5vbmUifX19XV0=
        \begin{tikzcd}[ampersand replacement=\&]
            Ef \& 0 \& A \& \bullet \& X \& 0 \\
            E \& 0 \& A \& B \& C \& 0
            \arrow[from=1-2, to=1-3]
            \arrow[dashed, from=1-3, to=1-4]
            \arrow[Rightarrow, no head, from=1-3, to=2-3]
            \arrow[""{name=0, anchor=center, inner sep=0}, dashed, from=1-4, to=1-5]
            \arrow[dashed, from=1-4, to=2-4]
            \arrow[from=1-5, to=1-6]
            \arrow["f", from=1-5, to=2-5]
            \arrow[dashed, from=2-1, to=1-1]
            \arrow[from=2-2, to=2-3]
            \arrow[from=2-3, to=2-4]
            \arrow[""{name=1, anchor=center, inner sep=0}, from=2-4, to=2-5]
            \arrow[from=2-5, to=2-6]
            \arrow["{\text{拉回}}"{description}, draw=none, from=0, to=1]
        \end{tikzcd}.
    \end{equation}
\end{definition}

\begin{proposition}[$\mathrm{Ext}$-群的长正合列]
    给定短正合列 $0\to A\xrightarrow{i} B \xrightarrow{c} C\to 0$, 则对任意 $X$ 均有长正合列
    \begin{equation}
        0\to (X,A)\xrightarrow{(X,i)} (X,B)\xrightarrow{(X,c)} (X,C)\xrightarrow \delta \mathrm{Ext}^1(X,A) \xrightarrow{\mathrm{Ext}^1(X,i)}\mathrm{Ext}^1(X,B) \xrightarrow{\mathrm{Ext}^1(X,c)}\mathrm{Ext}^1(X,C)\to\cdots
    \end{equation}
    \begin{proof}
        依次验证正合性即可.
        \begin{enumerate}
            \item ($(X,i)$ 是单射) 这是单射的泛性质.
            \item ($\mathrm{im}(X,i)=\ker(X,c)$) 这是核的泛性质.
            \item ($\mathrm{im}(X,c)=\mathrm{ker}(\delta)$) 任取 $(f\circ c)\in \mathrm{im}(X,c)$, 则有交换图
                  \begin{equation}
                      % https://q.uiver.app/#q=WzAsMTAsWzAsMSwiMCJdLFsxLDEsIkEiXSxbMiwxLCJCIl0sWzMsMSwiQyJdLFs0LDEsIjAiXSxbMywwLCJYIl0sWzIsMCwiWiJdLFs0LDAsIjAiXSxbMSwwLCJBIl0sWzAsMCwiMCJdLFswLDFdLFsxLDIsImkiLDJdLFsyLDMsImMiLDJdLFszLDRdLFs1LDMsImZjIl0sWzksOF0sWzgsNiwiIiwxLHsic3R5bGUiOnsiYm9keSI6eyJuYW1lIjoiZGFzaGVkIn19fV0sWzYsNSwiIiwxLHsic3R5bGUiOnsiYm9keSI6eyJuYW1lIjoiZGFzaGVkIn19fV0sWzUsN10sWzgsMSwiIiwxLHsibGV2ZWwiOjIsInN0eWxlIjp7ImhlYWQiOnsibmFtZSI6Im5vbmUifX19XSxbNiwyLCIiLDEseyJzdHlsZSI6eyJib2R5Ijp7Im5hbWUiOiJkYXNoZWQifX19XSxbNSwyLCJmIiwyLHsic3R5bGUiOnsiYm9keSI6eyJuYW1lIjoiZGFzaGVkIn19fV1d
                      \begin{tikzcd}[ampersand replacement=\&]
                          0 \& A \& Z \& X \& 0 \\
                          0 \& A \& B \& C \& 0
                          \arrow[from=1-1, to=1-2]
                          \arrow[dashed, from=1-2, to=1-3]
                          \arrow[Rightarrow, no head, from=1-2, to=2-2]
                          \arrow[dashed, from=1-3, to=1-4]
                          \arrow[dashed, from=1-3, to=2-3]
                          \arrow[from=1-4, to=1-5]
                          \arrow["f"', dashed, from=1-4, to=2-3]
                          \arrow["fc", from=1-4, to=2-4]
                          \arrow[from=2-1, to=2-2]
                          \arrow["i"', from=2-2, to=2-3]
                          \arrow["c"', from=2-3, to=2-4]
                          \arrow[from=2-4, to=2-5]
                      \end{tikzcd}.
                  \end{equation}
                  对 $X=X$ 与 $f:X\to B$ 使用拉回的泛性质, 此时得 $Z\twoheadrightarrow X$ 的右逆. 反之, 若正合列沿 $g:X\to C$ 的拉回是可裂短正合列, 则记 $X\to Z\to B$ 合成为 $f$. 此时 $cf=g$.
            \item ($\mathrm{im}(\delta)=\mathrm{ker}(\mathrm{Ext}^1(X,i))$) 任取正合列 $P\in \mathrm{ker}(\mathrm{Ext}^1(X,i))$, 考虑 $sl:Z\to B$ 在余核处的态射, 即得 $P=\delta(\widetilde{sl})$:
                  \begin{equation}
                      % https://q.uiver.app/#q=WzAsMTYsWzEsMiwiMCJdLFsyLDIsIkEiXSxbMywyLCJCIl0sWzQsMiwiQyJdLFs1LDIsIjAiXSxbNCwxLCJYIl0sWzMsMSwiWiJdLFs1LDEsIjAiXSxbMiwxLCJBIl0sWzEsMSwiMCJdLFsyLDAsIkIiXSxbMSwwLCIwIl0sWzMsMCwiQlxcb3BsdXMgWCJdLFs0LDAsIlgiXSxbNSwwLCIwIl0sWzAsMSwiUCJdLFswLDFdLFsxLDIsImkiLDJdLFsyLDMsImMiLDJdLFszLDRdLFs5LDhdLFs4LDZdLFs2LDVdLFs1LDddLFs4LDEsIiIsMSx7ImxldmVsIjoyLCJzdHlsZSI6eyJoZWFkIjp7Im5hbWUiOiJub25lIn19fV0sWzYsMiwic2wiLDAseyJzdHlsZSI6eyJib2R5Ijp7Im5hbWUiOiJkYXNoZWQifX19XSxbOCwxMCwiaSJdLFsxMSwxMF0sWzUsMTMsIiIsMix7ImxldmVsIjoyLCJzdHlsZSI6eyJoZWFkIjp7Im5hbWUiOiJub25lIn19fV0sWzEwLDEyXSxbMTIsMTNdLFsxMywxNF0sWzEyLDEwLCJzIiwwLHsiY3VydmUiOi0xfV0sWzYsMTIsImwiLDJdLFs1LDMsIlxcd2lkZXRpbGRlIHtzbH0iLDAseyJzdHlsZSI6eyJib2R5Ijp7Im5hbWUiOiJkYXNoZWQifX19XV0=
                      \begin{tikzcd}[ampersand replacement=\&]
                          \& 0 \& B \& {B\oplus X} \& X \& 0 \\
                          P \& 0 \& A \& Z \& X \& 0 \\
                          \& 0 \& A \& B \& C \& 0
                          \arrow[from=1-2, to=1-3]
                          \arrow[from=1-3, to=1-4]
                          \arrow["s", curve={height=-6pt}, from=1-4, to=1-3]
                          \arrow[from=1-4, to=1-5]
                          \arrow[from=1-5, to=1-6]
                          \arrow[from=2-2, to=2-3]
                          \arrow["i", from=2-3, to=1-3]
                          \arrow[from=2-3, to=2-4]
                          \arrow[Rightarrow, no head, from=2-3, to=3-3]
                          \arrow["l"', from=2-4, to=1-4]
                          \arrow[from=2-4, to=2-5]
                          \arrow["sl", dashed, from=2-4, to=3-4]
                          \arrow[Rightarrow, no head, from=2-5, to=1-5]
                          \arrow[from=2-5, to=2-6]
                          \arrow["{\widetilde {sl}}", dashed, from=2-5, to=3-5]
                          \arrow[from=3-2, to=3-3]
                          \arrow["i"', from=3-3, to=3-4]
                          \arrow["c"', from=3-4, to=3-5]
                          \arrow[from=3-5, to=3-6]
                      \end{tikzcd}.
                  \end{equation}
                  反之, 则可在左上构造 $Z\to B$, 因此推出是可裂单.
            \item ($\mathrm{im}(\mathrm{Ext}^1(X,i))=\mathrm{ker}(\mathrm{Ext}^1(X,c))$) 依照推出方块的复合律, $\subset$ 方向显然. 反之, 只需构造虚线处正合列
                  \begin{equation}
                      % https://q.uiver.app/#q=WzAsMTcsWzEsMSwiQSJdLFsyLDEsIkIiXSxbMywxLCJDIl0sWzQsMSwiMCJdLFsxLDMsIlgiXSxbMSwwLCIwIl0sWzEsMiwiWiJdLFsxLDQsIjAiXSxbMiwwLCIwIl0sWzIsMywiWCJdLFsyLDQsIjAiXSxbMiwyLCJXIl0sWzAsMSwiMCJdLFszLDIsIkNcXG9wbHVzIFgiXSxbMywzLCJYIl0sWzMsNCwiMCJdLFszLDAsIjAiXSxbMCwxLCJpIl0sWzEsMiwiYyJdLFsyLDNdLFs1LDAsIiIsMCx7InN0eWxlIjp7ImJvZHkiOnsibmFtZSI6ImRhc2hlZCJ9fX1dLFswLDYsIiIsMCx7InN0eWxlIjp7ImJvZHkiOnsibmFtZSI6ImRhc2hlZCJ9fX1dLFs2LDQsIiIsMCx7InN0eWxlIjp7ImJvZHkiOnsibmFtZSI6ImRhc2hlZCJ9fX1dLFs0LDcsIiIsMCx7InN0eWxlIjp7ImJvZHkiOnsibmFtZSI6ImRhc2hlZCJ9fX1dLFs4LDFdLFs0LDksIiIsMCx7ImxldmVsIjoyLCJzdHlsZSI6eyJib2R5Ijp7Im5hbWUiOiJkYXNoZWQifSwiaGVhZCI6eyJuYW1lIjoibm9uZSJ9fX1dLFsxMiwwXSxbNiwxMSwiIiwwLHsic3R5bGUiOnsiYm9keSI6eyJuYW1lIjoiZGFzaGVkIn19fV0sWzExLDldLFs5LDEwXSxbMTEsMTNdLFs5LDE0LCIiLDAseyJsZXZlbCI6Miwic3R5bGUiOnsiaGVhZCI6eyJuYW1lIjoibm9uZSJ9fX1dLFsxMywxNF0sWzE0LDE1XSxbMTYsMl0sWzEsMTFdLFsyLDEzXV0=
                      \begin{tikzcd}[ampersand replacement=\&]
                          \& 0 \& 0 \& 0 \\
                          0 \& A \& B \& C \& 0 \\
                          \& Z \& W \& {C\oplus X} \\
                          \& X \& X \& X \\
                          \& 0 \& 0 \& 0
                          \arrow[dashed, from=1-2, to=2-2]
                          \arrow[from=1-3, to=2-3]
                          \arrow[from=1-4, to=2-4]
                          \arrow[from=2-1, to=2-2]
                          \arrow["i", from=2-2, to=2-3]
                          \arrow[dashed, from=2-2, to=3-2]
                          \arrow["c", from=2-3, to=2-4]
                          \arrow[from=2-3, to=3-3]
                          \arrow[from=2-4, to=2-5]
                          \arrow[from=2-4, to=3-4]
                          \arrow[dashed, from=3-2, to=3-3]
                          \arrow[dashed, from=3-2, to=4-2]
                          \arrow[from=3-3, to=3-4]
                          \arrow[from=3-3, to=4-3]
                          \arrow[from=3-4, to=4-4]
                          \arrow[Rightarrow, dashed, no head, from=4-2, to=4-3]
                          \arrow[dashed, from=4-2, to=5-2]
                          \arrow[Rightarrow, no head, from=4-3, to=4-4]
                          \arrow[from=4-3, to=5-3]
                          \arrow[from=4-4, to=5-4]
                      \end{tikzcd}.
                  \end{equation}
                  依照 $3\times 3$ 引理, 下图三行与右两列是短正合列, 从而左列是短正合列:
                  \begin{equation}
                      % https://q.uiver.app/#q=WzAsMjEsWzEsMSwiQSJdLFsyLDEsIkIiXSxbMywxLCJDIl0sWzQsMSwiMCJdLFsxLDMsIlgiXSxbMSwwLCIwIl0sWzEsMiwiSyJdLFsxLDQsIjAiXSxbMiwwLCIwIl0sWzIsMywiWCJdLFsyLDQsIjAiXSxbMiwyLCJXIl0sWzAsMSwiMCJdLFszLDIsIkMiXSxbMywwLCIwIl0sWzMsMywiMCJdLFs0LDIsIjAiXSxbMCwzLCIwIl0sWzAsMiwiMCJdLFs0LDMsIjAiXSxbMyw0LCIwIl0sWzAsMSwiaSJdLFsxLDIsImMiXSxbMiwzXSxbNSwwLCIiLDAseyJzdHlsZSI6eyJib2R5Ijp7Im5hbWUiOiJkYXNoZWQifX19XSxbMCw2LCIiLDAseyJzdHlsZSI6eyJib2R5Ijp7Im5hbWUiOiJkYXNoZWQifX19XSxbNiw0LCIiLDAseyJzdHlsZSI6eyJib2R5Ijp7Im5hbWUiOiJkYXNoZWQifX19XSxbNCw3LCIiLDAseyJzdHlsZSI6eyJib2R5Ijp7Im5hbWUiOiJkYXNoZWQifX19XSxbOCwxXSxbNCw5LCIiLDAseyJsZXZlbCI6Miwic3R5bGUiOnsiaGVhZCI6eyJuYW1lIjoibm9uZSJ9fX1dLFsxMiwwXSxbNiwxMV0sWzExLDldLFs5LDEwXSxbMTEsMTNdLFsxNCwyXSxbMSwxMV0sWzIsMTMsIiIsMCx7ImxldmVsIjoyLCJzdHlsZSI6eyJoZWFkIjp7Im5hbWUiOiJub25lIn19fV0sWzksMTVdLFsxMywxNV0sWzEzLDE2XSxbMTcsNF0sWzE4LDZdLFsxNSwyMF0sWzE1LDE5XV0=
                      \begin{tikzcd}[ampersand replacement=\&]
                          \& 0 \& 0 \& 0 \\
                          0 \& A \& B \& C \& 0 \\
                          0 \& K \& W \& C \& 0 \\
                          0 \& X \& X \& 0 \& 0 \\
                          \& 0 \& 0 \& 0
                          \arrow[dashed, from=1-2, to=2-2]
                          \arrow[from=1-3, to=2-3]
                          \arrow[from=1-4, to=2-4]
                          \arrow[from=2-1, to=2-2]
                          \arrow["i", from=2-2, to=2-3]
                          \arrow[dashed, from=2-2, to=3-2]
                          \arrow["c", from=2-3, to=2-4]
                          \arrow[from=2-3, to=3-3]
                          \arrow[from=2-4, to=2-5]
                          \arrow[Rightarrow, no head, from=2-4, to=3-4]
                          \arrow[from=3-1, to=3-2]
                          \arrow[from=3-2, to=3-3]
                          \arrow[dashed, from=3-2, to=4-2]
                          \arrow[from=3-3, to=3-4]
                          \arrow[from=3-3, to=4-3]
                          \arrow[from=3-4, to=3-5]
                          \arrow[from=3-4, to=4-4]
                          \arrow[from=4-1, to=4-2]
                          \arrow[Rightarrow, no head, from=4-2, to=4-3]
                          \arrow[dashed, from=4-2, to=5-2]
                          \arrow[from=4-3, to=4-4]
                          \arrow[from=4-3, to=5-3]
                          \arrow[from=4-4, to=4-5]
                          \arrow[from=4-4, to=5-4]
                      \end{tikzcd}.
                  \end{equation}
        \end{enumerate}
    \end{proof}
\end{proposition}

\begin{example}[$3\times 3$ 表格, $\mathrm{Ext}^2$-群的又一刻画]
    以下 $3\times 3$ 蕴含三项 $\mathrm{Ext}^2(C_3,A_1)$ 中相同的元素: 
\begin{equation}
    % https://q.uiver.app/#q=WzAsMzMsWzEsMSwiQV8xIl0sWzIsMSwiQV8yIl0sWzMsMSwiQV8zIl0sWzEsMiwiQl8xIl0sWzIsMiwiQl8yIl0sWzMsMiwiQl8zIl0sWzEsMywiQ18xIl0sWzIsMywiQ18yIl0sWzMsMywiQ18zIl0sWzAsMSwiMCJdLFswLDIsIjAiXSxbMCwzLCIwIl0sWzEsNCwiMCJdLFsyLDQsIjAiXSxbMyw0LCIwIl0sWzQsMywiMCJdLFs0LDIsIjAiXSxbNCwxLCIwIl0sWzMsMCwiMCJdLFsyLDAsIjAiXSxbMSwwLCIwIl0sWzYsMSwiQV8xIl0sWzcsMSwiQV8yIl0sWzgsMSwiQl8zIl0sWzksMSwiQ18zIl0sWzksMiwiQ18zIl0sWzYsMiwiQV8xIl0sWzcsMiwiQl8yIl0sWzgsMiwiQ18yXFxvcGx1cyBCXzMiXSxbOCwzLCJDXzIiXSxbOSwzLCJDXzMiXSxbNiwzLCJBXzEiXSxbNywzLCJCXzEiXSxbMjAsMF0sWzAsM10sWzMsNl0sWzYsMTJdLFsxMSw2XSxbNiw3XSxbNyw4XSxbOCwxNV0sWzE4LDJdLFsyLDVdLFs1LDhdLFs4LDE0XSxbMTAsM10sWzMsNF0sWzQsNV0sWzUsMTZdLFs5LDBdLFsxLDJdLFsyLDE3XSxbMTksMV0sWzEsNF0sWzQsN10sWzcsMTNdLFswLDFdLFsxLDVdLFszLDddLFsyNSwyNCwiIiwxLHsibGV2ZWwiOjIsInN0eWxlIjp7ImhlYWQiOnsibmFtZSI6Im5vbmUifX19XSxbMjEsMjIsIiIsMSx7InN0eWxlIjp7InRhaWwiOnsibmFtZSI6Imhvb2siLCJzaWRlIjoiYm90dG9tIn19fV0sWzIyLDIzXSxbMjMsMjQsIiIsMSx7InN0eWxlIjp7ImhlYWQiOnsibmFtZSI6ImVwaSJ9fX1dLFsyNiwyMSwiIiwxLHsibGV2ZWwiOjIsInN0eWxlIjp7ImhlYWQiOnsibmFtZSI6Im5vbmUifX19XSxbMjYsMjcsIiIsMSx7InN0eWxlIjp7InRhaWwiOnsibmFtZSI6Imhvb2siLCJzaWRlIjoiYm90dG9tIn19fV0sWzI3LDI4LCJcXGJpbm9teyt9eyt9Il0sWzI4LDI1LCIoLSwrKSIsMCx7InN0eWxlIjp7ImhlYWQiOnsibmFtZSI6ImVwaSJ9fX1dLFsyMiwyNywiIiwxLHsic3R5bGUiOnsidGFpbCI6eyJuYW1lIjoiaG9vayIsInNpZGUiOiJib3R0b20ifX19XSxbMjMsMjgsIiIsMSx7InN0eWxlIjp7InRhaWwiOnsibmFtZSI6Imhvb2siLCJzaWRlIjoiYm90dG9tIn19fV0sWzMxLDMyLCIiLDEseyJzdHlsZSI6eyJ0YWlsIjp7Im5hbWUiOiJob29rIiwic2lkZSI6ImJvdHRvbSJ9fX1dLFszMiwyOV0sWzI5LDMwLCJcXGJveGVkIC0iLDIseyJzdHlsZSI6eyJoZWFkIjp7Im5hbWUiOiJlcGkifX19XSxbMzIsMjcsIiIsMSx7InN0eWxlIjp7InRhaWwiOnsibmFtZSI6Imhvb2siLCJzaWRlIjoiYm90dG9tIn19fV0sWzI5LDI4LCIiLDEseyJzdHlsZSI6eyJ0YWlsIjp7Im5hbWUiOiJob29rIiwic2lkZSI6ImJvdHRvbSJ9fX1dLFsyNiwzMSwiIiwxLHsibGV2ZWwiOjIsInN0eWxlIjp7ImhlYWQiOnsibmFtZSI6Im5vbmUifX19XSxbMjUsMzAsIiIsMSx7ImxldmVsIjoyLCJzdHlsZSI6eyJoZWFkIjp7Im5hbWUiOiJub25lIn19fV1d
\begin{tikzcd}[ampersand replacement=\&]
	\& 0 \& 0 \& 0 \\
	0 \& {A_1} \& {A_2} \& {A_3} \& 0 \&\& {A_1} \& {A_2} \& {B_3} \& {C_3} \\
	0 \& {B_1} \& {B_2} \& {B_3} \& 0 \&\& {A_1} \& {B_2} \& {C_2\oplus B_3} \& {C_3} \\
	0 \& {C_1} \& {C_2} \& {C_3} \& 0 \&\& {A_1} \& {B_1} \& {C_2} \& {C_3} \\
	\& 0 \& 0 \& 0
	\arrow[from=1-2, to=2-2]
	\arrow[from=1-3, to=2-3]
	\arrow[from=1-4, to=2-4]
	\arrow[from=2-1, to=2-2]
	\arrow[from=2-2, to=2-3]
	\arrow[from=2-2, to=3-2]
	\arrow[from=2-3, to=2-4]
	\arrow[from=2-3, to=3-3]
	\arrow[from=2-3, to=3-4]
	\arrow[from=2-4, to=2-5]
	\arrow[from=2-4, to=3-4]
	\arrow[hook', from=2-7, to=2-8]
	\arrow[from=2-8, to=2-9]
	\arrow[hook', from=2-8, to=3-8]
	\arrow[two heads, from=2-9, to=2-10]
	\arrow[hook', from=2-9, to=3-9]
	\arrow[from=3-1, to=3-2]
	\arrow[from=3-2, to=3-3]
	\arrow[from=3-2, to=4-2]
	\arrow[from=3-2, to=4-3]
	\arrow[from=3-3, to=3-4]
	\arrow[from=3-3, to=4-3]
	\arrow[from=3-4, to=3-5]
	\arrow[from=3-4, to=4-4]
	\arrow[Rightarrow, no head, from=3-7, to=2-7]
	\arrow[hook', from=3-7, to=3-8]
	\arrow[Rightarrow, no head, from=3-7, to=4-7]
	\arrow["{\binom{+}{+}}", from=3-8, to=3-9]
	\arrow["{(-,+)}", two heads, from=3-9, to=3-10]
	\arrow[Rightarrow, no head, from=3-10, to=2-10]
	\arrow[Rightarrow, no head, from=3-10, to=4-10]
	\arrow[from=4-1, to=4-2]
	\arrow[from=4-2, to=4-3]
	\arrow[from=4-2, to=5-2]
	\arrow[from=4-3, to=4-4]
	\arrow[from=4-3, to=5-3]
	\arrow[from=4-4, to=4-5]
	\arrow[from=4-4, to=5-4]
	\arrow[hook', from=4-7, to=4-8]
	\arrow[hook', from=4-8, to=3-8]
	\arrow[from=4-8, to=4-9]
	\arrow[hook', from=4-9, to=3-9]
	\arrow["{\boxed -}"', two heads, from=4-9, to=4-10]
\end{tikzcd}
\end{equation}
    因此 $A_1\to A_2\to B_3\to C_3$ 与 $A_1\to B_1\to C_2\to C_3$ 在 $\mathrm{Ext}^2$ 中是相反的元素. 特别地, $C_1=0$ 时以上四项正合列是 $\mathrm{Ext}^2$ 中的零元. 
\end{example}

\begin{definition}[$\mathrm{Ext}^n$ 群]
    归纳地进行 $\mathrm{Ext}^{m}(Y,Z)\times \mathrm{Ext}^n(X,Y)\to \mathrm{Ext}^{m+n}(X,Z)$. 定义
    \begin{align}
                      & [0\to Z\to P_1\to \cdots \to P_m\to Y\to 0] \quad \& \quad [0\to Y\to Q_1\to \cdots \to Q_n\to X\to 0] \\[6pt]
        \mapsto \quad & [0\to Z\to P_1\to \cdots \to P_m\to Q_1\to \cdots \to Q_n\to X\to 0].
    \end{align}
\end{definition}

\begin{remark}
    在可复合的意义下, 今后记 $\mathrm{Ext}^n$ 中长正合列为 $E^nE^{n-1}\cdots E^2E^1$, 此处每一 $E_i\in \mathrm{Ext}^1$.
\end{remark}

\begin{definition}[$\mathrm{Ext}^n$-群的运算]
    对 $\mathrm{Ext}^n$ 中长正合列 $E=E^nE^{n-1}\cdots E^2E^1$, 定义 $\alpha E\beta := (\alpha E^n)E^{n-1}\cdots E^2 (E^1\beta)$. 
\end{definition}

\begin{definition}[$\mathrm{Ext}^n$ 的等价关系]
    $\mathrm{Ext}^n(B,A)$ 中的``长度为 $1$ 的等价关系''描述作交换图
    \begin{equation}
        % https://q.uiver.app/#q=WzAsMTQsWzEsMCwiQSJdLFsyLDAsIlhfMSJdLFswLDAsIjAiXSxbMCwxLCIwIl0sWzEsMSwiQSJdLFsyLDEsIllfMSJdLFszLDAsIlxcY2RvdHMgIl0sWzQsMCwiWF9uIl0sWzQsMSwiWV9uIl0sWzUsMCwiQiJdLFszLDEsIlxcY2RvdHMgIl0sWzUsMSwiQiJdLFs2LDAsIjAiXSxbNiwxLCIwIl0sWzIsMF0sWzAsMV0sWzEsNl0sWzYsN10sWzcsOV0sWzksMTJdLFszLDRdLFs0LDVdLFs1LDEwXSxbMTAsOF0sWzgsMTFdLFsxMSwxM10sWzksMTEsIiIsMSx7ImxldmVsIjoyLCJzdHlsZSI6eyJoZWFkIjp7Im5hbWUiOiJub25lIn19fV0sWzcsOF0sWzEsNV0sWzAsNCwiIiwxLHsibGV2ZWwiOjIsInN0eWxlIjp7ImhlYWQiOnsibmFtZSI6Im5vbmUifX19XV0=
\begin{tikzcd}[ampersand replacement=\&]
	0 \& A \& {X_n} \& {\cdots } \& {X_1} \& B \& 0 \\
	0 \& A \& {Y_n} \& {\cdots } \& {Y_1} \& B \& 0
	\arrow[from=1-1, to=1-2]
	\arrow[from=1-2, to=1-3]
	\arrow[Rightarrow, no head, from=1-2, to=2-2]
	\arrow[from=1-3, to=1-4]
	\arrow[from=1-3, to=2-3]
	\arrow[from=1-4, to=1-5]
	\arrow[from=1-5, to=1-6]
	\arrow[from=1-5, to=2-5]
	\arrow[from=1-6, to=1-7]
	\arrow[Rightarrow, no head, from=1-6, to=2-6]
	\arrow[from=2-1, to=2-2]
	\arrow[from=2-2, to=2-3]
	\arrow[from=2-3, to=2-4]
	\arrow[from=2-4, to=2-5]
	\arrow[from=2-5, to=2-6]
	\arrow[from=2-6, to=2-7]
\end{tikzcd}.
    \end{equation}
    记以上第一条正合列为 $E_n\cdots E_1$, 其中 $E_k:[0\to \Omega_{k}\to X_k\to \Omega_{k-1}\to 0]$ 是短正合列, $\Omega_0=B$, $\Omega_n=A$. 记以上第二条正合列 $E_n'\cdots E_1'$, $\alpha_k:\Omega_k\to \Omega_k'$. 此时 $E_k'\alpha_{k-1}=\alpha_kE_k$: 
    \begin{equation}
        % https://q.uiver.app/#q=WzAsMTIsWzMsMCwiWF9rIl0sWzMsMSwiWV9rIl0sWzAsMCwiRV9rIl0sWzAsMSwiRV9rJyJdLFs0LDAsIlxcT21lZ2Ffe2stMX0iXSxbNCwxLCJcXE9tZWdhX3trLTF9JyJdLFsyLDAsIlxcT21lZ2Ffe2t9Il0sWzIsMSwiXFxPbWVnYV97a30nIl0sWzUsMCwiMCJdLFs1LDEsIjAiXSxbMSwwLCIwIl0sWzEsMSwiMCJdLFswLDFdLFs2LDBdLFswLDRdLFs0LDhdLFs3LDFdLFsxLDVdLFs1LDldLFsxMSw3XSxbMTAsNl0sWzYsNywiXFxhbHBoYV9rIl0sWzQsNSwiXFxhbHBoYV97ay0xfSJdXQ==
\begin{tikzcd}[ampersand replacement=\&]
	{E_k} \& 0 \& {\Omega_{k}} \& {X_k} \& {\Omega_{k-1}} \& 0 \\
	{E_k'} \& 0 \& {\Omega_{k}'} \& {Y_k} \& {\Omega_{k-1}'} \& 0
	\arrow[from=1-2, to=1-3]
	\arrow[from=1-3, to=1-4]
	\arrow["{\alpha_k}", from=1-3, to=2-3]
	\arrow[from=1-4, to=1-5]
	\arrow[from=1-4, to=2-4]
	\arrow[from=1-5, to=1-6]
	\arrow["{\alpha_{k-1}}", from=1-5, to=2-5]
	\arrow[from=2-2, to=2-3]
	\arrow[from=2-3, to=2-4]
	\arrow[from=2-4, to=2-5]
	\arrow[from=2-5, to=2-6]
\end{tikzcd}.
    \end{equation}
    以上关系生成 $\mathrm{Ext}^n$ 的等价关系 (锯齿图). 对 $\mathrm{Ext}^1$ 而言, 锯齿图可以拉直 (五引理). 
\end{definition}

\begin{remark}
    由定义, 若 $E\sim E'$, 则 $FEG\sim FE'G$, 从而米田积保持该等价关系.
\end{remark}

\begin{proposition}
    下图表明 $F(\varphi E')\sim (F\varphi )E'$: 
    \begin{equation}
        % https://q.uiver.app/#q=WzAsMTgsWzEsMCwiMCJdLFsyLDAsIlkiXSxbMywwLCJDIl0sWzQsMSwiQSJdLFsyLDEsIlkiXSxbMSwxLCIwIl0sWzMsMSwiQyciXSxbNCwyLCJBJyJdLFs1LDAsIkIiXSxbNSwxLCJCJyJdLFs2LDAsIlgnIl0sWzYsMSwiWCciXSxbNywwLCIwIl0sWzcsMSwiMCJdLFswLDAsIlxcbWF0aHJte0V4dH1eMShBLFkpXFxuaSBFIl0sWzAsMSwiXFxtYXRocm17RXh0fV4xKEEnLFkpXFxuaSBFJyJdLFs4LDAsIlxcbWF0aHJte0V4dH1eMShYJyxBKVxcbmkgRiJdLFs4LDEsIlxcbWF0aHJte0V4dH1eMShYJyxBJylcXG5pIEYnIl0sWzAsMV0sWzIsM10sWzUsNF0sWzQsNl0sWzYsN10sWzEsNF0sWzIsNl0sWzMsNywiXFx2YXJwaGkgIl0sWzcsOV0sWzMsOF0sWzgsMTBdLFs5LDExXSxbMTEsMTNdLFsxMCwxMl0sWzEwLDExLCIiLDEseyJsZXZlbCI6Miwic3R5bGUiOnsiaGVhZCI6eyJuYW1lIjoibm9uZSJ9fX1dLFsyLDgsIiIsMSx7ImN1cnZlIjoyLCJzdHlsZSI6eyJib2R5Ijp7Im5hbWUiOiJkYXNoZWQifX19XSxbNiw5LCIiLDEseyJjdXJ2ZSI6Miwic3R5bGUiOnsiYm9keSI6eyJuYW1lIjoiZGFzaGVkIn19fV0sWzgsOV0sWzEsMl1d
\begin{tikzcd}[ampersand replacement=\&]
	{\mathrm{Ext}^1(A,Y)\ni E} \& 0 \& Y \& C \&\& B \& {X'} \& 0 \& {\mathrm{Ext}^1(X',A)\ni F} \\
	{\mathrm{Ext}^1(A',Y)\ni E'} \& 0 \& Y \& {C'} \& A \& {B'} \& {X'} \& 0 \& {\mathrm{Ext}^1(X',A')\ni F'} \\
	\&\&\&\& {A'}
	\arrow[from=1-2, to=1-3]
	\arrow[from=1-3, to=1-4]
	\arrow[from=1-3, to=2-3]
	\arrow[curve={height=12pt}, dashed, from=1-4, to=1-6]
	\arrow[from=1-4, to=2-4]
	\arrow[from=1-4, to=2-5]
	\arrow[from=1-6, to=1-7]
	\arrow[from=1-6, to=2-6]
	\arrow[from=1-7, to=1-8]
	\arrow[Rightarrow, no head, from=1-7, to=2-7]
	\arrow[from=2-2, to=2-3]
	\arrow[from=2-3, to=2-4]
	\arrow[curve={height=12pt}, dashed, from=2-4, to=2-6]
	\arrow[from=2-4, to=3-5]
	\arrow[from=2-5, to=1-6]
	\arrow["{\varphi }", from=2-5, to=3-5]
	\arrow[from=2-6, to=2-7]
	\arrow[from=2-7, to=2-8]
	\arrow[from=3-5, to=2-6]
\end{tikzcd}.
    \end{equation}
    换言之, 态射乘法与正合列的米田积相容. 
\end{proposition}

\begin{proposition}[$\mathrm{Ext}^n$ 作为双函子]
    需要依次定义 $\mathrm{Ext}^1(B,A)$ 的加法结构, 并验证加法结构与来源, 去向的变换相容.
    \begin{enumerate}
        \item 给定 $E,E'\in \mathrm{Ext}^n(B,A)$, 定义 $E+E'=\nabla(E\oplus E')\Delta$. 
        \item 零对象即 $0\to A=A\to \cdots \to B=B\to 0$ 所在的等价类. 注: 这并不一定是可裂短正合列, 例如
        \begin{equation}
            0\to \mathbb R\to \mathbb R[x]/(x^2)\xrightarrow {\times x}\mathbb R[x]/(x^2)\to \mathbb R\to 0\quad (\mathrm{Mod}_{\mathbb R[x]})
        \end{equation}
        并非可裂短正合列. 
        \item (加法的函子性) 只验证一侧, 另一侧同理. (特别地, 长度为 $0$ 的正合列即态射.)
        \begin{enumerate}
            \item ($(F\oplus F')(E\oplus E')=FE\oplus F' E'$). 这由矩阵的乘法直接给出. 
            \item ($(E+E')F \sim (EF+E'F)$) 在等价意义下, $\Delta$ 与结合律相容. 依照 $\Delta F_k= (F_k\oplus F_k)\Delta$ 逐级计算即可. 
        \end{enumerate}
        \item (加法结合律) 在 $\sim$ 意义下, 将 $k$ 元和统一化作 $(1\,\cdots \,1)(E_1\oplus \cdots \oplus E_k)(1\,\cdots \,1)^T$ 即可. 
        \item (加法交换律) 在 $\sim$ 意义下, 依照 $\tau$ 与结合律相容, 逐级计算即可. 
    \end{enumerate}
\end{proposition}

\begin{remark}
    $E_n\cdots E_1$ 的加法逆元是 $E_n\cdots (-E_k)\cdots E_1$. 简单而言, 改变长正合列任意奇数次符号即可. 
\end{remark}

\begin{proposition}[$\mathrm{Ext}^2=0$ 的充要条件]
    给定正合列 $E:[0\to A\xrightarrow{i_X}X\xrightarrow{\pi_X} B\to 0]$, $F:[0\to B\xrightarrow{i_Y}Y\xrightarrow{\pi_Y}C\to 0]$, 以下条件等价. 
    \begin{enumerate}
        \item $EF$ 是 $\mathrm{Ext}^2(C,A)$ 中的零对象 ($EF\sim 0$); 
        \item 存在 $E\sim E'\alpha$ 使得 $\alpha F\sim 0$; 
        \item 存在 $E\sim E''\iota_Y$;
        \item 存在 $F\sim \beta F'$ 使得 $E\beta\sim0$; 
        \item 存在 $F\sim \pi_X F''$;
    \end{enumerate}
    \begin{proof}
        证明顺序: $\left[% https://q.uiver.app/#q=WzAsOCxbMSwwLCIyIl0sWzEsMSwiMyJdLFswLDAsIjQiXSxbMCwxLCI1Il0sWzMsMCwiMSJdLFsyLDAsIjJcXHdlZGdlIDMiXSxbMiwxLCIyXFx2ZWUgMyJdLFszLDEsIjEiXSxbMiwzLCIiLDEseyJzdHlsZSI6eyJ0YWlsIjp7Im5hbWUiOiJhcnJvd2hlYWQifX19XSxbMCwyLCIiLDEseyJzdHlsZSI6eyJ0YWlsIjp7Im5hbWUiOiJhcnJvd2hlYWQifX19XSxbMSwzLCIiLDEseyJzdHlsZSI6eyJ0YWlsIjp7Im5hbWUiOiJhcnJvd2hlYWQifX19XSxbNiw3XSxbNCw1XSxbNSw2LCIiLDEseyJzdHlsZSI6eyJ0YWlsIjp7Im5hbWUiOiJhcnJvd2hlYWQifSwiYm9keSI6eyJuYW1lIjoiZGFzaGVkIn19fV0sWzAsMSwiIiwxLHsic3R5bGUiOnsidGFpbCI6eyJuYW1lIjoiYXJyb3doZWFkIn0sImJvZHkiOnsibmFtZSI6ImRhc2hlZCJ9fX1dLFsxNCwxMywiIiwxLHsic2hvcnRlbiI6eyJzb3VyY2UiOjIwLCJ0YXJnZXQiOjIwfX1dXQ==
        \begin{tikzcd}[ampersand replacement=\&]
            4 \& 2 \& {2\wedge 3} \& 1 \\
            5 \& 3 \& {2\vee 3} \& 1
            \arrow[tail reversed, from=1-1, to=2-1]
            \arrow[tail reversed, from=1-2, to=1-1]
            \arrow[""{name=0, anchor=center, inner sep=0}, dashed, tail reversed, from=1-2, to=2-2]
            \arrow[""{name=1, anchor=center, inner sep=0}, dashed, tail reversed, from=1-3, to=2-3]
            \arrow[from=1-4, to=1-3]
            \arrow[tail reversed, from=2-2, to=2-1]
            \arrow[from=2-3, to=2-4]
            \arrow[shorten <=6pt, shorten >=6pt, Rightarrow, from=0, to=1]
        \end{tikzcd}\right]$. 此处 $\wedge$ 是逻辑``与'', $\vee$ 是逻辑``或''. 
        \begin{enumerate}
            \item ($2\to 1$. 同理地, $3\to 1$)
            \begin{equation}
                % https://q.uiver.app/#q=WzAsMTgsWzEsMCwiQSJdLFszLDAsIkIiXSxbNSwwLCJDIl0sWzIsMCwiWCJdLFs0LDAsIlkiXSxbMywxLCJCJyJdLFsxLDEsIkEiXSxbMiwxLCJYJyJdLFs0LDEsIkInXFxvcGx1cyBDIl0sWzUsMSwiQyJdLFswLDAsIkU9RSdcXGFscGhhICJdLFswLDEsIkUnIl0sWzYsMCwiRiJdLFs2LDEsIkYnPVxcYWxwaGEgRiJdLFsxLDIsIkEiXSxbMiwyLCJBIl0sWzQsMiwiQyJdLFs1LDIsIkMiXSxbMCwzLCIiLDEseyJzdHlsZSI6eyJ0YWlsIjp7Im5hbWUiOiJob29rIiwic2lkZSI6ImJvdHRvbSJ9fX1dLFszLDEsIiIsMSx7InN0eWxlIjp7ImhlYWQiOnsibmFtZSI6ImVwaSJ9fX1dLFsxLDQsIiIsMSx7InN0eWxlIjp7InRhaWwiOnsibmFtZSI6Imhvb2siLCJzaWRlIjoiYm90dG9tIn19fV0sWzQsMiwiIiwxLHsic3R5bGUiOnsiaGVhZCI6eyJuYW1lIjoiZXBpIn19fV0sWzEsNSwiXFxhbHBoYSIsMl0sWzYsNywiIiwwLHsic3R5bGUiOnsidGFpbCI6eyJuYW1lIjoiaG9vayIsInNpZGUiOiJib3R0b20ifX19XSxbNyw1LCIiLDAseyJzdHlsZSI6eyJoZWFkIjp7Im5hbWUiOiJlcGkifX19XSxbMyw3XSxbNiwwLCIiLDEseyJsZXZlbCI6Miwic3R5bGUiOnsiaGVhZCI6eyJuYW1lIjoibm9uZSJ9fX1dLFs1LDgsIiIsMSx7InN0eWxlIjp7InRhaWwiOnsibmFtZSI6Imhvb2siLCJzaWRlIjoiYm90dG9tIn19fV0sWzQsOF0sWzgsOSwiIiwxLHsic3R5bGUiOnsiaGVhZCI6eyJuYW1lIjoiZXBpIn19fV0sWzIsOSwiIiwxLHsibGV2ZWwiOjIsInN0eWxlIjp7ImhlYWQiOnsibmFtZSI6Im5vbmUifX19XSxbMTYsMTcsIiIsMSx7ImxldmVsIjoyLCJzdHlsZSI6eyJoZWFkIjp7Im5hbWUiOiJub25lIn19fV0sWzE0LDE1LCIiLDEseyJsZXZlbCI6Miwic3R5bGUiOnsiaGVhZCI6eyJuYW1lIjoibm9uZSJ9fX1dLFsxNSwxNiwiMCJdLFsxNyw5LCIiLDEseyJsZXZlbCI6Miwic3R5bGUiOnsiaGVhZCI6eyJuYW1lIjoibm9uZSJ9fX1dLFsxNiw4XSxbMTUsN10sWzE0LDYsIiIsMSx7ImxldmVsIjoyLCJzdHlsZSI6eyJoZWFkIjp7Im5hbWUiOiJub25lIn19fV1d
    \begin{tikzcd}[ampersand replacement=\&]
        {E=E'\alpha } \& A \& X \& B \& Y \& C \& F \\
        {E'} \& A \& {X'} \& {B'} \& {B'\oplus C} \& C \& {F'=\alpha F} \\
        \& A \& A \&\& C \& C
        \arrow[hook', from=1-2, to=1-3]
        \arrow[two heads, from=1-3, to=1-4]
        \arrow[from=1-3, to=2-3]
        \arrow[hook', from=1-4, to=1-5]
        \arrow["\alpha"', from=1-4, to=2-4]
        \arrow[two heads, from=1-5, to=1-6]
        \arrow[from=1-5, to=2-5]
        \arrow[Rightarrow, no head, from=1-6, to=2-6]
        \arrow[Rightarrow, no head, from=2-2, to=1-2]
        \arrow[hook', from=2-2, to=2-3]
        \arrow[two heads, from=2-3, to=2-4]
        \arrow[hook', from=2-4, to=2-5]
        \arrow[two heads, from=2-5, to=2-6]
        \arrow[Rightarrow, no head, from=3-2, to=2-2]
        \arrow[Rightarrow, no head, from=3-2, to=3-3]
        \arrow[from=3-3, to=2-3]
        \arrow["0", from=3-3, to=3-5]
        \arrow[from=3-5, to=2-5]
        \arrow[Rightarrow, no head, from=3-5, to=3-6]
        \arrow[Rightarrow, no head, from=3-6, to=2-6]
    \end{tikzcd}.
            \end{equation}
            \item ($2\leftrightarrow 3$. 同理地, $4\leftrightarrow 5$) 先看 $2\to 3$. 若 $E=E'\alpha$ 使得 $\alpha F=0$, 则存在分解 $\alpha = \varphi i_Y$. 取 $E'':=(E'\varphi)$. 再看 $3\to 2$. 只需证明 $\iota_Y F=0$. 依照``拉回是可裂单''等价于``被拉回的态射存在分解'', 因此 $\iota_YF$ 是可裂短正合列.
            \item ($3\leftrightarrow 5$) 这都等价于 ``$\star$ 是推出与拉回方块'': 
\begin{equation}
    % https://q.uiver.app/#q=WzAsMTIsWzEsMSwiQSJdLFsyLDEsIlgiXSxbMywxLCJCIl0sWzMsMiwiWSJdLFszLDMsIkMiXSxbMywwLCJGIl0sWzAsMSwiRSJdLFsyLDIsIlciXSxbMCwyLCJFJyciXSxbMiwwLCJGJyciXSxbMSwyLCJBIl0sWzIsMywiQyJdLFswLDEsImlfWCIsMCx7InN0eWxlIjp7InRhaWwiOnsibmFtZSI6Imhvb2siLCJzaWRlIjoiYm90dG9tIn19fV0sWzEsMiwiXFxwaV9YIiwwLHsic3R5bGUiOnsiaGVhZCI6eyJuYW1lIjoiZXBpIn19fV0sWzIsMywiaV9ZIiwwLHsic3R5bGUiOnsidGFpbCI6eyJuYW1lIjoiaG9vayIsInNpZGUiOiJib3R0b20ifX19XSxbMyw0LCJcXHBpX1kiLDAseyJzdHlsZSI6eyJoZWFkIjp7Im5hbWUiOiJlcGkifX19XSxbMTEsNCwiIiwxLHsibGV2ZWwiOjIsInN0eWxlIjp7ImhlYWQiOnsibmFtZSI6Im5vbmUifX19XSxbMCwxMCwiIiwxLHsibGV2ZWwiOjIsInN0eWxlIjp7ImhlYWQiOnsibmFtZSI6Im5vbmUifX19XSxbMTAsNywiIiwxLHsic3R5bGUiOnsidGFpbCI6eyJuYW1lIjoiaG9vayIsInNpZGUiOiJib3R0b20ifX19XSxbNywzLCIiLDEseyJzdHlsZSI6eyJoZWFkIjp7Im5hbWUiOiJlcGkifX19XSxbMSw3LCIiLDEseyJzdHlsZSI6eyJ0YWlsIjp7Im5hbWUiOiJob29rIiwic2lkZSI6ImJvdHRvbSJ9fX1dLFs3LDExLCIiLDEseyJzdHlsZSI6eyJoZWFkIjp7Im5hbWUiOiJlcGkifX19XSxbMjAsMTQsIlxcc3RhciAiLDEseyJzaG9ydGVuIjp7InNvdXJjZSI6MjAsInRhcmdldCI6MjB9LCJzdHlsZSI6eyJib2R5Ijp7Im5hbWUiOiJub25lIn0sImhlYWQiOnsibmFtZSI6Im5vbmUifX19XV0=
\begin{tikzcd}[ampersand replacement=\&]
	\&\& {F''} \& F \\
	E \& A \& X \& B \\
	{E''} \& A \& W \& Y \\
	\&\& C \& C
	\arrow["{i_X}", hook', from=2-2, to=2-3]
	\arrow[Rightarrow, no head, from=2-2, to=3-2]
	\arrow["{\pi_X}", two heads, from=2-3, to=2-4]
	\arrow[""{name=0, anchor=center, inner sep=0}, hook', from=2-3, to=3-3]
	\arrow[""{name=1, anchor=center, inner sep=0}, "{i_Y}", hook', from=2-4, to=3-4]
	\arrow[hook', from=3-2, to=3-3]
	\arrow[two heads, from=3-3, to=3-4]
	\arrow[two heads, from=3-3, to=4-3]
	\arrow["{\pi_Y}", two heads, from=3-4, to=4-4]
	\arrow[Rightarrow, no head, from=4-3, to=4-4]
	\arrow["{\star }"{description}, draw=none, from=0, to=1]
\end{tikzcd}
\end{equation}
        \item ($1\to (2\vee 4)=(2\wedge 4)$) 若 $EF\sim 0$, 则存在最小的 $k$ 使得
        \begin{align}
            EF &\overset \square= (E_1\alpha_1)F \overset \S \sim  E_1 (\alpha_1 F) \overset\dagger = E_1 (\beta_1 F_1) \overset \star \sim  (E_1\beta_1) F_1 = (E_2\alpha_2) F_1 \sim \\[6pt]
            \cdots &\sim E_{k}(\alpha_{k}F_{k-1}) \sim(E_{k}\beta_{k}) F_{k} \sim E_{k+1} (\alpha_{k+1} F_{k}) = E_{k+1} 0. 
        \end{align}
        依照数学归纳法, $1\to (2\vee 4)=(2\wedge 4)$ 对 $k=0$ 成立. 归纳步骤即命题: 若 $\square\S\dagger\star$ 中的等号后式满足 $(2\wedge 4)$, 则前式亦然. 
        \begin{itemize}
            \item ($\star$) 若 $(E_1\beta_1)F_1$ 满足``存在 $F_1\sim \gamma F'$ 使得 $(E_1\beta_1)\gamma\sim 0$'', 则 $E_1(\beta_1 F_1)$ 满足`` $\beta_1 F_1\sim \gamma \beta_1 F_1$, 使得 $E_1\gamma \beta_1 \sim 0$''. 
            \item ($\dagger$) 此处 $\alpha_1 F= \beta_1 F_1$ 是相同的对象, 故满足相同的命题. 
            \item ($\S$) 若 $E_1(\alpha_1 F)$ 满足``存在 $E_1\sim E'\delta$ 使得 $\delta\alpha_1 F\sim 0$'', 则 $(E_1\alpha)F$ 满足``$E_1\alpha_1\sim E_1\alpha_1\delta$, 使得 $\alpha_1\delta F\sim 0$''. 
            \item ($\square$) 此处 $E\alpha_1 = E$ 是相同的对象, 故满足相同的命题. 
        \end{itemize}
        \end{enumerate}
    \end{proof}
\end{proposition}

\begin{remark}
    对 $\mathrm{Ext}^n$ 的论证是同样的: 记 $E\in \mathrm{Ext}^p(B,A)$ 与 $F\in \mathrm{Ext}^q(C,B)$, 则 $EF\sim 0$ 的充要条件为: 
    \begin{enumerate}
        \item 存在 $E\sim E'\alpha$ 使得 $\alpha F\sim 0$; 或对偶地, 存在 $F\sim \beta F'$ 使得 $E\beta\sim 0$; 
        \item $E\sim E''\iota_Y$ ($\iota_Y$ 是正合列 $F$ 的首个单态射); 或对偶地, 存在 $F\sim \pi_X F''$ 使得 $E\pi_X\sim 0$ ($\pi_X$ 是正合列 $E$ 的末个满态射). 
    \end{enumerate} 
\end{remark}

\begin{proposition}[$\mathrm{Ext}^n$ 的长正合列]
    给定短正合列 $S:=0\to A\xrightarrow i B\xrightarrow c C\to 0$, 则有长正合列
    \begin{equation}
        \cdots \xrightarrow{\delta^{n-1}} \mathrm{Ext}^n(X,A)\xrightarrow{\mathrm{Ext}^n(X,i)}\mathrm{Ext}^n(X,B)\xrightarrow{\mathrm{Ext}^n(X,c)}\mathrm{Ext}^n(X,C)\xrightarrow{\delta^n} \mathrm{Ext}^{n+1}(X,A)\to \cdots .
    \end{equation}
    \begin{proof}
        依次证明 $\mathrm{Ext}^n(X,A)$, $\mathrm{Ext}^n(X,B)$ 与 $\mathrm{Ext}^n(X,C)$ 处正合性. 以上一切 $\delta^k$ 无非左复合 $S$. 
        \begin{enumerate}
            \item ($\mathrm{Ext}^n(X,A)$ 处正合) 任取 $SE\in \mathrm{im}(\delta^{n-1})$, 则自然有 $(iS)E=0E\sim 0$. 反之, 若 $iE\sim 0$, 则写 $E$ 作 $E_{n}E_{(n-1)\to 1}$. 此时存在 $E_{(n-1)\to 1}\sim \alpha F$ 使得 $iE_n\alpha\sim 0$. 依照前文对 $\delta^0$ 正合性的说明, 存在 $E_n\alpha\sim S\beta$. 因此 $E\sim S(\beta F)\in \mathrm{im}(\delta^{n-1})$. 
            \item ($\mathrm{Ext}^n(X,B)$ 处正合) 将第一问中的所有 $(i,S)$ 换作 $(c,i)$ 即可. 
            \item ($\mathrm{Ext}^n(X,C)$ 处正合) 将第一问中的所有 $(i,S)$ 换作 $(S,c)$ 即可. 
        \end{enumerate}
    \end{proof}
\end{proposition}

\begin{proposition}[拉直零对象的锯齿]
    若 $E\sim 0$, 则存在 $E\xrightarrow\sim F\xleftarrow\sim \mathcal O$ (等价地, $E\xleftarrow\sim F\xrightarrow\sim \mathcal O$). 此处 $\xrightarrow\sim$ 与 $\xleftarrow\sim $ 指``长度为 $1$ 的等价'',
    \begin{equation}
        \mathcal O:\quad 0\to X=X\to 0\to\cdots \to 0\quad \oplus \quad 0\to \cdots\to 0\to Y=Y\to 0. 
    \end{equation}
    \begin{proof}
        只看 $E\xrightarrow\sim F\xleftarrow\sim \mathcal O$. 存在 $F\xrightarrow\sim \mathcal O$ 的充要条件是, $F$ 最右端的满态射可裂. 此时取 $\alpha:E\to F$ 如下: 
        \begin{itemize}
            \item 记 $E=E_n\cdots E_1$, 则存在 $\alpha_n$ 使得 $E_n=F_n\alpha_n$, 且 $\alpha_n E_{(n-1)\to 1}\sim 0$; 
            \item 对 $\alpha_k E_{k-1}\cdots E_1\sim 0$, 存在 $\alpha_k E_{k-1}=F_{k-1}\alpha_{k-1}$ 使得 $\alpha_{k-1} E_{k-2}\cdots E_1\sim 0$; 
            \item 最后 $\alpha_2 E_1=:F_1$ 是可裂短正合列. 
        \end{itemize}
    \end{proof}
\end{proposition}

\begin{remark}
    下给出一例``长度不为 $1$ 的等价''. 在 $\mathbb R[x]$-模范畴中取长为 $2$ 的短正合列: 
    \begin{equation}
        0\to \mathbb R\to \mathbb R[x]/(x^2)\xrightarrow {\times x}\mathbb R[x]/(x^2)\to \mathbb R\to 0\quad (\mathrm{Mod}_{\mathbb R[x]}).
    \end{equation}
    显然以上正合列不可裂. 依照 $\mathrm{Ext}^2=0$ 的充要条件, 以上 $\times x$ 处态射的满-单分解是推出与拉回方块 
    \begin{equation}
        % https://q.uiver.app/#q=WzAsNCxbMCwwLCJcXG1hdGhiYiBSW3hdLyh4XjIpIl0sWzEsMSwiXFxtYXRoYmIgUlt4XS8oeF4yKSJdLFsxLDAsIlxcbWF0aGJiIFIiXSxbMCwxLCJcXG1hdGhiYiBSW3hdLyh4XjMpIl0sWzAsMiwiIiwxLHsic3R5bGUiOnsiaGVhZCI6eyJuYW1lIjoiZXBpIn19fV0sWzIsMSwiIiwxLHsic3R5bGUiOnsidGFpbCI6eyJuYW1lIjoiaG9vayIsInNpZGUiOiJib3R0b20ifX19XSxbMCwzLCIiLDEseyJzdHlsZSI6eyJ0YWlsIjp7Im5hbWUiOiJob29rIiwic2lkZSI6ImJvdHRvbSJ9fX1dLFszLDEsIiIsMSx7InN0eWxlIjp7ImhlYWQiOnsibmFtZSI6ImVwaSJ9fX1dLFswLDEsIlxcdGltZXMgeCIsMSx7InN0eWxlIjp7ImJvZHkiOnsibmFtZSI6ImRhc2hlZCJ9fX1dXQ==
\begin{tikzcd}[ampersand replacement=\&]
	{\mathbb R[x]/(x^2)} \& {\mathbb R} \\
	{\mathbb R[x]/(x^3)} \& {\mathbb R[x]/(x^2)}
	\arrow[two heads, from=1-1, to=1-2]
	\arrow[hook', from=1-1, to=2-1]
	\arrow["{\times x}"{description}, dashed, from=1-1, to=2-2]
	\arrow[hook', from=1-2, to=2-2]
	\arrow[two heads, from=2-1, to=2-2]
\end{tikzcd}.
    \end{equation}
    实际上, 由 $\mathbb R[x]$ 的整体维数是 $1$ 知 $\mathrm{Ext}^2=0$. 
\end{remark}

\begin{proposition}
    若 $E\sim F$, 则存在 $E\xrightarrow{\sim} \bullet\xleftarrow{\sim} \bullet\xrightarrow{\sim} F$ (对称表述略). 
\begin{proof}
    由于 $E+(-F)\sim 0$, 故存在 $0\xrightarrow\sim K\xleftarrow\sim E\oplus (-F)$. 在右侧直和 $\oplus F$, 依照 $(E\oplus F)\oplus G=E\oplus (F\oplus G)$, 取 $(-F)\oplus F\xrightarrow\sim 0$. 
    \begin{equation}
        F\xrightarrow\sim K\oplus F\xleftarrow\sim E\oplus (-F)\oplus F\xrightarrow \sim E
    \end{equation}
    即为所求. 
\end{proof}
\end{proposition}

\begin{remark}
    假若范畴有足够投射对象 (内射对象), 维数移位给出更简短的刻画: $E\xleftarrow\sim \bullet \xrightarrow \sim F$ ($E\xrightarrow\sim \bullet \xleftarrow \sim F$). 
\end{remark}

    

\subsection{譜序列簡介}
\begin{abstract}
    關於 Leray 早期工作, 層與譜序列嚆矢等參見 \cite{MR1775587}. 較 suite spectrale 之稱呼, anneau spectral 之稱更能體現其環結構. 

\end{abstract}

\subsubsection{譜序列的定義, 構造 I, 以及收斂性: 濾過復形}

\begin{abstract}
    先建立一條邏輯閉環: 用 $(A,d,F)$ 構造 $E$, 再證明其 $E_∞$ 收斂至 $(A,d,F)$. 往後加入``收斂至 $(A,d,F)$ 的譜序列''這一記號.  

    這是一條清晰的路, 既不涉及正合耦, 又不涉及雙複形. 
\end{abstract}

\begin{definition}[(上) 同調譜序列]
    先明確定向: $(1,0) = →$ 是往右, $(0,1) = ↑$ 是往上. 稱一組資料 $\{(E_r, d_r)\}$ 是同調譜序列, 若以下滿足. 
    \begin{enumerate}
        \item 凡 $E_r$ 均是 $(ℤ × ℤ)$-指標的對象, 凡 $d_r$ 均是 $(r,1-r)$ 朝向的態射, 滿足 $d_r ^2 =0$. 
        \item 凡 $E_{r+1}$ 均是 $E_r$ 在各分量取同調群所得, 即 $E_{r+1} = H(E_r)$. 
    \end{enumerate}
\end{definition}

\begin{remark}
    箭頭朝向可以整體轉置. 稱上述同調的譜序列, 是以微分逐級傾斜. 假定 $E_0$ 的支撐集在滿足某種有限性, 則存在穩定項 $E_r = E_∞$. 
\end{remark}

\begin{example}[譜序列的構造 I: 濾過複形 (微分分次模)]\label{filteredcomplex}
    給定微分分次模 $(A^∙ , d)$ (也就是上鏈複形). 稱 $F^∙(-)$ 是 $A^∙$ 的濾過, 當且僅當存在形如以下的 $(ℤ × ℤ)$-規格的交換圖\parnote{$↘$ 向减小 \\ $↑$ 向微分}
\begin{equation}
    % https://q.uiver.app/#q=WzAsOCxbMSwxLCJGXnBBXntwK3F9Il0sWzIsMiwiRl57cCsxfUFee3ArcX0iXSxbMCwwLCJGXntwLTF9QV57cCtxfSJdLFsxLDAsIkZee3B9QV57cCtxKzF9Il0sWzIsMSwiRl57cCsxfUFee3ArcSsxfSJdLFsxLDIsIkZee3B9QV57cCtxLTF9Il0sWzAsMSwiRl57cC0xfUFee3ArcS0xfSJdLFswLDIsIlxcYm94ZWR7XFx0ZXh0e+i8uOWFpX0oQSxGKX0iXSxbMCwyLCJcXHN1YnNldCIsMyx7InN0eWxlIjp7ImJvZHkiOnsibmFtZSI6ImRvdHRlZCJ9LCJoZWFkIjp7Im5hbWUiOiJub25lIn19fV0sWzEsMCwiXFxzdWJzZXQiLDMseyJzdHlsZSI6eyJib2R5Ijp7Im5hbWUiOiJkb3R0ZWQifSwiaGVhZCI6eyJuYW1lIjoibm9uZSJ9fX1dLFs0LDMsIlxcc3Vic2V0IiwzLHsic3R5bGUiOnsiYm9keSI6eyJuYW1lIjoiZG90dGVkIn0sImhlYWQiOnsibmFtZSI6Im5vbmUifX19XSxbNSw2LCJcXHN1YnNldCIsMyx7InN0eWxlIjp7ImJvZHkiOnsibmFtZSI6ImRvdHRlZCJ9LCJoZWFkIjp7Im5hbWUiOiJub25lIn19fV0sWzYsMl0sWzAsM10sWzUsMF0sWzEsNF1d
\begin{tikzcd}[ampersand replacement=\&]
	{F^{p-1}A^{p+q}} \& {F^{p}A^{p+q+1}} \\
	{F^{p-1}A^{p+q-1}} \& {F^pA^{p+q}} \& {F^{p+1}A^{p+q+1}} \\
	{\boxed{\text{輸入}(A,F)}} \& {F^{p}A^{p+q-1}} \& {F^{p+1}A^{p+q}}
	\arrow[from=2-1, to=1-1]
	\arrow["\subset"{marking, allow upside down}, dotted, no head, from=2-2, to=1-1]
	\arrow[from=2-2, to=1-2]
	\arrow["\subset"{marking, allow upside down}, dotted, no head, from=2-3, to=1-2]
	\arrow["\subset"{marking, allow upside down}, dotted, no head, from=3-2, to=2-1]
	\arrow[from=3-2, to=2-2]
	\arrow["\subset"{marking, allow upside down}, dotted, no head, from=3-3, to=2-2]
	\arrow[from=3-3, to=2-3]
\end{tikzcd}.
\end{equation}
此時, 定義 $E_0^{p,q} = \frac{F^p A^{p+q}}{F^{p+1}A^{p+q}}$, 相應的微分繼承自 $d$. \parnote{$\frac{s(↘)}{t(↘)}$} 
\begin{equation}
    % https://q.uiver.app/#q=WzAsOCxbMSwxLCJcXGZyYWN7Rl5wQV57cCtxfX17Rl57cCsxfUFee3ArcX19Il0sWzIsMiwiXFxmcmFje0Zee3ArMX1BXntwK3F9fXtGXntwKzJ9QV57cCtxfX0iXSxbMCwwLCJcXGZyYWN7Rl57cC0xfUFee3ArcX19e0ZecEFee3ArcX19Il0sWzEsMCwiXFxmcmFje0Zee3B9QV57cCtxKzF9fXtGXntwKzF9QV57cCtxKzF9fSJdLFsyLDEsIlxcZnJhY3tGXntwKzF9QV57cCtxKzF9fXtGXntwKzJ9QV57cCtxKzF9fSJdLFsxLDIsIlxcZnJhY3tGXntwfUFee3ArcS0xfX17Rl57cCsxfUFee3ArcS0xfX0iXSxbMCwxLCJcXGZyYWN7Rl57cC0xfUFee3ArcS0xfX17Rl57cH1BXntwK3EtMX19Il0sWzAsMiwiXFxib3hlZHtFXzAoQSxGKX0iXSxbMSw0XSxbNSwwXSxbMCwzXSxbNiwyXV0=
\begin{tikzcd}[ampersand replacement=\&]
	{\frac{F^{p-1}A^{p+q}}{F^pA^{p+q}}} \& {\frac{F^{p}A^{p+q+1}}{F^{p+1}A^{p+q+1}}} \\
	{\frac{F^{p-1}A^{p+q-1}}{F^{p}A^{p+q-1}}} \& {\frac{F^pA^{p+q}}{F^{p+1}A^{p+q}}} \& {\frac{F^{p+1}A^{p+q+1}}{F^{p+2}A^{p+q+1}}} \\
	{\boxed{E_0(A,F)}} \& {\frac{F^{p}A^{p+q-1}}{F^{p+1}A^{p+q-1}}} \& {\frac{F^{p+1}A^{p+q}}{F^{p+2}A^{p+q}}}
	\arrow[from=2-1, to=1-1]
	\arrow[from=2-2, to=1-2]
	\arrow[from=3-2, to=2-2]
	\arrow[from=3-3, to=2-3]
\end{tikzcd}.
\end{equation}
計算譜序列第一頁, 得 $E_1^{p,q} = H^{p+q}(F^pA / F^{p+1}A)$. \textbf{算第二頁? 我們卡住了: 微分需要人工定義!} \parnote{定義 $d_1$?}
\begin{itemize}
    \item 嘗試計算 $\frac{F^pA^{p+q}}{F^{p+1}A^{p+q}}$ 處同調群如下, 若將 $\ker$ 與 $\mathrm{im}$ 都看做商集 $\frac{F^pA^{p+q}}{F^{p+1}A^{p+q}}$ 的子集, 則 
    \begin{equation}
        E_1^{p,q} = H_0^{p,q} = \frac{[F^pA^{p+q} \quad ∩ \quad {\color{red}d^{-1}(F^{p+1}A^{p+q+1})}]\quad \bmod \quad F^{p+1}A^{p+q}}{[F^pA^{p+q} \quad ∩ \quad {\color{blue}d(F^{p}A^{p+q-1})}]\quad \bmod \quad F^{p+1}A^{p+q}}
    \end{equation}
    此處 $\bmod$ 亦可改作 $+$. 括號也可以省略: 模恆等式表明 $(U^♯ ∩ V) + U^♭ = U^♯ ∩ (V + U^♭)$. 
    
    將 $\boxed{F^pA^{p+q}}$ 處的濾過定位回複形, 得
    \begin{equation}
        % https://q.uiver.app/#q=WzAsNyxbMSwxLCJcXGJveGVke0ZecEFee3ArcX19Il0sWzIsMiwiRl57cCsxfUFee3ArcX0iXSxbMCwwLCJGXntwLTF9QV57cCtxfSJdLFsxLDAsIkZee3B9QV57cCtxKzF9IixbMjM1LDEwMCw2MCwxXV0sWzIsMSwiRl57cCsxfUFee3ArcSsxfSIsWzIzNSwxMDAsNjAsMV1dLFsxLDIsIkZee3B9QV57cCtxLTF9IixbMzU3LDEwMCw2MCwxXV0sWzAsMSwiRl57cC0xfUFee3ArcS0xfSIsWzM1NywxMDAsNjAsMV1dLFswLDIsIlxcc3Vic2V0IiwzLHsic3R5bGUiOnsiYm9keSI6eyJuYW1lIjoiZG90dGVkIn0sImhlYWQiOnsibmFtZSI6Im5vbmUifX19XSxbMSwwLCJcXHN1YnNldCIsMyx7InN0eWxlIjp7ImJvZHkiOnsibmFtZSI6ImRvdHRlZCJ9LCJoZWFkIjp7Im5hbWUiOiJub25lIn19fV0sWzQsMywiXFxzdWJzZXQiLDMseyJjb2xvdXIiOlsyMzUsMTAwLDYwXSwic3R5bGUiOnsiYm9keSI6eyJuYW1lIjoiZG90dGVkIn0sImhlYWQiOnsibmFtZSI6Im5vbmUifX19LFsyMzUsMTAwLDYwLDFdXSxbNSw2LCJcXHN1YnNldCIsMyx7ImNvbG91ciI6WzAsNjAsNjBdLCJzdHlsZSI6eyJib2R5Ijp7Im5hbWUiOiJkb3R0ZWQifSwiaGVhZCI6eyJuYW1lIjoibm9uZSJ9fX0sWzAsNjAsNjAsMV1dLFs2LDJdLFswLDNdLFs1LDBdLFsxLDRdXQ==
\begin{tikzcd}[ampersand replacement=\&]
	{F^{p-1}A^{p+q}} \& \textcolor{rgb,255:red,51;green,68;blue,255}{{F^{p}A^{p+q+1}}} \\
	\textcolor{rgb,255:red,255;green,51;blue,61}{{F^{p-1}A^{p+q-1}}} \& {\boxed{F^pA^{p+q}}} \& \textcolor{rgb,255:red,51;green,68;blue,255}{{F^{p+1}A^{p+q+1}}} \\
	\& \textcolor{rgb,255:red,255;green,51;blue,61}{{F^{p}A^{p+q-1}}} \& {F^{p+1}A^{p+q}}
	\arrow[from=2-1, to=1-1]
	\arrow["\subset"{marking, allow upside down}, dotted, no head, from=2-2, to=1-1]
	\arrow[from=2-2, to=1-2]
	\arrow["\subset"{marking, allow upside down}, color={rgb,255:red,51;green,68;blue,255}, dotted, no head, from=2-3, to=1-2]
	\arrow["\subset"{marking, allow upside down}, color={rgb,255:red,214;green,92;blue,92}, dotted, no head, from=3-2, to=2-1]
	\arrow[from=3-2, to=2-2]
	\arrow["\subset"{marking, allow upside down}, dotted, no head, from=3-3, to=2-2]
	\arrow[from=3-3, to=2-3]
\end{tikzcd}.
    \end{equation}
\end{itemize}
我們希望在 $(p,q)$-坐標處給出一族 $Z$ 與 $B$ 的濾過, 最好是以 $r$-爲指標的. 最終效果如下\parnote{$q$ 混在分次模的各分量中, 自行析出}
\begin{enumerate}
    \item ($Z$-部) $\underbracket{F^p ∩ d^{-1}(F^{p+1})}\limits_{Z_0^p} \quad ⊇ \quad \underbracket{F^p ∩ (d^{-1}(F^{p+r+1})) + F^{p+1}}\limits_{Z_r^p} \quad ⊇ \quad \underbracket{F^p ∩ (\ker d) + F^{p+1}}\limits_{Z_∞ ^p}$; 
    \item ($B$-部) $\underbracket{F^p ∩ (\mathrm{im}\ d) + F^{p+1}}\limits_{B_∞^p} \quad ⊇ \quad \underbracket{F^p ∩ (d(F^{p-r})) + F^{p+1}}\limits_{B_r^p} \quad ⊇ \quad \underbracket{F^p ∩ (0) + F^{p+1}}\limits_{B_0 ^p}$; 
    \item ($H$-群) $H_r = E_{r+1}$.  
\end{enumerate}
整合之, 得一條濾過鏈: 
\begin{equation}
    F^p ∩ d^{-1}(F^{p+1}) = \underbracket{Z_0 ^p ⊇ \cdots ⊇ Z_∞ ^p}\limits_{Z-\text{濾過}} = F^p ∩ (\ker d) + F^{p+1} ⊇  F^p ∩ (\mathrm{im} \  d) + F^{p+1} = \underbracket{B_∞ ^p ⊇ \cdots ⊇ B_0^p}\limits_{B-\text{濾過}} = F^{p+1}
\end{equation}
此時的微分如何選取? 先給出 $d_r : E_r → E_{r+1}$ 的滿-單分解. 由 Zassenhaus 引理的函子性, 
\begin{align}
    \frac{Z_{r-1} ^p}{Z_{r} ^p} &= \frac{F^p ∩ (d^{-1}(F^{p+r})) + F^{p+1}}{F^{p} ∩ (d^{-1}(F^{p+r+1})) + F^{p+1}} \\[6pt]
    \frac{A^♯ ∩ {\color{red} X^♯} + A^♭}{A^♯ ∩ {\color{red} X^♭} + A^♭} \ \ &≃ \frac{d(F^p) ∩ {\color{red} F^{p+r}} + d(F^{p+1})}{d(F^{p}) ∩ {\color{red} F^{p+r+1}} + d(F^{p+1})} \\[6pt]
    \frac{{\color{red} X^♯} ∩ A^♯ + {\color{red} X^♭}}{{\color{red} X^♯} ∩ A^♭ + {\color{red} X^♭}}  \ \  &≃ \frac{{\color{red} F^{p+r}} ∩ d(F^p) + {\color{red} F^{p+r+1}}}{{\color{red} F^{p+r}} ∩ d(F^{p+1}) + {\color{red} F^{p+r+1}}} \quad = \frac{B^{p+r}_r}{B^{p+r}_{r-1}}  
\end{align}

由以上, $E_r^p → E_r^{p+r}$ 自然繼承自 $d$, 即 \parnote{$B_{r-1} ⊆ Z_r$}
\begin{equation}
    E_r ^p = H_{r-1}^p = \frac{Z_{r-1}^p}{B_{r-1}^p} ↠ \frac{Z_{r-1}^p}{Z_{r}^p} ≃ \frac{B_{r-1}^{p+r}}{B_{r}^{p+r}} ↪ \frac{Z_{r}^{p+r}}{B_{r}^{p+r}} = E_{r+1} ^p. 
\end{equation} 
容易看出, $d_r$ 的朝向是右移 $(r,r-1)$. 
\end{example}

\begin{example}[收斂極限]
    對以上濾過復形計算 $E_∞$, 得 ($≃$ 使用 Zassenhaus)
    \begin{equation}
        E_∞^p = \frac{F^p ∩ (\ker d) + F^{p+1}}{F^p ∩ (\mathrm{im} \ d) + F^{p+1}} ≃ \frac{(\ker d) ∩ F^p + (\mathrm{im} \ d)}{(\ker d) ∩ F^{p+1} + (\mathrm{im} \ d)}.
    \end{equation}
    考慮極端情況 $F^0 = \mathrm{id}$ 與 $F^1 = 0$, 則 $E_∞^p = \frac{\ker (d)}{\mathrm{im} \ d}$ 就是同調群. 一般地, $E_∞^{∙,q}$ 給出 $H^q(A)$ 的濾過. 
\end{example}

\begin{definition}[收斂]
    稱 $(E, d)$ 收斂至複形 (微分分次模) $A$, 當且僅當\textbf{存在濾過} $F$ 使得
    \begin{equation}
        E_∞ ^{p,q} = \frac{F^p H^{p+q}(A)}{F^{p+1}H^{p+q}(A)} = \frac{(\ker d) ∩ F^p + (\mathrm{im} \ d)\quad \ \ \text{at $(p+q)$-th degree}}{(\ker d) ∩ F^{p+1} + (\mathrm{im} \ d)\quad \text{at $(p+q)$-th degree}}. 
    \end{equation}
    爲避免一些麻煩, 通常規定譜序列與複形濾過是有限型的. 
\end{definition}

\begin{theorem}
    依照構造, $(A, d,F)$ 的譜序列收斂至 $(A,d)$. 
\end{theorem}

\begin{example}[計算示例: 同調代數基本定理]
    給定濾過複形 $X ⊇ K ⊇ 0$, 譜序列收斂至 $E_2 = E_∞$: 
    \begin{equation}
        % https://q.uiver.app/#q=WzAsMjIsWzAsMSwiWF5wL0tecCJdLFswLDAsIlhee3ArMX0vS157cCsxfSJdLFswLDIsIlhee3AtMX0vS157cC0xfSJdLFsxLDAsIktee3ArMn0iXSxbMSwxLCJLXntwKzF9Il0sWzEsMiwiS15wIl0sWzAsMywiXFxib3hlZHtFXzB9Il0sWzMsMiwiSF57cC0xfShYL0spIl0sWzMsMSwiSF57cH0oWC9LKSJdLFszLDAsIkhee3ArMX0oWC9LKSJdLFs0LDIsIkhecCAoSykiXSxbNCwxLCJIXntwKzF9KEspIl0sWzQsMCwiSF57cCsyfShLKSJdLFszLDMsIlxcYm94ZWR7RV8xfSJdLFs2LDAsIlxca2VyIChcXGRlbHRhXntwKzF9KSJdLFs2LDEsIlxca2VyIChcXGRlbHRhXntwfSkiXSxbNiwyLCJcXGtlciAoXFxkZWx0YV57cC0xfSkiXSxbNywyLCJcXG1hdGhybXtjb2t9IChcXGRlbHRhXntwLTF9KSJdLFs3LDEsIlxcbWF0aHJte2Nva30gKFxcZGVsdGFee3B9KSJdLFs3LDAsIlxcbWF0aHJte2Nva30gKFxcZGVsdGFee3ArMX0pIl0sWzYsMywiXFxib3hlZHtFXzJ9Il0sWzcsMywiXFxib3hlZHtFX1xcaW5mdHl9Il0sWzIsMF0sWzAsMV0sWzUsNF0sWzQsM10sWzcsMTAsIlxcZGVsdGFee3AtMX0iXSxbOCwxMSwiXFxkZWx0YV57cH0iXSxbOSwxMiwiXFxkZWx0YV57cCsxfSJdLFsyMSwyMCwiIiwwLHsibGV2ZWwiOjIsInN0eWxlIjp7ImhlYWQiOnsibmFtZSI6Im5vbmUifX19XV0=
\begin{tikzcd}[ampersand replacement=\&, sep = small]
	{X^{p+1}/K^{p+1}} \& {K^{p+2}} \&\& {H^{p+1}(X/K)} \& {H^{p+2}(K)} \&\& {\ker (\delta^{p+1})} \& {\mathrm{cok} (\delta^{p+1})} \\
	{X^p/K^p} \& {K^{p+1}} \&\& {H^{p}(X/K)} \& {H^{p+1}(K)} \&\& {\ker (\delta^{p})} \& {\mathrm{cok} (\delta^{p})} \\
	{X^{p-1}/K^{p-1}} \& {K^p} \&\& {H^{p-1}(X/K)} \& {H^p (K)} \&\& {\ker (\delta^{p-1})} \& {\mathrm{cok} (\delta^{p-1})} \\
	{\boxed{E_0}} \&\&\& {\boxed{E_1}} \&\&\& {\boxed{E_2}} \& {\boxed{E_\infty}}
	\arrow["{\delta^{p+1}}", from=1-4, to=1-5]
	\arrow[from=2-1, to=1-1]
	\arrow[from=2-2, to=1-2]
	\arrow["{\delta^{p}}", from=2-4, to=2-5]
	\arrow[from=3-1, to=2-1]
	\arrow[from=3-2, to=2-2]
	\arrow["{\delta^{p-1}}", from=3-4, to=3-5]
	\arrow[equals, from=4-8, to=4-7]
\end{tikzcd}.
    \end{equation}
    收斂終點即 $X$ 的分次同調群, $↘$ 向分別是商與子, 即 $H^p(X)/\mathrm{cok}(δ^{p-1}) ≃ \ker (δ^p)$. 此時得到連接態射組成的長正合列
    \begin{equation}
        \cdots → H^{p-1}(X/K) → H^{p}(K) → [\mathrm{coker}(δ ^{p-1})] → H^p (X) → [\ker (δ ^p)] → H^p(X/K) → H^{p+1}(K) → \cdots . 
    \end{equation}
\end{example}

\begin{remark}
    一個特殊技巧: 算至 $E_2$ 時, 即可預判所有 $(1,-1)$-朝向的箭頭至多有兩處支撐, 從而將全複形的同調群直接嵌入即可.  
    \begin{equation}
        % https://q.uiver.app/#q=WzAsNyxbMCwyLCJIXntwLTF9KFgvSykiXSxbMCwxLCJIXntwfShYL0spIl0sWzAsMCwiSF57cCsxfShYL0spIl0sWzIsMiwiSF5wIChLKSJdLFsyLDEsIkhee3ArMX0oSykiXSxbMiwwLCJIXntwKzJ9KEspIl0sWzEsMywiXFxib3hlZHtFXzF9Il0sWzAsMywiXFxkZWx0YV57cC0xfSJdLFsxLDQsIlxcZGVsdGFee3B9Il0sWzIsNSwiXFxkZWx0YV57cCsxfSJdLFs0LDIsIkhee3ArMX0gKFgpIiwxXSxbMywxLCJIXnAgKFgpIiwxXV0=
\begin{tikzcd}[ampersand replacement=\&]
	{H^{p+1}(X/K)} \&\& {H^{p+2}(K)} \\
	{H^{p}(X/K)} \&\& {H^{p+1}(K)} \\
	{H^{p-1}(X/K)} \&\& {H^p (K)} \\
	\& {\boxed{E_2}}
	\arrow["{\delta^{p+1}}", from=1-1, to=1-3]
	\arrow["{\delta^{p}}", from=2-1, to=2-3]
	\arrow["{H^{p+1} (X)}"{description}, from=2-3, to=1-1]
	\arrow["{\delta^{p-1}}", from=3-1, to=3-3]
	\arrow["{H^p (X)}"{description}, from=3-3, to=2-1]
\end{tikzcd}.
    \end{equation}
\end{remark}

\begin{example}[乘法結構]
    \parnote{拓撲學: 譜序列是拿來乘的} 依照定義, 譜序列 $(E_r,d_r)$ 是一族滿足特殊條件的分次模. 依照動機
    \begin{itemize}
        \item $A_∞$-代數從 $H$ 還原原始信息, 揭示了同調群間 (且唯一的) 乘法結構. 關於 $A-∞$-代數的介紹見 \cite{keller2001introductionainfinityalgebrasmodules}. 
    \end{itemize}

    稱 $(E,d,μ ,ε )$ 是譜序列的一個乘法結構\parnote{$ε(p,q)$ 是符號}, 當且僅當 (假定 $d_0 \ ↑$): 
    \begin{enumerate}
        \item (初始設定) $(E_0,d,μ_0, ε)$ 是分次代數;
        \item (誘導結果) 依照 $H = \frac{Z}{B} = \frac{\text{子分次代數}}{\text{分次雙邊理想}}$, 所有 $(E_r, d_r, μ _r, ε )$ 都是分次代數, $d_r$ 的次數爲 $(r,1-r)$; 
        \item (相容條件) 分次模同構 $H^{p,q}(E_r) → E_{r+1}^{p,q}$ 建立了分次代數同構. 
    \end{enumerate}
特別地, 我們希望初始設定 $(E_0, d_0)$ 能夠誘導譜序列的乘法結構. 
\end{example}

\begin{theorem}[帶乘法結構的l濾過復形收斂定理]
    假定 $(A, d, μ)$ 是濾過分次代數 $|d| = 1$, 則收斂定理中的譜序列帶有乘法結構
    \begin{enumerate}
        \item \cite{filteredcomplex} 構造譜序列的帶有自然的乘法結構 $(E, d, μ , ε)$, 其中 $μ_r$ 由商關係誘導, $ε(p,q) = p+q$. 
        \item 收斂終點 $E_∞{p,q} ≃ (F^p H^{p+q})/ (F^{p+1} H^{p+q})$ 是分次模的同構, 同時也是分次代數的同構.
    \end{enumerate}
    \begin{proof}
        取代表元進行驗證即可. 關鍵使用了諸 $B_r$ 的雙邊理想性. 
    \end{proof}
\end{theorem}


\subsubsection{譜序列的構造 II: 雙複形}

\begin{definition}[雙複形]
    稱 $(A,d)$ 是雙復形, 若 $A$ 是 $(ℤ× ℤ)$-分次模, $d = \{d_→ , d_↑\}$ 滿足 \parnote{$\mathrm{Tot}(X)$ $=$ $\mathrm{Tot}(X^T)$}
    \begin{equation}
        d_→ : A^{p,q} → A^{p+1, q}, d_↑ : A^{p,q} → A^{p, q+1},\quad d_↑ ∘ d_↑ = d_→ ∘ d_→ = 0, d_→ ∘ d_↑ = d_↑ ∘ d_→.
    \end{equation}
    有時 (時常) 規定 $I := →$, $II := ↑$. 
\end{definition}

\begin{remark}
    \begin{pinked}
    特別注釋: 有些定義要求中間方塊交換, 微分時不給出 $(± 1)$-分次. 
    \end{pinked}
\end{remark}

\begin{example}[動機: double counting]
    願景: 先取橫向微分的同調群, 縱向的譜序列收斂至 $\mathrm{Tot}(X)$; 若先取縱向微分, 橫向譜序列亦收斂至 $\mathrm{Tot}(X)$. 
    
    先假定 $X$ 支撐有限. 顯然 
    \begin{equation}
        \left(\frac{F_→ ^p \mathrm{Tot}(X)}{F_→ ^{p+1} \mathrm{Tot}(X)}\right)^{p+q} ≃ X^{p+q} ≃ \left(\frac{F_↑ ^q \mathrm{Tot}(X)}{F_↑ ^{q+1} \mathrm{Tot}(X)}\right)^{p+q}.
    \end{equation}
    從而 ${_→}E_1^{p,q}=H^{p,q}(\mathrm{Tot}(X), F_→)$ 與 ${_↑}E_1^{p,q}=H^{p,q}(\mathrm{Tot}(X), F_↑)$ 都是收斂至 $\mathrm{Tot}(X)$ 的.  
\end{example}

\begin{theorem}
    以上 ${_→}E$ 與 ${_↑}E$ 有更好的性質: 在雙復形微分的自然誘導下, 
    \begin{enumerate}
        \item ${_→}E_2$ 恰是 $H_→(X)$ 的同調群, 換言之, ${_→}E_2^{p,q} = H\left(H_↑^{p,q-1}(X) → H_↑^{p,q}(X) → H_↑^{p,q+1}(X)\right)$; 
        \item ${_↑}E_2$ 恰是 $H_↑(X)$ 的同調群, 換言之, ${_↑}E_2^{p,q} = H\left(H_→^{p-1,q}(X) → H_→^{p,q}(X) → H_→^{p+1,q}(X)\right)$. 
    \end{enumerate}
    \begin{proof}
        追圖即可. 
    \end{proof}
\end{theorem}

\begin{remark}
    這告訴我們, 存在兩個收斂至 $\mathrm{Tot}(X)$ 的譜序列, 第二頁分別是雙複形的``橫向同調群的縱向同調群''與``縱向同調群的橫向同調群''. 
\end{remark}

\begin{example}[強形式蛇引理]
    給定正合列的同態 $f : X → Y$, 則範性質確定的典範態射 $\mathrm{ker}(f)[-1] → \mathrm{cok}(f)[1]$ 是擬同構. 
    \begin{proof}
        考慮縱向同調群, 計算橫向譜序列得 
\begin{equation}
    % https://q.uiver.app/#q=WzAsMzUsWzIsMSwiWF57bi0xfSJdLFszLDEsIlhee259Il0sWzQsMSwiWF57bisxfSJdLFsyLDAsIllee24tMX0iXSxbMywwLCJZXntufSJdLFs0LDAsIllee24rMX0iXSxbMSwxLCJYXntuLTJ9Il0sWzEsMCwiWV57bi0yfSJdLFs1LDAsIllee24rMn0iXSxbNSwxLCJYXntuKzJ9Il0sWzAsMSwiXFxib3hlZHtYX1xcdXBhcnJvdyB9Il0sWzEsMiwiXFxrZXIoZl57bi0yfSkiXSxbMiwyLCJcXGtlcihmXntuLTF9KSJdLFszLDIsIlxca2VyKGZee259KSJdLFs0LDIsIlxca2VyKGZee24rMX0pIl0sWzUsMiwiXFxrZXIoZl57bisyfSkiXSxbMSwzLCJcXG1hdGhybXtjb2t9KGZee24tMn0pIl0sWzIsMywiXFxtYXRocm17Y29rfShmXntuLTF9KSJdLFszLDMsIlxcbWF0aHJte2Nva30oZl57bn0pIl0sWzQsMywiXFxtYXRocm17Y29rfShmXntuKzF9KSJdLFs1LDMsIlxcbWF0aHJte2Nva30oZl57bisyfSkiXSxbMCwyLCJcXGJveGVke0hfXFx1cGFycm93IChYKX0iXSxbMCwzLCJcXGJveGVke3tfXFx0byB9RV8xfSJdLFswLDUsIlxcYm94ZWR7e19cXHRvIH1FXzJ9Il0sWzEsNCwiSF57bi0yfShcXGtlcikiXSxbMiw0LCJIXntuLTF9KFxca2VyKSJdLFszLDQsIkhee259KFxca2VyKSJdLFs0LDQsIkhee24rMX0oXFxrZXIpIl0sWzUsNCwiSF57bisyfShcXGtlcikiXSxbMSw1LCJIXntuLTJ9KFxcbWF0aHJte2Nva30pIl0sWzIsNSwiSF57bi0xfShcXG1hdGhybXtjb2t9KSJdLFszLDUsIkhee259KFxcbWF0aHJte2Nva30pIl0sWzQsNSwiSF57bisxfShcXG1hdGhybXtjb2t9KSJdLFs1LDUsIkhee24rMn0oXFxtYXRocm17Y29rfSkiXSxbMCw0LCJcXGJveGVke0hfXFx0byAoSF9cXHVwYXJyb3cgKFgpKX0iXSxbMCwzLCJmXntuLTF9Il0sWzEsNCwiZl57bn0iXSxbMiw1LCJmXntuKzF9Il0sWzYsNywiZl57bi0yfSJdLFs5LDgsImZee24rMn0iXSxbMTEsMTJdLFsxMiwxM10sWzEzLDE0XSxbMTQsMTVdLFsxNiwxN10sWzE3LDE4XSxbMTgsMTldLFsxOSwyMF0sWzIxLDIyLCIiLDAseyJsZXZlbCI6Miwic3R5bGUiOnsiaGVhZCI6eyJuYW1lIjoibm9uZSJ9fX1dLFsyNCwzMSwiXFx2YXJlcHNpbG9uIF57bi0xfSIsMV0sWzI1LDMyLCJcXHZhcmVwc2lsb24gXntufSIsMV0sWzI2LDMzLCJcXHZhcmVwc2lsb24gXntuKzF9IiwxXV0=
\begin{tikzcd}[ampersand replacement=\&, sep = small]
	\& {Y^{n-2}} \& {Y^{n-1}} \& {Y^{n}} \& {Y^{n+1}} \& {Y^{n+2}} \\
	{\boxed{X_\uparrow }} \& {X^{n-2}} \& {X^{n-1}} \& {X^{n}} \& {X^{n+1}} \& {X^{n+2}} \\
	{\boxed{H_\uparrow (X)}} \& {\ker(f^{n-2})} \& {\ker(f^{n-1})} \& {\ker(f^{n})} \& {\ker(f^{n+1})} \& {\ker(f^{n+2})} \\
	{\boxed{{_\to }E_1}} \& {\mathrm{cok}(f^{n-2})} \& {\mathrm{cok}(f^{n-1})} \& {\mathrm{cok}(f^{n})} \& {\mathrm{cok}(f^{n+1})} \& {\mathrm{cok}(f^{n+2})} \\
	{\boxed{H_\to (H_\uparrow (X))}} \& {H^{n-2}(\ker)} \& {H^{n-1}(\ker)} \& {H^{n}(\ker)} \& {H^{n+1}(\ker)} \& {H^{n+2}(\ker)} \\
	{\boxed{{_\to }E_2}} \& {H^{n-2}(\mathrm{cok})} \& {H^{n-1}(\mathrm{cok})} \& {H^{n}(\mathrm{cok})} \& {H^{n+1}(\mathrm{cok})} \& {H^{n+2}(\mathrm{cok})}
	\arrow["{f^{n-2}}", from=2-2, to=1-2]
	\arrow["{f^{n-1}}", from=2-3, to=1-3]
	\arrow["{f^{n}}", from=2-4, to=1-4]
	\arrow["{f^{n+1}}", from=2-5, to=1-5]
	\arrow["{f^{n+2}}", from=2-6, to=1-6]
	\arrow[equals, from=3-1, to=4-1]
	\arrow[from=3-2, to=3-3]
	\arrow[from=3-3, to=3-4]
	\arrow[from=3-4, to=3-5]
	\arrow[from=3-5, to=3-6]
	\arrow[from=4-2, to=4-3]
	\arrow[from=4-3, to=4-4]
	\arrow[from=4-4, to=4-5]
	\arrow[from=4-5, to=4-6]
	\arrow["{\varepsilon ^{n-1}}"{description}, from=5-2, to=6-4]
	\arrow["{\varepsilon ^{n}}"{description}, from=5-3, to=6-5]
	\arrow["{\varepsilon ^{n+1}}"{description}, from=5-4, to=6-6]
\end{tikzcd}
\end{equation}
    ${_→}E_3 = {_→}E_∞$, 因此 $0 → \mathrm{cok}(ε^{n-1}) → ? → \ker(ε ^n) → 0$ 是 $H(\mathrm{Tot})=0$ 的濾過. 故 $ε$ 是同構. 
    \end{proof}
\end{example}

\begin{theorem}[复形态射基本定理]
        若 $f^\bullet:Y^\bullet\to X^\bullet$ 是双边无界的复形的同态, 则存在复形 (可取作全复形) $E$ 使得下图是正合列的交换图
        \begin{equation}
            % https://q.uiver.app/#q=WzAsMTIsWzQsMCwiSF57a30oRV5cXGJ1bGxldCkiXSxbMywxLCJIXmsoXFxtYXRocm17a2VyfShmXlxcYnVsbGV0KSkiXSxbMiwxLCJIXntrLTJ9KFxcbWF0aHJte2Nva30oZl5cXGJ1bGxldCkpIl0sWzUsMCwiSF57a30oWV5cXGJ1bGxldCkiXSxbMSwwLCJIXntrLTF9KEVeXFxidWxsZXQpIl0sWzMsMCwiSF57ay0xfShYXlxcYnVsbGV0KSJdLFsyLDAsIkhee2stMX0oWV5cXGJ1bGxldCkiXSxbMCwwLCJIXntrLTJ9KFheXFxidWxsZXQpIl0sWzAsMSwiSF57ay0xfShcXG1hdGhybXtrZXJ9KGZeXFxidWxsZXQpKSJdLFs0LDEsIkhee2t9KEVeXFxidWxsZXQpIl0sWzEsMSwiSF57ay0xfShFXlxcYnVsbGV0KSJdLFs1LDEsIkhee2stMX0oXFxtYXRocm17Y29rfShmXlxcYnVsbGV0KSkiXSxbMiwxXSxbMCwzXSxbNCw2XSxbNiw1XSxbNSwwXSxbNyw0XSxbOCwxMF0sWzEwLDJdLFsxLDldLFs0LDEwLCIiLDIseyJsZXZlbCI6Miwic3R5bGUiOnsiaGVhZCI6eyJuYW1lIjoibm9uZSJ9fX1dLFswLDksIiIsMix7ImxldmVsIjoyLCJzdHlsZSI6eyJoZWFkIjp7Im5hbWUiOiJub25lIn19fV0sWzksMTFdXQ==
            \begin{tikzcd}[ampersand replacement=\&, row sep = small]
                {H^{k-2}(X^\bullet)} \& {H^{k-1}(E^\bullet)} \& {H^{k-1}(Y^\bullet)} \& {H^{k-1}(X^\bullet)} \& {H^{k}(E^\bullet)} \& {H^{k}(Y^\bullet)} \\
                {H^{k-1}(\mathrm{ker}(f^\bullet))} \& {H^{k-1}(E^\bullet)} \& {H^{k-2}(\mathrm{cok}(f^\bullet))} \& {H^k(\mathrm{ker}(f^\bullet))} \& {H^{k}(E^\bullet)} \& {H^{k-1}(\mathrm{cok}(f^\bullet))}
                \arrow[from=1-1, to=1-2]
                \arrow[from=1-2, to=1-3]
                \arrow[Rightarrow, no head, from=1-2, to=2-2]
                \arrow[from=1-3, to=1-4]
                \arrow[from=1-4, to=1-5]
                \arrow[from=1-5, to=1-6]
                \arrow[Rightarrow, no head, from=1-5, to=2-5]
                \arrow[from=2-1, to=2-2]
                \arrow[from=2-2, to=2-3]
                \arrow[from=2-3, to=2-4]
                \arrow[from=2-4, to=2-5]
                \arrow[from=2-5, to=2-6]
            \end{tikzcd}
        \end{equation}
        \begin{proof}
            视 $2\times n$ 方块为全复形 $E^\bullet$, 依次计算 $E^\bullet$ 各阶同调群的横向与纵向滤过即可.
            \end{proof}
\end{theorem}


    \begin{example}[收斂性定理的``反例'']\parnote{必須強調有界!}
        假定雙復形或濾過復形是 $(→ , ↑)$ 朝向的. 若對任意 $s$, 復形在一切 $\{(p,q) ∣ p+q = s\}$ 僅有限項非零, 則譜序列在有限步後必然穩定. \parnote{逐點收斂即可, 不必一致收斂} 常見的例子是``二/四象限-型譜序列''. 特別地, 
        \begin{equation}
            % https://q.uiver.app/#q=WzAsMTIsWzMsMiwiXFxtYXRoYmIgWiJdLFs0LDIsIlxcbWF0aGJiIFoiXSxbMywxLCJcXG1hdGhiYiBaIl0sWzIsMSwiXFxtYXRoYmIgWiJdLFsyLDAsIlxcbWF0aGJiIFoiXSxbMiwyLCIwIl0sWzEsMSwiMCJdLFszLDAsIjAiXSxbNCwxLCIwIl0sWzUsMiwiMCJdLFsxLDAsIlxcbWF0aGJiIFoiXSxbMCwwLCIwIl0sWzAsMSwiIiwwLHsibGV2ZWwiOjIsInN0eWxlIjp7ImhlYWQiOnsibmFtZSI6Im5vbmUifX19XSxbMCwyLCIiLDIseyJsZXZlbCI6Miwic3R5bGUiOnsiaGVhZCI6eyJuYW1lIjoibm9uZSJ9fX1dLFszLDIsIiIsMCx7ImxldmVsIjoyLCJzdHlsZSI6eyJoZWFkIjp7Im5hbWUiOiJub25lIn19fV0sWzMsNCwiIiwyLHsibGV2ZWwiOjIsInN0eWxlIjp7ImhlYWQiOnsibmFtZSI6Im5vbmUifX19XSxbMTAsNCwiIiwwLHsibGV2ZWwiOjIsInN0eWxlIjp7ImhlYWQiOnsibmFtZSI6Im5vbmUifX19XSxbNCw3XSxbNiwzXSxbNSwzXSxbMiw4XSxbMSw4XSxbMiw3XSxbMSw5XSxbNSwwXSxbMTEsMTBdLFs2LDEwXV0=
\begin{tikzcd}[ampersand replacement=\&, sep = small]
	0 \& {\mathbb Z} \& {\mathbb Z} \& 0 \\
	\& 0 \& {\mathbb Z} \& {\mathbb Z} \& 0 \\
	\&\& 0 \& {\mathbb Z} \& {\mathbb Z} \& 0
	\arrow[from=1-1, to=1-2]
	\arrow[equals, from=1-2, to=1-3]
	\arrow[from=1-3, to=1-4]
	\arrow[from=2-2, to=1-2]
	\arrow[from=2-2, to=2-3]
	\arrow[equals, from=2-3, to=1-3]
	\arrow[equals, from=2-3, to=2-4]
	\arrow[from=2-4, to=1-4]
	\arrow[from=2-4, to=2-5]
	\arrow[from=3-3, to=2-3]
	\arrow[from=3-3, to=3-4]
	\arrow[equals, from=3-4, to=2-4]
	\arrow[equals, from=3-4, to=3-5]
	\arrow[from=3-5, to=2-5]
	\arrow[from=3-5, to=3-6]
\end{tikzcd}
        \end{equation}
        中無界的復形各行列正合, 故譜序列收斂至 $0$; 但全復形的是微分爲 $0$ 的非零復形, 從而非正合.  
    \end{example}
    
\subsubsection{應用: AR 序列}

\begin{theorem}[Auslander defeat]
    選定好一些的代數 (例如有限維代數), 則任意短正合列 $0 → K → X → Y → 0$, 則有長正合列
    \begin{equation}
        0 → (-, K) → (-, X) → (-, Y) → \mathrm{Tr}(-) ⊗ K → \mathrm{Tr}(-) ⊗ X → \mathrm{Tr}(-) ⊗ Y → 0.
    \end{equation}
    \begin{proof}
        取 $0 → ν (M) → ν (P_0) → ν (P_1) → \mathrm{Tr}(M) → 0$, 使用蛇引理. 
    \end{proof}
\end{theorem}

\begin{theorem}[穩定 Hom]
    對賦值 $Y ⊗ X^t → (X, Y),\quad y⊗f ↦ [x ↦ y⋅ f(x)]$, 有正合列
    \begin{equation}
        0 → \mathrm{Tor}_2(\mathrm{Tr}(X), Y) → Y ⊗ X^t → (X, Y) → \mathrm{Tor}_1(\mathrm{Tr}(X), Y) → 0.
    \end{equation}
    特別地, $\mathrm{Tor}_1(\mathrm{Tr}(M), N) = \underline{\mathrm{Hom}(M,N)}$. 
    \begin{proof}
        對 $M$ 取平坦分解 (投射分解) $F^{-1} → F^0 → M → 0$, 類似取 $Q → N$. 此時
        \begin{equation}
            % https://q.uiver.app/#q=WzAsMjcsWzIsMCwiKEZeey0xfSledFxcb3RpbWVzIFFeey0xfSJdLFszLDAsIihGXnstMX0pXnRcXG90aW1lcyBRXnswfSJdLFszLDEsIihGXnswfSledFxcb3RpbWVzIFFeezB9Il0sWzIsMSwiKEZeezB9KV50XFxvdGltZXMgUV57LTF9Il0sWzEsMSwiKEZeezB9KV50XFxvdGltZXMgUV57LTJ9Il0sWzEsMCwiKEZeey0xfSledFxcb3RpbWVzIFFeey0yfSJdLFswLDAsIlxcY2RvdHMiXSxbMCwxLCJcXGNkb3RzIl0sWzQsMSwiXFxib3hlZHtFXzB9Il0sWzAsMiwiXFxjZG90cyAiXSxbMywyLCJcXG1hdGhybXtUcn0oTSlcXG90aW1lcyBRXjAiXSxbMywzLCJNXnRcXG90aW1lcyBRXnswfSJdLFsyLDIsIlxcbWF0aHJte1RyfShNKVxcb3RpbWVzIFFeey0xfSJdLFsxLDIsIlxcbWF0aHJte1RyfShNKVxcb3RpbWVzIFFeey0yfSJdLFsyLDMsIk1edFxcb3RpbWVzIFFeey0xfSJdLFsxLDMsIk1edFxcb3RpbWVzIFFeey0yfSJdLFswLDMsIlxcY2RvdHMiXSxbNCwzLCJcXGJveGVke0VfMX0iXSxbMyw0LCJcXG1hdGhybXtUcn0oTSlcXG90aW1lcyBOIl0sWzIsNCwiXFxtYXRocm17VG9yfV8xKFxcbWF0aHJte1RyfShNKSwgTikiXSxbMSw0LCJcXG1hdGhybXtUb3J9XzIoXFxtYXRocm17VHJ9KE0pLCBOKSJdLFswLDQsIlxcY2RvdHMiXSxbMyw1LCJNXnRcXG90aW1lcyBOIl0sWzIsNSwiXFxtYXRocm17VG9yfV8xKE1edCwgTikiXSxbMSw1LCJcXG1hdGhybXtUb3J9XzIoTV50LCBOKSJdLFswLDUsIlxcY2RvdHMgIl0sWzQsNSwiXFxib3hlZHtFXzJ9Il0sWzQsNV0sWzMsMF0sWzIsMV0sWzcsNl0sWzksMTNdLFsxMywxMl0sWzEyLDEwXSxbMTYsMTVdLFsxNSwxNF0sWzE0LDExXSxbMjAsMjJdLFsyMSwyM11d
\begin{tikzcd}[ampersand replacement=\&,sep=tiny]
	\cdots \& {(F^{-1})^t\otimes Q^{-2}} \& {(F^{-1})^t\otimes Q^{-1}} \& {(F^{-1})^t\otimes Q^{0}} \\
	\cdots \& {(F^{0})^t\otimes Q^{-2}} \& {(F^{0})^t\otimes Q^{-1}} \& {(F^{0})^t\otimes Q^{0}} \& {\boxed{E_0}} \\
	{\cdots } \& {\mathrm{Tr}(M)\otimes Q^{-2}} \& {\mathrm{Tr}(M)\otimes Q^{-1}} \& {\mathrm{Tr}(M)\otimes Q^0} \\
	\cdots \& {M^t\otimes Q^{-2}} \& {M^t\otimes Q^{-1}} \& {M^t\otimes Q^{0}} \& {\boxed{E_1}} \\
	\cdots \& {\mathrm{Tor}_2(\mathrm{Tr}(M), N)} \& {\mathrm{Tor}_1(\mathrm{Tr}(M), N)} \& {\mathrm{Tr}(M)\otimes N} \\
	{\cdots } \& {\mathrm{Tor}_2(M^t, N)} \& {\mathrm{Tor}_1(M^t, N)} \& {M^t\otimes N} \& {\boxed{E_2}}
	\arrow[from=2-1, to=1-1]
	\arrow[from=2-2, to=1-2]
	\arrow[from=2-3, to=1-3]
	\arrow[from=2-4, to=1-4]
	\arrow[from=3-1, to=3-2]
	\arrow[from=3-2, to=3-3]
	\arrow[from=3-3, to=3-4]
	\arrow[from=4-1, to=4-2]
	\arrow[from=4-2, to=4-3]
	\arrow[from=4-3, to=4-4]
	\arrow[from=5-1, to=6-3]
	\arrow[from=5-2, to=6-4]
\end{tikzcd}.
        \end{equation}
        $E_2$ 收斂至全復形的同調群; 或是將 $E_0$ 箭頭該做橫向, 得 $E_1 = \begin{tikzcd}[ampersand replacement=\&, sep = small]
            {(F^{-1})^t\otimes N} \\
            {(F^{0})^t\otimes N}
            \arrow[from=2-1, to=1-1]
        \end{tikzcd}$, 相應地, 
\begin{enumerate}
    \item $E_2$ 上項 $\mathrm{cok}[(F^0)^t → (F^{-1})^t] ⊗ N = \mathrm{Tr}(M) ⊗ N$; 
    \item $E_2$ 下項 $\ker [(F^0, N) → (F^{-1}, N)] = (M, N)$. 
\end{enumerate}
        雙復形的全同調群 $[\mathrm{Tr}(M)⊗ N\quad \mathrm{Tor}_1(\mathrm{Tr}(M), N)\quad \cdots \quad ]$. ``配上'' $E_2$ 的濾過, 得到
        \begin{equation}
            % https://q.uiver.app/#q=WzAsMTAsWzMsMCwiXFxtYXRocm17VHJ9KE0pXFxvdGltZXMgTiJdLFsyLDAsIlxcbWF0aHJte1Rvcn1fMShcXG1hdGhybXtUcn0oTSksIE4pIl0sWzEsMCwiXFxtYXRocm17VG9yfV8yKFxcbWF0aHJte1RyfShNKSwgTikiXSxbMCwwLCJcXGNkb3RzIl0sWzMsMiwiTV50XFxvdGltZXMgTiJdLFsyLDIsIlxcbWF0aHJte1Rvcn1fMShNXnQsIE4pIl0sWzEsMiwiXFxtYXRocm17VG9yfV8yKE1edCwgTikiXSxbMCwyLCJcXGNkb3RzICJdLFs0LDIsIjAiXSxbNSwyLCIwIl0sWzIsNF0sWzMsNV0sWzEsOF0sWzgsMCwiSF8wIiwxLHsic3R5bGUiOnsiYm9keSI6eyJuYW1lIjoiZG90dGVkIn19fV0sWzQsMSwiSF8xIiwxLHsic3R5bGUiOnsiYm9keSI6eyJuYW1lIjoiZG90dGVkIn19fV0sWzUsMiwiSF8yIiwxLHsic3R5bGUiOnsiYm9keSI6eyJuYW1lIjoiZG90dGVkIn19fV0sWzYsMywiSF8zIiwxLHsic3R5bGUiOnsiYm9keSI6eyJuYW1lIjoiZG90dGVkIn19fV0sWzAsOV1d
\begin{tikzcd}[ampersand replacement=\&,sep=tiny]
	\cdots \& {\mathrm{Tor}_2(\mathrm{Tr}(M), N)} \& {\mathrm{Tor}_1(\mathrm{Tr}(M), N)} \& {\mathrm{Tr}(M)\otimes N} \\
	\\
	{\cdots } \& {\mathrm{Tor}_2(M^t, N)} \& {\mathrm{Tor}_1(M^t, N)} \& {M^t\otimes N} \& 0 \& 0
	\arrow[from=1-1, to=3-3]
	\arrow[from=1-2, to=3-4]
	\arrow[from=1-3, to=3-5]
	\arrow[from=1-4, to=3-6]
	\arrow["{H_3}"{description}, dotted, from=3-2, to=1-1]
	\arrow["{H_2}"{description}, dotted, from=3-3, to=1-2]
	\arrow["{H_1}"{description}, dotted, from=3-4, to=1-3]
	\arrow["{H_0}"{description}, dotted, from=3-5, to=1-4]
\end{tikzcd}.
        \end{equation}
        $H_0$ 處顯然. $H_1$ 對應四項長正合列

        最後說明 $\underline{(M, N)} ≃ \mathrm{Tor}_1(\mathrm{Tr}(M), N)$. \parnote{穩定 Hom} 取投射蓋 $p : Q^0 → N$, 穩定 Hom 即 $\mathrm{coker}(M, p)$. 
        \begin{equation}
            % https://q.uiver.app/#q=WzAsOCxbMSwxLCJNXnRcXG90aW1lcyBOIl0sWzIsMSwiKE0sIE4pIl0sWzEsMCwiTV50IFxcb3RpbWVzIFAiXSxbMywxLCJcXG1hdGhybXtjb2t9KHApIl0sWzEsMiwiMCJdLFszLDIsIlxcdW5kZXJsaW5lIHsoTSwgTil9Il0sWzIsMCwiMCJdLFswLDEsIlxcdmRvdHMgIl0sWzAsMV0sWzIsMCwiIiwxLHsic3R5bGUiOnsiaGVhZCI6eyJuYW1lIjoiZXBpIn19fV0sWzIsMV0sWzAsNF0sWzEsNV0sWzEsM10sWzQsNSwiIiwxLHsic3R5bGUiOnsiYm9keSI6eyJuYW1lIjoiZGFzaGVkIn19fV0sWzUsMywiIiwxLHsic3R5bGUiOnsiYm9keSI6eyJuYW1lIjoiZGFzaGVkIn19fV0sWzMsNiwiIiwxLHsic3R5bGUiOnsiYm9keSI6eyJuYW1lIjoiZGFzaGVkIn19fV0sWzcsMF0sWzcsNCwiIiwxLHsic3R5bGUiOnsiYm9keSI6eyJuYW1lIjoiZGFzaGVkIn19fV1d
\begin{tikzcd}[ampersand replacement=\&,sep=small]
	\& {M^t \otimes P} \& 0 \\
	{\vdots } \& {M^t\otimes N} \& {(M, N)} \& {\mathrm{cok}(p)} \\
	\& 0 \&\& {\underline {(M, N)}}
	\arrow[two heads, from=1-2, to=2-2]
	\arrow[from=1-2, to=2-3]
	\arrow[from=2-1, to=2-2]
	\arrow[dashed, from=2-1, to=3-2]
	\arrow[from=2-2, to=2-3]
	\arrow[from=2-2, to=3-2]
	\arrow[from=2-3, to=2-4]
	\arrow[from=2-3, to=3-4]
	\arrow[dashed, from=2-4, to=1-3]
	\arrow[dashed, from=3-2, to=3-4]
	\arrow[dashed, from=3-4, to=2-4]
\end{tikzcd}
        \end{equation}
        對上述符合態射使用小-蛇引理, 得同構 $\underline {(M, N)} ≃ \mathrm{cok}(p)$. 
    \end{proof}
\end{theorem}

\begin{theorem}[穩定 Tensor ?]
    有典範四項正合列 
    \begin{equation}
        0 → \mathrm{Ext}^1(\mathrm{Tr}(M), N) → N ⊗ M → (M^t, N) → \mathrm{Ext}^2(\mathrm{Tr}(M), N) → 0. 
    \end{equation}
    依照 $M ≃ (M^t)^t$, 得正合列的同構 
    \begin{equation}
        % https://q.uiver.app/#q=WzAsMTIsWzEsMCwiXFxtYXRocm17VG9yfV8yKFxcbWF0aHJte1RyfShNKSwgTikiXSxbNCwwLCJcXG1hdGhybXtUb3J9XzEoXFxtYXRocm17VHJ9KE0pLCBOKSJdLFszLDAsIihNLCBOKSAiXSxbMiwwLCJOIOKKlyBNXnQgIl0sWzMsMSwiKE0sIE4pIl0sWzIsMSwiTiDiipcgTV50ICJdLFsxLDEsIlxcbWF0aHJte0V4dH1eMShcXG1hdGhybXtUcn0oTV50KSwgTikiXSxbNCwxLCJcXG1hdGhybXtFeHR9XjIoXFxtYXRocm17VHJ9KE1edCksIE4pIl0sWzAsMCwiMCJdLFs1LDAsIjAiXSxbMCwxLCIwIl0sWzUsMSwiMCJdLFs4LDBdLFswLDNdLFszLDJdLFsyLDFdLFsxLDldLFsxMCw2XSxbNiw1XSxbNSw0XSxbNCw3XSxbNywxMV0sWzMsNSwiIiwxLHsibGV2ZWwiOjIsInN0eWxlIjp7ImhlYWQiOnsibmFtZSI6Im5vbmUifX19XSxbMiw0LCIiLDEseyJsZXZlbCI6Miwic3R5bGUiOnsiaGVhZCI6eyJuYW1lIjoibm9uZSJ9fX1dLFswLDYsIlxcc2ltZXEgIiwyXSxbMSw3LCJcXHNpbWVxICIsMl1d
\begin{tikzcd}[ampersand replacement=\&,sep=small]
	0 \& {\mathrm{Tor}_2(\mathrm{Tr}(M), N)} \& {N ⊗ M^t } \& {(M, N) } \& {\mathrm{Tor}_1(\mathrm{Tr}(M), N)} \& 0 \\
	0 \& {\mathrm{Ext}^1(\mathrm{Tr}(M^t), N)} \& {N ⊗ M^t } \& {(M, N)} \& {\mathrm{Ext}^2(\mathrm{Tr}(M^t), N)} \& 0
	\arrow[from=1-1, to=1-2]
	\arrow[from=1-2, to=1-3]
	\arrow["{\simeq }"', from=1-2, to=2-2]
	\arrow[from=1-3, to=1-4]
	\arrow[equals, from=1-3, to=2-3]
	\arrow[from=1-4, to=1-5]
	\arrow[equals, from=1-4, to=2-4]
	\arrow[from=1-5, to=1-6]
	\arrow["{\simeq }"', from=1-5, to=2-5]
	\arrow[from=2-1, to=2-2]
	\arrow[from=2-2, to=2-3]
	\arrow[from=2-3, to=2-4]
	\arrow[from=2-4, to=2-5]
	\arrow[from=2-5, to=2-6]
\end{tikzcd}.
    \end{equation}
\end{theorem}

\subsubsection{應用: 超同調代數}

\begin{remark}
    同調代數的一個重要構造是對象的投射消解與內射餘消解, 如果將對象換上組合性質 (通過 $𝒜 → C(𝒜)$), 能否繼續建立相應的投射分解? 
\end{remark}

\begin{theorem}[Eilenburg-Cartan 消解, 超-投射分解/內射分解]
    給定復形 $X^∙$ (坐標 $(0,∙ )$), 則存在投射複形的消解
    \begin{equation}
        [\cdots → P^{-1, ∙} →P^{0, ∙} →X^∙ → 0]\quad =: \quad [P → X → 0].
    \end{equation} 
    特別地, 若 $P$ 關於 $↘ ↖$ 方向有限, 則 $\mathrm{Tot}(P) → X$ 是擬同構. 
    \begin{proof}
        對復形 $X^{p-1} → X^p → X^{p+1}$ 之中項提出 $0 → \mathrm{ker}(d^p) → X^p → \mathrm{im}(d^p) → 0$, 轉化得 $0 → \mathrm{im}(d^{p-1}) → \mathrm{ker}(d^p) → H^p(X) → 0$. 更清晰地, 有下圖
\begin{equation}
    % https://q.uiver.app/#q=WzAsMTIsWzAsMCwiWF57cC0xfSJdLFs0LDAsIlhecCJdLFs4LDAsIlhee3ArMX0iXSxbMiwwLCJcXG1hdGhybXtpbX0oZF57cC0xfSkiXSxbNiwwLCJcXG1hdGhybXtpbX0oZF57cH0pIl0sWzEsMSwiXFxtYXRocm17Y29rfShkXntwLTJ9KSJdLFszLDEsIlxca2VyIChkXnApIl0sWzcsMSwiXFxrZXIgKGRee3ArMX0pIl0sWzUsMSwiXFxtYXRocm17Y29rfShkXntwLTF9KSJdLFswLDIsIkhee3AtMX0oWCkiXSxbNCwyLCJIXntwfShYKSJdLFs4LDIsIkhee3ArMX0oWCkiXSxbOSw1LCIiLDAseyJzdHlsZSI6eyJ0YWlsIjp7Im5hbWUiOiJob29rIiwic2lkZSI6InRvcCJ9fX1dLFs1LDMsIiIsMCx7InN0eWxlIjp7ImhlYWQiOnsibmFtZSI6ImVwaSJ9fX1dLFszLDYsIiIsMCx7InN0eWxlIjp7InRhaWwiOnsibmFtZSI6Imhvb2siLCJzaWRlIjoidG9wIn19fV0sWzYsMTAsIiIsMCx7InN0eWxlIjp7ImhlYWQiOnsibmFtZSI6ImVwaSJ9fX1dLFsxMCw4LCIiLDAseyJzdHlsZSI6eyJ0YWlsIjp7Im5hbWUiOiJob29rIiwic2lkZSI6InRvcCJ9fX1dLFs4LDQsIiIsMCx7InN0eWxlIjp7ImhlYWQiOnsibmFtZSI6ImVwaSJ9fX1dLFs0LDcsIiIsMCx7InN0eWxlIjp7InRhaWwiOnsibmFtZSI6Imhvb2siLCJzaWRlIjoidG9wIn19fV0sWzcsMTEsIiIsMCx7InN0eWxlIjp7ImhlYWQiOnsibmFtZSI6ImVwaSJ9fX1dLFswLDUsIiIsMCx7InN0eWxlIjp7ImJvZHkiOnsibmFtZSI6ImRvdHRlZCJ9LCJoZWFkIjp7Im5hbWUiOiJlcGkifX19XSxbMSw4LCIiLDAseyJzdHlsZSI6eyJib2R5Ijp7Im5hbWUiOiJkb3R0ZWQifSwiaGVhZCI6eyJuYW1lIjoiZXBpIn19fV0sWzYsMSwiIiwwLHsic3R5bGUiOnsidGFpbCI6eyJuYW1lIjoiaG9vayIsInNpZGUiOiJ0b3AifSwiYm9keSI6eyJuYW1lIjoiZG90dGVkIn19fV0sWzcsMiwiIiwwLHsic3R5bGUiOnsidGFpbCI6eyJuYW1lIjoiaG9vayIsInNpZGUiOiJ0b3AifSwiYm9keSI6eyJuYW1lIjoiZG90dGVkIn19fV0sWzAsMywiIiwwLHsic3R5bGUiOnsiYm9keSI6eyJuYW1lIjoiZG90dGVkIn0sImhlYWQiOnsibmFtZSI6ImVwaSJ9fX1dLFsxLDQsIiIsMSx7InN0eWxlIjp7ImJvZHkiOnsibmFtZSI6ImRvdHRlZCJ9LCJoZWFkIjp7Im5hbWUiOiJlcGkifX19XSxbMywxLCIiLDAseyJzdHlsZSI6eyJ0YWlsIjp7Im5hbWUiOiJob29rIiwic2lkZSI6InRvcCJ9LCJib2R5Ijp7Im5hbWUiOiJkb3R0ZWQifX19XSxbNCwyLCIiLDAseyJzdHlsZSI6eyJ0YWlsIjp7Im5hbWUiOiJob29rIiwic2lkZSI6InRvcCJ9LCJib2R5Ijp7Im5hbWUiOiJkb3R0ZWQifX19XV0=
\begin{tikzcd}[ampersand replacement=\&, sep = tiny]
	{X^{p-1}} \&\& {\mathrm{im}(d^{p-1})} \&\& {X^p} \&\& {\mathrm{im}(d^{p})} \&\& {X^{p+1}} \\
	\& {\mathrm{cok}(d^{p-2})} \&\& {\ker (d^p)} \&\& {\mathrm{cok}(d^{p-1})} \&\& {\ker (d^{p+1})} \\
	{H^{p-1}(X)} \&\&\&\& {H^{p}(X)} \&\&\&\& {H^{p+1}(X)}
	\arrow[dotted, two heads, from=1-1, to=1-3]
	\arrow[dotted, two heads, from=1-1, to=2-2]
	\arrow[dotted, hook, from=1-3, to=1-5]
	\arrow[hook, from=1-3, to=2-4]
	\arrow[dotted, two heads, from=1-5, to=1-7]
	\arrow[dotted, two heads, from=1-5, to=2-6]
	\arrow[dotted, hook, from=1-7, to=1-9]
	\arrow[hook, from=1-7, to=2-8]
	\arrow[two heads, from=2-2, to=1-3]
	\arrow[dotted, hook, from=2-4, to=1-5]
	\arrow[two heads, from=2-4, to=3-5]
	\arrow[two heads, from=2-6, to=1-7]
	\arrow[dotted, hook, from=2-8, to=1-9]
	\arrow[two heads, from=2-8, to=3-9]
	\arrow[hook, from=3-1, to=2-2]
	\arrow[hook, from=3-5, to=2-6]
\end{tikzcd}
\end{equation}
        先對 $H$ 與 $\mathrm{im}$ 進行投射分解, 使用馬蹄引理構造 $\ker$ 或 $\mathrm{cok}$ 的投射分解, 最後再使用一次馬蹄引理構造 $X$ 的投射分解即可. \textbf{構造出的 $I^{p, ∙}$ 甚至都是可裂的!}\parnote{超可裂消解}

        雙復形 $P$ 所有橫行在 $p ≠ 0$ 時正合, $p=0$ 處的同調群恰好是 $X$. 假若該雙復形在 $↘ ↖$ 方向有限, 由譜序列收斂性定理知 $H(X) = {↑}E_2 ⇒ H(\mathrm{Tot}(P))$, 因此 $\mathrm{Tot}(P) → X$ 誘導了擬同構. 
    \end{proof}
\end{theorem}

\begin{definition}[超-導出函子]
    對左正合函子 $F : 𝒜 → ℬ$ 考察 $i$-次右導出, 實際上是復合函子
\begin{equation}
    [R^i F] = [𝒜 ↪ D(𝒜) \xrightarrow {RF} D(ℬ) \xrightarrow{H^i(-)}ℬ]. 
\end{equation}
將第一處 $↪$ 捨去, 可定義 $R^i F :D(𝒜) → ℬ$. 右正合的左導出亦然. 
\end{definition}

\begin{example}[Kunneth 譜序列]
    給定上有界復形 $C$ 與可裂超投射分解 $P → C$. \parnote{可裂!} 此時, $R^{i}F(C) = R^{i}F(\mathrm{Tot}(P))$. 可以計算以下譜序列
\begin{equation}
    % https://q.uiver.app/#q=WzAsMjMsWzEsMSwiRihJXnswLHB9KSJdLFsxLDIsIkYoSV57MCxwLTF9KSJdLFsxLDAsIkYoSV57MCxwKzF9KSJdLFswLDAsIkYoSV57LTEscCsxfSkiXSxbMCwxLCJGKEleey0xLHB9KSJdLFswLDIsIkYoSV57LTEscC0xfSkiXSxbMCwzLCJcXGJveGVke0VfMH0iXSxbMiwzLCJcXGJveGVke0VfMX0iXSxbMiwyLCJGKEheey0xLHAtMX0oSSkpIl0sWzIsMSwiRihIXnstMSxwfShJKSkiXSxbMiwwLCJGKEheey0xLHArMX0oSSkpIl0sWzMsMiwiRihIXnswLHAtMX0oSSkpIl0sWzMsMSwiRihIXnswLHB9KEkpKSJdLFszLDAsIkYoSF57MCxwLTF9KEkpKSJdLFs0LDAsIlJeMUYoSF57cCsxfShDKSkiXSxbNCwxLCJSXjFGKEhee3B9KEMpKSJdLFs0LDIsIlJeMUYoSF57cC0xfShDKSkiXSxbNSwwLCJGKEhee3ArMX0oQykpIl0sWzUsMSwiRihIXntwfShDKSkiXSxbNSwyLCJGKEhee3AtMX0oQykpIl0sWzYsMSwiMCJdLFs2LDIsIjAiXSxbNCwzLCJcXGJveGVke0VfMn0iXSxbMSwwXSxbMCwyXSxbNSw0XSxbNCwzXSxbMTAsMTNdLFs5LDEyXSxbOCwxMV0sWzE0LDIwXSxbMTUsMjFdXQ==
\begin{tikzcd}[ampersand replacement=\&, sep = small]
	{F(I^{-1,p+1})} \& {F(I^{0,p+1})} \& {F(H^{-1,p+1}(I))} \& {F(H^{0,p-1}(I))} \& {R^1F(H^{p+1}(C))} \& {F(H^{p+1}(C))} \\
	{F(I^{-1,p})} \& {F(I^{0,p})} \& {F(H^{-1,p}(I))} \& {F(H^{0,p}(I))} \& {R^1F(H^{p}(C))} \& {F(H^{p}(C))} \& 0 \\
	{F(I^{-1,p-1})} \& {F(I^{0,p-1})} \& {F(H^{-1,p-1}(I))} \& {F(H^{0,p-1}(I))} \& {R^1F(H^{p-1}(C))} \& {F(H^{p-1}(C))} \& 0 \\
	{\boxed{E_0}} \&\& {\boxed{E_1}} \&\& {\boxed{E_2}}
	\arrow[from=1-3, to=1-4]
	\arrow[from=1-5, to=2-7]
	\arrow[from=2-1, to=1-1]
	\arrow[from=2-2, to=1-2]
	\arrow[from=2-3, to=2-4]
	\arrow[from=2-5, to=3-7]
	\arrow[from=3-1, to=2-1]
	\arrow[from=3-2, to=2-2]
	\arrow[from=3-3, to=3-4]
\end{tikzcd}.
\end{equation}
特別地, $E_1$ 使用了消解的可裂性, 即 $FH = HF$. 綜上, $R^{q}F(H^p(C))$ 給出 $R^{p+q}F(C)$ 的濾過 \parnote{$p$ 位置反了, 今後再改吧}
\end{example}

\begin{theorem}[Kunneth 譜序列定理]
選用此處 (\cite{rotman2008introduction}) 版本. 稱 $X$ 是正 (負) 的復形, 當且僅當 $X$ 的非零像僅能落在 $ℤ_{>0}$ ($ℤ_{<0}$) 分支. 
\begin{enumerate}
    \item 記 $A$ 與 $C$ 均是負的復形, 且 $A$ 或 $C$ 一者平坦, 則有譜序列\parnote{第一象限}
    \begin{equation}
        E^{p,q}_2 = ∐ _{s + t = q} \mathrm{Tor}_p (H^s(A), H^t(C)) ⇒ H^{p+q} (\mathrm{Tot}(A⊗C)); 
    \end{equation}
    \item 記 $A$ 是負復形, $C$ 是正復形, 假定 $A$ 投射或 $C$ 內射, 則有第三象限譜序列 
    \begin{equation}
        E_2 ^{p,q} = ∐ _{s + t = q} \mathrm{Ext}^p (H^{-s}(A), H^t(C)) ⇒ H^{p+q} (ℋ(A,C)). 
    \end{equation}
\end{enumerate}
\begin{proof}
    對第一問, 不妨假設 $C$ 平坦, 此時 $C ⊗ -$ 是復形至復形的函子. 記 $F → A$ 是平坦分解 (投射分解), 對雙復形 $∐ _{i+j = p}(F^{q,i} ⊗ C^j)$ 計算譜序列得
    \begin{equation}
        % https://q.uiver.app/#q=WzAsMzYsWzIsNywiXFxjb3Byb2QgX3tpK2ogPSBwfShBXmkgXFxvdGltZXMgQ15qICkiXSxbMyw3LCJcXGNvcHJvZCBfe2kraiA9IHArMX0oQV5pIFxcb3RpbWVzIENeaiApIl0sWzEsNywiXFxjb3Byb2QgX3tpK2ogPSBwLTF9KEFeaSBcXG90aW1lcyBDXmogKSJdLFswLDcsIlxcYm94ZWR7RV8xfSJdLFswLDYsIlxcYm94ZWR7RV8yfSJdLFsxLDYsIkhee3AtMX0oQSBcXG90aW1lcyBDKSJdLFsyLDYsIkhee3B9KEEgXFxvdGltZXMgQykiXSxbMyw2LCJIXntwKzF9KEEgXFxvdGltZXMgQykiXSxbMSw4LCJcXGNvcHJvZCBfe2kraiA9IHAtMX0oRl57MCxpfSBcXG90aW1lcyBDXmogKSJdLFsyLDgsIlxcY29wcm9kIF97aStqID0gcH0oRl57MCxpfSBcXG90aW1lcyBDXmogKSJdLFszLDgsIlxcY29wcm9kIF97aStqID0gcCsxfShGXnswLGl9IFxcb3RpbWVzIENeaiApIl0sWzEsOSwiXFxjb3Byb2QgX3tpK2ogPSBwLTF9KEZeey0xLGl9IFxcb3RpbWVzIENeaiApIl0sWzIsOSwiXFxjb3Byb2QgX3tpK2ogPSBwfShGXnstMSxpfSBcXG90aW1lcyBDXmogKSJdLFszLDksIlxcY29wcm9kIF97aStqID0gcCsxfShGXnstMSxpfSBcXG90aW1lcyBDXmogKSJdLFszLDUsIlxcY29wcm9kIF97aStqID0gcCsxfShGXnstMSxpfSBcXG90aW1lcyBDXmogKSJdLFszLDQsIlxcY29wcm9kIF97aStqID0gcCsxfShGXnswLGl9IFxcb3RpbWVzIENeaiApIl0sWzIsNSwiXFxjb3Byb2QgX3tpK2ogPSBwfShGXnstMSxpfSBcXG90aW1lcyBDXmogKSJdLFsyLDQsIlxcY29wcm9kIF97aStqID0gcH0oRl57MCxpfSBcXG90aW1lcyBDXmogKSJdLFsxLDUsIlxcY29wcm9kIF97aStqID0gcC0xfShGXnstMSxpfSBcXG90aW1lcyBDXmogKSJdLFsxLDQsIlxcY29wcm9kIF97aStqID0gcC0xfShGXnswLGl9IFxcb3RpbWVzIENeaiApIl0sWzEsMywiXFxjb3Byb2QgX3tpK2ogPSBwLTF9KEZeey0xLCBpfVxcb3RpbWVzIEheaiAoQykpIl0sWzIsMywiXFxjb3Byb2QgX3tpK2ogPSBwfShGXnstMSwgaX1cXG90aW1lcyBIXmogKEMpKSJdLFszLDMsIlxcY29wcm9kIF97aStqID0gcCsxfShGXnstMSwgaX1cXG90aW1lcyBIXmogKEMpKSJdLFsxLDIsIlxcY29wcm9kIF97aStqID0gcC0xfShGXnswLCBpfVxcb3RpbWVzIEheaiAoQykpIl0sWzIsMiwiXFxjb3Byb2QgX3tpK2ogPSBwfShGXnswLCBpfVxcb3RpbWVzIEheaiAoQykpIl0sWzMsMiwiXFxjb3Byb2QgX3tpK2ogPSBwKzF9KEZeezAsIGl9XFxvdGltZXMgSF5qIChDKSkiXSxbMSwwLCJcXGNvcHJvZCBfe2kraiA9IHAtMX0oQV5pIFxcb3RpbWVzIEheaiAoQykpIl0sWzIsMCwiXFxjb3Byb2QgX3tpK2ogPSBwfShBXmkgXFxvdGltZXMgSF5qIChDKSkiXSxbMywwLCJcXGNvcHJvZCBfe2kraiA9IHArMX0oQV5pIFxcb3RpbWVzIEheaiAoQykpIl0sWzEsMSwiXFxjb3Byb2QgX3tpK2ogPSBwLTF9XFxtYXRocm17VG9yfV8xKEFeaSwgSF5qIChDKSkiXSxbMiwxLCJcXGNvcHJvZCBfe2kraiA9IHB9XFxtYXRocm17VG9yfV8xKEFeaSwgSF5qIChDKSkiXSxbMywxLCJcXGNvcHJvZCBfe2kraiA9IHArMX1cXG1hdGhybXtUb3J9XzEoQV5pLCBIXmogKEMpKSJdLFswLDgsIlxcYm94ZWR7RV8wIH0iXSxbNCw0LCJcXGJveGVke0VfMCB9Il0sWzQsMiwiXFxib3hlZHtFXzF9Il0sWzQsMCwiXFxib3hlZHtFXzJ9Il0sWzIsMF0sWzAsMV0sWzEyLDldLFsxMSw4XSxbMTgsMTZdLFsxNiwxNF0sWzE5LDE3XSxbMTcsMTVdLFsyMiwyNV0sWzIxLDI0XSxbMjAsMjNdLFsxMywxMF0sWzMyLDMsIiIsMix7ImxldmVsIjoyfV0sWzMsNCwiIiwyLHsibGV2ZWwiOjJ9XSxbMzMsMzQsIiIsMix7ImxldmVsIjoyfV0sWzM0LDM1LCIiLDIseyJsZXZlbCI6Mn1dXQ==
\begin{tikzcd}[ampersand replacement=\&,sep=tiny]
	\& {\coprod _{i+j = p-1}(A^i \otimes H^j (C))} \& {\coprod _{i+j = p}(A^i \otimes H^j (C))} \& {\coprod _{i+j = p+1}(A^i \otimes H^j (C))} \& {\boxed{E_2}} \\
	\& {\coprod _{i+j = p-1}\mathrm{Tor}_1(A^i, H^j (C))} \& {\coprod _{i+j = p}\mathrm{Tor}_1(A^i, H^j (C))} \& {\coprod _{i+j = p+1}\mathrm{Tor}_1(A^i, H^j (C))} \\
	\& {\coprod _{i+j = p-1}(F^{0, i}\otimes H^j (C))} \& {\coprod _{i+j = p}(F^{0, i}\otimes H^j (C))} \& {\coprod _{i+j = p+1}(F^{0, i}\otimes H^j (C))} \& {\boxed{E_1}} \\
	\& {\coprod _{i+j = p-1}(F^{-1, i}\otimes H^j (C))} \& {\coprod _{i+j = p}(F^{-1, i}\otimes H^j (C))} \& {\coprod _{i+j = p+1}(F^{-1, i}\otimes H^j (C))} \\
	\& {\coprod _{i+j = p-1}(F^{0,i} \otimes C^j )} \& {\coprod _{i+j = p}(F^{0,i} \otimes C^j )} \& {\coprod _{i+j = p+1}(F^{0,i} \otimes C^j )} \& {\boxed{E_0 }} \\
	\& {\coprod _{i+j = p-1}(F^{-1,i} \otimes C^j )} \& {\coprod _{i+j = p}(F^{-1,i} \otimes C^j )} \& {\coprod _{i+j = p+1}(F^{-1,i} \otimes C^j )} \\
	{\boxed{E_2}} \& {H^{p-1}(A \otimes C)} \& {H^{p}(A \otimes C)} \& {H^{p+1}(A \otimes C)} \\
	{\boxed{E_1}} \& {\coprod _{i+j = p-1}(A^i \otimes C^j )} \& {\coprod _{i+j = p}(A^i \otimes C^j )} \& {\coprod _{i+j = p+1}(A^i \otimes C^j )} \\
	{\boxed{E_0 }} \& {\coprod _{i+j = p-1}(F^{0,i} \otimes C^j )} \& {\coprod _{i+j = p}(F^{0,i} \otimes C^j )} \& {\coprod _{i+j = p+1}(F^{0,i} \otimes C^j )} \\
	\& {\coprod _{i+j = p-1}(F^{-1,i} \otimes C^j )} \& {\coprod _{i+j = p}(F^{-1,i} \otimes C^j )} \& {\coprod _{i+j = p+1}(F^{-1,i} \otimes C^j )}
	\arrow[Rightarrow, from=3-5, to=1-5]
	\arrow[from=4-2, to=3-2]
	\arrow[from=4-3, to=3-3]
	\arrow[from=4-4, to=3-4]
	\arrow[from=5-2, to=5-3]
	\arrow[from=5-3, to=5-4]
	\arrow[Rightarrow, from=5-5, to=3-5]
	\arrow[from=6-2, to=6-3]
	\arrow[from=6-3, to=6-4]
	\arrow[Rightarrow, from=8-1, to=7-1]
	\arrow[from=8-2, to=8-3]
	\arrow[from=8-3, to=8-4]
	\arrow[Rightarrow, from=9-1, to=8-1]
	\arrow[from=10-2, to=9-2]
	\arrow[from=10-3, to=9-3]
	\arrow[from=10-4, to=9-4]
\end{tikzcd}. 
    \end{equation}
    略去第二問的圖表, 證明框架如下. 
    \begin{enumerate}
        \item 若 $A$ 投射, 取 $C$ 的投射分解 $C → I$. 計算 $∐ _{i + j = p} ℋ (A^{-i}, I^{q, j})$ 的兩向的譜序列, 得
        \begin{equation}
            ∐ _{i + j = p} \mathrm{Ext}^q(H^{-i}(A), H^j(C)) ⇒ H^{p+q}(ℋ(A,C)). 
        \end{equation}
        \item 若 $C$ 內射, 取 $A$ 的投射分解 $P → A$. 計算 $∐ _{i + j = p} ℋ (P^{-q, -i}, C^{j})$ 的兩向的譜序列, 得
        \begin{equation}
            ∐ _{i + j = p} \mathrm{Ext}^q(H^{-i}(A), H^j(C)) ⇒ H^{p+q}(ℋ(A,C)). 
        \end{equation}
    \end{enumerate}
\end{proof}
\end{theorem}

\begin{remark}
    Kunneth 譜序列處在是第一或第三象限, 從而滿足有界性. 一般地, 若規定整體維度有限, 則相應的譜序列的支撐在一個長條內, 從而也滿足一些收斂性定理. 
\end{remark}

\begin{example}[Kunneth 公式]
    若 Kunneth 譜序列中的``高階導出函子''均消失, 則全復形同調群之濾過會無比簡單. 
    \begin{enumerate}
        \item 若 $\mathrm{Tor}_{≥ 2}(H(A), -)=0$ 或 $\mathrm{Tor}_{≥ 2}(-, H(C))=0$, 則 
        \begin{equation}
            0 → ∐ _{i+j = p}(H^i (A) ⊗ H^j (C)) → H^p (A ⊗ C) → ∐ _{i+j = p+1}\mathrm{Tor}_1(H^i (A) ⊗ H^j (C)) → 0
        \end{equation}
        \item 若 $\mathrm{Ext}^{≥ 2} (H^{-i} (A), -)=0$ 或 $\mathrm{Ext}^{≥ 2} (-, H^{-i} (C))=0$, 則 
        \begin{equation}
            0 → ∐ _{i+j = p+1}\mathrm{Ext}^1(H^{-i} (A) , H^j (C)) → H^p (A ⊗ C) → ∐ _{i+j = p}\mathrm{Hom}(H^{-i} (A) ⊗ H^j (C)) → 0
        \end{equation}
    \end{enumerate}
    若 $X$ 與 $\mathrm{im}(d)$ 均是投射/內射/平坦對象, 則 $\mathrm{Ext}^{≥ 2}(H,-)$/$\mathrm{Ext}^{≥ 2}(-, H)$/$\mathrm{Tor}_{≥ 2}(H,-)$ 消失. \parnote{何時可裂? }
\end{example}


\subsubsection{合成函子的譜序列}



\begin{theorem}[Grothendieck 譜序列]
    假定 $𝒜\xrightarrow F ℬ \xrightarrow G 𝒞$ 是 Abel 範疇間的左正合函子. 假定
    \begin{enumerate}
        \item $𝒜$ 有足夠投射對象, 即任意 $X ∈ A$ 存在投射分解; 
        \item 對投射對象 $P ∈ 𝒜$, 像 $F(P)$ 關於右導出 $R^{≥ 1} G$ 消失. 
    \end{enumerate}
此時存在收斂的譜序列:
    \begin{equation}
        E_2^{p,q} := R^pG (R^q F(X)) ⇒ (R^{p+q} (G ∘ F)) (X).  
    \end{equation}
\begin{proof}
    記 $X$ 的投射分解 $Q → X$ ($x$-負半軸), 繼而依馬蹄引理取 $F(Q)$ 的投射分解, 得行可裂雙復形 $P$. 示意圖如下: 
    \begin{equation}
        % https://q.uiver.app/#q=WzAsMTYsWzMsMCwiRlgiLFszMCw2MCw2MCwxXV0sWzIsMCwiRlFeMCIsWzE4MCw2MCw2MCwxXV0sWzEsMCwiRlFeey0xfSIsWzE4MCw2MCw2MCwxXV0sWzAsMCwiRlFeey0yfSIsWzE4MCw2MCw2MCwxXV0sWzAsMSwiUF57LTIsMH0iXSxbMiwxLCJQXnswLDB9Il0sWzEsMSwiUF57LTEsMH0iXSxbMiwyLCJQXnswLC0xfSJdLFsyLDMsIlBeezAsLTJ9Il0sWzEsMiwiUF57LTEsLTF9Il0sWzEsMywiUF57LTEsLTJ9Il0sWzAsMiwiUF57LTIsLTF9Il0sWzAsMywiUF57LTIsLTJ9Il0sWzMsMSwiUF57MSwwfSIsWzE4MCw2MCw2MCwxXV0sWzMsMiwiUF57MSwtMX0iLFsxODAsNjAsNjAsMV1dLFszLDMsIlBeezEsLTJ9IixbMTgwLDYwLDYwLDFdXSxbNCw2XSxbNiw1XSxbMTEsOV0sWzksN10sWzEyLDEwXSxbMTAsOF0sWzEyLDExXSxbMTEsNF0sWzEwLDldLFs5LDZdLFs4LDddLFs3LDVdLFszLDIsIiIsMSx7ImNvbG91ciI6WzE4MCw2MCw2MF19XSxbMiwxLCIiLDEseyJjb2xvdXIiOlsxODAsNjAsNjBdfV0sWzEsMCwiIiwxLHsiY29sb3VyIjpbMTgwLDYwLDYwXSwic3R5bGUiOnsiYm9keSI6eyJuYW1lIjoiZG90dGVkIn19fV0sWzE1LDE0LCIiLDEseyJjb2xvdXIiOlsxODAsNjAsNjBdfV0sWzE0LDEzLCIiLDEseyJjb2xvdXIiOlsxODAsNjAsNjBdfV0sWzEzLDAsIiIsMSx7ImNvbG91ciI6WzE4MCw2MCw2MF0sInN0eWxlIjp7ImJvZHkiOnsibmFtZSI6ImRvdHRlZCJ9fX1dLFs0LDMsIiIsMSx7InN0eWxlIjp7ImJvZHkiOnsibmFtZSI6ImRvdHRlZCJ9fX1dLFs2LDIsIiIsMSx7InN0eWxlIjp7ImJvZHkiOnsibmFtZSI6ImRvdHRlZCJ9fX1dLFs1LDEsIiIsMSx7InN0eWxlIjp7ImJvZHkiOnsibmFtZSI6ImRvdHRlZCJ9fX1dLFs3LDE0LCIiLDEseyJzdHlsZSI6eyJib2R5Ijp7Im5hbWUiOiJkb3R0ZWQifX19XSxbNSwxMywiIiwxLHsic3R5bGUiOnsiYm9keSI6eyJuYW1lIjoiZG90dGVkIn19fV0sWzgsMTUsIiIsMSx7InN0eWxlIjp7ImJvZHkiOnsibmFtZSI6ImRvdHRlZCJ9fX1dXQ==
\begin{tikzcd}[ampersand replacement=\&, sep = small]
	\textcolor{rgb,255:red,92;green,214;blue,214}{{FQ^{-2}}} \& \textcolor{rgb,255:red,92;green,214;blue,214}{{FQ^{-1}}} \& \textcolor{rgb,255:red,92;green,214;blue,214}{{FQ^0}} \& \textcolor{rgb,255:red,214;green,153;blue,92}{FX} \\
	{P^{-2,0}} \& {P^{-1,0}} \& {P^{0,0}} \& \textcolor{rgb,255:red,92;green,214;blue,214}{{P^{1,0}}} \\
	{P^{-2,-1}} \& {P^{-1,-1}} \& {P^{0,-1}} \& \textcolor{rgb,255:red,92;green,214;blue,214}{{P^{1,-1}}} \\
	{P^{-2,-2}} \& {P^{-1,-2}} \& {P^{0,-2}} \& \textcolor{rgb,255:red,92;green,214;blue,214}{{P^{1,-2}}}
	\arrow[color={rgb,255:red,92;green,214;blue,214}, from=1-1, to=1-2]
	\arrow[color={rgb,255:red,92;green,214;blue,214}, from=1-2, to=1-3]
	\arrow[color={rgb,255:red,92;green,214;blue,214}, dotted, from=1-3, to=1-4]
	\arrow[dotted, from=2-1, to=1-1]
	\arrow[from=2-1, to=2-2]
	\arrow[dotted, from=2-2, to=1-2]
	\arrow[from=2-2, to=2-3]
	\arrow[dotted, from=2-3, to=1-3]
	\arrow[dotted, from=2-3, to=2-4]
	\arrow[color={rgb,255:red,92;green,214;blue,214}, dotted, from=2-4, to=1-4]
	\arrow[from=3-1, to=2-1]
	\arrow[from=3-1, to=3-2]
	\arrow[from=3-2, to=2-2]
	\arrow[from=3-2, to=3-3]
	\arrow[from=3-3, to=2-3]
	\arrow[dotted, from=3-3, to=3-4]
	\arrow[color={rgb,255:red,92;green,214;blue,214}, from=3-4, to=2-4]
	\arrow[from=4-1, to=3-1]
	\arrow[from=4-1, to=4-2]
	\arrow[from=4-2, to=3-2]
	\arrow[from=4-2, to=4-3]
	\arrow[from=4-3, to=3-3]
	\arrow[dotted, from=4-3, to=4-4]
	\arrow[color={rgb,255:red,92;green,214;blue,214}, from=4-4, to=3-4]
\end{tikzcd}.
    \end{equation}
    繼而取 $G(P)$ 的雙向譜序列.
    \begin{enumerate}
        \item ($→$) 得 $E_2$ 如下: 
        \begin{equation}
            % https://q.uiver.app/#q=WzAsNDcsWzQsMCwiR0ZYIixbMzAsNjAsNjAsMV1dLFszLDAsIkdGUV4wIixbMTgwLDYwLDYwLDFdXSxbMiwwLCJHRlFeey0xfSIsWzE4MCw2MCw2MCwxXV0sWzEsMCwiR0ZRXnstMn0iLFsxODAsNjAsNjAsMV1dLFsxLDEsIkcoUF57LTIsMH0pIl0sWzMsMSwiRyhQXnswLDB9KSJdLFsyLDEsIkcoUF57LTEsMH0pIl0sWzMsMiwiRyhQXnswLC0xfSkiXSxbMywzLCJHKFBeezAsLTJ9KSJdLFsyLDIsIkcoUF57LTEsLTF9KSJdLFsyLDMsIkcoUF57LTEsLTJ9KSJdLFsxLDIsIkcoUF57LTIsLTF9KSJdLFsxLDMsIkcoUF57LTIsLTJ9KSJdLFs0LDEsIkcoUF57MSwwfSkiLFsxODAsNjAsNjAsMV1dLFs0LDIsIkcoUF57MSwtMX0pIixbMTgwLDYwLDYwLDFdXSxbNCwzLCJHKFBeezEsLTJ9KSIsWzE4MCw2MCw2MCwxXV0sWzAsMywiXFxib3hlZHtFXzB9Il0sWzAsNywiXFxib3hlZHtFXzF9Il0sWzQsNSwiRyhQXnsxLDB9KSIsWzE4MCw2MCw2MCwxXV0sWzQsNiwiRyhQXnsxLC0xfSkiLFsxODAsNjAsNjAsMV1dLFs0LDcsIkcoUF57MSwtMn0pIixbMTgwLDYwLDYwLDFdXSxbNCw0LCJHRlgiLFszMCw2MCw2MCwxXV0sWzMsNSwiRyhIX1xcdG8gXnswLDB9KSJdLFszLDYsIkcoSF9cXHRvIF57MCwtMX0pIl0sWzMsNywiRyhIX1xcdG8gXnswLC0yfSkiXSxbMiw1LCJHKEhfXFx0byBeey0xLDB9KSJdLFsyLDYsIkcoSF9cXHRvIF57LTEsLTF9KSJdLFsyLDcsIkcoSF9cXHRvIF57LTEsLTJ9KSJdLFsxLDUsIkcoSF9cXHRvIF57LTIsMH0pIl0sWzEsNiwiRyhIX1xcdG8gXnstMiwtMX0pIl0sWzEsNywiRyhIX1xcdG8gXnstMiwtMn0pIl0sWzMsOCwiRyhGWCkiXSxbMiw4LCJHKFJeMUZYKSJdLFsxLDgsIkcoUl4yRlgpIl0sWzMsOSwiKFJeMUcpKEZYKSJdLFszLDEwLCIoUl4yRykoRlgpIl0sWzIsOSwiKFJeMUcpKFJeMUZYKSJdLFsxLDksIihSXjFHKShSXjJGWCkiXSxbMiwxMCwiKFJeMkcpKFJeMUZYKSJdLFsxLDEwLCIoUl4yRykoUl4yRlgpIl0sWzAsMTAsIlxcYm94ZWR7RV8yfSJdLFs0LDEwLCIwIl0sWzQsOCwiMCJdLFs0LDksIjAiXSxbMyw0LCJHRlgiLFsxODAsNjAsNjAsMV1dLFsyLDQsIkcoUl4xIEZYKSIsWzE4MCw2MCw2MCwxXV0sWzEsNCwiRyhSXjIgRlgpIixbMTgwLDYwLDYwLDFdXSxbNCw2XSxbNiw1XSxbMTEsOV0sWzksN10sWzEyLDEwXSxbMTAsOF0sWzMsMiwiIiwxLHsiY29sb3VyIjpbMTgwLDYwLDYwXX1dLFsyLDEsIiIsMSx7ImNvbG91ciI6WzE4MCw2MCw2MF19XSxbMSwwLCIiLDEseyJjb2xvdXIiOlsxODAsNjAsNjBdLCJzdHlsZSI6eyJib2R5Ijp7Im5hbWUiOiJkb3R0ZWQifX19XSxbNywxNCwiIiwxLHsic3R5bGUiOnsiYm9keSI6eyJuYW1lIjoiZG90dGVkIn19fV0sWzUsMTMsIiIsMSx7InN0eWxlIjp7ImJvZHkiOnsibmFtZSI6ImRvdHRlZCJ9fX1dLFs4LDE1LCIiLDEseyJzdHlsZSI6eyJib2R5Ijp7Im5hbWUiOiJkb3R0ZWQifX19XSxbMjAsMTldLFsxOSwxOF0sWzI0LDIzXSxbMjMsMjJdLFsyNywyNl0sWzI2LDI1XSxbMzAsMjldLFsyOSwyOF0sWzIyLDE4LCIiLDEseyJsZXZlbCI6Miwic3R5bGUiOnsiYm9keSI6eyJuYW1lIjoiZG90dGVkIn0sImhlYWQiOnsibmFtZSI6Im5vbmUifX19XSxbMjMsMTksIiIsMSx7ImxldmVsIjoyLCJzdHlsZSI6eyJib2R5Ijp7Im5hbWUiOiJkb3R0ZWQifSwiaGVhZCI6eyJuYW1lIjoibm9uZSJ9fX1dLFsyNCwyMCwiIiwxLHsibGV2ZWwiOjIsInN0eWxlIjp7ImJvZHkiOnsibmFtZSI6ImRvdHRlZCJ9LCJoZWFkIjp7Im5hbWUiOiJub25lIn19fV0sWzE4LDIxLCIiLDEseyJzdHlsZSI6eyJib2R5Ijp7Im5hbWUiOiJkb3R0ZWQifX19XSxbMzUsMzJdLFszOCwzM10sWzQxLDMxXSxbMjgsNDYsIiIsMSx7InN0eWxlIjp7ImJvZHkiOnsibmFtZSI6ImRvdHRlZCJ9fX1dLFsyNSw0NSwiIiwxLHsic3R5bGUiOnsiYm9keSI6eyJuYW1lIjoiZG90dGVkIn19fV0sWzIyLDQ0LCIiLDEseyJzdHlsZSI6eyJib2R5Ijp7Im5hbWUiOiJkb3R0ZWQifX19XV0=
\begin{tikzcd}[ampersand replacement=\&, sep = tiny]
	\& \textcolor{rgb,255:red,92;green,214;blue,214}{{GFQ^{-2}}} \& \textcolor{rgb,255:red,92;green,214;blue,214}{{GFQ^{-1}}} \& \textcolor{rgb,255:red,92;green,214;blue,214}{{GFQ^0}} \& \textcolor{rgb,255:red,214;green,153;blue,92}{GFX} \\
	\& {G(P^{-2,0})} \& {G(P^{-1,0})} \& {G(P^{0,0})} \& \textcolor{rgb,255:red,92;green,214;blue,214}{{G(P^{1,0})}} \\
	\& {G(P^{-2,-1})} \& {G(P^{-1,-1})} \& {G(P^{0,-1})} \& \textcolor{rgb,255:red,92;green,214;blue,214}{{G(P^{1,-1})}} \\
	{\boxed{E_0}} \& {G(P^{-2,-2})} \& {G(P^{-1,-2})} \& {G(P^{0,-2})} \& \textcolor{rgb,255:red,92;green,214;blue,214}{{G(P^{1,-2})}} \\
	\& \textcolor{rgb,255:red,92;green,214;blue,214}{{G(R^2 FX)}} \& \textcolor{rgb,255:red,92;green,214;blue,214}{{G(R^1 FX)}} \& \textcolor{rgb,255:red,92;green,214;blue,214}{GFX} \& \textcolor{rgb,255:red,214;green,153;blue,92}{GFX} \\
	\& {G(H_\to ^{-2,0})} \& {G(H_\to ^{-1,0})} \& {G(H_\to ^{0,0})} \& \textcolor{rgb,255:red,92;green,214;blue,214}{{G(P^{1,0})}} \\
	\& {G(H_\to ^{-2,-1})} \& {G(H_\to ^{-1,-1})} \& {G(H_\to ^{0,-1})} \& \textcolor{rgb,255:red,92;green,214;blue,214}{{G(P^{1,-1})}} \\
	{\boxed{E_1}} \& {G(H_\to ^{-2,-2})} \& {G(H_\to ^{-1,-2})} \& {G(H_\to ^{0,-2})} \& \textcolor{rgb,255:red,92;green,214;blue,214}{{G(P^{1,-2})}} \\
	\& {G(R^2FX)} \& {G(R^1FX)} \& {G(FX)} \& 0 \\
	\& {(R^1G)(R^2FX)} \& {(R^1G)(R^1FX)} \& {(R^1G)(FX)} \& 0 \\
	{\boxed{E_2}} \& {(R^2G)(R^2FX)} \& {(R^2G)(R^1FX)} \& {(R^2G)(FX)} \& 0
	\arrow[color={rgb,255:red,92;green,214;blue,214}, from=1-2, to=1-3]
	\arrow[color={rgb,255:red,92;green,214;blue,214}, from=1-3, to=1-4]
	\arrow[color={rgb,255:red,92;green,214;blue,214}, dotted, from=1-4, to=1-5]
	\arrow[from=2-2, to=2-3]
	\arrow[from=2-3, to=2-4]
	\arrow[dotted, from=2-4, to=2-5]
	\arrow[from=3-2, to=3-3]
	\arrow[from=3-3, to=3-4]
	\arrow[dotted, from=3-4, to=3-5]
	\arrow[from=4-2, to=4-3]
	\arrow[from=4-3, to=4-4]
	\arrow[dotted, from=4-4, to=4-5]
	\arrow[dotted, from=6-2, to=5-2]
	\arrow[dotted, from=6-3, to=5-3]
	\arrow[dotted, from=6-4, to=5-4]
	\arrow[equals, dotted, from=6-4, to=6-5]
	\arrow[dotted, from=6-5, to=5-5]
	\arrow[from=7-2, to=6-2]
	\arrow[from=7-3, to=6-3]
	\arrow[from=7-4, to=6-4]
	\arrow[equals, dotted, from=7-4, to=7-5]
	\arrow[from=7-5, to=6-5]
	\arrow[from=8-2, to=7-2]
	\arrow[from=8-3, to=7-3]
	\arrow[from=8-4, to=7-4]
	\arrow[equals, dotted, from=8-4, to=8-5]
	\arrow[from=8-5, to=7-5]
	\arrow[from=11-3, to=9-2]
	\arrow[from=11-4, to=9-3]
	\arrow[from=11-5, to=9-4]
\end{tikzcd}.
        \end{equation}
        $E_0 ⇒ E_1$ 是由于 $P$ 横向可裂. $E_1 ⇒ E_2$ 是由於 $H^{p, ∙} → G(R^pFX)$ 是投射分解, 詳細而言 
        \begin{enumerate}
            \item 所有 $H^{p,q}$ 均是 $P$ 的直和項, 從而是投射對象; 
            \item 若干列投射分解 $[P ↑ Q]$ 取上同調所得的 $[H_→(P) ↑ H_→(Q)]$ 仍是若干列投射分解. \parnote{不用證明! \\ CE-分解自帶!}
        \end{enumerate}
        此時 $E_2$ 確實是圖中所述, $E_2^{p,q}=R^{-q}G(R^{-p}FX)$. 
        \item ($↑$) $E_2$ 的計算比較簡單: 
\begin{equation}
    % https://q.uiver.app/#q=WzAsMzAsWzQsMiwiRyhQXnsxLC0xfSkiLFsxODAsNjAsNjAsMV1dLFszLDMsIkcoUF57MCwtMn0pIl0sWzQsMywiRyhQXnsxLC0yfSkiLFsxODAsNjAsNjAsMV1dLFsyLDIsIkcoUF57LTEsLTF9KSJdLFszLDIsIkcoUF57MCwtMX0pIl0sWzIsMywiRyhQXnstMSwtMn0pIl0sWzEsMywiRyhQXnstMiwtMn0pIl0sWzEsMiwiRyhQXnstMiwtMX0pIl0sWzEsMSwiRyhQXnstMiwwfSkiXSxbMiwxLCJHKFBeey0xLDB9KSJdLFszLDEsIkcoUF57MCwwfSkiXSxbNCwxLCJHKFBeezEsMH0pIixbMTgwLDYwLDYwLDFdXSxbNCwwLCJHRlgiLFszMCw2MCw2MCwxXV0sWzMsMCwiR0ZRXjAiLFsxODAsNjAsNjAsMV1dLFsyLDAsIkdGUV57LTF9IixbMTgwLDYwLDYwLDFdXSxbMSwwLCJHRlFeey0yfSIsWzE4MCw2MCw2MCwxXV0sWzAsMywiXFxib3hlZHtFXzB9Il0sWzMsNCwiR0ZRXjAiXSxbMyw1LCIoUl4xRylGUV4wIixbMCwwLDc1LDFdXSxbMiw0LCJHRlFeMSJdLFsxLDQsIkdGUV4yIl0sWzIsNSwiKFJeMUcpRlFeMSIsWzAsMCw3NSwxXV0sWzEsNSwiKFJeMUcpRlFeMiIsWzAsMCw3NSwxXV0sWzAsNSwiXFxib3hlZHtFXzF9Il0sWzQsNSwiXFx0ZXh0e+a2iOWksX0iLFswLDAsNzUsMV1dLFs0LDQsIkdGWCIsWzMwLDYwLDYwLDFdXSxbMyw2LCIoR0YpWCJdLFsyLDYsIlJeMShHRilYIl0sWzEsNiwiUl4yKEdGKVgiXSxbMCw2LCJcXGJveGVke0VfMn0iXSxbMiwwLCIiLDEseyJjb2xvdXIiOlsxODAsNjAsNjBdfV0sWzEsNF0sWzUsM10sWzYsN10sWzAsMTEsIiIsMSx7ImNvbG91ciI6WzE4MCw2MCw2MF19XSxbNCwxMF0sWzMsOV0sWzcsOF0sWzExLDEyLCIiLDEseyJjb2xvdXIiOlsxODAsNjAsNjBdLCJzdHlsZSI6eyJib2R5Ijp7Im5hbWUiOiJkb3R0ZWQifX19XSxbMTAsMTMsIiIsMSx7InN0eWxlIjp7ImJvZHkiOnsibmFtZSI6ImRvdHRlZCJ9fX1dLFs5LDE0LCIiLDEseyJzdHlsZSI6eyJib2R5Ijp7Im5hbWUiOiJkb3R0ZWQifX19XSxbOCwxNSwiIiwxLHsic3R5bGUiOnsiYm9keSI6eyJuYW1lIjoiZG90dGVkIn19fV0sWzIwLDE5XSxbMTksMTddLFsyMiwyMSwiIiwwLHsiY29sb3VyIjpbMCwwLDc1XX1dLFsyMSwxOCwiIiwwLHsiY29sb3VyIjpbMCwwLDc1XX1dLFsxNywyNSwiIiwwLHsic3R5bGUiOnsiYm9keSI6eyJuYW1lIjoiZG90dGVkIn19fV1d
\begin{tikzcd}[ampersand replacement=\&, sep = small]
	\& \textcolor{rgb,255:red,92;green,214;blue,214}{{GFQ^{-2}}} \& \textcolor{rgb,255:red,92;green,214;blue,214}{{GFQ^{-1}}} \& \textcolor{rgb,255:red,92;green,214;blue,214}{{GFQ^0}} \& \textcolor{rgb,255:red,214;green,153;blue,92}{GFX} \\
	\& {G(P^{-2,0})} \& {G(P^{-1,0})} \& {G(P^{0,0})} \& \textcolor{rgb,255:red,92;green,214;blue,214}{{G(P^{1,0})}} \\
	\& {G(P^{-2,-1})} \& {G(P^{-1,-1})} \& {G(P^{0,-1})} \& \textcolor{rgb,255:red,92;green,214;blue,214}{{G(P^{1,-1})}} \\
	{\boxed{E_0}} \& {G(P^{-2,-2})} \& {G(P^{-1,-2})} \& {G(P^{0,-2})} \& \textcolor{rgb,255:red,92;green,214;blue,214}{{G(P^{1,-2})}} \\
	\& {GFQ^2} \& {GFQ^1} \& {GFQ^0} \& \textcolor{rgb,255:red,214;green,153;blue,92}{GFX} \\
	{\boxed{E_1}} \& \textcolor{rgb,255:red,191;green,191;blue,191}{{(R^1G)FQ^2}} \& \textcolor{rgb,255:red,191;green,191;blue,191}{{(R^1G)FQ^1}} \& \textcolor{rgb,255:red,191;green,191;blue,191}{{(R^1G)FQ^0}} \& \textcolor{rgb,255:red,191;green,191;blue,191}{{\text{消失}}} \\
	{\boxed{E_2}} \& {R^2(GF)X} \& {R^1(GF)X} \& {(GF)X}
	\arrow[dotted, from=2-2, to=1-2]
	\arrow[dotted, from=2-3, to=1-3]
	\arrow[dotted, from=2-4, to=1-4]
	\arrow[color={rgb,255:red,92;green,214;blue,214}, dotted, from=2-5, to=1-5]
	\arrow[from=3-2, to=2-2]
	\arrow[from=3-3, to=2-3]
	\arrow[from=3-4, to=2-4]
	\arrow[color={rgb,255:red,92;green,214;blue,214}, from=3-5, to=2-5]
	\arrow[from=4-2, to=3-2]
	\arrow[from=4-3, to=3-3]
	\arrow[from=4-4, to=3-4]
	\arrow[color={rgb,255:red,92;green,214;blue,214}, from=4-5, to=3-5]
	\arrow[from=5-2, to=5-3]
	\arrow[from=5-3, to=5-4]
	\arrow[dotted, from=5-4, to=5-5]
	\arrow[color={rgb,255:red,191;green,191;blue,191}, from=6-2, to=6-3]
	\arrow[color={rgb,255:red,191;green,191;blue,191}, from=6-3, to=6-4]
\end{tikzcd}.
\end{equation}
        此處依照假定, $(R^{≥}G)(FQ^0)$ 消失. $E_2 = E_∞$ 穩定.  
    \end{enumerate}
\end{proof}
\end{theorem}

\begin{remark}
    若觀察 Hom 與 $⊗$-函子, 則``奇怪的假定''是很自然的. 
\end{remark}

\begin{example}[前幾項]
    考慮下圖: 
    \begin{equation}
        % https://q.uiver.app/#q=WzAsMTQsWzIsNCwiRyhGWCkiXSxbMiwyLCJHKFJeMUZYKSJdLFsyLDAsIkcoUl4yRlgpIl0sWzMsNCwiKFJeMUcpKEZYKSJdLFs0LDQsIihSXjJHKShGWCkiXSxbMywyLCIoUl4xRykoUl4xRlgpIl0sWzMsMCwiKFJeMUcpKFJeMkZYKSJdLFs0LDIsIihSXjJHKShSXjFGWCkiXSxbNCwwLCIoUl4yRykoUl4yRlgpIl0sWzUsNCwiKFJeM0cpKEZYKSJdLFs1LDIsIihSXjNHKShSXjFGWCkiXSxbMSwyLCIwIl0sWzEsMCwiMCJdLFswLDIsIjAiXSxbNCwxXSxbNywyXSxbOSw1XSxbMTAsNl0sWzMsMTFdLFs1LDEyXSxbMTEsMCwiR0ZYIiwxLHsic3R5bGUiOnsiYm9keSI6eyJuYW1lIjoiZGFzaGVkIn19fV0sWzAsMTNdLFsxLDMsIlJeMShHRilYIiwxLHsic3R5bGUiOnsiYm9keSI6eyJuYW1lIjoiZGFzaGVkIn19fV0sWzUsNCwiUl4yKEdGKVgiLDEseyJzdHlsZSI6eyJib2R5Ijp7Im5hbWUiOiJkYXNoZWQifX19XV0=
\begin{tikzcd}[ampersand replacement=\&, sep = small ]
	\& 0 \& {G(R^2FX)} \& {(R^1G)(R^2FX)} \& {(R^2G)(R^2FX)} \\
	\\
	0 \& 0 \& {G(R^1FX)} \& {(R^1G)(R^1FX)} \& {(R^2G)(R^1FX)} \& {(R^3G)(R^1FX)} \\
	\\
	\&\& {G(FX)} \& {(R^1G)(FX)} \& {(R^2G)(FX)} \& {(R^3G)(FX)}
	\arrow["GFX"{description}, dashed, from=3-2, to=5-3]
	\arrow["{R^1(GF)X}"{description}, dashed, from=3-3, to=5-4]
	\arrow[from=3-4, to=1-2]
	\arrow["{R^2(GF)X}"{description}, dashed, from=3-4, to=5-5]
	\arrow[from=3-5, to=1-3]
	\arrow[from=3-6, to=1-4]
	\arrow[from=5-3, to=3-1]
	\arrow[from=5-4, to=3-2]
	\arrow[from=5-5, to=3-3]
	\arrow[from=5-6, to=3-4]
\end{tikzcd}.
    \end{equation}
    此時有五項正合列 
    \begin{equation}
        R^2(GF)X → (R^2G)(FX) → G(R^1FX) → R^1(GF)X → (R^1G)(FX) → 0. 
    \end{equation}
\begin{enumerate}
    \item 若進一步要求 $R^{≥ 2}FX = 0$, 則可以進一步左接``三週期''長正合列 
    \begin{equation}
        \cdots → (R^kG)(R^1FX) → R^{k+1}(GF)X → (R^{k+1}G)(FX) → \cdots  
    \end{equation}
    簡單地寫作 $(GF)^2 → G^2F → GF^1 → (GF)^1 → G^1F → 0$. \parnote{縱有界}
    \item 若進一步要求 $R^{≥ 2}G = 0$, 則有短正合列 (復合函子求導法則)\parnote{橫有界}
    \begin{equation}
        0 → GF^{k+1} → (GF)^{k+1} → G^1 F^k → 0. 
    \end{equation}
\end{enumerate}
\end{example}

\begin{remark}
    類似地, 左導出函子適合 $0 → F^1R → (FR)^1 → FR^1 → F^2R → (FR)^2$
\end{remark}

\begin{example}[函子符合求導: 雙模結構]
    假定 $M$ 是 $(A,B)$-雙模, $N$ 是 $(B,C)$-雙模, 則有右正合函子 
    \begin{equation}
        𝐦𝐨𝐝_A \xrightarrow{- ⊗ M} 𝐦𝐨𝐝_B \xrightarrow{- ⊗ N} 𝐦𝐨𝐝_C. 
    \end{equation}
    此時 $\mathrm{Tor}^{-q}_B(\mathrm{Tor}^{-p}_A(-, M), N) ⇒ \mathrm{Tor}_A^{p+q} (-, M ⊗ N)$. 左導出的前五項
    \begin{align}
        0 → \mathrm{Tor}_A^1(-, M) ⊗_B N → \mathrm{Tor}_A^1(-, M ⊗_B N) → \mathrm{Tor}_B^1(- ⊗_A M, N) → \\ 
        → \mathrm{Tor}_A^2(-, M) ⊗_B N → \mathrm{Tor}_A^2(-, M ⊗_B N). 
    \end{align}
    此時有一些特例可探索. 
    \begin{enumerate}
        \item $M$ 作爲 $A$-模, 其平坦維數 $≤ 1$. 此時有三週期長正合列 (略). 
        \item 若 $M = B$, 其左 $A$-模結構由環同態 $A → B$ 實現, 則又有一些可玩的 (例如整體維數 $≤ 1$, 或更直接的). \parnote{faithful flat?}
        \item 依照拓撲學習慣, 時常引入 PID 環. 此時得各種萬有係數定理. 
    \end{enumerate}
    另有 $(X, -) \& (Y, -)$, 以及 $(-, X) \& (- ⊗ Y)$ 兩種推廣. 
\end{example}

\begin{example}
    群的 MacLane 四項正合列. \parnote{補充?}
\end{example}

\begin{example}[Grothendieck 譜序列的自然性]
	譜序列 (分次復形定義) 誘導的態射是自然的. 如何刻畫 Grothendieck 譜序列的前五項是一個問題. 例如, 給定內射分解誘導的 
	\begin{equation}
		0 → (R^1G)FX → R^1(GF)X → G(R^1F)X → (R^2G)FX → R^2(GF)X . 
	\end{equation}
	\begin{enumerate}
		\item 態射 $(R^pG)FX → R^p(GF)X$ 由投射分解誘導的復形態射 $F[X → I] ⇒ [FX → J]$ 給出: 
		\begin{equation}
			% https://q.uiver.app/#q=WzAsMTYsWzAsMSwiRlgiXSxbMSwxLCJKXjAgIl0sWzIsMSwiSl4xIl0sWzAsMCwiRlgiXSxbMSwwLCJGSV4wICJdLFsyLDAsIkZJXjEiXSxbMywwLCJcXGNkb3RzICJdLFszLDEsIlxcY2RvdHMgIl0sWzQsMSwiSl5wIl0sWzQsMCwiRklecCAiXSxbNSwwLCJGSV5cXGJ1bGxldCAiXSxbNSwxLCJKXlxcYnVsbGV0ICJdLFs2LDAsIkdGIEleXFxidWxsZXQgIl0sWzYsMSwiR0peXFxidWxsZXQgIl0sWzcsMCwiSF5wIChHRkleXFxidWxsZXQgKSJdLFs3LDEsIkhecChHSl5cXGJ1bGxldCkiXSxbMCwxLCIiLDAseyJzdHlsZSI6eyJib2R5Ijp7Im5hbWUiOiJkb3R0ZWQifX19XSxbMSwyXSxbMyw0LCIiLDAseyJzdHlsZSI6eyJib2R5Ijp7Im5hbWUiOiJkb3R0ZWQifX19XSxbNCw1XSxbMywwLCIiLDEseyJsZXZlbCI6Miwic3R5bGUiOnsiaGVhZCI6eyJuYW1lIjoibm9uZSJ9fX1dLFs0LDEsIiIsMSx7InN0eWxlIjp7ImJvZHkiOnsibmFtZSI6ImRhc2hlZCJ9fX1dLFs1LDIsIiIsMSx7InN0eWxlIjp7ImJvZHkiOnsibmFtZSI6ImRhc2hlZCJ9fX1dLFs1LDZdLFsyLDddLFs2LDldLFs5LDgsIiIsMSx7InN0eWxlIjp7ImJvZHkiOnsibmFtZSI6ImRhc2hlZCJ9fX1dLFs3LDhdLFsxMCwxMSwiXFx0aGV0YSAiLDIseyJzdHlsZSI6eyJib2R5Ijp7Im5hbWUiOiJkYXNoZWQifX19XSxbMTIsMTMsIkcoXFx0aGV0YSkiXSxbMTQsMTUsIkhecCAoRyhcXHRoZXRhKSkiXV0=
\begin{tikzcd}[ampersand replacement=\&, sep = small]
	FX \& {FI^0 } \& {FI^1} \& {\cdots } \& {FI^p } \& {FI^\bullet } \& {GF I^\bullet } \& {H^p (GFI^\bullet )} \\
	FX \& {J^0 } \& {J^1} \& {\cdots } \& {J^p} \& {J^\bullet } \& {GJ^\bullet } \& {H^p(GJ^\bullet)}
	\arrow[dotted, from=1-1, to=1-2]
	\arrow[equals, from=1-1, to=2-1]
	\arrow[from=1-2, to=1-3]
	\arrow[dashed, from=1-2, to=2-2]
	\arrow[from=1-3, to=1-4]
	\arrow[dashed, from=1-3, to=2-3]
	\arrow[from=1-4, to=1-5]
	\arrow[dashed, from=1-5, to=2-5]
	\arrow["{\theta }"', dashed, from=1-6, to=2-6]
	\arrow["{G(\theta)}", from=1-7, to=2-7]
	\arrow["{H^p (G(\theta))}", from=1-8, to=2-8]
	\arrow[dotted, from=2-1, to=2-2]
	\arrow[from=2-2, to=2-3]
	\arrow[from=2-3, to=2-4]
	\arrow[from=2-4, to=2-5]
\end{tikzcd}.
		\end{equation}
		此處 $H^p (G(θ)) : (R^pG)FX → R^p(GF)X$. 
		\item 態射 $R^p(GF) → G(R^p F)$ 由 Kan 延拓的泛性質給出: 
		\begin{equation}
			% https://q.uiver.app/#q=WzAsNSxbNCwwLCJcXG1hdGhzY3IgQyJdLFs0LDEsIkRcXG1hdGhzY3IgQyJdLFsyLDAsIlxcbWF0aHNjciBCIl0sWzIsMSwiRCBcXG1hdGhzY3IgQiJdLFswLDAsIlxcbWF0aHNjciBBIl0sWzAsMV0sWzIsM10sWzIsMCwiRyJdLFszLDEsIkRHIiwyXSxbNCwyLCJGIl0sWzQsMywiUl5cXGJ1bGxldCBGIiwyXSxbNCwxLCIiLDAseyJzdHlsZSI6eyJib2R5Ijp7Im5hbWUiOiJkb3R0ZWQifX19XV0=
\begin{tikzcd}[ampersand replacement=\&, sep = small]
	{\mathscr A} \&\& {\mathscr B} \&\& {\mathscr C} \\
	\&\& {D \mathscr B} \&\& {D\mathscr C}
	\arrow["F", from=1-1, to=1-3]
	\arrow["{R^\bullet F}"', from=1-1, to=2-3]
	\arrow[dotted, from=1-1, to=2-5]
	\arrow["G", from=1-3, to=1-5]
	\arrow[from=1-3, to=2-3]
	\arrow[from=1-5, to=2-5]
	\arrow["DG"', from=2-3, to=2-5]
\end{tikzcd}.
		\end{equation}
		特別地, 自然變換 $α ∘ (GF)^∙ ⇒ DG ∘ (R^∙ F)$ 給出 $H^p(α_X) : (GF)^p X→ G(F^p X)$. 
		\item 態射 $G(R^1 F)X → (R^2 G)(FX)$ 由內射分解 $FI^∙ → J^∙$ 作用 $G$ 後給出. 特別地, 
		\begin{equation}
			% https://q.uiver.app/#q=WzAsMTMsWzUsMCwiRyhSXjEgRilYIl0sWzUsMywiKFJeMiBHKShGWCkiXSxbMCwyLCJHRlgiXSxbMiwyLCJHSl4xIl0sWzMsMiwiR0peMiJdLFsxLDIsIkdKXjAiXSxbNCwyLCJHSl4zIl0sWzMsMywiSF4yIChHSl5cXGJ1bGxldCkiXSxbMywwLCJHSF4xKEZJXlxcYnVsbGV0ICkiXSxbMywxLCJHRkleMSJdLFsyLDEsIkdGSV4wIl0sWzQsMSwiR0ZJXjIiXSxbMSwxLCJHRlgiXSxbMCwxLCJIXjEoXFxzaWdtYSApIiwwLHsic3R5bGUiOnsiYm9keSI6eyJuYW1lIjoiZGFzaGVkIn19fV0sWzUsM10sWzQsNl0sWzMsNF0sWzEwLDldLFs5LDExXSxbMiw1XSxbMTIsMTBdLFs5LDQsIlxcc2lnbWFeMSIsMCx7InN0eWxlIjp7ImJvZHkiOnsibmFtZSI6ImRhc2hlZCJ9fX1dLFsxMiw1XSxbMTAsMywiXFxzaWdtYSBeMCIsMCx7InN0eWxlIjp7ImJvZHkiOnsibmFtZSI6ImRhc2hlZCJ9fX1dLFs4LDAsIiIsMix7ImxldmVsIjoyLCJzdHlsZSI6eyJoZWFkIjp7Im5hbWUiOiJub25lIn19fV0sWzcsMSwiIiwwLHsibGV2ZWwiOjIsInN0eWxlIjp7ImhlYWQiOnsibmFtZSI6Im5vbmUifX19XV0=
\begin{tikzcd}[ampersand replacement=\&,sep=small]
	\&\&\& {GH^1(FI^\bullet )} \&\& {G(R^1 F)X} \\
	\& GFX \& {GFI^0} \& {GFI^1} \& {GFI^2} \\
	GFX \& {GJ^0} \& {GJ^1} \& {GJ^2} \& {GJ^3} \\
	\&\&\& {H^2 (GJ^\bullet)} \&\& {(R^2 G)(FX)}
	\arrow[equals, from=1-4, to=1-6]
	\arrow["{H^1(\sigma )}", dashed, from=1-6, to=4-6]
	\arrow[from=2-2, to=2-3]
	\arrow[from=2-2, to=3-2]
	\arrow[from=2-3, to=2-4]
	\arrow["{\sigma ^0}", dashed, from=2-3, to=3-3]
	\arrow[from=2-4, to=2-5]
	\arrow["{\sigma^1}", dashed, from=2-4, to=3-4]
	\arrow[from=3-1, to=3-2]
	\arrow[from=3-2, to=3-3]
	\arrow[from=3-3, to=3-4]
	\arrow[from=3-4, to=3-5]
	\arrow[equals, from=4-4, to=4-6]
\end{tikzcd}.
		\end{equation}
	\end{enumerate}
\end{example}



\subsubsection{譜序列的構造 III: ``纖維化塔 ''(正合耦) 直接誘導譜序列}

\begin{abstract}
	Given a tower of fibrations of homotopy types, its degreewise homotopy groups naturally form an exact couple. The induced spectral sequence is the spectral sequence of the tower. 觀點來自 \cite{nlab:spectral_sequence_of_a_tower_of_fibrations}.  

	主要是爲了 Adams 濾過系統服務. 暫時不大常用. 
\end{abstract}

\begin{example}[正合耦]
    給定短正合列 $0 → X \xrightarrow f X \xrightarrow g Y → 0$, 則有``基本定理''導出的長正合列. 
    \begin{equation}
        % https://q.uiver.app/#q=WzAsMTQsWzEsMiwiSF57bi0xfShYKSJdLFsxLDAsIkhee24tMX0oWCkiXSxbMywxLCJIXntuLTF9KFkpIl0sWzIsMiwiSF57bn0oWCkiXSxbMiwwLCJIXntufShYKSJdLFs0LDEsIkhee259KFkpIl0sWzMsMiwiSF57bisxfShYKSJdLFszLDAsIkhee24rMX0oWCkiXSxbNSwxLCJcXGNkb3RzICJdLFsyLDEsIkhee24tMn0oWSkiXSxbMCwwLCJcXGNkb3RzICJdLFs2LDAsIkgoWCkiXSxbNiwyLCJIKFgpIl0sWzcsMSwiSChZKSJdLFswLDFdLFsxLDJdLFsyLDNdLFszLDRdLFs0LDVdLFs1LDZdLFs2LDddLFs3LDhdLFs5LDBdLFsxMCw5XSxbMTIsMTEsImYiXSxbMTEsMTMsImciXSxbMTMsMTIsIlxcZGVsdGEgIl1d
\begin{tikzcd}[ampersand replacement=\&, sep = small]
	{\cdots } \& {H^{n-1}(X)} \& {H^{n}(X)} \& {H^{n+1}(X)} \&\&\& {H(X)} \\
	\&\& {H^{n-2}(Y)} \& {H^{n-1}(Y)} \& {H^{n}(Y)} \& {\cdots } \&\& {H(Y)} \\
	\& {H^{n-1}(X)} \& {H^{n}(X)} \& {H^{n+1}(X)} \&\&\& {H(X)}
	\arrow[from=1-1, to=2-3]
	\arrow[from=1-2, to=2-4]
	\arrow[from=1-3, to=2-5]
	\arrow[from=1-4, to=2-6]
	\arrow["g", from=1-7, to=2-8]
	\arrow[from=2-3, to=3-2]
	\arrow[from=2-4, to=3-3]
	\arrow[from=2-5, to=3-4]
	\arrow["{\delta }", from=2-8, to=3-7]
	\arrow[from=3-2, to=1-2]
	\arrow[from=3-3, to=1-3]
	\arrow[from=3-4, to=1-4]
	\arrow["f", from=3-7, to=1-7]
\end{tikzcd}.
    \end{equation}
    如果用``微分分次模''一筆帶過, 則得到一個 $3$-circle 態射鏈. 
\end{example}

\begin{definition}
    給定給定 (分次) 模 $(D,E)$ 與態射 $(i,j,k)$. 稱 $(D,E,i,j,k)$ 是正合耦, 當且僅當 
    \begin{equation}
        % https://q.uiver.app/#q=WzAsMyxbMCwwLCJEIl0sWzIsMCwiRCJdLFsxLDEsIkUiXSxbMCwxLCJpIl0sWzEsMiwiaiJdLFsyLDAsImsiXV0=
\begin{tikzcd}[ampersand replacement=\&, sep= small]
	D \&\& D \\
	\& E
	\arrow["i", from=1-1, to=1-3]
	\arrow["j", from=1-3, to=2-2]
	\arrow["k", from=2-2, to=1-1]
\end{tikzcd}, 
    \end{equation}
    其中三處 $\mathrm{im} = \ker$. \parnote{用 $D$ 濾過 $E$}
\end{definition}

\begin{example}[導出正合耦]
    大正合耦 $(D,E, i,j,k)$ 給出外微分 $(j ∘ k) : E → E$. 此時導出外側小正合耦: 
    \begin{equation}
% https://q.uiver.app/#q=WzAsNixbMiwxLCJEIl0sWzQsMSwiRCJdLFszLDIsIkUiXSxbMCwwLCJcXG1hdGhybXtpbX0oaSkiXSxbNiwwLCJcXG1hdGhybXtpbX0oaSkiXSxbMywzLCJIKEUsIGsgXFxjaXJjIGopIl0sWzAsMSwiaSJdLFsxLDIsImoiXSxbMiwwLCJrIl0sWzMsNCwiZCBcXG1hcHN0byBpKGQpIl0sWzMsMCwiIiwwLHsic3R5bGUiOnsidGFpbCI6eyJuYW1lIjoiaG9vayIsInNpZGUiOiJib3R0b20ifX19XSxbNCwxLCIiLDAseyJzdHlsZSI6eyJ0YWlsIjp7Im5hbWUiOiJob29rIiwic2lkZSI6ImJvdHRvbSJ9fX1dLFs0LDUsImkoZCkgXFxtYXBzdG8gaihkKSJdLFs1LDMsImUgXFxtYXBzdG8gayhlKSJdXQ==
\begin{tikzcd}[ampersand replacement=\&, sep = small]
	{\mathrm{im}(i)} \&\&\&\&\&\& {\mathrm{im}(i)} \\
	\&\& D \&\& D \\
	\&\&\& E \\
	\&\&\& {H(E, k \circ j)}
	\arrow["{d \mapsto i(d)}", from=1-1, to=1-7]
	\arrow[hook', from=1-1, to=2-3]
	\arrow[hook', from=1-7, to=2-5]
	\arrow["{i(d) \mapsto j(d)}", from=1-7, to=4-4]
	\arrow["i", from=2-3, to=2-5]
	\arrow["j", from=2-5, to=3-4]
	\arrow["k", from=3-4, to=2-3]
	\arrow["{e \mapsto k(e)}", from=4-4, to=1-1]
\end{tikzcd}.
    \end{equation}
    \begin{enumerate}
        \item (良定義) 對象 $D' = \mathrm{im}(i)$, $E' = H((j ∘ k): E → E)$, 態射 $k'$ 良定義, 因爲 $k : (j ∘ k)(e) ↦ 0$, 態射 $j'$ 良定義, 因為 $j|_{\mathrm{im}(i)}=0$. 
        \item (左 $D'$ 處正合性) 核 $\mathrm{im}(i) ∩ \ker (i)$, 像 $k(\ker (k ∘ j))= \ker (j) ∩ \mathrm{im}(k) = \mathrm{im}(i) ∩ \ker (i)$.  
        \item (右 $D'$ 處正合性) 核 $\{i(d) ∣ j(d) ∈ \mathrm{im}(j ∘ k)\} = \frac{\{x ∣ j(x) ∈ \mathrm{im}(j ∘ k)\}}{\ker (i)} ≃ \frac{j^{-1}(\mathrm{im}(j ∘ k))}{\ker (i)} = \frac{\mathrm{im}(k) + \ker (j)}{\mathrm{ker}(i)}$ (第一同構定理), 繼而 $\frac{\mathrm{im}(k) + \ker (j)}{\mathrm{ker}(i)} = \frac{\ker (i) + \mathrm{im}(i)}{\ker(i)} ≃ i(\mathrm{im}(i))$ (第一同構定理). 故核等於像. 
        \item ($E$ 處正合性) 核 $\frac{\ker (k) ∩ \ker (j ∘ k)}{\mathrm{im}(j ∘ k)} ≃ \frac{\ker (k)}{\mathrm{im}(j ∘ k)} = \frac{\mathrm{im}(j)}{\mathrm{im}(j ∘ k)}$, 像 $\mathrm{im}(j')=\frac{\mathrm{im}(j)}{\mathrm{im}(j ∘ k)}$. 
    \end{enumerate}
\end{example}

\begin{definition}[$n$-次導出]
    給定正合耦 $ℰ = (D,E,i,j,k)$, 記 $()^1 = ()'$. 歸納地給出 $ℰ ^n$. 稱正合耦是冪零的, 若 $i :D → D$ 冪零. 
\end{definition}

\begin{theorem}[正合耦誘導譜序列]
    給定微分分次雙模 $(D,E)$, 若存在正合耦 $(D,E, i,j,k)$, 其態射次數分別是 
    \begin{equation}
        % https://q.uiver.app/#q=WzAsMyxbMCwwLCJEIl0sWzIsMCwiRCJdLFsxLDEsIkUiXSxbMCwxLCJpKC0xLDEpIl0sWzEsMiwiaigwLDApIl0sWzIsMCwiaygxLDApIl1d
\begin{tikzcd}[ampersand replacement=\&, sep = small]
	D \&\& D \\
	\& E
	\arrow["{i(-1,1)}", from=1-1, to=1-3]
	\arrow["{j(0,0)}", from=1-3, to=2-2]
	\arrow["{k(1,0)}", from=2-2, to=1-1]
\end{tikzcd},
    \end{equation}
則 $(E, (j ∘ k))$ 是譜序列. 
\begin{equation}
    % https://q.uiver.app/#q=WzAsMTAsWzAsMiwiRF57cCxxfSJdLFsyLDIsIkRee3ArMSxxfSJdLFs0LDIsIkRee3ArMixxfSJdLFswLDAsIkRee3AscSsxfSJdLFsyLDAsIkRee3ArMSxxKzF9Il0sWzQsMCwiRF57cCsyLHErMX0iXSxbMSwzLCJFXntwLHF9Il0sWzEsMSwiRV57cCxxKzF9Il0sWzMsMSwiRV57cCsxLHErMX0iXSxbMywzLCJFXntwKzEscSsxfSJdLFswLDFdLFsxLDJdLFswLDNdLFsxLDMsImkiLDIseyJjdXJ2ZSI6LTUsInN0eWxlIjp7ImJvZHkiOnsibmFtZSI6ImRhc2hlZCJ9fX1dLFsyLDQsImkiLDIseyJjdXJ2ZSI6LTUsInN0eWxlIjp7ImJvZHkiOnsibmFtZSI6ImRhc2hlZCJ9fX1dLFszLDRdLFs0LDVdLFsxLDRdLFsyLDVdLFswLDYsImoiLDJdLFs2LDEsImsiLDJdLFszLDcsImoiLDJdLFs3LDQsImsiLDJdLFs0LDgsImoiLDJdLFs4LDUsImsiLDJdLFsxLDksImoiLDJdLFs5LDIsImsiLDJdXQ==
\begin{tikzcd}[ampersand replacement=\&, sep = small]
	{D^{p,q+1}} \&\& {D^{p+1,q+1}} \&\& {D^{p+2,q+1}} \\
	\& {E^{p,q+1}} \&\& {E^{p+1,q+1}} \\
	{D^{p,q}} \&\& {D^{p+1,q}} \&\& {D^{p+2,q}} \\
	\& {E^{p,q}} \&\& {E^{p+1,q+1}}
	\arrow[from=1-1, to=1-3]
	\arrow["j"', from=1-1, to=2-2]
	\arrow[from=1-3, to=1-5]
	\arrow["j"', from=1-3, to=2-4]
	\arrow["k"', from=2-2, to=1-3]
	\arrow["k"', from=2-4, to=1-5]
	\arrow[from=3-1, to=1-1]
	\arrow[from=3-1, to=3-3]
	\arrow["j"', from=3-1, to=4-2]
	\arrow["i"', curve={height=-30pt}, dashed, from=3-3, to=1-1]
	\arrow[from=3-3, to=1-3]
	\arrow[from=3-3, to=3-5]
	\arrow["j"', from=3-3, to=4-4]
	\arrow["i"', curve={height=-30pt}, dashed, from=3-5, to=1-3]
	\arrow[from=3-5, to=1-5]
	\arrow["k"', from=4-2, to=3-3]
	\arrow["k"', from=4-4, to=3-5]
\end{tikzcd}.
\end{equation}
沿 $↑$ 方向投影, 大致得 $% https://q.uiver.app/#q=WzAsNSxbMCwwLCJEXntwLFxcYnVsbGV0fSJdLFsyLDAsIkRee3ArMSxcXGJ1bGxldH0iXSxbNCwwLCJEXntwKzIsXFxidWxsZXR9Il0sWzEsMSwiRV57cCxcXGJ1bGxldH0iXSxbMywxLCJFXntwKzEsXFxidWxsZXR9Il0sWzEsMCwiaSIsMix7InN0eWxlIjp7ImJvZHkiOnsibmFtZSI6ImRhc2hlZCJ9fX1dLFsyLDEsImkiLDIseyJzdHlsZSI6eyJib2R5Ijp7Im5hbWUiOiJkYXNoZWQifX19XSxbMCwzLCJqIiwyXSxbMywxLCJrIiwyXSxbMSw0LCJqIiwyXSxbNCwyLCJrIiwyXV0=
\begin{tikzcd}[ampersand replacement=\&,sep=small]
	{D^{p,\bullet}} \&\& {D^{p+1,\bullet}} \&\& {D^{p+2,\bullet}} \\
	\& {E^{p,\bullet}} \&\& {E^{p+1,\bullet}}
	\arrow["j"', from=1-1, to=2-2]
	\arrow["i"', dashed, from=1-3, to=1-1]
	\arrow["j"', from=1-3, to=2-4]
	\arrow["i"', dashed, from=1-5, to=1-3]
	\arrow["k"', from=2-2, to=1-3]
	\arrow["k"', from=2-4, to=1-5]
\end{tikzcd}$. 
\begin{proof}
    暫時從略, 看著容易接受. 
\end{proof}
\end{theorem}

\begin{definition}[有足夠的投射類的三角範疇]
    這是相對同調代數 (\cite{eilenberg1965foundations}) 的三角範疇版本. 稱 $(𝒯 ,𝒫 , 𝒩 )$ 是有足夠投射對象的三角範疇, 當且僅當 
    \begin{enumerate}
        \item (資料) $𝒯$ 是三角範疇, $𝒫$ 是對象類, $𝒩$ 是態射類; 
        \item (三角封閉) $𝒯$ 與 $𝒫$ 關於三角雙向平移 $[±]$-封閉; 
        \item (垂直關係) 關於 $\mathrm{Hom}_𝒯(-,-)$, 恰有 $𝒫^⟂ = 𝒩$ 與 $𝒫 = {^⟂}𝒩$; 
        \begin{itemize}
            \item $𝒫$-消失態射恰是 $𝒩$; $𝒩$-投射對象恰是 $𝒫$. 
        \end{itemize}
        \item (足夠投射) 任意 $X$ 可嵌入好三角 $P → X \xrightarrow i Y$, 其中 $P ∈ 𝒫$ 且 $i ∈ 𝒩$. 
    \end{enumerate}
\end{definition}

\begin{remark}
    以上條件可以適當減弱. 同時, 以下三者彼此確定 (\cite{CHRISTENSEN1998284})
    \begin{enumerate}
        \item 有足夠投射對象的 $(𝒫, 𝒩)$-垂直對, 垂直條件是 $(P,i)=0$; 
        \item 有足夠投射對象的 $(𝒫, ℰ)$-垂直對, 垂直條件是 $(P,e)$ 爲滿態射; 
        \item 有足夠投射對象的 $(𝒫, 𝒞)$-垂直對, 垂直條件是 $(P,C^∙)= [∙ → ∙ → ∙]$ 在中間項正合. 
    \end{enumerate}
\end{remark}

\begin{proposition}[一些封閉性條件]
    $𝒩$ 是範疇的雙邊理想, 關於極限封閉; $ℰ$ 關於``滿態射性二推三'', 形變收縮核, 極限封閉; $𝒫$ 關於餘極限, 直和項 (形變收縮核) 封閉. \parnote{證明見筆記}
\end{proposition}

\begin{example}[如何構造 $(𝒫 , 𝒩)$-對?]
    一種方法: 選用 J.P. May 公理體系 (\cite{JPMayAdd01}) 下的幺半三角範疇, 找到不必滿足結合律的乘法對象 $M$, 則有誘導的內射對. 改用餘乘對象, 則有投射對. \parnote{單位 spec $𝕊$ 給出 Phantom 態射}
\end{example}

\begin{proposition}[$(𝒫 , 𝒩)$-對的 Bool 運算]
    給定一組 $(𝒫 _α , 𝒩 _α )$, 則 $(\mathrm{Sum}({∐ _{α}𝒫 _α}) , ⋂_α 𝒩 _α)$ 也是 $(𝒫, 𝒩)$-對. \parnote{直接驗證}
\end{proposition}

\begin{proposition}[$(𝒫 , 𝒩 )$-對的三角運算]
    給定一族 $(𝒫 _i, 𝒩 _i)$, 則 $ (𝒫 _1 ⋆ 𝒫 _2 , 𝒩 _2 ∘ 𝒩 _1)$ 的新的 $(𝒫_1 ⋆ 𝒫 _2)$ 對. 其中, 對象類的運算定義作
    \begin{equation}
        𝒳 ⋆ 𝒴 := \{A ∣ ∃ X ∈ 𝒳 \ ∃ Y ∈ 𝒴 \ ∃ [X → A → Y] =: Δ \ (Δ \  \text{是好三角})\}. 
    \end{equation}
    \begin{proof}
        細節從略. 主要原理是八面體公理
        \begin{equation}
            % https://q.uiver.app/#q=WzAsMTIsWzEsMSwiUF8xIl0sWzMsMSwiUF8yIl0sWzIsMSwiPyJdLFsyLDIsIlgiXSxbMiwzLCJCIl0sWzEsMiwiWCJdLFsxLDMsIkEiXSxbMCwxLCJQXzJbLTFdIl0sWzEsMCwiQVstMV0iXSxbMiwwLCJCWy0xXSJdLFswLDAsIlBfMlstMV0iXSxbMywwLCJQXzIiXSxbMiwzXSxbMyw0LCJmXzIgXFxjaXJjIGZfMSJdLFszLDUsIiIsMSx7ImxldmVsIjoyLCJzdHlsZSI6eyJoZWFkIjp7Im5hbWUiOiJub25lIn19fV0sWzAsMl0sWzIsMV0sWzUsNiwiZl8xIiwyXSxbMCw1XSxbNiw0LCJmXzIiLDJdLFs4LDksImZfMlstMV0iXSxbNywwXSxbMTAsNywiIiwxLHsibGV2ZWwiOjIsInN0eWxlIjp7ImhlYWQiOnsibmFtZSI6Im5vbmUifX19XSxbMTAsOF0sWzExLDEsIiIsMSx7ImxldmVsIjoyLCJzdHlsZSI6eyJoZWFkIjp7Im5hbWUiOiJub25lIn19fV0sWzksMTFdLFs5LDJdLFs4LDBdXQ==
\begin{tikzcd}[ampersand replacement=\&]
	{P_2[-1]} \& {A[-1]} \& {B[-1]} \& {P_2} \\
	{P_2[-1]} \& {P_1} \& {?} \& {P_2} \\
	\& X \& X \\
	\& A \& B
	\arrow[from=1-1, to=1-2]
	\arrow[equals, from=1-1, to=2-1]
	\arrow["{f_2[-1]}", from=1-2, to=1-3]
	\arrow[from=1-2, to=2-2]
	\arrow[from=1-3, to=1-4]
	\arrow[from=1-3, to=2-3]
	\arrow[equals, from=1-4, to=2-4]
	\arrow[from=2-1, to=2-2]
	\arrow[from=2-2, to=2-3]
	\arrow[from=2-2, to=3-2]
	\arrow[from=2-3, to=2-4]
	\arrow[from=2-3, to=3-3]
	\arrow["{f_1}"', from=3-2, to=4-2]
	\arrow[equals, from=3-3, to=3-2]
	\arrow["{f_2 \circ f_1}", from=3-3, to=4-3]
	\arrow["{f_2}"', from=4-2, to=4-3]
\end{tikzcd}.
        \end{equation}
    \end{proof}
\end{proposition}

\begin{remark}
    結合律 $(𝒳 ⋆ 𝒴) ⋆ 𝒵 = 𝒳 ⋆ (𝒴 ⋆ 𝒵)$ 也是八面體公理的推論. 
\end{remark}

\begin{definition}[三角的 Adams 投射分解]
    給定對象 $X = X^0$, 則有``投射覆蓋''的好三角 $P^0 → X^0 → X^{-1}$. 繼而考慮 $P^{-1} → X^{-1} → X^{-2}$ 等等, 最終得
    \begin{equation}
        % https://q.uiver.app/#q=WzAsOCxbMCwwLCJYXjAiXSxbMSwxLCJQXjAiXSxbMiwwLCJYXnstMX0iXSxbMywxLCJQXnstMX0iXSxbNCwwLCJYXnstMn0iXSxbNywwLCJcXGNkb3RzIl0sWzUsMSwiUF57LTJ9Il0sWzYsMCwiWF57LTN9Il0sWzEsMF0sWzAsMl0sWzMsMl0sWzIsNF0sWzQsN10sWzcsNV0sWzYsNF0sWzIsMSwiIiwxLHsiY29sb3VyIjpbMjQwLDYwLDYwXSwic3R5bGUiOnsiYm9keSI6eyJuYW1lIjoiYmFycmVkIn19fV0sWzQsMywiIiwxLHsiY29sb3VyIjpbMjQwLDYwLDYwXSwic3R5bGUiOnsiYm9keSI6eyJuYW1lIjoiYmFycmVkIn19fV0sWzcsNiwiIiwxLHsiY29sb3VyIjpbMjQwLDYwLDYwXSwic3R5bGUiOnsiYm9keSI6eyJuYW1lIjoiYmFycmVkIn19fV1d
\begin{tikzcd}[ampersand replacement=\&, sep = small]
	{X^0} \&\& {X^{-1}} \&\& {X^{-2}} \&\& {X^{-3}} \& \cdots \\
	\& {P^0} \&\& {P^{-1}} \&\& {P^{-2}}
	\arrow[from=1-1, to=1-3]
	\arrow[from=1-3, to=1-5]
	\arrow["\shortmid"{marking, text={rgb,255:red,92;green,92;blue,214}}, color={rgb,255:red,92;green,92;blue,214}, from=1-3, to=2-2]
	\arrow[from=1-5, to=1-7]
	\arrow["\shortmid"{marking, text={rgb,255:red,92;green,92;blue,214}}, color={rgb,255:red,92;green,92;blue,214}, from=1-5, to=2-4]
	\arrow[from=1-7, to=1-8]
	\arrow["\shortmid"{marking, text={rgb,255:red,92;green,92;blue,214}}, color={rgb,255:red,92;green,92;blue,214}, from=1-7, to=2-6]
	\arrow[from=2-2, to=1-1]
	\arrow[from=2-4, to=1-3]
	\arrow[from=2-6, to=1-5]
\end{tikzcd}.
    \end{equation}
    特別地, 每一好三角在 $(P, -)$ 分裂做一族短正合列 ($P^∙$-對邊斷開), 因此有 $(P,-)$-相對投射分解 
    \begin{equation}
        \cdots → P^{-2}[-2] → P^{-1}[-1] → P^0 → X → 0. 
    \end{equation}
\end{definition}

\begin{proposition}[Adams 分解的逆命題]
    若 $X$ 存在 $𝒫$-相對投射分解, 即 
    \begin{enumerate}
        \item 一族態射鏈 $θ : 𝒫 → X → 0$, 不必正合; 
        \item 對任意 $P ∈ 𝒫$, $(P, θ)$ 是長正合列. 
    \end{enumerate}
    此時存在 Adams 投射分解系統. 
    \begin{proof}
        考慮長 $𝒫$-正合列 $\cdots → P_2 → P_1 → P_0 → X → 0$. 
        \begin{enumerate}
            \item 取好三角 $P(X) → X \xrightarrow {f} Y$. 由滿態射 $(P(X),P_0) ↠ (P(X), X)$ 得提升 $P_0 → X$
            \begin{equation}
                % https://q.uiver.app/#q=WzAsOCxbMCwxLCJQKFgpIl0sWzEsMSwiWCJdLFsyLDEsIlkiXSxbMSwwLCJQXzAiXSxbMCwwLCJQKFgpIl0sWzEsMiwiWF57LTF9Il0sWzIsMCwiXFxidWxsZXQiXSxbMiwyLCJYXnstMX0iXSxbMCwxXSxbMSwyLCJmIl0sWzMsMV0sWzQsMCwiIiwxLHsibGV2ZWwiOjIsInN0eWxlIjp7ImhlYWQiOnsibmFtZSI6Im5vbmUifX19XSxbNCwzLCIiLDAseyJzdHlsZSI6eyJib2R5Ijp7Im5hbWUiOiJkYXNoZWQifX19XSxbMSw1XSxbMyw2XSxbNiwyXSxbMiw3XSxbNSw3LCIiLDEseyJsZXZlbCI6Miwic3R5bGUiOnsiaGVhZCI6eyJuYW1lIjoibm9uZSJ9fX1dXQ==
\begin{tikzcd}[ampersand replacement=\&]
	{P(X)} \& {P_0} \& \bullet \\
	{P(X)} \& X \& Y \\
	\& {X^{-1}} \& {X^{-1}}
	\arrow[dashed, from=1-1, to=1-2]
	\arrow[equals, from=1-1, to=2-1]
	\arrow[from=1-2, to=1-3]
	\arrow[from=1-2, to=2-2]
	\arrow[from=1-3, to=2-3]
	\arrow[from=2-1, to=2-2]
	\arrow["f", from=2-2, to=2-3]
	\arrow[from=2-2, to=3-2]
	\arrow[from=2-3, to=3-3]
	\arrow[equals, from=3-2, to=3-3]
\end{tikzcd}.
            \end{equation}
            上圖 (八面體公理的局部) 給出 $P_0 → X^0 \xrightarrow{∈ 𝒩} X^{-1}$. 
            \item 繼而證明 $P(X^{-1}) → X^{-1}$ 通過 $P_1[1]$ 分解. 實際上, $X^{-1}$ 是正合列的 $𝒫$-相對 syzygy. 因此, 後繼歸納與初始設定的證明步驟相同. 
    \end{enumerate}
    \end{proof}
\end{proposition}

\begin{theorem}
    Adams-投射分解系統和 $𝒫$-相對投射分解互相轉換. $𝒫$-相對投射分解的復形同態延拓至 Adams 投射分解系統. \parnote{態射範疇!}
\end{theorem}

\begin{example}[Adams 濾過]
    相對 syzygy $X^{-n}$, 上同調函子 $H$ 的導出, 
    \textcolor{red}{有用時再補上吧}
\end{example}





































\newpage


\section{Tilting theory}

\subsection{Torsion Pair}
\begin{abstract}
    爲研究複雜環 $A$, 有時可以尋找一類斜置模 $T_A$, 使得 $\mathrm{End}(T_A)$ 是較爲簡單的代數, 同時 $𝐦𝐨𝐝 _A$ 與 $𝐦𝐨𝐝 _B$ 等價. 關於 Tilting 早期工作的文章見手冊 \cite{Angeleri_Hügel_Happel_Krause_2007}, 及其\href{https://www.mathematik.uni-muenchen.de/~tilting/}{對應的網站}. \cite{Keller1994} 使用 tilting 理論清晰地解釋一個導出等價問題 (Theorem 5), 十分有趣.

本節介紹基本概念 torsion pair 和 tilting module. 解釋 tilting module 如何創造 torsion pair, 最後目標是 Brenner-Butler 定理 (\cite{Brenner1980GeneralizationsOT}). 
\end{abstract}

\subsubsection{扭對的基本性質, 結構, 以及等價定義}

\begin{definition}[扭對]\label{torsionpari}
    稱 $(ℱ, 𝒯)$ 是扭對, 當且僅當 $𝒯 ⟂_{\mathrm{Hom}} ℱ$, 且 $𝒯 ^{⟂ ⟂ } = 𝒯$, $ℱ^{⟂ ⟂ } = ℱ$. \parnote{Hom 垂直, 正交閉集}
\begin{enumerate}
    \item $ℱ$ 稱作無扭類 (torsion-free class); \parnote{類似自由群}
    \item $𝒯$ 稱作扭類 (torsion class). \parnote{類似扭群}
\end{enumerate}
\end{definition}

\begin{remark}
    給定對象類 $𝒳$, 則 $(𝒳^{⟂}, {^⟂}(𝒳^⟂))$ 與 $((^⟂ 𝒳)^⟂ , {^⟂ }𝒳 )$ 是擾對; 給定 $A$ 的擾對 $(ℱ, 𝒯 )$, 則有 $DA$ 的擾對 $(D𝒯, Dℱ )$. 
\end{remark}

\begin{remark}
    \begin{pinked}
        順序選用 $(ℱ, 𝒯)$. 爲了符合 $\mathrm{Hom}(ℱ, 𝒯)$ 與 $\mathrm{Ext}^1(ℱ, 𝒯)$ 等順序. 
    \end{pinked}
\end{remark}

\begin{theorem}[扭對結構: radical 函子 $t$]
    稱 $t$ 是冪等根函子, 當且僅當 $t$ 是 $\mathrm{id}$ 的子函子, 且\parnote{$\mathrm{Rad}∘ \mathrm{Top} = 0$}
\begin{equation}
    t : \ \ [0 → tM → M → M / tM → 0] \ \  ⟹ \ \ [0 → tM = tM → 0 → 0].
\end{equation}
此時, 以下關於對象類 $𝒯$ 的論斷等價: 
\begin{enumerate}
    \item 存在擾對 $(ℱ , 𝒯)$, 換言之, $(^⟂𝒯)^⟂ = 𝒯$;
    \item $𝒯$ 對餘極限 (商 + 餘積), 以及擴張封閉; \parnote{``扭模''}
    \item 存在冪等根函子 $t$ 使得 $t|_𝒯 = \mathrm{id}$. 
\end{enumerate}
類似地, 以下關於對象類 $ℱ$ 的論斷等價: 
\begin{enumerate}
    \item 存在擾對 $(ℱ , 𝒯)$, 換言之, ${^⟂}(ℱ^⟂) = ℱ$;
    \item $ℱ$ 對極限 (子 + 積), 以及擴張封閉; \parnote{``無扭模''}  
    \item 存在冪等根函子 $t$ 使得 $t|_ℱ = 0$. 
\end{enumerate}
\end{theorem}

\begin{remark}
    冪等根函子建立如下映像: 
    \begin{equation}
        [0 → tM → M → M/tM → 0] ⟺ [0 → T → M → F → 0]. 
    \end{equation}
    對 $𝐚𝐛$ 等簡單範疇而言, 以上正合列可裂, 從而對象類就是 $𝒯⊕ ℱ$. 
\end{remark}

\begin{theorem}[可裂扭對的結構]
    以下是可裂扭對 (有时记作 $ℱ ∨ 𝒯$) 的等價表述: 
    \begin{enumerate}
        \item 對象類就是 $𝒯⊕ ℱ$, 即所有 $M$ 分解作 $T ∈ 𝒯$ 與 $F ∈ ℱ$ 的直和; 
        \item $ℱ ⟂_{\mathrm{Ext}^1} 𝒯$, 即所有 $0 → T → M → F → 0$ 可裂, 且垂直閉; 
        \item $ℱ$ 關於 $τ$ 封閉, 此處 $τ$ 往``投射方向''走; 
        \item $𝒯$ 關於 $τ⁻¹$ 封閉, 此處 $τ ⁻¹$ 往``內射方向''走,   
    \end{enumerate}
\end{theorem}

\subsubsection{(預備定義) 斜置模生成扭對}

\begin{definition}[預備記號: $\mathrm{Gen}$]
    給定對象 $T$. 稱 $M ∈ \mathrm{Gen}(T)$, 當且僅當以下等價條件成立. 
    \begin{enumerate}
        \item 存在 $d ≥ 0$ 使得有滿射 $T^d ↠ M$; \parnote{像可達}
        \item $(T, M) ⊗_{\mathrm{End}(T)} T → M,\ ∑ f⊗ x ↦ ∑ f(x)$ 是滿射. 
    \end{enumerate}
這一記號爲創造 $𝒯$ 服務.\parnote{$≈ 𝒯$}
\end{definition}

\begin{definition}[預備記號: $\mathrm{Cogen}$]
    給定對象 $T$. 稱 $M ∈ \mathrm{Cogen}(T)$, 當且僅當以下等價條件成立. 
    \begin{enumerate}
        \item 存在 $d ≥ 0$ 使得有單射 $M ↪ T^d$. ; \parnote{泛函可分}
        \item $M → ((M,T),T)_{\mathrm{End}(T)}, \ m ↦[g ↦ g(m)]$ 是單射. 
    \end{enumerate}
這一記號爲創造 $ℱ$ 服務. \parnote{$≈ ℱ$}
\end{definition}

\begin{definition}
    我們關心正則模本身 $A ∈ \mathrm{Gen}(?)$, $A ∈ \mathrm{Cogen}(?)$. 稱 $M$ 是忠實的, 若以下等價命題成立: \parnote{忠實模}
\begin{enumerate}
    \item $A ↪ \mathrm{End}(M)$ 是單射; 右乘不同的 $a ∈ A$ 給出不同的態射; $\mathrm{Ann}_A(M)$ 是零理想; \parnote{環性質}
    \item $A ∈ \mathrm{Cogen}(M)$; $DA ∈ \mathrm{Gen}(DM)$. \parnote{生成模性質}
\end{enumerate}
\end{definition}

\begin{definition}[(預備定義) 偏斜置模]\label{tilting}
    稱 $T$ 是偏斜置 (partial tilting) $A$-模, 若以下兩點同時成立:
    \begin{enumerate}
        \item $p.\dim T ≤ 1$; \parnote{類似投射模}
        \item $\mathrm{Ext}^1(T,T)=0$. \parnote{相對投射, 相對內射}
    \end{enumerate} 
\end{definition}

\subsubsection{(偏) 斜置模生成扭對}

\begin{abstract}
    核心定理: 偏斜置模誘導扭對 $(ℱ, 𝒯)$; 左模對稱地誘導扭對 $(𝒳, 𝒴)$; $ℱ≃𝒳$ 與 $𝒯≃𝒴$.
\end{abstract}

\begin{theorem}[偏斜置模生成扭對]
    給定偏斜置模 $T$, 則 
    \begin{enumerate}
        \item $\mathrm{Ext}^1 (T, \mathrm{Gen}(T)) = 0$; 
        \item $\mathrm{Gen}(T)$ 是擾對的 $𝒯$-部分;
        \begin{itemize}
            \item 注: $M ∈ 𝒯$ 當且僅當 $(T, M) ⊗_{\mathrm{End}(T)} T → M, \ ∑ f ⊗ x ↦ ∑ f(x)$ 是雙射.
        \end{itemize}
        \item $\mathrm{Cogen}(τ T)$ 是擾對的 $ℱ$ 部分.
    \end{enumerate}
    \begin{pinked}
        $(ℱ,𝒯) = (\mathrm{Cogen}(τ T), \mathrm{Gen}(T))$. 
    \end{pinked}
\end{theorem}

\begin{definition}[斜置模]
    稱偏斜置模 $T$ 是斜置的, 當且僅當 $\mathrm{Add}(T) = \mathrm{Add}(A)$. 等價地, 
    \begin{enumerate}
        \item $A ∈ \mathrm{Cogen} (T)$, 換言之, $T$ 是忠實模;
        \item 任何 $M ∈ (T)^{⟂, 1}$ 通過 $T$ 有限表現, 即下一條;
        \item $\mathrm{Gen}(T) = (T)^{⟂, 1}$; \parnote{$\mathrm{Ext}^1(T,-)$}
        \item $\mathrm{Cogen}(T) = (τ T) ^{⟂, 0}$. \parnote{$\mathrm{Hom}(τT,-)$}
    \end{enumerate} 
\end{definition}

\begin{remark}
    ``偏斜置''包含了結構與性質, 扔掉``偏''無非篩選 (取全子範疇).
\end{remark}

\begin{theorem}[斜置模誘導扭對]
    對斜置模 $T$, 其作爲偏斜置模誘導了扭對.
    \begin{equation}
        𝒯 = \mathrm{Gen}(T) = (T)^{⟂, 1}, \ ℱ = \mathrm{Cogen}(τ T) = (T)^{⟂, 0}.
    \end{equation}
\end{theorem}

\begin{proposition}[$T$ 的左模結構, BB 定理]
    取斜置模 $T$, 記 $B := \mathrm{End}(T)$. 此時 $T$ 是 $(B,A)$ 雙模. 
    \begin{enumerate}
        \item $T$ 作爲左 $B$-模, 也是斜置的; $A → \mathrm{End}_B(T)^{\mathrm{op}},\quad a ↦ [(-)⋅ a]$ 是代數同構. \parnote{雙側斜置}
        \item 依照 $T$ 的雙側斜置結構, 得四類對象: 
\begin{enumerate}
    \item ($A$-扭元類) $𝒯 := \mathrm{Gen}(T) = \ker \mathrm{Ext}^1_A(T, -)$, 
    \item ($A$-無扭元類) $ℱ := \mathrm{Cogen}(τ T) = \mathrm{Hom}_A(T, -)$, 
    \item ($B$-扭元類) $𝒳 := D ℱ(_BT) = \ker \mathrm{Hom}_B(-, DT) = \ker (- ⊗ _BT)$, 
    \item ($B$-無扭元類) $𝒴 := D 𝒯(_BT) = \ker \mathrm{Ext}_B(-, DT) = \ker \mathrm{Tor}_1^B (-, T)$. 
\end{enumerate}
        \item (BB) 對應關係 $ℱ  ↔ 𝒳$, $𝒯 ↔ 𝒴$\parnote{導出的核, 用零次函子鏈接, 反之亦然}
        \begin{equation}
            % https://q.uiver.app/#q=WzAsMTQsWzEsMiwiXFxtYXRocm17Q29nZW59KM+EIFQpICJdLFswLDIsIlxca2VyIFxcbWF0aHJte0hvbX1fQShULCAtKSJdLFsyLDIsIuKEsSAiXSxbMiwwLCLwnZKvICJdLFswLDAsIlxca2VyIFxcbWF0aHJte0V4dH1eMV9BKFQsIC0pIl0sWzEsMCwiXFxtYXRocm17R2VufShUKSAiXSxbMywyLCLwnZKzICJdLFs1LDIsIlxca2VyICgtIOKKlyBfQlQpIl0sWzQsMiwiXFxrZXIgXFxtYXRocm17SG9tfV9CKC0sIERUKSJdLFszLDAsIvCdkrQgIl0sWzUsMCwiXFxrZXIgXFxtYXRocm17VG9yfV8xXkIgKC0sIFQpIl0sWzQsMCwiXFxrZXIgXFxtYXRocm17RXh0fV9CKC0sIERUKSJdLFsxLDEsIvCdkKbwnZCo8J2QnV9BIl0sWzQsMSwi8J2QpvCdkKjwnZCdX0IiXSxbMiwwLCIiLDAseyJsZXZlbCI6Miwic3R5bGUiOnsiaGVhZCI6eyJuYW1lIjoibm9uZSJ9fX1dLFswLDEsIiIsMCx7ImxldmVsIjoyLCJzdHlsZSI6eyJoZWFkIjp7Im5hbWUiOiJub25lIn19fV0sWzMsNSwiIiwwLHsibGV2ZWwiOjIsInN0eWxlIjp7ImhlYWQiOnsibmFtZSI6Im5vbmUifX19XSxbNSw0LCIiLDAseyJsZXZlbCI6Miwic3R5bGUiOnsiaGVhZCI6eyJuYW1lIjoibm9uZSJ9fX1dLFs2LDgsIiIsMCx7ImxldmVsIjoyLCJzdHlsZSI6eyJoZWFkIjp7Im5hbWUiOiJub25lIn19fV0sWzksMTEsIiIsMCx7ImxldmVsIjoyLCJzdHlsZSI6eyJoZWFkIjp7Im5hbWUiOiJub25lIn19fV0sWzExLDEwLCIiLDAseyJsZXZlbCI6Miwic3R5bGUiOnsiaGVhZCI6eyJuYW1lIjoibm9uZSJ9fX1dLFs4LDcsIiIsMCx7ImxldmVsIjoyLCJzdHlsZSI6eyJoZWFkIjp7Im5hbWUiOiJub25lIn19fV0sWzIsNiwiXFxtYXRocm17RXh0fV9BXjEgKFQsIC0pIiwyLHsib2Zmc2V0Ijo1fV0sWzYsMiwiXFxtYXRocm17VG9yfV8xXkIoLSwgVCkiLDIseyJvZmZzZXQiOjV9XSxbOSwzLCIoLSDiipcgX0IgVCkiLDIseyJvZmZzZXQiOjV9XSxbMyw5LCJcXG1hdGhybXtIb219X0EoVCwgLSkiLDIseyJvZmZzZXQiOjV9XV0=
\begin{tikzcd}[ampersand replacement=\&, sep = tiny]
	{\ker \mathrm{Ext}^1_A(T, -)} \& {\mathrm{Gen}(T) } \& {𝒯 } \& {𝒴 } \& {\ker \mathrm{Ext}_B(-, DT)} \& {\ker \mathrm{Tor}_1^B (-, T)} \\
	\& {𝐦𝐨𝐝_A} \&\&\& {𝐦𝐨𝐝_B} \\
	{\ker \mathrm{Hom}_A(T, -)} \& {\mathrm{Cogen}(τ T) } \& {ℱ } \& {𝒳 } \& {\ker \mathrm{Hom}_B(-, DT)} \& {\ker (- ⊗ _BT)}
	\arrow[equals, from=1-2, to=1-1]
	\arrow[equals, from=1-3, to=1-2]
	\arrow["{\mathrm{Hom}_A(T, -)}"', shift right=3, from=1-3, to=1-4]
	\arrow["{(- ⊗ _B T)}"', shift right=3, from=1-4, to=1-3]
	\arrow[equals, from=1-4, to=1-5]
	\arrow[equals, from=1-5, to=1-6]
	\arrow[equals, from=3-2, to=3-1]
	\arrow[equals, from=3-3, to=3-2]
	\arrow["{\mathrm{Ext}_A^1 (T, -)}"', shift right=3, from=3-3, to=3-4]
	\arrow["{\mathrm{Tor}_1^B(-, T)}"', shift right=3, from=3-4, to=3-3]
	\arrow[equals, from=3-4, to=3-5]
	\arrow[equals, from=3-5, to=3-6]
\end{tikzcd}.
        \end{equation}
        \item 假定 $M$ 是 $A$ 模, 且 $X$ 是 $B$ 模, 則有同構 \parnote{$T$ 類似投射}
        \begin{enumerate}
            \item $(T,M) ⊗ T ≃ M, \ ∑ f ⊗ t ↦ ∑ f(t)$, 以及
            \item $X ≃ (T, X ⊗_B T), \ m ↦ [t ↦ m ⊗ t]$. 
        \end{enumerate}
        \item 作爲推論, 得混合係數公式: \parnote{來源 $≠$ 去向, 結果爲 $0$}
        \begin{enumerate}
            \item $\mathrm{Tor}_1^B (\mathrm{Hom}_A(T,M), T) = 0$, 
            \item $\mathrm{Ext}^1_A(T,M) ⊗ _BT = 0$,
            \item $\mathrm{Hom}_A(T, X) ⊗_B T = 0$,
            \item $\mathrm{Ext}_A^1(T, X ⊗_B T) = 0$. 
        \end{enumerate}
        所謂``混合'', 一處 $⊗$ 一處 $\mathrm{Hom}$, 一處導出一處原是也. 結果均是 $0$. 
        \item 存在類似``拓撲六函子''的兩條正合列, 以刻畫兩個範疇中的冪等根函子: 
        \begin{enumerate}
            \item $(𝒯, ℱ)$, $𝐦𝐨𝐝_A$ 中的扭對: 
            \begin{equation}
                0 → \underbracket{(T, M)_A ⊗_B T}\limits_{𝐦𝐨𝐝_A → 𝒴 → 𝒯} → M → \underbracket{\mathrm{Tor}_1 ^B(\mathrm{Ext}_A^1 (T,M), T)}\limits_{𝐦𝐨𝐝_A → 𝒳 → ℱ} → 0; 
            \end{equation}
            \item $(𝒳, 𝒴)$, $𝐦𝐨𝐝_B$ 中的扭對: 
            \begin{equation}
                0 → \underbracket{\mathrm{Ext}_A^1 (T, \mathrm{Tor}_1 ^B(X, T))}\limits_{𝐦𝐨𝐝_B → ℱ → 𝒳} → X → \underbracket{\mathrm{Hom}_A(T, X ⊗ _B T)}\limits_{𝐦𝐨𝐝_B → 𝒯 → 𝒴} → 0
            \end{equation}
        \end{enumerate}
    \end{enumerate}
\end{proposition}











\subsection{Some spectral sequence}
\begin{abstract}
    原始的斜置模 $T$ 滿足
    \begin{enumerate}
        \item $T$ 的 $𝐦𝐨𝐝(A)$-投射維度 $≤ 1$; 
        \item $A$ 的 $𝐦𝐨𝐝(T)$-內射維度 $≤ 1$; 
        \item $\mathrm{Ext}^1(T,T)=0$. 
    \end{enumerate}
    有限維度的斜置模將以上 $1$ 換做了 $< ∞$. \parnote{有什麼用?}

    以上推廣由宮下洋一 (Yōichi Miyashita) 在 \cite{Miyashita1986} 中首次提及. 
    \begin{itemize}
    \item 宮下在寫完此篇引用 400+ 的文章後杳然無蹤, \href{https://www.mathgenealogy.org/index.php}{數學族譜網}找不到宮下與其老師 \cite{Nagahara1995} 的任何消息, 大抵是功成名就後淡出了學術舞台. 
    \end{itemize}
    本節先概括宮下的系列工作, 之後使用 B-B 的譜序列算法 (第四章, \cite{Angeleri_Hügel_Happel_Krause_2007}) 將結論串通一遍. 

    \begin{enumerate}
        \item 定義投射維度有限的斜置模; 
        \item 相對投射 (內射) 消解的對偶, 恰好是左模的相對投射 (內射) 消解; 
        \item 給出四個相同的``消解維度'', 以及相應的斜置函子; 
        \item (重點) 原始斜置理論的``四函子''很簡單, 復合求導公式的所有二階導數均為 $0$ (因爲 $p.\dim T ≤ 1$), 將投射維數提升後, 更好的精細來自譜序列. 
    \end{enumerate}
    重要問題: 傳統的斜置模蘊含扭對, 投射維數有限的斜置模如何變化? 
\end{abstract}

\subsubsection{投射維度有限的斜置模 (斜置模的推廣)}

\begin{definition}[(餘) 消解]\label{resol}
    稱上鏈複形 $X^∙$ 是對象 $A$ 的消解, 若 $X = X^{≤ 0}$, 且 $H^0(X) = A$. \parnote{投射分解的推廣} 餘消解類似. 
\end{definition}

\begin{definition}[投射維度有限的斜置模]
    稱 $T ∈ 𝐦𝐨𝐝 _A$ 是投射維度有限的斜置模, 當且僅當以下三點成立. 
    \begin{enumerate}
        \item $T$ 具有有限長度的 $𝐚𝐝𝐝 (A)$-消解; \parnote{$𝐩𝐫𝐨𝐣 (A)$}
        \item $A$ 具有有限長度的 $𝐚𝐝𝐝 (T)$-餘消解; 
        \item 對任意 $n ≥ 1$, 總有 $\mathrm{Ext}^n (T,T)=0$. 
    \end{enumerate}
\end{definition}

\begin{remark}
    若 $T$ 有二項 $𝐚𝐝𝐝 (A)$-消解, 且 $A$ 有二項 $𝐚𝐝𝐝 (T)$-餘消解, 則 $T$ 是通常的斜置模. 
\end{remark}

\begin{example}
    \text{bf}{往後使用斜置模簡稱``投射維度有限的斜置模''}. 仿照斜置模的一般研究方式, 我們希望 $T$ 的左 $\mathrm{End}(T)$ ($B$)-模也是斜置的. 更進一步地, 能否直接從 $𝐦𝐨𝐝 _A$ 的消解直接給出 $𝐦𝐨𝐝 _{B}$ 的消解? 
\end{example}

\begin{definition}[$T$-對偶]
    記 $h_T: 𝐦𝐨𝐝_A → {_B}𝐦𝐨𝐝$ 與 $h_T: {_B}𝐦𝐨𝐝 → 𝐦𝐨𝐝 _A$ 是 $T$-對偶函子. 必要時強調左右模: 
    \begin{enumerate}
        \item $((-)_A,{_B}T_A)_A : 𝐦𝐨𝐝 _A → 𝐦𝐨𝐝_{B^{\mathrm{op}}}$; 
        \item $({_B}(-), {_B}T_A)_{B^{\mathrm{op}}} : 𝐦𝐨𝐝_{B^{\mathrm{op}}} → 𝐦𝐨𝐝 _A$. 
    \end{enumerate}
\end{definition}

\begin{theorem}[$T$-對偶模消解定理]
    給定斜置模 $T$. 假定
    \begin{itemize}
        \item  $𝐦𝐨𝐝 _A$-範疇中, $P^∙$ 是 $T$ 的有限 $𝐚𝐝𝐝 (A)$-消解, $T^∙$ 是 $A$ 的有限 $𝐚𝐝𝐝 (T)$-餘消解. 
    \end{itemize}
    此時
    \begin{itemize}
        \item ${_B}𝐦𝐨𝐝$-範疇中 $h_T(T^∙)$ 是 $T$ 的有限 $𝐚𝐝𝐝 (B)$-消解, $h_T(P^∙)$ 是 $B$ 的有限 $𝐚𝐝𝐝 (T)$-余消解.  
    \end{itemize}
    同時, ${_B}T$ 是斜置左模. 也可以從 ${_B}T$ 推到 $T_A$: 只需發現 $h_T ∘ h_T$ 建立了 $𝐚𝐝𝐝 (A ⊕ T)$ 的同構. 
    \begin{proof}
        依次證明對象類的對應, 對偶消解成立, ${_B}T$ 是斜置左模, 以及二次對偶同構. 
        \begin{enumerate}
            \item 直接地, $h_T: 𝐚𝐝𝐝 (A) → 𝐚𝐝𝐝 (T)$, $h_T: 𝐚𝐝𝐝 (T) → 𝐚𝐝𝐝 (B)$, 以及 $h_T: 𝐚𝐝𝐝 $. \parnote{檢驗對象類}
            \item 先說明 $(T^∙,T) ⇒ T$ 是有限 $𝐚𝐝𝐝 (B)$ 消解. 對象已說明, 只需檢驗正合性. 
            \begin{equation}
                % https://q.uiver.app/#q=WzAsMTgsWzYsMiwiKFReMCwgVCkiXSxbNCwyLCIoVF4xLCBUKSJdLFsyLDIsIihUXjIsIFQpIl0sWzAsMiwiKFReMywgVCkiXSxbNywyLCJUIl0sWzYsMSwiVF4wICJdLFs3LDEsIkEiXSxbNCwxLCJUXjEiXSxbMiwxLCJUXjIiXSxbMCwxLCJUXjMiXSxbNSwwLCJcXE9tZWdhXjAgIl0sWzMsMCwiXFxPbWVnYSBeMSJdLFsxLDAsIlxcT21lZ2FeMiJdLFs1LDMsIihcXE9tZWdhXjAgLFQpIl0sWzMsMywiKFxcT21lZ2EgXjEsIFQpIl0sWzEsMywiKFxcT21lZ2EgXjIsIFQpIl0sWzgsMSwiXFxtYXRoYmZ7YWRkfShBKSJdLFs4LDIsIlxcbWF0aGJme2FkZH0oVCkiXSxbMywyXSxbMiwxXSxbMCw0LCIiLDAseyJsZXZlbCI6Miwic3R5bGUiOnsiaGVhZCI6eyJuYW1lIjoibm9uZSJ9fX1dLFs2LDUsIiIsMCx7ImxldmVsIjoyLCJzdHlsZSI6eyJoZWFkIjp7Im5hbWUiOiJub25lIn19fV0sWzUsN10sWzcsOF0sWzgsOV0sWzcsMTEsIiIsMCx7InN0eWxlIjp7ImhlYWQiOnsibmFtZSI6ImVwaSJ9fX1dLFs4LDEyLCIiLDAseyJzdHlsZSI6eyJoZWFkIjp7Im5hbWUiOiJlcGkifX19XSxbMTIsOSwiIiwxLHsic3R5bGUiOnsidGFpbCI6eyJuYW1lIjoiaG9vayIsInNpZGUiOiJib3R0b20ifX19XSxbMTEsOCwiIiwxLHsic3R5bGUiOnsidGFpbCI6eyJuYW1lIjoiaG9vayIsInNpZGUiOiJib3R0b20ifX19XSxbMTAsNywiIiwxLHsic3R5bGUiOnsidGFpbCI6eyJuYW1lIjoiaG9vayIsInNpZGUiOiJib3R0b20ifX19XSxbMSwxM10sWzMsMTVdLFsyLDE0XSxbMTUsMl0sWzE0LDFdLFsxNiwxNywiaF9UIiwyXSxbNSwxMCwiIiwwLHsibGV2ZWwiOjIsInN0eWxlIjp7ImhlYWQiOnsibmFtZSI6Im5vbmUifX19XSxbMSwwXSxbMTMsMCwiIiwwLHsibGV2ZWwiOjIsInN0eWxlIjp7ImhlYWQiOnsibmFtZSI6Im5vbmUifX19XV0=
\begin{tikzcd}[ampersand replacement=\&, sep = tiny]
	\& {\Omega^2} \&\& {\Omega ^1} \&\& {\Omega^0 } \\
	{T^3} \&\& {T^2} \&\& {T^1} \&\& {T^0 } \& A \& {\mathbf{add}(A)} \\
	{(T^3, T)} \&\& {(T^2, T)} \&\& {(T^1, T)} \&\& {(T^0, T)} \& T \& {\mathbf{add}(T)} \\
	\& {(\Omega ^2, T)} \&\& {(\Omega ^1, T)} \&\& {(\Omega^0 ,T)}
	\arrow[hook', from=1-2, to=2-1]
	\arrow[hook', from=1-4, to=2-3]
	\arrow[hook', from=1-6, to=2-5]
	\arrow[two heads, from=2-3, to=1-2]
	\arrow[from=2-3, to=2-1]
	\arrow[two heads, from=2-5, to=1-4]
	\arrow[from=2-5, to=2-3]
	\arrow[equals, from=2-7, to=1-6]
	\arrow[from=2-7, to=2-5]
	\arrow[equals, from=2-8, to=2-7]
	\arrow["{h_T}"', from=2-9, to=3-9]
	\arrow[from=3-1, to=3-3]
	\arrow[from=3-1, to=4-2]
	\arrow[from=3-3, to=3-5]
	\arrow[from=3-3, to=4-4]
	\arrow[from=3-5, to=3-7]
	\arrow[from=3-5, to=4-6]
	\arrow[equals, from=3-7, to=3-8]
	\arrow[from=4-2, to=3-3]
	\arrow[from=4-4, to=3-5]
	\arrow[equals, from=4-6, to=3-7]
\end{tikzcd}.
            \end{equation}
            爲了由上正合列推得下者, 只需證明 $[0 → (Ω^{k+1}, T) → (T^{k+1}, T) → (Ω ^k, T) → 0]$ 是正合列, 也就是 $\mathrm{Ext}^1(Ω^{k+1}, T) = 0$. 依照 $\mathrm{Ext}^{≥ 1} (T,T) = 0$, 必然有 
            \begin{equation}
                \mathrm{Ext}^1(Ω^{k+1}, T) = \mathrm{Ext}^2(Ω^{k+2}, T) = \cdots = 0. 
            \end{equation}
            \begin{pinked}
                對有限長度正合列 $C^∙$, 若 $\mathrm{Ext}^{≥ 1} (C^∙ , M) =0$, 則 $\mathrm{Hom}(C^ ∙ , M)$ 正合. 
            \end{pinked}\parnote{臨時關鍵引理}
            \item 再說明 $B ⇒ (P^∙ , T)$ 是有限 $𝐚𝐝𝐝 (T)$-餘消解. 顯然 $\mathrm{Ext}^{≥ 1} (A, T)=0$ 恆成立. 由臨時關鍵引理, $(P^∙ , T)$ 正合. 
            \item 先說明 ${_B}T$ 的 $(≥ 1)$-自垂直性. 由上, $h_T(T^∙)$ 是 ${_B}T$ 的有限投射分解. 相應地, 導出群 $\mathrm{Ext}^{≥ 1}({_B}T, {_B}T)$ 由複形 $({_B}(h_T(T^∙)), {_B}T)$ 決定. 由於 $T^{≥ 1}$ 以 $𝐚𝐝𝐝 (T)$ 爲分量, 故 
            \begin{equation}
                ({_B}(h_T(T^∙)), {_B}T) ≃ [\underbracket{\cdots → T^2 → T^1 }\limits_{\text{正合}}→ B = \mathrm{End}_A(T) → 0].
            \end{equation}
            從而 $\mathrm{Ext}^{≥ 1} ({_BT}, {_BT}) = 0 $. 
            \item 二次對偶建立了 $A ≃ \mathrm{End}_A(T) = B$, 因爲 $T^∙$ 是兩者共同的有限消解. 此時 $h_T ∘ h_T$ 是限制在 $𝐚𝐝𝐝 (X ⊕ T)$ 上的同構. \parnote{$A ≃ B$}
        \end{enumerate}
    \end{proof}
\end{theorem}

\begin{remark}
    若以四要件 $(A, T, P^∙ , T^∙)$ 描述斜置模, 則 $(B^{\mathrm{op}}, T, h_T(T^∙), h_T(P^∙))$ 也是斜置模. 
\end{remark}

\begin{proposition}
    來自章節 2.2,\cite{Happel_1988}. $T_A$ 與 ${_B}T$ 投射維度相同. 
    \begin{proof}
        思路是說明, 所有極小 (餘) 消解 $ℓ (P^∙) ≥ ℓ (T^∙) = ℓ (h_T(T^∙)) ≥ ℓ (h_T(P^∙)) = ℓ (P^∙)$, 從而不等號取等. 故而只需證明如下問題:
        \begin{itemize}
            \item 若有足夠投射對象的 Abel 範疇存在 $p.\dim T =: d < ∞$, 且 $\mathrm{Ext}^{≥ 1}(T,T)=0$, 若另有 $X$ 存在有限 $𝐚𝐝𝐝 (T)$-餘消解, 則 $X$ 的餘消解維度 $≤ d$. 
        \end{itemize}
        取 $X$ 的 $𝐚𝐝𝐝 (T)$-餘消解, 由 $\mathrm{Ext}^{≥ 1}(T,T)=0$ 得維數位移 $\mathrm{Ext}^{i+1}(T, Ω^k) = \mathrm{Ext}^{i}(T, Ω^{k+1})$. 
        \begin{equation}
            % https://q.uiver.app/#q=WzAsOSxbMCwxLCIwIl0sWzEsMSwiTSJdLFszLDEsIlReMSJdLFs1LDEsIlReMiJdLFs3LDEsIlReMyJdLFsyLDAsIlxcT21lZ2EgXjAiXSxbNCwwLCJcXE9tZWdhIF4xIl0sWzYsMCwiXFxPbWVnYSBeMiJdLFs4LDEsIlxcY2RvdHMgIl0sWzAsMV0sWzEsMl0sWzIsM10sWzMsNF0sWzEsNV0sWzUsMl0sWzIsNl0sWzYsM10sWzMsN10sWzcsNF0sWzQsOF1d
\begin{tikzcd}[ampersand replacement=\&,sep=small]
	\&\& {\Omega ^0} \&\& {\Omega ^1} \&\& {\Omega ^2} \\
	0 \& M \&\& {T^1} \&\& {T^2} \&\& {T^3} \& {\cdots }
	\arrow[from=1-3, to=2-4]
	\arrow[from=1-5, to=2-6]
	\arrow[from=1-7, to=2-8]
	\arrow[from=2-1, to=2-2]
	\arrow[from=2-2, to=1-3]
	\arrow[from=2-2, to=2-4]
	\arrow[from=2-4, to=1-5]
	\arrow[from=2-4, to=2-6]
	\arrow[from=2-6, to=1-7]
	\arrow[from=2-6, to=2-8]
	\arrow[from=2-8, to=2-9]
\end{tikzcd}.
        \end{equation}
        如果極小餘消解維度 $l > d$ (至多 $T^l ≠ 0$), 則 $\mathrm{Ext}^1(T, Ω^{l-1}) = \mathrm{Ext}^{l-1}(T,Ω^1)=0$. 此時 $Ω^{l-1}↪ T^l$ 可裂, 與極小餘消解矛盾. 
    \end{proof}
\end{proposition}

\begin{theorem}
    取以上四种斜置模的極小 (餘) 消解, 記作 $(A, T, P^∙ , T^∙)$ 與 $(B^{\mathrm{op}}, T, h_T(T^∙), h_T(P^∙))$. 此時, 四條鏈長度相同. \parnote{四鏈等長}
\end{theorem}

\subsubsection{譜序列的應用}

\begin{notation}
此節記號: $T$ 是斜置右 $A$-模, 从而也是斜置左 $B = \mathrm{End}_A(T)$-模. 定义函子
\begin{enumerate}
    \item (右正合, 左伴随, 左导出) $G(-) := - ⊗_B T : 𝐦𝐨𝐝 _B → 𝐦𝐨𝐝 _A$; \parnote{$L_{-n} G$}
    \item (左正合, 右伴随, 右导出) $F(-) := (T, -)_A : 𝐦𝐨𝐝 _A → 𝐦𝐨𝐝 _B$. \parnote{$R^n F$}
\end{enumerate}
特别地, 取上述四种消解, 则 $L_{< -n}G$ 与 $R^{> n}F$ 消失. 
\end{notation}

\begin{remark}
    为了统一双复形朝向, 记左导出 $L_p =: L_{-p}$.
\end{remark}

% \begin{theorem}[Grothendieck 谱序列定理]
%     由于 $F$ 将投射对象映至 $G$-导出消失对象 (回顾 $\mathrm{Ext}^{≥ 1}(T,T)=0$). 此时存在函子的谱序列使得对任意 $M ∈ 𝐦𝐨𝐝 _A$, 
%     \begin{equation}
%         E_2 = L_{-p}G ∘ R^qF(M) ⇒ R^{p+q}(G ∘ F) = H^{p+q}(M). 
%     \end{equation}
%     考虑支撑, 则 $E_{n} = E_∞$. 特别地, 
% \end{theorem}

\begin{theorem}[$LR$-型 Grothendieck 谱序列]
    存在函子的谱序列使得对任意 $M ∈ 𝐦𝐨𝐝 _A$, 
    \begin{equation}
        E_2 = L_{-p}G ∘ R^qF(M) ⇒ H^{p+q}(M). 
    \end{equation}
    \begin{proof}
        取 $M$ 的内射分解 $M → I$, 並將 $G(-) ≃ (-) ⊗_B T$ 的分解選作 $(-) ⊗ _B (T^∙, T)$. \parnote{$T^∙$ 是 $A$ 的 $T$-餘消解} 依照 
		\begin{equation}
			(T′, X)_A ⊗ _B (Y, T′)_A ≃ (X, Y)_A \quad T′ ∈ 𝐚𝐝𝐝 T. 
		\end{equation}
		此時有兩個同構的雙複形
\begin{equation}
	% https://q.uiver.app/#q=WzAsMjcsWzAsMCwiR0ZNIixbMzAsNjAsNjAsMV1dLFsxLDAsIkdGSV4wICIsWzE4MCw2MCw2MCwxXV0sWzIsMCwiR0ZJXjEiLFsxODAsNjAsNjAsMV1dLFszLDAsIkdGSV4yIixbMTgwLDYwLDYwLDFdXSxbMCwxLCJGTVxcb3RpbWVzIChUXjAsIFQpIixbMTgwLDYwLDYwLDFdXSxbMCwyLCJGTVxcb3RpbWVzIChUXjEsIFQpIixbMTgwLDYwLDYwLDFdXSxbMCwzLCJGTVxcb3RpbWVzIChUXjIsIFQpIixbMTgwLDYwLDYwLDFdXSxbMSwxLCJGSV4wXFxvdGltZXMgKFReMCwgVCkiXSxbMSwyLCJGSV4wXFxvdGltZXMgKFReMSwgVCkiXSxbMSwzLCJGSV4wXFxvdGltZXMgKFReMiwgVCkiXSxbMiwxLCJGSV4xXFxvdGltZXMgKFReMCwgVCkiXSxbMiwyLCJGSV4xXFxvdGltZXMgKFReMSwgVCkiXSxbMiwzLCJGSV4xXFxvdGltZXMgKFReMiwgVCkiXSxbMywxLCJGSV4yXFxvdGltZXMgKFReMCwgVCkiXSxbMywyLCJGSV4yXFxvdGltZXMgKFReMSwgVCkiXSxbMywzLCJGSV4yXFxvdGltZXMgKFReMiwgVCkiXSxbMSw0LCIoVF4wLEleMCkiXSxbMSw1LCIoVF4xLEleMCkiXSxbMiw0LCIoVF4wLEleMSkiXSxbMyw0LCIoVF4wLEleMikiXSxbMiw1LCIoVF4xLEleMSkiXSxbMyw1LCIoVF4xLEleMikiXSxbMSw2LCIoVF4yLEleMCkiXSxbMiw2LCIoVF4yLEleMSkiXSxbMyw2LCIoVF4yLEleMikiXSxbNCwzLCJcXGJveGVke0VfMH0iXSxbNCw2LCJcXGJveGVke0UnXzB9Il0sWzAsMSwiIiwyLHsiY29sb3VyIjpbMTgwLDYwLDYwXX1dLFsxLDIsIiIsMix7ImNvbG91ciI6WzE4MCw2MCw2MF19XSxbMiwzLCIiLDIseyJjb2xvdXIiOlsxODAsNjAsNjBdfV0sWzUsNCwiIiwyLHsiY29sb3VyIjpbMTgwLDYwLDYwXX1dLFs0LDAsIiIsMix7ImNvbG91ciI6WzE4MCw2MCw2MF19XSxbNywxLCIiLDIseyJzdHlsZSI6eyJib2R5Ijp7Im5hbWUiOiJkb3R0ZWQifX19XSxbOCw3XSxbOSw4XSxbMTAsMiwiIiwyLHsic3R5bGUiOnsiYm9keSI6eyJuYW1lIjoiZG90dGVkIn19fV0sWzExLDEwXSxbMTIsMTFdLFsxMywzLCIiLDAseyJzdHlsZSI6eyJib2R5Ijp7Im5hbWUiOiJkb3R0ZWQifX19XSxbMTQsMTNdLFsxNSwxNF0sWzQsNywiIiwxLHsic3R5bGUiOnsiYm9keSI6eyJuYW1lIjoiZG90dGVkIn19fV0sWzcsMTBdLFsxMCwxM10sWzUsOCwiIiwxLHsic3R5bGUiOnsiYm9keSI6eyJuYW1lIjoiZG90dGVkIn19fV0sWzgsMTFdLFsxMSwxNF0sWzYsOSwiIiwxLHsic3R5bGUiOnsiYm9keSI6eyJuYW1lIjoiZG90dGVkIn19fV0sWzksMTJdLFsxMiwxNV0sWzE2LDE4XSxbMTgsMTldLFsyMSwxOV0sWzI0LDIxXSxbMTcsMTZdLFsyMiwxN10sWzIyLDIzXSxbMjMsMjRdLFsxNywyMF0sWzIwLDIxXSxbMjAsMThdLFsyMywyMF0sWzI1LDI2LCJcXHNpbWVxICIsMCx7ImxldmVsIjoyfV0sWzYsNSwiIiwyLHsiY29sb3VyIjpbMTgwLDYwLDYwXX1dXQ==
\begin{tikzcd}[ampersand replacement=\&,sep=small]
	\textcolor{rgb,255:red,214;green,153;blue,92}{GFM} \& \textcolor{rgb,255:red,92;green,214;blue,214}{{GFI^0 }} \& \textcolor{rgb,255:red,92;green,214;blue,214}{{GFI^1}} \& \textcolor{rgb,255:red,92;green,214;blue,214}{{GFI^2}} \\
	\textcolor{rgb,255:red,92;green,214;blue,214}{{FM\otimes (T^0, T)}} \& {FI^0\otimes (T^0, T)} \& {FI^1\otimes (T^0, T)} \& {FI^2\otimes (T^0, T)} \\
	\textcolor{rgb,255:red,92;green,214;blue,214}{{FM\otimes (T^1, T)}} \& {FI^0\otimes (T^1, T)} \& {FI^1\otimes (T^1, T)} \& {FI^2\otimes (T^1, T)} \\
	\textcolor{rgb,255:red,92;green,214;blue,214}{{FM\otimes (T^2, T)}} \& {FI^0\otimes (T^2, T)} \& {FI^1\otimes (T^2, T)} \& {FI^2\otimes (T^2, T)} \& {\boxed{E_0}} \\
	\& {(T^0,I^0)} \& {(T^0,I^1)} \& {(T^0,I^2)} \\
	\& {(T^1,I^0)} \& {(T^1,I^1)} \& {(T^1,I^2)} \\
	\& {(T^2,I^0)} \& {(T^2,I^1)} \& {(T^2,I^2)} \& {\boxed{E'_0}}
	\arrow[color={rgb,255:red,92;green,214;blue,214}, from=1-1, to=1-2]
	\arrow[color={rgb,255:red,92;green,214;blue,214}, from=1-2, to=1-3]
	\arrow[color={rgb,255:red,92;green,214;blue,214}, from=1-3, to=1-4]
	\arrow[color={rgb,255:red,92;green,214;blue,214}, from=2-1, to=1-1]
	\arrow[dotted, from=2-1, to=2-2]
	\arrow[dotted, from=2-2, to=1-2]
	\arrow[from=2-2, to=2-3]
	\arrow[dotted, from=2-3, to=1-3]
	\arrow[from=2-3, to=2-4]
	\arrow[dotted, from=2-4, to=1-4]
	\arrow[color={rgb,255:red,92;green,214;blue,214}, from=3-1, to=2-1]
	\arrow[dotted, from=3-1, to=3-2]
	\arrow[from=3-2, to=2-2]
	\arrow[from=3-2, to=3-3]
	\arrow[from=3-3, to=2-3]
	\arrow[from=3-3, to=3-4]
	\arrow[from=3-4, to=2-4]
	\arrow[color={rgb,255:red,92;green,214;blue,214}, from=4-1, to=3-1]
	\arrow[dotted, from=4-1, to=4-2]
	\arrow[from=4-2, to=3-2]
	\arrow[from=4-2, to=4-3]
	\arrow[from=4-3, to=3-3]
	\arrow[from=4-3, to=4-4]
	\arrow[from=4-4, to=3-4]
	\arrow["{\simeq }", Rightarrow, from=4-5, to=7-5]
	\arrow[from=5-2, to=5-3]
	\arrow[from=5-3, to=5-4]
	\arrow[from=6-2, to=5-2]
	\arrow[from=6-2, to=6-3]
	\arrow[from=6-3, to=5-3]
	\arrow[from=6-3, to=6-4]
	\arrow[from=6-4, to=5-4]
	\arrow[from=7-2, to=6-2]
	\arrow[from=7-2, to=7-3]
	\arrow[from=7-3, to=6-3]
	\arrow[from=7-3, to=7-4]
	\arrow[from=7-4, to=6-4]
\end{tikzcd}.
\end{equation}
繼而計算雙向的譜序列. 
\begin{enumerate}
	\item 先保留 $↑$. 由 $(-, I^p)$ 是正合函子, 得譜序列
	\begin{equation}
		% https://q.uiver.app/#q=WzAsMjksWzAsMCwiR0ZNIixbMzAsNjAsNjAsMV1dLFsxLDAsIkdGSV4wICIsWzE4MCw2MCw2MCwxXV0sWzIsMCwiR0ZJXjEiLFsxODAsNjAsNjAsMV1dLFszLDAsIkdGSV4yIixbMTgwLDYwLDYwLDFdXSxbMCwxLCJGTVxcb3RpbWVzIChUXjAsIFQpIixbMTgwLDYwLDYwLDFdXSxbMCwyLCJGTVxcb3RpbWVzIChUXjEsIFQpIixbMTgwLDYwLDYwLDFdXSxbMCwzLCJGTVxcb3RpbWVzIChUXjIsIFQpIixbMTgwLDYwLDYwLDFdXSxbMSwxLCIoVF4wLEleMCkiXSxbMSwyLCIoVF4xLEleMCkiXSxbMiwxLCIoVF4wLEleMSkiXSxbMywxLCIoVF4wLEleMikiXSxbMiwyLCIoVF4xLEleMSkiXSxbMywyLCIoVF4xLEleMikiXSxbMSwzLCIoVF4yLEleMCkiXSxbMiwzLCIoVF4yLEleMSkiXSxbMywzLCIoVF4yLEleMikiXSxbNCwzLCJcXGJveGVke0VfMH0iXSxbMSw0LCIoSF4wKFQpLCBJXjAgKSJdLFsxLDUsIihIXjEoVCksIEleMCkiLFswLDAsNzUsMV1dLFsyLDQsIihIXjAoVCksIEleMSkiXSxbMyw0LCIoSF4wKFQpLCBJXjIpIl0sWzIsNSwiKEheMShUKSwgSV4xKSIsWzAsMCw3NSwxXV0sWzMsNSwiKEheMShUKSwgSV4yKSIsWzAsMCw3NSwxXV0sWzQsNSwiXFxib3hlZHtFXzF9Il0sWzAsNSwiXFx0ZXh0e+a2iOWksSF9IixbMCwwLDc1LDFdXSxbMSw2LCIoQSxNKSJdLFsyLDYsIjAiLFswLDAsNzUsMV1dLFszLDYsIjAiLFswLDAsNzUsMV1dLFs0LDYsIlxcYm94ZWR7RV8yfSJdLFswLDEsIiIsMix7ImNvbG91ciI6WzE4MCw2MCw2MF19XSxbMSwyLCIiLDIseyJjb2xvdXIiOlsxODAsNjAsNjBdfV0sWzIsMywiIiwyLHsiY29sb3VyIjpbMTgwLDYwLDYwXX1dLFs1LDQsIiIsMix7ImNvbG91ciI6WzE4MCw2MCw2MF19XSxbNCwwLCIiLDIseyJjb2xvdXIiOlsxODAsNjAsNjBdfV0sWzEyLDEwXSxbMTUsMTJdLFs4LDddLFsxMyw4XSxbMTEsOV0sWzYsNSwiIiwyLHsiY29sb3VyIjpbMTgwLDYwLDYwXX1dLFsxNCwxMV0sWzE3LDE5XSxbMTksMjBdLFsxOCwyMSwiIiwyLHsiY29sb3VyIjpbMCwwLDc1XX1dLFsyMSwyMiwiIiwyLHsiY29sb3VyIjpbMCwwLDc1XX1dXQ==
\begin{tikzcd}[ampersand replacement=\&,sep=small]
	\textcolor{rgb,255:red,214;green,153;blue,92}{GFM} \& \textcolor{rgb,255:red,92;green,214;blue,214}{{GFI^0 }} \& \textcolor{rgb,255:red,92;green,214;blue,214}{{GFI^1}} \& \textcolor{rgb,255:red,92;green,214;blue,214}{{GFI^2}} \\
	\textcolor{rgb,255:red,92;green,214;blue,214}{{FM\otimes (T^0, T)}} \& {(T^0,I^0)} \& {(T^0,I^1)} \& {(T^0,I^2)} \\
	\textcolor{rgb,255:red,92;green,214;blue,214}{{FM\otimes (T^1, T)}} \& {(T^1,I^0)} \& {(T^1,I^1)} \& {(T^1,I^2)} \\
	\textcolor{rgb,255:red,92;green,214;blue,214}{{FM\otimes (T^2, T)}} \& {(T^2,I^0)} \& {(T^2,I^1)} \& {(T^2,I^2)} \& {\boxed{E_0}} \\
	\& {(H^0(T), I^0 )} \& {(H^0(T), I^1)} \& {(H^0(T), I^2)} \\
	\textcolor{rgb,255:red,191;green,191;blue,191}{{\text{消失!}}} \& \textcolor{rgb,255:red,191;green,191;blue,191}{{(H^1(T), I^0)}} \& \textcolor{rgb,255:red,191;green,191;blue,191}{{(H^1(T), I^1)}} \& \textcolor{rgb,255:red,191;green,191;blue,191}{{(H^1(T), I^2)}} \& {\boxed{E_1}} \\
	\& {(A,M)} \& \textcolor{rgb,255:red,191;green,191;blue,191}{0} \& \textcolor{rgb,255:red,191;green,191;blue,191}{0} \& {\boxed{E_2}}
	\arrow[color={rgb,255:red,92;green,214;blue,214}, from=1-1, to=1-2]
	\arrow[color={rgb,255:red,92;green,214;blue,214}, from=1-2, to=1-3]
	\arrow[color={rgb,255:red,92;green,214;blue,214}, from=1-3, to=1-4]
	\arrow[color={rgb,255:red,92;green,214;blue,214}, from=2-1, to=1-1]
	\arrow[color={rgb,255:red,92;green,214;blue,214}, from=3-1, to=2-1]
	\arrow[from=3-2, to=2-2]
	\arrow[from=3-3, to=2-3]
	\arrow[from=3-4, to=2-4]
	\arrow[color={rgb,255:red,92;green,214;blue,214}, from=4-1, to=3-1]
	\arrow[from=4-2, to=3-2]
	\arrow[from=4-3, to=3-3]
	\arrow[from=4-4, to=3-4]
	\arrow[from=5-2, to=5-3]
	\arrow[from=5-3, to=5-4]
	\arrow[color={rgb,255:red,191;green,191;blue,191}, from=6-2, to=6-3]
	\arrow[color={rgb,255:red,191;green,191;blue,191}, from=6-3, to=6-4]
\end{tikzcd}.
	\end{equation}
	從 $E_0$ 至 $E_1$: 內射模給出的 $(-,I)$ 是正合函子, 從而與同調群交換. 而 $H^∙ (T) = A$, $H^∙ (I)=M$, 故 $\boxed{E_2}$ 只留下 $M$ 一項. 從而全復形的濾過上同調是 $H^0 = M$ 與 $H^≠ 0 = 0$. 
	\item 繼而計算 $→$. 由 $(T^p, T)$ 是投射 $B$-模, 得 $⊗ (T^p, T)$ 是正合函子. 此時有 
	\begin{equation}
		% https://q.uiver.app/#q=WzAsNDAsWzAsMCwiR0ZNIixbMzAsNjAsNjAsMV1dLFsxLDAsIkdGSV4wICIsWzE4MCw2MCw2MCwxXV0sWzIsMCwiR0ZJXjEiLFsxODAsNjAsNjAsMV1dLFszLDAsIkdGSV4yIixbMTgwLDYwLDYwLDFdXSxbMCwxLCJGTVxcb3RpbWVzIChUXjAsIFQpIixbMTgwLDYwLDYwLDFdXSxbMCwyLCJGTVxcb3RpbWVzIChUXjEsIFQpIixbMTgwLDYwLDYwLDFdXSxbMCwzLCJGTVxcb3RpbWVzIChUXjIsIFQpIixbMTgwLDYwLDYwLDFdXSxbMSwxLCJGSV4wXFxvdGltZXMgKFReMCwgVCkiXSxbMSwyLCJGSV4wXFxvdGltZXMgKFReMSwgVCkiXSxbMSwzLCJGSV4wXFxvdGltZXMgKFReMiwgVCkiXSxbMiwxLCJGSV4xXFxvdGltZXMgKFReMCwgVCkiXSxbMiwyLCJGSV4xXFxvdGltZXMgKFReMSwgVCkiXSxbMiwzLCJGSV4xXFxvdGltZXMgKFReMiwgVCkiXSxbMywxLCJGSV4yXFxvdGltZXMgKFReMCwgVCkiXSxbMywyLCJGSV4yXFxvdGltZXMgKFReMSwgVCkiXSxbMywzLCJGSV4yXFxvdGltZXMgKFReMiwgVCkiXSxbNCwzLCJcXGJveGVke0VfMH0iXSxbMiw1LCIoUl4xIEZNKSBcXG90aW1lcyAoVF4wLCBUKSJdLFsyLDYsIihSXjEgRk0pIFxcb3RpbWVzIChUXjEsIFQpIl0sWzIsNywiKFJeMSBGTSkgXFxvdGltZXMgKFReMiwgVCkiXSxbMSw1LCJGTVxcb3RpbWVzIChUXjAsIFQpIl0sWzEsNiwiRk1cXG90aW1lcyAoVF4xLCBUKSJdLFsxLDcsIkZNXFxvdGltZXMgKFReMiwgVCkiXSxbMyw1LCIoUl4yIEZNKSBcXG90aW1lcyAoVF4wLCBUKSJdLFszLDYsIihSXjIgRk0pIFxcb3RpbWVzIChUXjEsIFQpIl0sWzMsNywiKFJeMiBGTSkgXFxvdGltZXMgKFReMiwgVCkiXSxbNCw3LCJcXGJveGVke0VfMX0iXSxbMSw4LCJHRk0iXSxbMSw0LCJGTSBcXG90aW1lcyBUIixbMTgwLDYwLDYwLDFdXSxbMiw0LCJSXjFGTSBcXG90aW1lcyBUIixbMTgwLDYwLDYwLDFdXSxbMyw0LCJSXjJGTSBcXG90aW1lcyBUIixbMTgwLDYwLDYwLDFdXSxbMiw4LCJHKFJeMUYpTSJdLFszLDgsIkcoUl4yRilNIl0sWzIsOSwiKExfey0xfUcpKFJeMUYpTSJdLFsxLDksIihMX3stMX1HKUZNIl0sWzMsOSwiKExfey0xfSlHKFJeMkYpTSJdLFsyLDEwLCIoTF97LTJ9RykoUl4xRilNIl0sWzEsMTAsIihMX3stMn1HKUZNIl0sWzMsMTAsIihMX3stMn1HKShSXjJGKU0iXSxbNCwxMCwiXFxib3hlZHtFXzJ9Il0sWzAsMSwiIiwyLHsiY29sb3VyIjpbMTgwLDYwLDYwXX1dLFsxLDIsIiIsMix7ImNvbG91ciI6WzE4MCw2MCw2MF19XSxbMiwzLCIiLDIseyJjb2xvdXIiOlsxODAsNjAsNjBdfV0sWzQsNywiIiwxLHsic3R5bGUiOnsiYm9keSI6eyJuYW1lIjoiZG90dGVkIn19fV0sWzcsMTBdLFsxMCwxM10sWzUsOCwiIiwxLHsic3R5bGUiOnsiYm9keSI6eyJuYW1lIjoiZG90dGVkIn19fV0sWzgsMTFdLFsxMSwxNF0sWzYsOSwiIiwxLHsic3R5bGUiOnsiYm9keSI6eyJuYW1lIjoiZG90dGVkIn19fV0sWzksMTJdLFsxMiwxNV0sWzI1LDI0XSxbMjQsMjNdLFsxOSwxOF0sWzE4LDE3XSxbMjIsMjFdLFsyMSwyMF0sWzM2LDI3XSxbMzgsMzFdXQ==
\begin{tikzcd}[ampersand replacement=\&, sep = small]
	\textcolor{rgb,255:red,214;green,153;blue,92}{GFM} \& \textcolor{rgb,255:red,92;green,214;blue,214}{{GFI^0 }} \& \textcolor{rgb,255:red,92;green,214;blue,214}{{GFI^1}} \& \textcolor{rgb,255:red,92;green,214;blue,214}{{GFI^2}} \\
	\textcolor{rgb,255:red,92;green,214;blue,214}{{FM\otimes (T^0, T)}} \& {FI^0\otimes (T^0, T)} \& {FI^1\otimes (T^0, T)} \& {FI^2\otimes (T^0, T)} \\
	\textcolor{rgb,255:red,92;green,214;blue,214}{{FM\otimes (T^1, T)}} \& {FI^0\otimes (T^1, T)} \& {FI^1\otimes (T^1, T)} \& {FI^2\otimes (T^1, T)} \\
	\textcolor{rgb,255:red,92;green,214;blue,214}{{FM\otimes (T^2, T)}} \& {FI^0\otimes (T^2, T)} \& {FI^1\otimes (T^2, T)} \& {FI^2\otimes (T^2, T)} \& {\boxed{E_0}} \\
	\& \textcolor{rgb,255:red,92;green,214;blue,214}{{FM \otimes T}} \& \textcolor{rgb,255:red,92;green,214;blue,214}{{R^1FM \otimes T}} \& \textcolor{rgb,255:red,92;green,214;blue,214}{{R^2FM \otimes T}} \\
	\& {FM\otimes (T^0, T)} \& {(R^1 FM) \otimes (T^0, T)} \& {(R^2 FM) \otimes (T^0, T)} \\
	\& {FM\otimes (T^1, T)} \& {(R^1 FM) \otimes (T^1, T)} \& {(R^2 FM) \otimes (T^1, T)} \\
	\& {FM\otimes (T^2, T)} \& {(R^1 FM) \otimes (T^2, T)} \& {(R^2 FM) \otimes (T^2, T)} \& {\boxed{E_1}} \\
	\& GFM \& {G(R^1F)M} \& {G(R^2F)M} \\
	\& {(L_{-1}G)FM} \& {(L_{-1}G)(R^1F)M} \& {(L_{-1})G(R^2F)M} \\
	\& {(L_{-2}G)FM} \& {(L_{-2}G)(R^1F)M} \& {(L_{-2}G)(R^2F)M} \& {\boxed{E_2}}
	\arrow[draw={rgb,255:red,92;green,214;blue,214}, from=1-1, to=1-2]
	\arrow[draw={rgb,255:red,92;green,214;blue,214}, from=1-2, to=1-3]
	\arrow[draw={rgb,255:red,92;green,214;blue,214}, from=1-3, to=1-4]
	\arrow[dotted, from=2-1, to=2-2]
	\arrow[from=2-2, to=2-3]
	\arrow[from=2-3, to=2-4]
	\arrow[dotted, from=3-1, to=3-2]
	\arrow[from=3-2, to=3-3]
	\arrow[from=3-3, to=3-4]
	\arrow[dotted, from=4-1, to=4-2]
	\arrow[from=4-2, to=4-3]
	\arrow[from=4-3, to=4-4]
	\arrow[from=7-2, to=6-2]
	\arrow[from=7-3, to=6-3]
	\arrow[from=7-4, to=6-4]
	\arrow[from=8-2, to=7-2]
	\arrow[from=8-3, to=7-3]
	\arrow[from=8-4, to=7-4]
	\arrow[from=11-3, to=9-2]
	\arrow[from=11-4, to=9-3]
\end{tikzcd}.
	\end{equation}
\end{enumerate}
以上證明了存在譜序列 $\mathrm{Tor}^B_{-q}(T, \mathrm{Ext}^p_A(T, -)) ⇒ δ_{p+q, 0}⋅ \mathrm{id}$. 
    \end{proof}
\end{theorem}

\begin{remark}
	一些粗淺的視角: 右上角處 
\begin{equation}
	% https://q.uiver.app/#q=WzAsOSxbMCwxLCJHKFJee24tMX1GKU0iLFszMCw2MCw2MCwxXV0sWzEsMSwiRyhSXm5GKU0iLFsxODAsNjAsNjAsMV1dLFswLDIsIihMX3stMX1HKShSXntuLTF9RilNIixbMCw2MCw2MCwxXV0sWzEsMiwiKExfey0xfSlHKFJebkYpTSIsWzE4MCw2MCw2MCwxXV0sWzAsMywiKExfey0yfUcpKFJee24tMX1GKU0iXSxbMSwzLCIoTF97LTJ9RykoUl5uRilNIixbMzAsNjAsNjAsMV1dLFsxLDQsIihMX3stM31HKShSXm5GKU0iLFswLDYwLDYwLDFdXSxbMCw0LCIoTF97LTN9RykoUl57bi0xfUYpTSJdLFswLDAsIjAiXSxbNSwwLCJcXHNpbWVxIiwxLHsiY29sb3VyIjpbMzAsNjAsNjBdfSxbMzAsNjAsNjAsMV1dLFs2LDIsIlxcdmFyZXBzaWxvbiAiLDEseyJjb2xvdXIiOlswLDYwLDYwXX0sWzAsNjAsNjAsMV1dLFswLDMsIjAiLDEseyJzdHlsZSI6eyJib2R5Ijp7Im5hbWUiOiJkb3R0ZWQifX19XSxbMiw1LCIwIiwxLHsic3R5bGUiOnsiYm9keSI6eyJuYW1lIjoiZG90dGVkIn19fV0sWzMsOCwiIiwxLHsiY29sb3VyIjpbMTgwLDYwLDYwXX1dLFs4LDEsIjAiLDEseyJjb2xvdXIiOlsxODAsNjAsNjBdLCJzdHlsZSI6eyJib2R5Ijp7Im5hbWUiOiJkb3R0ZWQifX19LFsxODAsNjAsNjAsMV1dXQ==
\begin{tikzcd}[ampersand replacement=\&]
	0 \\
	\textcolor{rgb,255:red,214;green,153;blue,92}{{G(R^{n-1}F)M}} \& \textcolor{rgb,255:red,92;green,214;blue,214}{{G(R^nF)M}} \\
	\textcolor{rgb,255:red,214;green,92;blue,92}{{(L_{-1}G)(R^{n-1}F)M}} \& \textcolor{rgb,255:red,92;green,214;blue,214}{{(L_{-1})G(R^nF)M}} \\
	{(L_{-2}G)(R^{n-1}F)M} \& \textcolor{rgb,255:red,214;green,153;blue,92}{{(L_{-2}G)(R^nF)M}} \\
	{(L_{-3}G)(R^{n-1}F)M} \& \textcolor{rgb,255:red,214;green,92;blue,92}{{(L_{-3}G)(R^nF)M}}
	\arrow["0"{description}, color={rgb,255:red,92;green,214;blue,214}, dotted, from=1-1, to=2-2]
	\arrow["0"{description}, dotted, from=2-1, to=3-2]
	\arrow["0"{description}, dotted, from=3-1, to=4-2]
	\arrow[color={rgb,255:red,92;green,214;blue,214}, from=3-2, to=1-1]
	\arrow["\simeq"{description}, color={rgb,255:red,214;green,153;blue,92}, from=4-2, to=2-1]
	\arrow["{\varepsilon }"{description}, color={rgb,255:red,214;green,92;blue,92}, from=5-2, to=3-1]
\end{tikzcd}.
\end{equation}
	得到以下兩則結果 (假定 $n > 3$): 
	\begin{enumerate}
		\item $\mathrm{Ext}_A^n(T, M) ⊗_B T = 0 = \mathrm{Tor}_1^B(\mathrm{Ext}_A^n(T, M), T)$; 
		\item $\mathrm{Tor}_2^B(\mathrm{Ext}_A^n(T, M), T) ≃ \mathrm{Ext}^{n-1}(T, M) ⊗ T$; 
		\item $(L_{-3}G)(R^nF)M \overset ε ↠ (L_{-1}G)(R^{n-1}F)M → H^{n-2} = 0$ 給出滿射 $ε$.
	\end{enumerate}
\end{remark}

\begin{proposition}[$RL$-型 Grothendieck 譜序列]
	對 $𝐦𝐨𝐝_B → 𝐦𝐨𝐝_A → 𝐦𝐨𝐝_B$ 型的函子, 有收斂 
	\begin{equation}
		\mathrm{Ext}_A^p(T, \mathrm{Tor}^B_{-q}(-, T)) ⇒ δ_{p+q, 0} ⋅ \mathrm{id}. 
	\end{equation}
	\begin{proof}
		證明略. 對 $X ⊗_B (P,T)_A ≃ (P, X ⊗_B T)_A$ 計算譜序列即可. 
	\end{proof}
\end{proposition}

\begin{example}[特例: $n=1$, 通常的斜置理論]
	混合係數公式來自譜序列: 
	\begin{equation}
		% https://q.uiver.app/#q=WzAsMTYsWzAsMiwiR0ZNIl0sWzEsMiwiRyhSXjFGKU0iXSxbMSwzLCIoTF97LTF9RykoUl4xRilNIl0sWzAsMywiKExfey0xfUcpRk0iXSxbMCwxLCIwIl0sWzEsNCwiMCJdLFswLDAsIjAiXSxbMSw1LCIwIl0sWzMsMiwiRkdOIl0sWzMsMywiKFJeMUYpR04iXSxbNCwzLCIoUl4xRikoTF97LTF9RylOIl0sWzQsMiwiRihMX3stMX1HKU4iXSxbNCw0LCIwIl0sWzMsMSwiMCJdLFszLDAsIjAiXSxbNCw1LCIwIl0sWzIsNF0sWzUsMF0sWzEsNl0sWzcsM10sWzQsMSwiMCIsMSx7InN0eWxlIjp7ImJvZHkiOnsibmFtZSI6ImRvdHRlZCJ9fX1dLFswLDIsIk0iLDEseyJzdHlsZSI6eyJib2R5Ijp7Im5hbWUiOiJkb3R0ZWQifX19XSxbMyw1LCIwIiwxLHsic3R5bGUiOnsiYm9keSI6eyJuYW1lIjoiZG90dGVkIn19fV0sWzEwLDgsIk4iLDEseyJzdHlsZSI6eyJib2R5Ijp7Im5hbWUiOiJkb3R0ZWQifX19XSxbOCwxMl0sWzEzLDEwXSxbOSwxNV0sWzE0LDExXSxbMTEsMTMsIiIsMSx7InN0eWxlIjp7ImJvZHkiOnsibmFtZSI6ImRvdHRlZCJ9fX1dLFsxMiw5LCIiLDEseyJzdHlsZSI6eyJib2R5Ijp7Im5hbWUiOiJkb3R0ZWQifX19XV0=
\begin{tikzcd}[ampersand replacement=\&, sep = small]
	0 \&\&\& 0 \\
	0 \&\&\& 0 \\
	GFM \& {G(R^1F)M} \&\& FGN \& {F(L_{-1}G)N} \\
	{(L_{-1}G)FM} \& {(L_{-1}G)(R^1F)M} \&\& {(R^1F)GN} \& {(R^1F)(L_{-1}G)N} \\
	\& 0 \&\&\& 0 \\
	\& 0 \&\&\& 0
	\arrow[from=1-4, to=3-5]
	\arrow["0"{description}, dotted, from=2-1, to=3-2]
	\arrow[from=2-4, to=4-5]
	\arrow["M"{description}, dotted, from=3-1, to=4-2]
	\arrow[from=3-2, to=1-1]
	\arrow[from=3-4, to=5-5]
	\arrow[dotted, from=3-5, to=2-4]
	\arrow["0"{description}, dotted, from=4-1, to=5-2]
	\arrow[from=4-2, to=2-1]
	\arrow[from=4-4, to=6-5]
	\arrow["N"{description}, dotted, from=4-5, to=3-4]
	\arrow[from=5-2, to=3-1]
	\arrow[dotted, from=5-5, to=4-4]
	\arrow[from=6-2, to=4-1]
\end{tikzcd}.
	\end{equation}
\end{example}

\begin{example}[特例: $n=2$]
	此時 $E_3 = E_∞$. 特別地, 計算譜序列 
\begin{equation}
	% https://q.uiver.app/#q=WzAsMzEsWzEsMCwiR0ZNIl0sWzIsMCwiRyhSXjFGKU0iXSxbMiwxLCJcXGJveGVkeyhMX3stMX1HKShSXjFGKU19Il0sWzEsMSwiXFxib3hlZHsoTF97LTF9RylGTX0iXSxbMywyLCIoTF97LTJ9RykoUl4yRilNIl0sWzMsMSwiXFxib3hlZHsoTF97LTF9KUcoUl4yRilNfSJdLFszLDAsIlxcYm94ZWR7RyhSXjJGKU19Il0sWzIsMiwiKExfey0yfUcpKFJeMUYpTSJdLFsxLDIsIlxcYm94ZWR7KExfey0yfUcpRk19Il0sWzAsMiwiXFxib3hlZHtFXzJ9Il0sWzAsNiwiXFxib3hlZHtFX1xcaW5mdHl9Il0sWzEsNSwiXFxib3hlZHsoTF97LTJ9RylGTX0iLFswLDAsNzUsMV1dLFsxLDQsIlxcYm94ZWR7KExfey0xfUcpRk19IixbMCwwLDc1LDFdXSxbMiw0LCJcXGJveGVkeyhMX3stMX1HKShSXjFGKU19Il0sWzMsNCwiXFxib3hlZHsoTF97LTF9KUcoUl4yRilNfSIsWzAsMCw3NSwxXV0sWzMsMywiXFxib3hlZHtHKFJeMkYpTX0iLFswLDAsNzUsMV1dLFsyLDMsIlxcbWF0aHJte2Nva31cXCBhIixbMCwwLDc1LDFdXSxbMSwzLCJcXG1hdGhybXtjb2t9XFwgYiJdLFszLDUsIlxca2VyIGEiXSxbMiw1LCJcXGtlciBiIixbMCwwLDc1LDFdXSxbMSw2LCJcXG1hdGhybXtjb2t9XFwgYiJdLFsyLDYsIj8iXSxbMyw2LCJcXGJveGVkeyhMX3stMX1HKShSXjFGKU19Il0sWzIsOCwiXFxrZXIgYSJdLFsyLDcsIk0iXSxbMyw4LCIoTF97LTJ9RykoUl4yRilNIl0sWzQsOCwiRyhSXjFGKU0iXSxbMSw3LCIoTF97LTJ9RykoUl4xRilNIl0sWzEsOCwiR0ZNIl0sWzAsNywiRV9cXGluZnR5XnswfSJdLFswLDUsIlxcYm94ZWR7RV8zfSJdLFs0LDEsImEiLDEseyJzdHlsZSI6eyJoZWFkIjp7Im5hbWUiOiJlcGkifX19XSxbNywwLCJiIiwxLHsic3R5bGUiOnsidGFpbCI6eyJuYW1lIjoiaG9vayIsInNpZGUiOiJ0b3AifX19XSxbMjAsMjEsIiIsMSx7InN0eWxlIjp7InRhaWwiOnsibmFtZSI6Imhvb2siLCJzaWRlIjoidG9wIn19fV0sWzIxLDIyLCIiLDEseyJzdHlsZSI6eyJoZWFkIjp7Im5hbWUiOiJlcGkifX19XSxbMjEsMjQsIiIsMSx7InN0eWxlIjp7InRhaWwiOnsibmFtZSI6Imhvb2siLCJzaWRlIjoiYm90dG9tIn19fV0sWzIzLDI1LCIiLDEseyJzdHlsZSI6eyJ0YWlsIjp7Im5hbWUiOiJob29rIiwic2lkZSI6InRvcCJ9fX1dLFsyNSwyNiwiIiwxLHsic3R5bGUiOnsiaGVhZCI6eyJuYW1lIjoiZXBpIn19fV0sWzI4LDI3LCIiLDEseyJzdHlsZSI6eyJ0YWlsIjp7Im5hbWUiOiJob29rIiwic2lkZSI6ImJvdHRvbSJ9fX1dLFsyNCwyMywiIiwxLHsic3R5bGUiOnsiaGVhZCI6eyJuYW1lIjoiZXBpIn19fV0sWzI3LDIwLCIiLDEseyJzdHlsZSI6eyJoZWFkIjp7Im5hbWUiOiJlcGkifX19XSxbMzAsMTAsIiIsMSx7ImxldmVsIjoyLCJzdHlsZSI6eyJoZWFkIjp7Im5hbWUiOiJub25lIn19fV1d
\begin{tikzcd}[ampersand replacement=\&, sep = small]
	\& GFM \& {G(R^1F)M} \& {\boxed{G(R^2F)M}} \\
	\& {\boxed{(L_{-1}G)FM}} \& {\boxed{(L_{-1}G)(R^1F)M}} \& {\boxed{(L_{-1})G(R^2F)M}} \\
	{\boxed{E_2}} \& {\boxed{(L_{-2}G)FM}} \& {(L_{-2}G)(R^1F)M} \& {(L_{-2}G)(R^2F)M} \\
	\& {\mathrm{cok}\ b} \& \textcolor{rgb,255:red,191;green,191;blue,191}{{\mathrm{cok}\ a}} \& \textcolor{rgb,255:red,191;green,191;blue,191}{{\boxed{G(R^2F)M}}} \\
	\& \textcolor{rgb,255:red,191;green,191;blue,191}{{\boxed{(L_{-1}G)FM}}} \& {\boxed{(L_{-1}G)(R^1F)M}} \& \textcolor{rgb,255:red,191;green,191;blue,191}{{\boxed{(L_{-1})G(R^2F)M}}} \\
	{\boxed{E_3}} \& \textcolor{rgb,255:red,191;green,191;blue,191}{{\boxed{(L_{-2}G)FM}}} \& \textcolor{rgb,255:red,191;green,191;blue,191}{{\ker b}} \& {\ker a} \\
	{\boxed{E_\infty}} \& {\mathrm{cok}\ b} \& {?} \& {\boxed{(L_{-1}G)(R^1F)M}} \\
	{E_\infty^{0}} \& {(L_{-2}G)(R^1F)M} \& M \\
	\& GFM \& {\ker a} \& {(L_{-2}G)(R^2F)M} \& {G(R^1F)M}
	\arrow["b"{description}, hook, from=3-3, to=1-2]
	\arrow["a"{description}, two heads, from=3-4, to=1-3]
	\arrow[equals, from=6-1, to=7-1]
	\arrow[hook, from=7-2, to=7-3]
	\arrow[two heads, from=7-3, to=7-4]
	\arrow[hook', from=7-3, to=8-3]
	\arrow[two heads, from=8-2, to=7-2]
	\arrow[two heads, from=8-3, to=9-3]
	\arrow[hook', from=9-2, to=8-2]
	\arrow[hook, from=9-3, to=9-4]
	\arrow[two heads, from=9-4, to=9-5]
\end{tikzcd}.
\end{equation}
特別地, 以下三者等價: 
\begin{enumerate}
	\item $GF M → (L_{-2}G)(R^1F)M$ 是滿射; 
	\begin{itemize}
		\item $(T, M) ⊗ T → \mathrm{Tor}_2(\mathrm{Ext}^1(T, M), T)$ 
	\end{itemize}
	\item $GF M → (L_{-2}G)(R^1F)M$ 是同構; 
	\item $0 → (L_{-1}G)(R^1F) M → M → (L_{-2}G)(R^2F) → G(R^1F)M → 0$ 是四項正合列. 
\end{enumerate}
對偶命題略. \parnote{映射構造?}
\end{example}

\begin{definition}[導出垂直]
	原始版本的斜置理論中, $(ℱ,𝒯,𝒳,𝒴)$ 通過四個函子的 $\ker$ 定義. 今推廣 
	\begin{enumerate}
		\item $K^p(A) := ⋂_{0 ≤ k ≤ n}^{k ≠ p}\mathrm{Ext}_A^k(T, -)$; 
		\item $K_p(B) := ⋂_{0 ≤ k ≤ n}^{k ≠ p}\mathrm{Tor}^B_k(-, T)$. 
	\end{enumerate}
\end{definition}

\begin{theorem}
	存在全子範疇間的互逆函子 $R^tF : K^t (A) ≃ K_t (B) : L_{-t}G$. 
	\begin{proof}
		對 $M ∈ K^p(A)$, 由譜序列的濾過知 $E_2^{p,q} = (L_{-p}G)(R^qF)M$ 僅在 $t$-列非零, 這也蘊含 $E_2 = E_∞$. 此時
\begin{equation}
	(L_{-∙}G)(R^tF)(M) = [0 ∣ \cdots ∣ 0 ∣ M ∣ 0 ∣ \cdots ∣ 0 ]. 
\end{equation}
	這說明 $(R^tF)(M) ∈ K_p (B)$. 逆函子等顯然. 
	\end{proof}
\end{theorem}

\begin{example}[王憲鍾序列]
	王憲鍾序列是一類特殊的譜序列: $E_2$ 中僅有兩行非零. 這說明, 可以對某一 $E_2 = E_r$ 使用``小技巧'', 從而導出長正合列. 

	記 $K^{i,j}(A) = ⋂ _{0 ≤ k ≤ n}^{k ≠ i,j}\mathrm{Ext}^k_A(T, -)$, \parnote{$i < j$} 則 $E_2$ 中僅有兩縱列非零. 計算得 
	\begin{enumerate}
		\item 存在五項正合列 
		\begin{align}
			0 → (L_{-j+1}G)(R^j F)M → (L_{-i}G)(R^i F)M → M \qquad \\ 
			\qquad → (L_{-j}G)(R^j F)M → (L_{-i-1}G)(R^i F)M → 0
		\end{align}
		\item $(L_{p-j+1}G)(R^j F)M ≃ (L_{p-i}G)(R^i F)M$ 對 $p ≠ 0, -1$ 成立. 
		\item $(L_{-([0,n-(j-i)])}G)(R^iF)M$ 與 $(L_{-([(j-i),n])}G)(R^jF)M$ 或非零; 其餘 $(L_{-p}G)(R^qF)$ 必爲 $0$.  
	\end{enumerate}
\end{example}

\begin{remark}
	王憲鍾的生平參考 \cite{严志达}.
\end{remark}

\begin{example}[$(i,j) = (0,n)$]
	此時有函子圖 
	\begin{equation}
		% https://q.uiver.app/#q=WzAsNixbMCwxLCJLXnswLG59KEEpIl0sWzAsMCwiS18wKEIpIl0sWzAsMiwiS19uIChCKSJdLFszLDAsIkteMCAoQSkiXSxbMywyLCJLXm4oQSkiXSxbMywxLCJLX3swLG59KEIpIl0sWzAsMSwiKFQsLSkiXSxbMCwyLCJcXG1hdGhybXtFeHR9Xm4oVCwtKSIsMl0sWzEsMywiLSBcXG90aW1lcyBUIiwyLHsib2Zmc2V0IjozfV0sWzIsNCwiXFxtYXRocm17VG9yfV9uKC0sVCkiLDIseyJvZmZzZXQiOjN9XSxbNSwzLCItIFxcb3RpbWVzIFQiLDJdLFs1LDQsIlxcbWF0aHJte1Rvcn1fbigtLFQpIl0sWzMsMSwiKFQsLSkiLDIseyJvZmZzZXQiOjN9XSxbNCwyLCJcXG1hdGhybXtFeHR9Xm4oVCwtKSIsMix7Im9mZnNldCI6M31dLFswLDUsIj8/PyIsMSx7InN0eWxlIjp7InRhaWwiOnsibmFtZSI6ImFycm93aGVhZCJ9fX1dLFsxMiw4LCJcXGNvbmcgIiwxLHsic2hvcnRlbiI6eyJzb3VyY2UiOjIwLCJ0YXJnZXQiOjIwfSwic3R5bGUiOnsiYm9keSI6eyJuYW1lIjoibm9uZSJ9LCJoZWFkIjp7Im5hbWUiOiJub25lIn19fV0sWzEzLDksIlxcY29uZyAiLDEseyJzaG9ydGVuIjp7InNvdXJjZSI6MjAsInRhcmdldCI6MjB9LCJzdHlsZSI6eyJib2R5Ijp7Im5hbWUiOiJub25lIn0sImhlYWQiOnsibmFtZSI6Im5vbmUifX19XV0=
\begin{tikzcd}[ampersand replacement=\&]
	{K_0(B)} \&\&\& {K^0 (A)} \\
	{K^{0,n}(A)} \&\&\& {K_{0,n}(B)} \\
	{K_n (B)} \&\&\& {K^n(A)}
	\arrow[""{name=0, anchor=center, inner sep=0}, "{- \otimes T}"', shift right=3, from=1-1, to=1-4]
	\arrow[""{name=1, anchor=center, inner sep=0}, "{(T,-)}"', shift right=3, from=1-4, to=1-1]
	\arrow["{(T,-)}", from=2-1, to=1-1]
	\arrow["{???}"{description}, tail reversed, from=2-1, to=2-4]
	\arrow["{\mathrm{Ext}^n(T,-)}"', from=2-1, to=3-1]
	\arrow["{- \otimes T}"', from=2-4, to=1-4]
	\arrow["{\mathrm{Tor}_n(-,T)}", from=2-4, to=3-4]
	\arrow[""{name=2, anchor=center, inner sep=0}, "{\mathrm{Tor}_n(-,T)}"', shift right=3, from=3-1, to=3-4]
	\arrow[""{name=3, anchor=center, inner sep=0}, "{\mathrm{Ext}^n(T,-)}"', shift right=3, from=3-4, to=3-1]
	\arrow["{\cong }"{description}, draw=none, from=1, to=0]
	\arrow["{\cong }"{description}, draw=none, from=3, to=2]
\end{tikzcd}.
	\end{equation}
	\begin{pinked}
		$K^{0,n}(A)$ 與 $K_{0,n}(B)$ 有何聯繫? 
	\end{pinked}
\end{example}

\begin{example}[$j - i = Δ$]
	此時有函子圖
	\begin{equation}
		% https://q.uiver.app/#q=WzAsNyxbMywyLCJLXntpLGp9KEEpIl0sWzYsMiwiS197WzAsIG4gLSBcXERlbHRhIF19KEIpIl0sWzAsMiwiS197W1xcRGVsdGEgLG5dfShCKSJdLFszLDEsIktee2ktMSxqLTF9KEEpIl0sWzMsMywiS157aSsxLGorMX0oQSkiXSxbMywwLCJcXHZkb3RzICJdLFszLDQsIlxcdmRvdHMgIl0sWzAsMiwiXFxtYXRocm17RXh0fV5qKFQsLSkiLDFdLFswLDEsIlxcbWF0aHJte0V4dH1eaShULC0pIiwxXSxbMywyLCJcXG1hdGhybXtFeHR9XntqLTF9KFQsLSkiLDEseyJjdXJ2ZSI6NX1dLFszLDEsIlxcbWF0aHJte0V4dH1ee2ktMX0oVCwtKSIsMSx7ImN1cnZlIjotNX1dLFs0LDIsIlxcbWF0aHJte0V4dH1ee2orMX0oVCwtKSIsMSx7ImN1cnZlIjotNX1dLFs0LDEsIlxcbWF0aHJte0V4dH1ee2krMX0oVCwtKSIsMSx7ImN1cnZlIjo1fV0sWzMsMCwiPz8/IiwxLHsic3R5bGUiOnsidGFpbCI6eyJuYW1lIjoiYXJyb3doZWFkIn19fV0sWzAsNCwiPz8/IiwxLHsic3R5bGUiOnsidGFpbCI6eyJuYW1lIjoiYXJyb3doZWFkIn19fV0sWzUsMywiPz8/IiwxLHsic3R5bGUiOnsidGFpbCI6eyJuYW1lIjoiYXJyb3doZWFkIn19fV0sWzQsNiwiPz8/IiwxLHsic3R5bGUiOnsidGFpbCI6eyJuYW1lIjoiYXJyb3doZWFkIn19fV1d
\begin{tikzcd}[ampersand replacement=\&]
	\&\&\& {\vdots } \\
	\&\&\& {K^{i-1,j-1}(A)} \\
	{K_{[\Delta ,n]}(B)} \&\&\& {K^{i,j}(A)} \&\&\& {K_{[0, n - \Delta ]}(B)} \\
	\&\&\& {K^{i+1,j+1}(A)} \\
	\&\&\& {\vdots }
	\arrow["{???}"{description}, tail reversed, from=1-4, to=2-4]
	\arrow["{\mathrm{Ext}^{j-1}(T,-)}"{description}, curve={height=30pt}, from=2-4, to=3-1]
	\arrow["{???}"{description}, tail reversed, from=2-4, to=3-4]
	\arrow["{\mathrm{Ext}^{i-1}(T,-)}"{description}, curve={height=-30pt}, from=2-4, to=3-7]
	\arrow["{\mathrm{Ext}^j(T,-)}"{description}, from=3-4, to=3-1]
	\arrow["{\mathrm{Ext}^i(T,-)}"{description}, from=3-4, to=3-7]
	\arrow["{???}"{description}, tail reversed, from=3-4, to=4-4]
	\arrow["{\mathrm{Ext}^{j+1}(T,-)}"{description}, curve={height=-30pt}, from=4-4, to=3-1]
	\arrow["{\mathrm{Ext}^{i+1}(T,-)}"{description}, curve={height=30pt}, from=4-4, to=3-7]
	\arrow["{???}"{description}, tail reversed, from=4-4, to=5-4]
\end{tikzcd}.
	\end{equation}
	\begin{pinked}
		此時諸 $K^{i-s, j-s}$ 有何聯繫? 
	\end{pinked}
\end{example}


\begin{example}[Happel 定理]
	对双模 $_BT_A$, 若
	\begin{enumerate}
		\item $B ≃ \mathrm{End}_A(T,T)$, \parnote{此處的題設}
		\item $T$ 有有限的 $𝐚𝐝𝐝(A)$-消解, 
		\item $A$ 有有限的 $𝐚𝐝𝐝(T)$-餘消解, 
		\item $\mathrm{Ext}_A^{≥ 1}(T,T) = 0$. 
	\end{enumerate}
	回憶兩則經典定理. 
	\begin{enumerate}
		\item (習題 5.10, \cite{2015三角范畴与导出范畴}) 給定 Abel 範疇 $𝒜$. 若對象 $X ∈ 𝒜$ 滿足 $\mathrm{Ext}^{≥ 1}(X,X)=0$, 則典範函子 $K^b(𝐚𝐝𝐝(M)) → D^b𝒜$ 是全忠實的. \parnote{見筆記, 歸納長度即可}
		\item (習題 5.4.1, \cite{2015三角范畴与导出范畴}) 假定 Abel 範疇有無限餘積和足夠投射對象, 則 $D^b𝒜 = D^b(𝒫(𝒜))$ 當且僅當 $𝒜$ 的整體維數有限, 亦當且僅當 $𝒜$ 中任意對象的投射維數有限. 
	\end{enumerate}
此時, $K^b (𝐚𝐝𝐝 (T)) → D^b(𝐦𝐨𝐝_A) = D^b(A)$ 是範疇等價. $(T,-)$ 與 $-⊗ T$ 給出了導出等價
\begin{equation}
	% https://q.uiver.app/#q=WzAsNixbMiwwLCJLXmIgKFxcbWF0aGJme2FkZH0oVCkpIl0sWzMsMCwiRF5iIChBKSJdLFsyLDEsIkteYiAoXFxtYXRoYmZ7YWRkfShCKSkiXSxbMywxLCJEXmIgKEIpIl0sWzAsMCwiVCJdLFswLDEsIihULCBUKSJdLFswLDEsIlxcc2ltZXEiLDAseyJzdHlsZSI6eyJ0YWlsIjp7Im5hbWUiOiJhcnJvd2hlYWQifX19XSxbMiwzLCJcXHNpbWVxICIsMix7InN0eWxlIjp7InRhaWwiOnsibmFtZSI6ImFycm93aGVhZCJ9fX1dLFs0LDUsIihULC0pX0EiLDIseyJvZmZzZXQiOjV9XSxbNSw0LCItXFxvdGltZXNfQiBUIiwyLHsib2Zmc2V0Ijo1fV0sWzAsMiwiXFxzaW1lcSAiLDAseyJzdHlsZSI6eyJ0YWlsIjp7Im5hbWUiOiJhcnJvd2hlYWQifX19XSxbMSwzLCJcXHNpbWVxICIsMCx7InN0eWxlIjp7InRhaWwiOnsibmFtZSI6ImFycm93aGVhZCJ9LCJib2R5Ijp7Im5hbWUiOiJkYXNoZWQifX19XSxbOCw5LCJcXGNvbmcgIiwxLHsic2hvcnRlbiI6eyJzb3VyY2UiOjIwLCJ0YXJnZXQiOjIwfSwic3R5bGUiOnsiYm9keSI6eyJuYW1lIjoibm9uZSJ9LCJoZWFkIjp7Im5hbWUiOiJub25lIn19fV1d
\begin{tikzcd}[ampersand replacement=\&]
	T \&\& {K^b (\mathbf{add}(T))} \& {D^b (A)} \\
	{(T, T)} \&\& {K^b (\mathbf{add}(B))} \& {D^b (B)}
	\arrow[""{name=0, anchor=center, inner sep=0}, "{(T,-)_A}"', shift right=5, from=1-1, to=2-1]
	\arrow["\simeq", tail reversed, from=1-3, to=1-4]
	\arrow["{\simeq }", tail reversed, from=1-3, to=2-3]
	\arrow["{\simeq }", dashed, tail reversed, from=1-4, to=2-4]
	\arrow[""{name=1, anchor=center, inner sep=0}, "{-\otimes_B T}"', shift right=5, from=2-1, to=1-1]
	\arrow["{\simeq }"', tail reversed, from=2-3, to=2-4]
	\arrow["{\cong }"{description}, draw=none, from=0, to=1]
\end{tikzcd}.
\end{equation}
\end{example}











\newpage







\section{First Section}
This document is an example of BibTeX using in bibliography management. Three items are cited: \textit{The \LaTeX\ Companion} book \cite{latexcompanion}, the Einstein journal paper \cite{einstein}, and the Donald Knuth's website \cite{knuthwebsite}. The \LaTeX\ related items are \cite{latexcompanion,knuthwebsite}. 
    

\newpage


\section{Terminologies (?)}

此處是部分中英文詞彙對照. % 插入文件 ``voc.csv'' 

\begin{center}
    \begin{tabular}{|| c | c | c | l ||}%
        \bfseries English & \bfseries 中文 & 出處 & 補註 % specify table head
        \csvreader[head to column names, separator=semicolon]{voc.csv}{}% use head of csv as column names
        {\\\hline\ \Eng & \CN & \REF & \NOT}% specify your columns here
        \end{tabular}
\end{center}

\medskip

%Sets the bibliography style to UNSRT and imports the 
%bibliography file "sample.bib".
\bibliographystyle{unsrt}
\bibliography{sample}

\end{document}