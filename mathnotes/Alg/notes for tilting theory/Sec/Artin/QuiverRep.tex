\begin{definition}[箭圖]
    略. 有限型, 有限, 以及 $𝐫𝐞𝐩(Q, I) ≃ 𝐦𝐨𝐝_{kQ/I}$ \parnote{日後補充}
\end{definition}

\begin{definition}[容許理想]\label{admissibleideal}
    容許理想 (Admissible ideal), 即增速可控的理想, 形如 $\mathrm{Rad}^2 ⊆ I ⊆ \mathrm{Rad}^k$ ($∃ n$). 
\end{definition}

\begin{remark}
    換言之, 生成關係是由``長 $≥2$ 的邊''決定. 
\end{remark}

\begin{theorem}[Artin 代數實現作 $(Q,I)$ 的方式]
    給定 $A$. 不妨設 $A$ 是基礎代數, 同時聯通. \parnote{造 $(kQ, I)$}
    \begin{enumerate}
        \item (點) 取正交冪等分解 $A = ⨁ A e_i$, 
        \item (邊) 找到 $\mathrm{Rad}(A) / \mathrm{Rad}^2(A)$ 的一組基, 
        \item (rel) 依照需求, 對前兩步決定的 quiver without rel 進行商. 
    \end{enumerate}
    特別地, 
    \begin{enumerate}
        \item 不可分解投射模 $P(i)$ 是 $i$ 出發的路代數;
        \item 不可分解內射模 $I(i)$ 是 $i$ 收尾的路代數 (投射的商);
        \item 單模 $S(i)$ 是 $i$ 單點所示的路代數; 
        \item $\mathrm{Rad}$ 是路理想 $kQ_1$-模.
    \end{enumerate}
\end{theorem}

\begin{theorem}[$𝐫𝐞𝐩(Q, I)$ 模結構]
    給定 $M ∈ 𝐫𝐞𝐩(Q, I)$, 則
    \begin{enumerate}
        \item 當且僅當 $M$ 半單, 則 $φ_α=0$ 對一切 $α ∈ Q_1$ 成立. \parnote{半單 $⟺$ 無邊}
        \item $\mathrm{Soc}(M)$ 在第 $a ∈ Q_0$ 位的分量是 $⋂_{f : a → ?} \ker (f)$. 特別地, $⋂_∅ = 0$. 
        \begin{pinked}
            極大單子對象, 即向外箭頭之公共核.
        \end{pinked}
        \item $\mathrm{Top}(M)$ 在第 $a ∈ Q_0$ 位的分量是 $⋀_{g : ? → a} \mathrm{coker}(g)$. 特別地, $⋀_∅ = \text{全}$. 此處商模之 $∧$ 即推出 (等價關係之并). $φ$ 由對象的取法誘導, 自然是 $0$.
        \begin{pinked}
            極大單商對象, 即向內箭頭之公共等價類.
        \end{pinked}
        \item $\mathrm{Rad}(M)$ 在第 $a ∈ Q_0$ 位的分量是 $∑_{h : ? → a} \mathrm{im}(h)$. 特別地, $∑_∅ = 0$.
        \begin{pinked}
            根即入勢之公共像, 亦可視爲 $kQ_1$ 誘導的東西.
        \end{pinked}
    \end{enumerate}
\end{theorem}

\begin{proposition}[擴張]
    $\dim \mathrm{Ext}^1 (S(a), S(b))$ 是 $a$ 至 $b$ 的邊數. 實現方式: 
    \begin{equation}
        \mathrm{Ext}{^1}(S(a), S(b)) ≃ \left([0 → S(b) → E → S(a) → 0] / ∼\right) ≃ e_a ⋅ \frac{\mathrm{Rad}(A)}{\mathrm{Rad}^2(A)} ⋅ e_b.
    \end{equation}
\end{proposition}












