\begin{abstract}
    總結 F.P. 模, 內射模, 平坦模, 投射模的等價定義(判准). 
    \begin{enumerate}
        \item (F.P.) 有限表現模的等價判據. $\mathrm{coker}(R^{n × m})$; $(M, -)$ 保持濾過餘極限; $M ⊗ -$ 保持積.
        \item (Plat.) 平坦模的等價判據. 
        \begin{enumerate}
            \item[a] $M ⊗ -$ 保單; $M ⊗ -$ 保 F.G. 模之單; $M ⊗ -$ 正合; $\mathrm{Tor}(M, - )=0$; 
            \item[b] $I ⊗ M ≃ IM$; $I ⊗ M ≃ IM$ 保 F.G. 理想之單; 
            \item[c] $Rm=0$ 蘊含 $R A =0$ 且 $m = A m′$; $0 → (A, M) → M^n → M^m$ 可補全作 ses $M^l → M^n → M^m$; 
            \item[d] $\text{F.P.} → M$ 定被 $R^n$ 分解; 自由模的濾過餘極限
        \end{enumerate}
        \item (特征模函子) $(-)^+$ 正合忠實; $X$ 的平坦維度等於 $X^+$ 的內射維度.
        \item (Inj.) 內射模的等價判據. 
        \begin{enumerate}
            \item[a] $(-, M)$ 正合, $\mathrm{Ext}^1(-, M)=0$; $\mathrm{Ext}^1(\text{商環},M)=0$;
            \item[b] 提升單射; 僅提升單射 $I ↪ R$; 本性擴張平凡; $M ↪ ?$ 可裂; 
            \item[c] $(R^{⊕})^+$ 直和項; $∏ R^+$ 的直和項. 
        \end{enumerate}
        \item (Proj.) 投射模的等價判據. 
        \begin{enumerate}
            \item[a] $(M, -)$ 正合, $\mathrm{Ext}^1(M, -)=0$; 
            \item[b] 提升滿射; $? ↠ M$ 可裂; 
            \item[c] $(R^{⊕})^+$ 直和項; 存在``投射基''; 
            \item[d] 投射模是可數生成投射模的直和. 
        \end{enumerate} 
    \end{enumerate}
\end{abstract}

\subsubsection{有限表現模的結構}

\begin{theorem}
    記 $I$ 濾過, $J$ 有限, $F: I × J → 𝐒𝐞𝐭𝐬$ 爲函子, 則典範態射 \parnote{濾過餘極限和極限交換}
    \begin{equation}\label{exchangelimcolim}
        \varinjlim_I \varprojlim_J F → \varprojlim_J \varinjlim_I F 
    \end{equation}
    是同構. 
    \begin{proof}
        §IX.2, \cite{lane1998categories} (GTM5).

        極限 (積的子) 和濾過餘極限 (有限歸納之等價類) 是兩大邏輯要件. 
    \end{proof}
\end{theorem}

\begin{remark}\label{LeX}
    若存在函子 $U : 𝒞 → 𝐒𝐞𝐭𝐬$, 其忠實, 保持濾過餘極限和小極限, 且返一切同構, 則上述交換定理對 $F: I × J → 𝒞$ 成立. 特例: 帶點集合, 環, 模等代數結構 (按 \cite{algstructure}). 
\end{remark}

\begin{definition}[緊對象]
    稱 $K$ 是範疇 $𝒞$ 之緊對象, 當且僅當 $(K, -)$ 與濾過餘極限交換. \parnote{緊 = 兼容歸納}
\end{definition}

\begin{remark}
    對 \ref{LeX} 中的 $(U, 𝒞)$, 應用式 \ref{exchangelimcolim} 知緊對象對有限極限封閉. 對於緊對象組成的範疇, 見 \cite{nlab:lex} 及其關聯內容.
\end{remark}

\begin{proposition}
    任意模是有限表現模的濾過餘極限. 
    \begin{proof}
        任意 $X$ 適用 ses: $0→ K→ R^{⊕ X}→ X→ 0$. 記濾過系統爲二元組 $\{(S, f)\}$, 其中 $S ⊆ X$ 是有限子集, $f : K_f → R^{⊕ S}$, $K_f$ 有限生成. 如此一來, 
        \begin{equation}
            \varinjlim_{(S, f)} (\mathrm{coker} (f)) = X
        \end{equation}
    \end{proof}
\end{proposition}

\begin{theorem}[A criterion for compact modules]
    以下條件等價. \parnote{f.p. 模刻畫}
    \begin{enumerate}
        \item $X$ 是有限表現模, 即生成元與生成關係有限的模; \parnote{f.p. 定義}
        \item $(X, -)$ 與濾過餘極限交換; \parnote{Hom 刻畫}
        \item $X ⊗ -$ 與任意積交換.  \parnote{$⊗$ 刻畫}
    \end{enumerate}
\begin{proof}
    順序 $1 ⟺ 2$, $1 ⟺ 3$. 
    \begin{enumerate}
        \item ($1 ⟹ 2$). 給定表現 $R^m → R^n → X → 0$, 由濾過餘極限和 Hom 的左正合性得
        \begin{equation}
            % https://q.uiver.app/#q=WzAsOCxbMCwwLCIwIl0sWzEsMCwiKFgsIFxcdmFyaW5qbGltICgtKSkiXSxbMiwwLCIoUl5uLCBcXHZhcmluamxpbSAoLSkpIl0sWzMsMCwiKFJebSwgXFx2YXJpbmpsaW0gKC0pKSJdLFswLDEsIjAiXSxbMSwxLCJcXHZhcmluamxpbSAoWCwgLSkiXSxbMiwxLCJcXHZhcmluamxpbSAoUl5uLCAtKSJdLFszLDEsIlxcdmFyaW5qbGltIChSXm0sIC0pIl0sWzAsMV0sWzEsMl0sWzIsM10sWzQsNV0sWzUsNl0sWzYsN10sWzMsNywiXFxzaW1lcSAiXSxbMiw2LCJcXHNpbWVxICJdLFsxLDVdXQ==
\begin{tikzcd}[ampersand replacement=\&,sep=small]
	0 \& {(X, \varinjlim (-))} \& {(R^n, \varinjlim (-))} \& {(R^m, \varinjlim (-))} \\
	0 \& {\varinjlim (X, -)} \& {\varinjlim (R^n, -)} \& {\varinjlim (R^m, -)}
	\arrow[from=1-1, to=1-2]
	\arrow[from=1-2, to=1-3]
	\arrow[from=1-2, to=2-2]
	\arrow[from=1-3, to=1-4]
	\arrow["{\simeq }", from=1-3, to=2-3]
	\arrow["{\simeq }", from=1-4, to=2-4]
	\arrow[from=2-1, to=2-2]
	\arrow[from=2-2, to=2-3]
	\arrow[from=2-3, to=2-4]
\end{tikzcd}.
        \end{equation}
        依照五引理, 得同構. 
        \item ($2 ⟹ 1$). 將緊模 $X$ 寫作有限表現模的濾過餘極限, 則
        \begin{equation}
            \mathrm{id}_X ∈ (X, X) = (X, \varinjlim K) ≃ \varinjlim (X, K). 
        \end{equation}
        因此, $\mathrm{id}_X$ 通過某一有限表現模分解. 這說明 $X$ 是有限表現模的直和項.
        \item ($1 ⟹ 3$) 給定表現 $R^m → R^n → X → 0$, 由 AB4* 和 $⊗$ 的右正合性得 
        \begin{equation}
            % https://q.uiver.app/#q=WzAsOCxbMCwxLCIoXFxwcm9kX2kgUl5tIFxcb3RpbWVzIE1faSkiXSxbMSwxLCIoXFxwcm9kX2kgUl5uIFxcb3RpbWVzIE1faSkiXSxbMiwxLCIoXFxwcm9kX2kgWCBcXG90aW1lcyBNX2kpIl0sWzMsMSwiMCJdLFswLDAsIlJebSBcXG90aW1lcyAoXFxwcm9kX2kgTV9pKSJdLFsxLDAsIlJebiBcXG90aW1lcyAoXFxwcm9kX2kgTV9pKSJdLFsyLDAsIlggXFxvdGltZXMgKFxccHJvZF9pIE1faSkiXSxbMywwLCIwIl0sWzAsMV0sWzEsMl0sWzIsM10sWzQsNV0sWzUsNl0sWzYsN10sWzQsMCwiXFxzaW1lcSAiXSxbNSwxLCJcXHNpbWVxICJdLFs2LDJdXQ==
\begin{tikzcd}[ampersand replacement=\&,sep=small]
	{R^m \otimes (\prod_i M_i)} \& {R^n \otimes (\prod_i M_i)} \& {X \otimes (\prod_i M_i)} \& 0 \\
	{(\prod_i R^m \otimes M_i)} \& {(\prod_i R^n \otimes M_i)} \& {(\prod_i X \otimes M_i)} \& 0
	\arrow[from=1-1, to=1-2]
	\arrow["{\simeq }", from=1-1, to=2-1]
	\arrow[from=1-2, to=1-3]
	\arrow["{\simeq }", from=1-2, to=2-2]
	\arrow[from=1-3, to=1-4]
	\arrow[from=1-3, to=2-3]
	\arrow[from=2-1, to=2-2]
	\arrow[from=2-2, to=2-3]
	\arrow[from=2-3, to=2-4]
\end{tikzcd}.
        \end{equation}
        依照五引理, 得同構. 
        \item ($3 ⟹ 1$) 類似以上對 $\mathrm{id}_X$ 的操作, 考慮 
        \begin{equation}
            \mathrm{id}_X ∈ X^X ≃ ∏ _X (X ⊗ R) ≃ X ⊗ ∏ _XR ∋ ∑ x_i ⊗ f_i. 
        \end{equation}
        追從同構, 恆有 $x = ∑ x_i ⋅ f_i(x)$, 故 $X$ 有限生成. 考慮 ses $0 → K → R^l → X → 0$, 得 
        \begin{equation}
            % https://q.uiver.app/#q=WzAsMTAsWzAsMSwiMCJdLFsxLDEsIlxccHJvZCBfSyBLIl0sWzIsMSwiXFxwcm9kIF9LIFJebCAiXSxbMywxLCJcXHByb2QgX0sgWCJdLFs0LDEsIjAiXSxbMywwLCJYIFxcb3RpbWVzIFxccHJvZCBfSyBSIl0sWzIsMCwiUl5sICBcXG90aW1lcyBcXHByb2QgX0sgUiJdLFsxLDAsIksgXFxvdGltZXMgXFxwcm9kIF9LIFIiXSxbNCwwLCIwIl0sWzAsMCwiPyJdLFswLDFdLFsxLDJdLFsyLDNdLFszLDRdLFs3LDZdLFs2LDVdLFs1LDhdLFs5LDddLFs5LDAsIiIsMSx7InN0eWxlIjp7ImhlYWQiOnsibmFtZSI6ImVwaSJ9fX1dLFs2LDIsIlxcc2ltZXEgIl0sWzcsMSwiIiwxLHsic3R5bGUiOnsiYm9keSI6eyJuYW1lIjoiZGFzaGVkIn19fV0sWzUsMywiXFxzaW1lcSAiXV0=
\begin{tikzcd}[ampersand replacement=\&,sep=small]
	{?} \& {K \otimes \prod _K R} \& {R^l  \otimes \prod _K R} \& {X \otimes \prod _K R} \& 0 \\
	0 \& {\prod _K K} \& {\prod _K R^l } \& {\prod _K X} \& 0
	\arrow[from=1-1, to=1-2]
	\arrow[two heads, from=1-1, to=2-1]
	\arrow[from=1-2, to=1-3]
	\arrow[dashed, from=1-2, to=2-2]
	\arrow[from=1-3, to=1-4]
	\arrow["{\simeq }", from=1-3, to=2-3]
	\arrow[from=1-4, to=1-5]
	\arrow["{\simeq }", from=1-4, to=2-4]
	\arrow[from=2-1, to=2-2]
	\arrow[from=2-2, to=2-3]
	\arrow[from=2-3, to=2-4]
	\arrow[from=2-4, to=2-5]
\end{tikzcd}.
        \end{equation}
        由五引理, 虛線處爲滿, 從而 $K$ 有限生成. 
    \end{enumerate}
\end{proof}
\end{theorem}

\subsubsection{有限生成模的結構}


\subsubsection{內射模的結構}

\begin{theorem}[內射模的 Baer 判別]
    $M$ 是內射模, 當且僅當對任意理想的包含 $i : I ⊆ R$, 態射 \parnote{Baer, 內射模}
    \begin{equation}
        (i, M) : (R, M) → (I, M) ,\quad f ↦ f ∘ i
    \end{equation}
    是滿的. 換言之, 任意模同態 $I → M$ 通過 $i$ 分解. 
    \begin{proof}
        僅證 $⇐$. 茲檢驗 $M$ 對任意單態射 $L ↪ N$ 之提升性, 不妨視 $L$ 爲 $N$ 的子模. 記部分 $N$-子模構成的集合\parnote{$𝒮$: 能提升哪些擴張?}
\begin{equation}
    𝒮 := \{K ∣ \text{$L ⊆ K ⊆ N$ 爲子模的包含鏈, 且任意 $L → M$ 通過 $K$ 分解}\}. 
\end{equation}
顯然 $L ∈ 𝒮$ 非空. 任取極大元 $Q ∈ 𝒮$ (Zorn 引理), 只需證明 \parnote{後繼歸納}
\begin{itemize}
    \item 若 $Q ≠ N$, 則對存在 $(n ∈ N) ∧ (n ∉ Q)$ 使得 $(⟨ n ⟩ + Q) ∈ 𝒮$, 得矛盾.  
\end{itemize}
實際上, 任意選定 $n$, 記理想 $I := \{r ∣ n ⋅ r  ∈  Q\}$. 對任意 $φ: Q → M$, 定義 
\begin{equation}
    (⟨ n ⟩ + Q) → M , \quad (n ⋅ r + q) ↦ α (r) + φ (q); \quad \text{其中,} \ \begin{tikzcd}[ampersand replacement=\&, sep = small]
        I \& R \\
        Q \& M 
        \arrow[hook, from=1-1, to=1-2]
        \arrow["{n ⋅ }"', from=1-1, to=2-1]
        \arrow["{α }", dashed, from=1-2, to=2-2]
        \arrow["{φ }"', from=2-1, to=2-2]
    \end{tikzcd}.
\end{equation}
這說明提升至 $Q$ 的態射可進一步地提升至 $⟨ n ⟩ + Q$. 
    \end{proof}
    以上證明使用最簡單的超限歸納, 對後繼序數的證明是關鍵, 對極限序數的證明由 Zorn 引理一筆帶過. 
\end{theorem}

\begin{remark}
    Baer's criterion 的意圖非常簡單: 驗證內射模本需檢驗對``所有單射''之提升性, 現將``所有單射''簡化至``所有形如 $I ⊆ R$''的包含. \parnote{所有單 $⟹$ 特殊單}
\end{remark}

\begin{example}
    將``全體理想''換做``全體有限生成的理想'', 命題不正確. 提示: $R := ℝ[x_{≥ 0}]$, 構造略. 
\end{example}

\subsubsection{平坦模的結構}

\begin{proposition}
    一般記號 $A$ 環, $I$ 理想, $\{M,N,L\}$ 左模, $\{X,Y,Z\}$ 右模. 以下關於 $M$ 的表述等價. \parnote{Baer, 平坦模}
    \begin{enumerate}
        \item 對任意單射 $f : X ↪ Y$, 態射 $f ⊗ \mathrm{id}_M : X ⊗ M → Y ⊗ M$ 單. 
        \item 對任意有限生成模的單射 $f : X ↪ Y$, 態射 $f ⊗ \mathrm{id}_M : X ⊗ M → Y ⊗ M$ 單. \parnote{僅需看有限生成模}
        \item 對任意理想 $I ⊆ A$, 自然態射 $I ⊗ M → IM$ 是同構. 
        \item 對任意有限生成的理想 $I ⊆ A$, 自然態射 $I ⊗ M → IM$ 是同構. 
    \end{enumerate}
    \begin{proof}
        過程: $1 ⟺ 2$ 且 $3 ⟺ 4$, $3 ⟹ 2$, $1 ⟹ 4$. \parnote{強 $⟹$ 弱}
\begin{enumerate}
    \item ($1 ⟺ 2$). 顯然 $1 ⇒ 2$, 下證明 $2 ⇒ 1$: 檢驗 $∑ x_i ⊗ m_i$ 與 $∑ x_j ⊗ m_j$ 在 $f ⊗ \mathrm{id}_M$ 下的像相同與否, 僅需考慮 $f$ 在有限生成模 $⟨\{x_i\} ∪ \{x_j\}⟩$ 上的限制, 遂得證. 同理, $3 ⟺ 4$. 
    \item ($3 ⟹ 2$). 不妨設子模 $X ⊆ Y$ 通過添加一個生成元得到, 則存在理想 $I$ 使得 $Y / X ≅ A / I$. 任取提升 $\begin{tikzcd}[ampersand replacement=\&, sep = small]
        A \& {A/ I} \\
        Y \& {Y / X}
        \arrow[two heads, from=1-1, to=1-2]
        \arrow["{\varphi }"', dashed, from=1-1, to=2-1]
        \arrow["{\simeq }", from=1-2, to=2-2]
        \arrow[two heads, from=2-1, to=2-2]
    \end{tikzcd}$, 得滿射 $(φ , ⊆ ) : A ⊕ X → Y$, 進而補全滿態射的推出拉回方塊\parnote{理想 $⟹$ 模}
\begin{equation}
    % https://q.uiver.app/#q=WzAsMTYsWzIsMiwiQSBcXG9wbHVzIFgiXSxbMiwzLCJZIl0sWzIsNCwiMCJdLFsxLDMsIlgiXSxbMywzLCJBIC8gSSJdLFsxLDIsIlgiXSxbMywyLCJBIl0sWzMsMSwiSSJdLFszLDQsIjAiXSxbNCwzLCIwIl0sWzAsMywiMCJdLFszLDAsIjAiXSxbNCwyLCIwIl0sWzAsMiwiMCJdLFsyLDEsIkkiXSxbMiwwLCIwIl0sWzAsMSwiKFxcdmFycGhpICwgXFxzdWJzZXRlcSApIiwyXSxbMSwyXSxbNSwwLCJcXGJpbm9tIDAxIl0sWzAsNiwiKDEsMCkiXSxbNSwzLCIiLDIseyJsZXZlbCI6Miwic3R5bGUiOnsiaGVhZCI6eyJuYW1lIjoibm9uZSJ9fX1dLFszLDEsIlxcc3Vic2V0ZXEgIiwyXSxbMSw0XSxbMTAsM10sWzQsOV0sWzExLDddLFs3LDYsIiIsMix7InN0eWxlIjp7ImJvZHkiOnsibmFtZSI6ImRhc2hlZCJ9fX1dLFs2LDRdLFs0LDhdLFsxMyw1XSxbNiwxMl0sWzE1LDE0XSxbMTQsMCwiIiwxLHsic3R5bGUiOnsiYm9keSI6eyJuYW1lIjoiZGFzaGVkIn19fV0sWzE0LDcsIiIsMSx7ImxldmVsIjoyLCJzdHlsZSI6eyJoZWFkIjp7Im5hbWUiOiJub25lIn19fV1d
\begin{tikzcd}[ampersand replacement=\&, sep = small]
	\&\& 0 \& 0 \\
	\&\& I \& I \\
	0 \& X \& {A \oplus X} \& A \& 0 \\
	0 \& X \& Y \& {A / I} \& 0 \\
	\&\& 0 \& 0
	\arrow[from=1-3, to=2-3]
	\arrow[from=1-4, to=2-4]
	\arrow[equals, from=2-3, to=2-4]
	\arrow[dashed, from=2-3, to=3-3]
	\arrow[dashed, from=2-4, to=3-4]
	\arrow[from=3-1, to=3-2]
	\arrow["{\binom 01}", from=3-2, to=3-3]
	\arrow[equals, from=3-2, to=4-2]
	\arrow["{(1,0)}", from=3-3, to=3-4]
	\arrow["{(\varphi , \subseteq )}"', from=3-3, to=4-3]
	\arrow[from=3-4, to=3-5]
	\arrow[from=3-4, to=4-4]
	\arrow[from=4-1, to=4-2]
	\arrow["{\subseteq }"', from=4-2, to=4-3]
	\arrow[from=4-3, to=4-4]
	\arrow[from=4-3, to=5-3]
	\arrow[from=4-4, to=4-5]
	\arrow[from=4-4, to=5-4]
\end{tikzcd}.
\end{equation}
作用 $M ⊗ -$, 可裂短正合列 R3 ses, 依歸納假設 C4 ses. 遂有蛇引理模型
\begin{equation}
    % https://q.uiver.app/#q=WzAsMTcsWzIsMiwiTVxcb3BsdXMgKE0gXFxvdGltZXMgWCkiXSxbMiwzLCJNIFxcb3RpbWVzIFkiXSxbMiw0LCIwIl0sWzEsMywiTSBcXG90aW1lcyBYIl0sWzMsMywiTSBcXG90aW1lcyAoQSAvIEkpIl0sWzEsMiwiTSBcXG90aW1lcyBYIl0sWzMsMiwiTSJdLFszLDEsIk0gXFxvdGltZXMgSSJdLFszLDQsIjAiXSxbNCwzLCIwIl0sWzQsMiwiMCJdLFswLDIsIjAiXSxbMiwxLCJNIFxcb3RpbWVzIEkiXSxbMywwLCIwIl0sWzEsMSwiMCJdLFsyLDAsIj8iXSxbNCwxLCIwIl0sWzEsMl0sWzUsMywiIiwyLHsibGV2ZWwiOjIsInN0eWxlIjp7ImhlYWQiOnsibmFtZSI6Im5vbmUifX19XSxbMSw0XSxbNCw5XSxbNiw0XSxbNCw4XSxbMTEsNV0sWzYsMTBdLFsxMiwwLCIiLDEseyJzdHlsZSI6eyJib2R5Ijp7Im5hbWUiOiJkYXNoZWQifX19XSxbMTIsNywiIiwxLHsibGV2ZWwiOjIsInN0eWxlIjp7ImhlYWQiOnsibmFtZSI6Im5vbmUifX19XSxbMCwxXSxbNSwwXSxbMywxXSxbMCw2XSxbMTMsN10sWzcsNl0sWzE0LDEyXSxbMTQsNV0sWzE1LDEyXSxbMTUsMTMsIiIsMSx7InN0eWxlIjp7ImhlYWQiOnsibmFtZSI6ImVwaSJ9fX1dLFsxMywzLCIiLDEseyJzdHlsZSI6eyJib2R5Ijp7Im5hbWUiOiJkb3R0ZWQifX19XSxbNywxNl1d
\begin{tikzcd}[ampersand replacement=\&,sep=small]
	\&\& {?} \& 0 \\
	\& 0 \& {M \otimes I} \& {M \otimes I} \& 0 \\
	0 \& {M \otimes X} \& {M\oplus (M \otimes X)} \& M \& 0 \\
	\& {M \otimes X} \& {M \otimes Y} \& {M \otimes (A / I)} \& 0 \\
	\&\& 0 \& 0
	\arrow[two heads, from=1-3, to=1-4]
	\arrow[from=1-3, to=2-3]
	\arrow[from=1-4, to=2-4]
	\arrow[dotted, from=1-4, to=4-2]
	\arrow[from=2-2, to=2-3]
	\arrow[from=2-2, to=3-2]
	\arrow[equals, from=2-3, to=2-4]
	\arrow[dashed, from=2-3, to=3-3]
	\arrow[from=2-4, to=2-5]
	\arrow[from=2-4, to=3-4]
	\arrow[from=3-1, to=3-2]
	\arrow[from=3-2, to=3-3]
	\arrow[equals, from=3-2, to=4-2]
	\arrow[from=3-3, to=3-4]
	\arrow[from=3-3, to=4-3]
	\arrow[from=3-4, to=3-5]
	\arrow[from=3-4, to=4-4]
	\arrow[from=4-2, to=4-3]
	\arrow[from=4-3, to=4-4]
	\arrow[from=4-3, to=5-3]
	\arrow[from=4-4, to=4-5]
	\arrow[from=4-4, to=5-4]
\end{tikzcd}.
\end{equation}
因此 $M ⊗ X → M ⊗ Y$ 是單態射. 
\item ($1 ⟹ 4$). 顯然 $R ⊗ M ≃ M$, $IM ⊆ M$. 依假設, $I ⊗ M$ 視同 $R ⊗ M$ 的子模. 鑒於\parnote{模 $⟹$ 理想}
\begin{enumerate}
    \item $R ⊗ M → M, ∑ r_i ⊗ m_i ↦ ∑ r_im_i$ 映 $I ⊗ M$ 至 $IM$, 以及
    \item $M → R ⊗ M, m ↦ 1 ⊗ m$ 映 $IM$ 至 $I ⊗ M$, 
\end{enumerate}
因此 $R ⊗ M ≃ M$ 誘導了子對象同構 $IM ≃ I ⊗ M$. 
\end{enumerate}
    \end{proof}
\end{proposition}

\begin{proposition}[平坦模的計算性質 I]
    $M$ 是平坦模, 當且僅當對任意矩陣式 $𝐑 ⋅ \overrightarrow{𝐦} = \overrightarrow{\mathbf 0}$, 存在 $𝐀$ 使得 \parnote{零化式平凡}
    \begin{equation}
        𝐑 ⋅ \overrightarrow{𝐦} = 𝐑 ⋅ \underbracket{(𝐀  ⋅ \overrightarrow{𝐦 ′ })}\limits_{ = 𝐦 } =  \underbracket{(𝐑 ⋅𝐀)}\limits_{ = 𝐎}⋅ \overrightarrow{𝐦′} = \overrightarrow{\mathbf 0}
    \end{equation}
    \begin{proof}
        若 $M$ 平坦, 由 ses $0 → \ker (𝐑 ⋅ ) → R^m → R^n → 0$ 得 ses
        \begin{equation}
            0 → \ker (𝐑 ⋅ ) ⊗ M → M^m \xrightarrow{𝐑 ⋅ } M^n → 0. 
        \end{equation}
        由正合性, 任意 $\overrightarrow{𝐦 } ∈ \ker (R ⋅ )$ 總是形如 $𝐀 ⊗ \overrightarrow{𝐦 ′}$; 其中 $𝐀 ∈ \ker (𝐑 ⋅ )$, 即 $𝐑⋅ 𝐀 = 𝐎$. 

        反之, 若任意矩陣式 $𝐑 ⋅ \overrightarrow{𝐦} = \overrightarrow{\mathbf 0}$ 允許上述媒介 $𝐀$, 往證對有限生成理想 $I$ 總有 $IM ≃ I ⊗ M$ (同構繼承自 $M ≃ R ⊗ M$). 顯然 $I ⊗ M ↠  IM$, 下證難點 $I ⊗ M ↪ IM$: 若 $\overrightarrow 𝐚^T ⋅ \overrightarrow 𝐦 = 0$, 則存在 $𝐀$ 與 $\overrightarrow{𝐦 ′ }$ 使得 $\overrightarrow 𝐚^T ⋅ 𝐀 =\overrightarrow{\mathbf 0}^T$ 與 $\overrightarrow{𝐦} = 𝐀 ⋅ \overrightarrow{𝐦 ′ }$. 這意味著 
        \begin{equation}
            \text{原輸入} =  ∑ a ⊗ m = ∑ (aA) ⊗ m′ = 0.  
        \end{equation} 
        從而單. 
    \end{proof}
\begin{pinked}
    $𝐑 ⋅ \overrightarrow{𝐦}$ 是 $IM$ 的元素, 而二元組所在的等價類 $(𝐑, \overrightarrow{𝐦}) / ∼ _𝐀$ 是 $I ⊗ M$ 的元素. 平坦模的內蘊性質即``$∼ _A$ 與 $⋅$ 是相同的等價關係''. 
\end{pinked}
\end{proposition}

\begin{proposition}[平坦模的計算性質 II]
    $M$ 是平坦模, 當且僅當一切有限表現模出發的態射 $X → M$ 通過某一有限自由模分解. \parnote{Plat → (Libre) → F.P.}
    \begin{proof}
        假定 $M$ 平坦, $R^m \xrightarrow {B ⋅ } R^n → X → 0$ 正合. 態射 $φ : X → M$ 滿足以下性質
        \begin{enumerate}
            \item $φ$ 由 $\{x_i ↦ m_i\}$ 決定, 其中 $x_i$ 是生成元, 即, $e_i ∈ R^n$ 在 $X$ 中的像.  
            \item 正合列 $0 → (X, M) → M^n → M^m$ 說明向量 $φ ∼ (m_i) ∈ M^n$ 恰使 $B^T ⋅ (m_i)=0$. 
        \end{enumerate}
        取媒介 $A$ 使得 $B^T ⋅ A^T =O$ 且 $A^T⋅ (m′ _j) = m_i$. 此時 $(A⋅ ): R^n → R^l$ 即爲所求. 
        
    反之, 證明過程類似. 
    \end{proof}
\begin{pinked}
    以上給出``séquence plat''中 representable syzygy 的性質: 若 $0 → (X, M )→ M^n → M^m$, 則必然能填充作 
    \begin{equation}
        % https://q.uiver.app/#q=WzAsNCxbMCwwLCJGXmwgIl0sWzIsMCwiRl5uICJdLFs0LDAsIkZebSJdLFsxLDEsIihYLCBNKSJdLFswLDFdLFsxLDJdLFszLDEsIiIsMix7InN0eWxlIjp7InRhaWwiOnsibmFtZSI6Imhvb2siLCJzaWRlIjoidG9wIn19fV0sWzAsMywiIiwxLHsic3R5bGUiOnsiaGVhZCI6eyJuYW1lIjoiZXBpIn19fV1d
\begin{tikzcd}[ampersand replacement=\&]
	{F^l } \&\& {F^n } \&\& {F^m} \\
	\& {(X, M)}
	\arrow[from=1-1, to=1-3]
	\arrow[two heads, from=1-1, to=2-2]
	\arrow[from=1-3, to=1-5]
	\arrow[hook, from=2-2, to=1-3]
\end{tikzcd}.
    \end{equation}
\end{pinked} \parnote{函子語言表述}
\end{proposition}

\begin{theorem}[Lazard 定理, 見 \cite{BSMF_1969__97__81_0}]
    平坦模恰是有限自由模的濾過餘極限. 
    \begin{proof}
        顯然 $R^n ⊗ -$ 和濾過餘極限正合, 從而有限自由模的濾過餘極限是平坦模. 反之, 將平坦模寫作有限表現模的濾過餘極限, 將每一態射 $X → M$ 通過有限自由模分解 (濾過子系統), 仔細驗證子系統共尾 (cofinal) 即可. 
    \end{proof}
\end{theorem}

\begin{remark}
    同理, 平坦模恰是自由模 (resp. 投射模, 有限生成投射模) 的濾過餘極限. 
\end{remark}

\subsection{特征模理論}

\begin{definition}[特征模] 
    函子 $(-)^+ := (-, ℚ / ℤ )_ℤ: 𝐌𝐨𝐝_R → 𝐌𝐨𝐝_{R^{\mathrm{op}}}$ 映 $M$ 至特征模. 
\end{definition}

\begin{proposition}[特征模函子的性質]\label{ChaMod}
    $ℚ / ℤ$ 是內射餘生成子; 相應地, $(-) ^+$ 是正合的忠實函子. 作爲推論: 
    \begin{enumerate}
        \item $A = 0$ 當且僅當 $A^+ = 0$; \parnote{特征模更像線性泛函!}
        \item $f$ 單射 (resp. 滿射) 當且僅當 $f^+$ 滿射 (resp. 單射); 
        \item $θ$ 正合當且僅當 $θ ^+$ 正合; 
        \item $X$ 平坦當且僅當 $X^+$ 內射; \parnote{Lambek}
        \item $(X ⊗ ^L Y)^+ ≃ R\mathrm{Hom}(X, Y^+)$; 
        \item $X$ 的平坦維度等於 $X^+$ 的內射維度. 
    \end{enumerate}
\end{proposition}

\begin{example}[應用: 對偶商空間]
    子模 $N ⊆ M$ 誘導了單射 $(M / N)^+ → M^+$. 泛函 $f ∈ M^+$ 屬於該像, 當且僅當 $f|_N =0$;
\end{example}

\begin{theorem}[應用: 模範疇有足夠的多內射對象]
    即任意對象存是某內射模的子模. \parnote{Bare 定理}
    \begin{proof}
        考慮 $R^{⊕ M^+} ↠ M^+$, 得 $M ↪ M^{++} ↪ (R^{⊕ M^+})^+$.
    \end{proof}
\end{theorem}

\begin{theorem}[應用: 內射模結構定理]
    內射 $R$-模形如``自由 $R^{\mathrm{op}}$-模之特征模''的直和項. 
    \begin{proof}
        $⟹$ 由以上 $M ↪ (R^{⊕ M^+})^+$ 給出. 另一方向顯然. 
    \end{proof}
\end{theorem}

\subsection{投射模的結構定理 (簡單的 transfinite dévissage)}

\begin{definition}[$α$-濾過與分次化]
    給定序數 $α$, 稱 $M_α$ 是模 $M$ 關於 $α$ 的濾過, 當且僅當 
    \begin{enumerate}
        \item $M_0 = 0$; \parnote{初始}
        \item $M_γ ⊆ M_{γ+1}$ 是子模; \parnote{後繼}
        \item 對極限序數 $γ$, 有 $M_γ := ⋃_{δ < γ }M_δ$. \parnote{極限}
    \end{enumerate}
\end{definition}

\begin{example}[超限歸納的一些例子]
    \begin{adjustwidth}{20pt}{}
        \begin{proposition}[簡答例子]
            定義 $M$ 關於濾過 $M_α$ 的分次模爲 $\mathrm{Gr}(M, α) := ⨁_{γ < α } M_{γ +1 }/ M_{γ}$. 若對任意序數 $γ < α$, $M_γ$ 總是 $M_{γ+1}$ 的直和項, 則 $\mathrm{Gr}(M, α) ≃ M$. 
            \begin{proof}
                爲證明上述對任意序數成立, 只需考慮初始, 後繼, 極限三類序數. 
                \begin{enumerate}
                    \item $M_0 = 0 = \mathrm{Gr}(M_0, 0)$, 顯然. 
                    \item 若 $\mathrm{Gr}(M_β, β) ≃ M_β$, 則 
                    \begin{equation}
                        \mathrm{Gr}(M_{β +1}, β+1) ≃  \mathrm{Gr}(M_{β}, β)  ⊕ \frac{M_{β +1}}{M_β }  ≃ M_β ⊕ \frac{M_{β +1}}{M_β } ≃ M_{β +1 }. 
                    \end{equation}
                    \item 若命題對一切 $γ < β$ 成立, 則 
                    \begin{equation}
                        \mathrm{Gr}(M_β , β ) ≃ ⋃ _{γ < β} \mathrm{Gr}(M_γ  , γ) ≃⋃ _{γ < β} M_γ  ≃ M_β. 
                    \end{equation}
                \end{enumerate}
            \end{proof}
        \end{proposition}      
        \end{adjustwidth}

        \begin{adjustwidth}{20pt}{}
\begin{proposition}[可數生成模的結構]\label{cgMod}
    假定 $M = ⨁_{i ∈ I} M_i$ 是可數生成模的任意直和, $\{f_n\}_{n ∈ ℕ}$ 是可數個自同態. 則存在濾過 $\{M_γ\}_{γ ≤ β}$ 使得 
    \begin{enumerate}
        \item $M_{γ +1} = M_γ ⊕ ⨁ (\text{countably many $M_i$'s})$. \parnote{新添可數}
        \item 任意 $(γ , n)$, $f_n$ 總是限制在 $M_γ$ 上的自同態. \parnote{子-自同態}
    \end{enumerate}
    \begin{proof}
        依次構造 $0$, 後繼, 極限. 
        \begin{enumerate}
            \item $M_0 = 0$. 略. 
            \item ($β → β +1 $) 若 $M_{≤ β}$ 構造成功, 且 $M_{β} ≠ M$. 下構造 $M_{β +1 }$. 任取 $M_β$ 之外的可數生成直和項 $M^0$, 則 $∑ _{n ∈ ℕ}\mathrm{im}(f_n(M^0))$ 可數生成, 因此是某一可數生成直和項 $M^1$ 的子模. 類似地構造 $M^k$, 記 $M_{β +1} = ⨁_{k≥ 0} M^k$ 即可. 
            \item ($(< β) → β$) 無需構造, 僅需驗證自同態的歸納極限是自同態, 非常顯然. 
        \end{enumerate}
    \end{proof}
\end{proposition} 
\begin{remark}
    約定``可數生成''表示至多可數生成, 不必是無限的.  
\end{remark}
            \end{adjustwidth}

        \begin{adjustwidth}{20pt}{}
            \begin{proposition}[有趣的例子: 雙點集問題; 問題 7.12, \cite{PBMSet}; \cite{Bienias2013}]
                是否存在 $ℝ²$ 的子集 $A$, 其與任意直線恰交於兩點? 
                \begin{proof}
                    用序數 $α$ 標記 $ℝ²$ 的全體直線 $\{ℓ _β \}_{β < α}$, 並對應地構造 $\{A_β\}$, 滿足 
                    \begin{enumerate}
                        \item 對任意 $β < α$, $|A_β| ≤ 2$; 
                        \item 對任意 $β < α$, $⋃_{γ ≤  β} A_γ$ 不存在共線的三點; 
                        \item 對任意 $β < α$, $⋃_{γ ≤  β} A_γ$ 與 $ℓ _β$ 有且僅有兩個交點. 
                    \end{enumerate}
                    若能暢通無阻地進行後繼歸納和極限歸納, 則原命題爲真. 
                    \begin{enumerate}
                        \item ($β =0$). 任取 $ℓ _0$ 與 $A_0$ 即可, 無較話頭. 
                        \item ($β → β +1$). $ℓ _{β +1}$ (新線) 與 $⋃_{γ ≤  β} A_γ$ (舊集) 的交點數量只能爲 $\{0,1,2\}$. 
                        \begin{enumerate}
                            \item 若交點數爲 $2$, 則直接取 $A_{β  +1 } = ∅$.\parnote{基數定義爲序數的 $|-|$-等價類的極小元; 對無窮基數: \\ $λ + κ = \max{}$, \\ $λ ⋅ κ = {\max{}}$. } 
                            \item 若交點數爲 $1$, 則需有 $|A_{β  +1 }| = 1$. 新的點既要落在落在 $ℓ _{β +1}$ (available) 之中, 又要落在 $⋃_{γ ≤  β} A_γ$ 的``兩點成線集'' (non-available) 之外. 由於 $|\text{non-avali}|$ 不超過 $|β|$ 的加法乘法式, 從而不超過 $\max \{ℵ_0, |β|\}$. 不妨設 $|α|$ 是基數, 則 $|β| < |α|$. 這說明新點有處可落.  
                            \item 若交點數爲 $0$, 同理以上. 
                        \end{enumerate}
                        \item ($\{< β\} → β$, $β$ 是極限序數) 同上, $ℓ _{β}$ (新線) 與 $⋃_{γ  <  β} A_γ$ (全體舊集之並) 的交點數量只能爲 $\{0,1,2\}$. 
                        \begin{enumerate}
                            \item 若交點數爲 $2$, 則 $A_β = ∅$. 
                            \item 若交點數爲 $0, 1$, 參考以上對``新點有處可落''之證明. 
                        \end{enumerate}
                    \end{enumerate}
                \end{proof}
            \end{proposition}
        \end{adjustwidth}
\end{example}

\begin{proposition}[投射模的結構 I] 
    投射模是可數生成投射模的直和. 約定: 可數 = 至多可數. \parnote{I. Kaplansky}
    \begin{proof}
        投射模 $P$ 是自由模 $M$ 的直和項, 即某一冪等自同態 $e$ 的像. 對資料 $(M, \{\mathrm{id}, e\})$ 使用 \ref{cgMod}, 得 $P$ 的可數直和濾過. 
    \end{proof}
\end{proposition}

\begin{proposition}[投射模的結構 II]
    $P$ 是投射模, 當且僅當存在一組 $\{(e_i, f_i)\}_{i ∈ I} ⊆ P × (P, R)$, 使得\parnote{對偶基}
    \begin{equation}
        ∀ x ∈ R \  (x = ∑ _{i ∈ I} e_i ⋅ f_i(x)\quad \text{爲有限和}). 
    \end{equation}
\begin{proof}
    若有上述``基'', 則有態射
    \begin{enumerate}
        \item $R^{⊕ I} → P, \quad r_i ↦ e_i ⋅ r_i$; 
        \item $P → R^{⊕ I}, \quad p ↦ (f_i(p))_{i ∈ I}$. 
    \end{enumerate}
    由於 $P → R^{⊕ I} → P,\quad p ↦ e_i ⋅ f_i (p)$ 是恆等, 得 $P$ 是截面. \parnote{本質直和項}

    反之, 若 $P$ 是投射模, 依照直和項 $P → R^{⊕ I} → P$ 直接構造即可. 
\end{proof}
\end{proposition}

\begin{remark}
    對投射模, 似乎沒有很好的 Baer 判別法. 見 \cite{Trlifaj_2018}. 
\end{remark}
