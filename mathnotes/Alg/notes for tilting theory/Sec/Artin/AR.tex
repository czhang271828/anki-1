\begin{abstract}
    使用不可約態射, 左右極小幾乎可裂態射, AR 平移三種表述以刻畫\textbf{幾乎可裂短正合列}. 最後使用函子視角, 對已知的語言重述某些``不明所以然''的概念. 所有未給出的證明都能在 \cite{e1}, \cite{auslander1997representation} 中找到, 具體頁碼暫時從略. 
\end{abstract}

\subsubsection{範疇的 Rad, 不可約態射}

\begin{proposition}
    同環態射, 範疇之 $k$-函子保持 $\mathrm{Rad}$. 特別地, 有以下特例. \parnote{等同 \ref{Rad}}
    \begin{enumerate}
        \item $\mathrm{Rad}(X,X)$ 就是 $\mathrm{Rad}(\mathrm{End}(X))$. 
        \item 若 $\mathrm{End}(X)$ 與 $\mathrm{End}(Y)$ 是局部環, 則 $\mathrm{Rad}(X, Y)$ 恰是 $X → Y$ 的非同構. 
        \item 若 $\mathrm{End}(X)$ 與 $\mathrm{End}(Y)$ 是局部環, 且 $X ≇ Y$, 則 $\mathrm{Rad}(X, Y) = \mathrm{Hom}(X,Y)$.
    \end{enumerate}
簡單地說, $\mathrm{Rad}(X, Y) = \mathrm{Rad}(Y,Y) ∘ \mathrm{Hom}(X,Y) ∘ \mathrm{Rad}(X,X)$. 
\end{proposition}

\begin{remark}
    $\mathrm{End}(X)$ 局部 $⇒$ $X$ 不可分解; 反之, 通常需要 $𝒞$ Abel 以及 $\dim_k \mathrm{End}(X) < ∞$ 等條件. \parnote{$𝐦𝐨𝐝 _A$ 好}
\end{remark}

\begin{definition}[不可約態射]
    約定 $\mathrm{Rad} := \mathrm{Rad}_A := \mathrm{Rad}_{𝐦𝐨𝐝_A}$, 歸納地定義 
    \begin{enumerate}
        \item $\mathrm{Rad}^0 = \mathrm{Hom}$, $\mathrm{Rad}^1 = \mathrm{Rad}$, 以及
        \item $\mathrm{Rad}^{n+1} = \mathrm{Rad} ∘ \mathrm{Rad}^n$. \parnote{路}
    \end{enumerate}
\end{definition}

\begin{example}[$\mathrm{Rad}$ 的結構]
    定 $X$ 與 $Y$ 不可分解. 
    \begin{enumerate}
        \item 若 $X = Y$, 則 $\mathrm{Rad}(X, Y)$ 是 $\mathrm{Rad}(\mathrm{End}(X))$; 
        \item 若 $X ≇ Y$, 則 $\mathrm{Rad}(X, Y) = \mathrm{Hom}(X, Y)$; 
        \item 假定 $X$ 不可分解, 則 $X \xrightarrow f Y$ 非可裂單, 當且僅當任意 $X \xrightarrow f Y\xrightarrow g X$ 屬於 $\mathrm{Rad}(\mathrm{End}(X))$; \parnote{$\mathrm{Rad}$ 吸收非可裂單, 滿的特定復合}
        \item 假定 $Y$ 不可分解, 則 $X \xrightarrow f Y$ 非可裂滿, 當且僅當任意 $Y \xrightarrow g X\xrightarrow f Y$ 屬於 $\mathrm{Rad}(\mathrm{End}(Y))$; 
        \item $f ∈ (\mathrm{Rad}(X, Y) \backslash \mathrm{Rad}^2(X, Y))$ (稱作``不可約態射'') 具有以下性質: \parnote{不可約態射}
        \begin{enumerate}
            \item $f$ 既不是可裂單, 又不是可裂滿; 
            \item 對任何分解 $X → Z → Y$, 或前箭頭是可裂單, 或後箭頭是可裂滿.
        \end{enumerate}
        總之, 若 $X → Z$ 可裂單, 則 $X → Z → X$ 相當于回頭路. 不可約態射 = 不能分作兩段單向路. 
    \end{enumerate}
\end{example}

\begin{theorem}[不可約單 (滿) 態射的結構]\label{irrmonoepi}
    給定不可裂的正合列 $0 → L \overset f → M \overset g → N → 0$. 
    \begin{enumerate}
        \item $f$ 是不可約的, 當且僅當任意 $? → N$ 分解 $g$, 或被 $g$ 分解. 推論: $N$ 不可分解. \parnote{$g$ 幾乎可裂滿}
        \item $g$ 是不可約的, 當且僅當任意 $L → ?$ 分解 $f$, 或被 $f$ 分解. 推論: $L$ 不可分解. \parnote{$f$ 幾乎可裂單}
    \end{enumerate}
\end{theorem}

\begin{pinked}
    \begin{remark}
        幾乎可裂單 (滿): 僅次於可裂單 (滿). $a$ 的不可約, 對應 $b$ 的幾乎可裂性. 
    \end{remark}
\end{pinked}\parnote{交叉表述} 

\subsubsection{極小態射}

\begin{definition}[左 (右) 極小態射]
    大概會有
    \begin{equation}
        0 → L \xrightarrow{\text{左極小, 左幾乎可裂}} M \xrightarrow{\text{右極小, 右幾乎可裂}} N → 0.
    \end{equation}
    . 此 ses 僅供輔助記憶, 實際上未必有單, 滿的假定. 
    \begin{enumerate}
        \item 稱 $f : L → M$ 是左極小的, 若 $M$ 的自同構 $α$ 方使 $α ∘ f = f$. \parnote{若吸收了``大地方''某自同態, 則該自同態必爲同構}
        \item 稱 $g : M → N$ 是右極小的, 若 $M$ 的自同構 $α$ 方使 $g ∘ α = g$. 
        \item 稱 $f : L → M$ 是左幾乎可裂的, 若 $f$ 非可裂滿, 且任意非可裂滿 $L → ?$ 必經 $f$ 分解; \parnote{如幾乎可裂單}
        \item 稱 $g : M → N$ 是右幾乎可裂的, 若 $g$ 非可裂單, 且任意非可裂單 $? → N$ 必經 $g$ 分解;  \parnote{如幾乎可裂滿}
    \end{enumerate}
\end{definition}

\begin{proposition}
    對 Abel 範疇, 以下是左 (右) 幾乎可裂態射單 (滿) 的充要條件:
    \begin{enumerate}
        \item 若 $g : M → N$ 是右幾乎可裂態射, 則 $g$ 滿當且僅當 $N$ 非投射對象; 
        \item 若 $f : L → M$ 是左幾乎可裂態射, 則 $f$ 單當且僅當 $K$ 非投射對象. 
    \end{enumerate}
    \begin{proof}
        僅看第一者, 假定 $g$ 是右幾乎可裂態射. 若 $g$ 滿, 則 $g$ 非可裂滿, 從而 $N$ 非投射. 若 $g$ 非滿, 下證 $N$ 投射. 
        \begin{adjustwidth}{10pt}{}
            (使用反證法) 假定 $N$ 非投射, 則存在滿態射 $ψ : A ↠ B$ 使得 $N$ 無法提升之. 提升性等價於 $ψ$ 的拉回 (記作 $α$) 可裂. 由極小幾乎可裂的定義知, 存在 $φ$ 使得下圖交換: 
            \begin{equation}
% https://q.uiver.app/#q=WzAsNyxbMSwyLCJBIl0sWzIsMiwiQiJdLFsyLDEsIk4iXSxbMCwwLCJNIl0sWzEsMSwiXFxidWxsZXQiXSxbMywxLCJcXG1hdGhybXtjb2tlcn0oZikiXSxbNCwxLCIwIl0sWzAsMSwiXFxwc2kgIiwyLHsic3R5bGUiOnsiaGVhZCI6eyJuYW1lIjoiZXBpIn19fV0sWzMsMiwiZiIsMCx7ImN1cnZlIjotMn1dLFsyLDFdLFs0LDIsIlxcYWxwaGEgIiwwLHsic3R5bGUiOnsiYm9keSI6eyJuYW1lIjoiZGFzaGVkIn19fV0sWzQsMCwiIiwxLHsic3R5bGUiOnsiYm9keSI6eyJuYW1lIjoiZGFzaGVkIn19fV0sWzQsMSwiXFx0ZXh0e1BCfSIsMSx7InN0eWxlIjp7ImJvZHkiOnsibmFtZSI6Im5vbmUifSwiaGVhZCI6eyJuYW1lIjoibm9uZSJ9fX1dLFs0LDMsIlxcdmFycGhpICJdLFsyLDVdLFs1LDYsIlxcbmVxICIsMSx7InN0eWxlIjp7ImJvZHkiOnsibmFtZSI6Im5vbmUifSwiaGVhZCI6eyJuYW1lIjoibm9uZSJ9fX1dXQ==
\begin{tikzcd}[ampersand replacement=\&,sep=small]
	M \\
	\& \bullet \& N \& {\mathrm{coker}(f)} \& 0 \\
	\& A \& B
	\arrow["f", curve={height=-12pt}, from=1-1, to=2-3]
	\arrow["{\varphi }", from=2-2, to=1-1]
	\arrow["{\alpha }", dashed, from=2-2, to=2-3]
	\arrow[dashed, from=2-2, to=3-2]
	\arrow["{\text{PB}}"{description}, draw=none, from=2-2, to=3-3]
	\arrow[from=2-3, to=2-4]
	\arrow[from=2-3, to=3-3]
	\arrow["{\neq }"{description}, draw=none, from=2-4, to=2-5]
	\arrow["{\psi }"', two heads, from=3-2, to=3-3]
\end{tikzcd}.
            \end{equation}
            由 $α$ 是滿態射, $∙ → N → \mathrm{coker}(f)$ 亦滿. 這一復合是零態射. 從而 $\mathrm{coker}(f)=0$, 矛盾. 
        \end{adjustwidth}
    \end{proof}
\end{proposition}

\begin{example}
    對 Artin 代數的設定, 右極小幾乎可裂態射要麼是非投射模的投射蓋, 要麼是投射模 $P$ 的單態射 $\mathrm{Rad}(P) ↪ P$. 另一方向同理. \parnote{極小幾乎可裂, 二擇}
\end{example}

\begin{theorem}[不可約 v.s. 極小幾乎可裂]
    左 (右) 極小幾乎可裂態射的來源 (去向) 不可分解. 特別地, 
    \begin{enumerate}
        \item 若 $f : L → M$ 左極小幾乎可裂, 則 $f$ 不可約; 
        \item 若 $f : L → M$ 不可約 ($M ≠ 0$), 當且僅當 $f$ 可以補全作左極小幾乎可裂態射 $\binom{f}{g} : L → M ⊕ \overline M$; 
        \item 若 $g : M → N$ 不可約 ($N ≠ 0$), 當且僅當 $g$ 可以補全作右極小幾乎可裂態射 $(g,h) : M ⊕ \widetilde M → N$. 
    \end{enumerate}
    \begin{pinked}
        極小幾乎可裂 $⊆$ 不可約; 不可約可反向補全作極小幾乎可裂態射. 此處沒有單射滿射的限定!
    \end{pinked}
\end{theorem}

\subsubsection{AR 平移, \texorpdfstring{$τ$}{PDFstring}}

\begin{definition}[AR 轉置 $\mathrm{Tr}(-)$]\label{Tr}
    考慮極小投射表現 $M = \mathrm{coker}(P_1 \xrightarrow f P_0)$, 定義 $\mathrm{Tr}(M) := \mathrm{coker}(f^t)$. 
    \begin{equation}
        % https://q.uiver.app/#q=WzAsMTAsWzIsMCwiUF8xICJdLFszLDAsIlBfMCJdLFs0LDAsIk0iXSxbNSwwLCIwIl0sWzUsMSwiMCJdLFs0LDEsIk1edCAiXSxbMywxLCJQXzEgXnQiXSxbMiwxLCJQXzEgXnQiXSxbMSwxLCJcXG1hdGhybXtUcn0oTSkiXSxbMCwxLCIwIl0sWzAsMSwiZiJdLFsxLDJdLFsyLDNdLFs0LDVdLFs1LDZdLFs2LDcsImZedCIsMl0sWzcsOF0sWzgsOV1d
\begin{tikzcd}[ampersand replacement=\&,sep=small]
	\&\& {P_1 } \& {P_0} \& M \& 0 \\
	0 \& {\mathrm{Tr}(M)} \& {P_1 ^t} \& {P_0 ^t} \& {M^t } \& 0
	\arrow["f", from=1-3, to=1-4]
	\arrow[from=1-4, to=1-5]
	\arrow[from=1-5, to=1-6]
	\arrow[from=2-2, to=2-1]
	\arrow[from=2-3, to=2-2]
	\arrow["{f^t}"', from=2-4, to=2-3]
	\arrow[from=2-5, to=2-4]
	\arrow[from=2-6, to=2-5]
\end{tikzcd}.
    \end{equation}\parnote{$\mathrm{Tr}(\mathrm{cok}(f))$ → $\mathrm{cok}(f^t)$}
\end{definition}

\begin{proposition}[$\mathrm{Tr}$ 基本性質]
    $\mathrm{Tr}$ 與直和交換. 以下假定 $M$ 不可分解. 
    \begin{enumerate}
        \item 當且僅當 $M$ 投射, $\mathrm{Tr}(M) = 0$; 
        \item 若 $M$ 非投射, 則 $\mathrm{Tr}(\mathrm{Tr}(M)) ≃ M$ 是典範同構. 
    \end{enumerate}
\end{proposition}

\begin{pinked}
    \begin{remark}
        $\mathrm{Tr}(-)$ 暫時沒有函子性 (畢竟 $\mathrm{Tr}(P)=0$), 需要藉助穩定範疇. 
    \end{remark}
\end{pinked}

\begin{definition}[AR 平移]
    給定極小投射表現 $θ : P_1 → P_0 → M → 0$, 考慮 $D(θ^t)$, 得, 
    \begin{equation}
        0 → τ(M) → ν (P_1) → ν (P_0) → ν (M) → 0. 
    \end{equation}\parnote{$τ = D ∘ \mathrm{Tr}$}
    類似地, 對極小內射表現 $η : 0 → M → E_1 → E_0$, 考慮 $τ⁻¹ ; \mathrm{Tr} ∘ D$, 得 
    \begin{equation}
        0 → ν^{-1}(M) → ν ^{-1}(E_1) → ν ^{-1}(E_0) → τ ^{-1}(M) → 0 . 
    \end{equation}
\end{definition}

\begin{remark}
    中山函子對偶投射模與內射模; AR 平移對偶極小投射表現與極小內射表現? \parnote{待解釋...}
\end{remark}

\subsubsection{幾乎可裂短正合列}

\begin{theorem}
    給定不可裂的短正合列 
\begin{equation}
    0 → L \xrightarrow f M \xrightarrow g N → 0. 
\end{equation}
以下是幾乎可裂的等價命題. 
\begin{enumerate}
    \item $f$ 左極小幾乎可裂, 且 $g$ 右極小幾乎可裂; 
    \item $f$ 左極小幾乎可裂; $g$ 右極小幾乎可裂; \parnote{單邊}
    \item $f$ 左幾乎可裂, 且 $N$ 不可分解; $L$ 不可分解, 且 $g$ 右幾乎可裂. 
    \item $L$ 与 $N$ 不可分解, $f$ 与 $g$ 不可约. 
\end{enumerate}
\end{theorem}

\begin{proposition}\label{PreAR}
    函子性定理: 若 $0 → L₀ → M₀ → N₀ → 0$ 是另一幾乎可列短正合列, 则
\begin{equation}
    \text{正合列同构} ⟺ (L ≃ L₀) ⟺ (N ≃ N₀).
\end{equation}
實際上, 幾乎可裂短正合列必形如以下: \parnote{$τ (\mathrm{Top}(M)) = \mathrm{Rad}(M)$}
\begin{equation}
    0 → N → ? → τ ⁻¹ N → 0\quad (\mathrm{Ext}^1(τ ⁻¹ N, N) ≃ k).
\end{equation} 
\end{proposition}

\begin{theorem}[中項結構]
    假定 $L$ 不可約, 則 $L → ⨁ M_i^{⊕ n_i}$ 左極小幾乎可裂等價於以下同時成立.
    \begin{enumerate}
        \item 所有分量 $L →  M_i$ 屬於 $\mathrm{Rad}$. 
        \item 若有不可分解 $M′$ 使得 $\mathrm{Rad}(L, M′) ≠ 0$, 則 $M′$ 是直和項. \parnote{恰 Rad}
        \item 對任意 $i$, 以上 $n_i$ 個態射恰是 $\frac{\mathrm{Rad}(L, M_i)}{\mathrm{Rad}^2(L, M_i)}$ 的一組基. 
    \end{enumerate}
\end{theorem}

\subsection{AR 大定理}

\begin{definition}[穩定範疇,穩定 $\mathrm{Hom}$]
    定義 $\underline{𝐦𝐨𝐝_A}$ 爲 $𝐦𝐨𝐝_A$ 的投射穩定範疇 (加法商範疇, 局部化範疇). 特別地, 
    \begin{equation}
        \underline{(X , Y)} = (X, Y) / \{\text{通過投射對象分解者}\}.
    \end{equation}
\end{definition}

\begin{theorem}[穩定等價大定理 I]
    $\mathrm{Tr}: \underline{𝐦𝐨𝐝 _A} → \underline{𝐦𝐨𝐝 _{A^{\mathrm{op}}}}$ 是穩定範疇的等價. \parnote{非投射單}
\end{theorem}

\begin{example}
    穩定 $\mathrm{Hom}$ 是賦值的餘核, 即以下正合 
    \begin{equation}
        Y ⊗ X^t \xrightarrow α  (X, Y) → \underline{(X, Y)} → 0;\quad α : [y ⊗ f] ↦ [x ↦ y⋅ f(x)]. 
    \end{equation}
\end{example}

\begin{proposition}[穩定等價大定理 II]
    對不可分解模 $M$ 與 $N$. 有穩定等價 
    \begin{equation}
        τ : \underline{𝐦𝐨𝐝_A} → \overline{𝐦𝐨𝐝_A} : τ ⁻¹ .
    \end{equation}
    \begin{enumerate}
        \item $τ M = 0$ 當且僅當 $M$ 投射; 
        \item 若 $M$ 不可約且非投射, 則 $τ M$ 不可約且非内射, 此時 $τ ⁻¹ τ M ≃ M$; 
        \item 作爲上一條的推論, $M ≃ M₀$ 當且僅當 $τ M ≃ τ M₀$;
        \item $τ⁻¹ N = 0$ 當且僅當 $N$ 内射; 
        \item 若 $N$ 不可約且非内射, 則 $τ⁻¹ M$ 不可約且非投射, 此時 $τ τ ⁻¹ N ≃ N$; 
        \item 作爲上一條的推論, $N ≃ N₀$ 當且僅當 $τ ⁻¹ N ≃ τ ⁻¹ N₀$. 
    \end{enumerate}
\end{proposition}

\begin{remark}
    $τ$ 將模向投射方向推移, $τ ⁻¹$ 將模向内射方向推移. 示意圖如下: 
    \begin{equation}
        % https://q.uiver.app/#q=WzAsNSxbMSwwLCJcXHRleHR75oqV5bCEfSJdLFszLDAsIlxcdGV4dHvlhaflsIR9Il0sWzIsMCwiXFxidWxsZXQiXSxbMCwwLCIwIl0sWzQsMCwiMCJdLFswLDIsIlxcdGF1XnstMX0iLDAseyJvZmZzZXQiOi01fV0sWzIsMSwiXFx0YXVeey0xfSIsMCx7Im9mZnNldCI6LTV9XSxbMSw0LCJcXHRhdV57LTF9IiwwLHsib2Zmc2V0IjotNX1dLFsxLDIsIlxcdGF1IiwwLHsib2Zmc2V0IjotNX1dLFsyLDAsIlxcdGF1IiwwLHsib2Zmc2V0IjotNX1dLFswLDMsIlxcdGF1IiwwLHsib2Zmc2V0IjotNX1dXQ==
\begin{tikzcd}[ampersand replacement=\&,sep=small]
	0 \& {\text{投射}} \& \bullet \& {\text{內射}} \& 0
	\arrow["\tau", shift left=5, from=1-2, to=1-1]
	\arrow["{\tau^{-1}}", shift left=5, from=1-2, to=1-3]
	\arrow["\tau", shift left=5, from=1-3, to=1-2]
	\arrow["{\tau^{-1}}", shift left=5, from=1-3, to=1-4]
	\arrow["\tau", shift left=5, from=1-4, to=1-3]
	\arrow["{\tau^{-1}}", shift left=5, from=1-4, to=1-5]
\end{tikzcd}.
    \end{equation}
\end{remark}

\begin{theorem}[AR 引理]
    有函子同構 $\mathrm{Ext}^1 (M, N) ≃ D\underline{(τ ⁻¹ N, M)} ≃ D\overline {(N , τ M)}$. 特別地, \parnote{摘線?}
    \begin{enumerate}
        \item 若 $p.\dim M ≤ 1$, 則可以摘掉上劃綫, 得 $\mathrm{Ext}^1 (M, N) ≃ D {(N , τ M)}$; 
        \item 若 $i.\dim N ≤ 1$, 則可以摘掉下劃綫, 得 $\mathrm{Ext}^1 (M, N) ≃ D{(τ ⁻¹ N, M)}$; 
        \item 若 $p.\dim M ≤ 1$ 且 $i.\dim N ≤ 1$, 則 ${(τ ⁻¹ N, M)} ≃ {(N , τ M)}$; 
        \item 若 $p.\dim M ≤ 1$ 且 $i.\dim τ N ≤ 1$, 則 ${(N, M)} ≃ {(τ N , τ M)}$; 
        \item 若 $p.\dim τ ⁻¹ M ≤ 1$ 且 $i.\dim N ≤ 1$, 則 ${(τ ⁻¹ N, τ ⁻¹ M)} ≃ {(N , M)}$. 
    \end{enumerate}
\end{theorem}

\begin{theorem}[AR 大定理]
    回顧 \ref{PreAR} 中的正合列. 
    \begin{enumerate}
        \item 若 $M$ 是非投射的不可分解模, 則存在幾乎可列短正合列 $0 → τM → ? → M → 0$. 參考 $\mathrm{Ext}^1 (M, τ M ) ≃ D\underline{\mathrm{End}(M)}$.
        \item 若 $N$ 是非内射的不可分解模, 則存在幾乎可裂短正合列 $0 → N → ? → τ ⁻¹ N → 0$. 參考 $\mathrm{Ext}^1 (τ ⁻¹ N , N) ≃ D\overline {\mathrm{End}(N)}$. 
    \end{enumerate}
\end{theorem}

\subsubsection{函子角度}

\begin{definition}
    $A  ∈ 𝐚𝐥𝐠 _k$, 默認 
\begin{enumerate}
    \item $\mathrm{Fun}^{\mathrm{op}}(A) :=\mathrm{Funct}_k (𝐦𝐨𝐝 _A^{\mathrm{op}}, 𝐦𝐨𝐝 _k)$ 是預層; \parnote{$(-, M)$}
    \item $\mathrm{Fun}(A) :=\mathrm{Funct}_k (𝐦𝐨𝐝 _A, 𝐦𝐨𝐝 _k)$ 是餘預層; \parnote{$(M, -)$}
\end{enumerate}
這些均是 $k$-範疇 ($k$-充實的 Abel 範疇).
\end{definition}

\begin{proposition}[單函子結構]
選定不可分解模 $M$, 相應的可表函子是函子範疇的不可分解投射對象. 
\begin{enumerate}
    \item $\mathrm{Fun}^{\mathrm{op}}(A)$ 的單函子恰形如 $S^M := (-, M) / \mathrm{Rad}(-, M)$; \parnote{均是投射蓋}
    \item $\mathrm{Fun}(A)$ 的單函子恰形如 $S_M := (M, -) / \mathrm{Rad}(M, -)$;
    \item $S^M(M) = S_M(M)$ 是 $\mathrm{End}(M)$ 的剩餘域; \parnote{示性函子}
    \item 若 $X ≇ M$ 是不可分解的, 則 $S^M(X) = 0$ 且 $S_M(X) = 0$. 
\end{enumerate}
\end{proposition}

\begin{proposition}[米田嵌入的伴隨]
    米田引理表明, 函子範疇的``(不可分解) 有限表現投射對象''恰好是``(不可分解) 可表對象''; 米田嵌入的右伴隨 (一向 $\mathrm{coker}(-, f) ↦ \mathrm{coker}(f)$) 有何性質? \parnote{單 → 不可分解}
    \begin{enumerate}
        \item $f$ (極小) 左幾乎可裂, 當且僅當 $(-,f)$ 誘導了 (極小) 投射表現. 此時,
        \begin{enumerate}
            \item $s(f) \xrightarrow f t(f) → \mathrm{coker}(f) → 0$ 正合, 且 $\mathrm{coker}(f)$ 不可分解;
            \item $(-,s(f)) \xrightarrow{(-,f)} (-, t(f)) → S^{t(f)} → 0$ 正合, 且 $\mathrm{coker} ((-,f))$ 是單對象. 
        \end{enumerate}
        \item $f$ (極小) 右幾乎可裂, 當且僅當 $(f,-)$ 誘導了 (極小) 投射表現. 此時,
        \begin{enumerate}
            \item $0 → \ker (f) → s(f) \xrightarrow f t(f)$ 正合, 且 $\ker (f)$ 不可分解; 
            \item $(t(f),-) \xrightarrow{(f,-)} (s(f),-) → S_{s(f)} → 0$ 正合, 且 $\ker ((f,-))$ 是單對象. 
        \end{enumerate}
    \end{enumerate}
\end{proposition}

