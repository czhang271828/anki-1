\begin{abstract}
    給出 $\mathrm{Ext}$ 與 $\mathrm{Tor}$ 的 Baer 和構造. 前者的系統性描述見 \cite{mitchell1965theory} 之 Chapter VII, 後者見 \cite{maclane2012homology} 之 Chapter five. 

    加入 satellite $\mathrm{Nat}(\mathrm{Ext}^n_A(X, - ), Y ⊗_A - ) ≃ \mathrm{Tor}_n^A(Y, X)$ 以串通之? 

    $\mathrm{Ext}$-部分照抄個人筆記. 茲將提綱暨主要結論悉列如下: 
    \begin{enumerate}
        \item 形式化定義雙函子 $\mathrm{Hom}(-,-)$ 的``加群''結構: $f + g := ∇ ∘ (f ⊕ g) ∘ Δ$; \parnote{真類亦可}
        \item 形式化定義雙函子 $\mathrm{Ext}^1(-,-)$ 的``加群''結構: $θ + τ = ∇ ∘ (θ ⊕ τ) ∘ Δ$; \parnote{擴張同構類} 
        \item $\mathrm{Ext}^n$ 的雙函子性, $\mathrm{Ext}$-乘法結構;  
        \item 連接態射 $δ^0$, $δ^{≥ 1}$ 保持長正合列 (附: $\mathrm{Ext}^1=0$, $\mathrm{Ext}^2=0$ 的充要條件); \parnote{$≥ 1$ 與 $1$ 證明相同, 歸納}
        \item 小技巧: $(3 × 3)$-模型的一對逆元 (推論: $\mathrm{Ext}^2$ 中零元恰補全作推出拉回); 
        \item $\mathrm{Ext}$-群的 zigzag 等價方式: 最多等價三次. 
        \item 假定局部小範疇, 足夠投射對象, 則以上與導出範疇相同. 
    \end{enumerate}
\end{abstract}

\begin{example}[$\mathrm{Ext}^0=\mathrm{Hom}$ 的双函子性]
    以下验证 $\mathrm{Hom}(-,-)$ 的双函子性 (从 Abel 范畴 $\mathcal A$ 至加群 (包括类) 范畴), 其包括两点: 一切 $\mathrm{Hom}(-,-)$ 是加群, 加法结构与来源, 去向的变换相容.
    \begin{enumerate}
        \item (定义 $(B,A)$ 的加法结构) 定义对角变换 $\Delta_X:X\xrightarrow{\binom 11} X\oplus X$ \parnote{$Δ$: 分開} 与余对角变换 $\nabla_X:X\oplus X\xrightarrow{(1\,\,1)} X$ \parnote{$∇$ 合上}. 既将其视作自然变换, 往后省略 $\Delta$ 与 $\nabla$ 的角标. 态射 $f,g\in (B,A)$ 的加法定义如下
              \begin{equation}
                  % https://q.uiver.app/#q=WzAsNyxbMCwyLCJCXFxvcGx1cyBCIl0sWzIsMiwiQVxcb3BsdXMgQSJdLFsyLDEsIkEiXSxbMCwxLCJCXFxvcGx1cyBCIl0sWzAsMCwiQiJdLFsyLDAsIkEiXSxbMywyLCJcXG1hdGhybXt9Il0sWzAsMSwiXFxiaW5vbXtmXFxxdWFkIH17XFxxdWFkIGd9Il0sWzIsMSwiXFxEZWx0YSAiXSxbMCwzLCIiLDAseyJsZXZlbCI6Miwic3R5bGUiOnsiaGVhZCI6eyJuYW1lIjoibm9uZSJ9fX1dLFszLDIsIihmXFxxdWFkIGcpIiwwLHsic3R5bGUiOnsiYm9keSI6eyJuYW1lIjoiZGFzaGVkIn19fV0sWzMsNCwiXFxuYWJsYSAiXSxbMiw1LCIiLDIseyJsZXZlbCI6Miwic3R5bGUiOnsiaGVhZCI6eyJuYW1lIjoibm9uZSJ9fX1dLFs0LDUsImZcXG9wbHVzIGciXV0=
                  \begin{tikzcd}[ampersand replacement=\&]
                      B \&\& A \\
                      {B\oplus B} \&\& A \\
                      {B\oplus B} \&\& {A\oplus A} \& {\mathrm{}}
                      \arrow["{f\oplus g}", from=1-1, to=1-3]
                      \arrow["{\nabla }", from=2-1, to=1-1]
                      \arrow["{(f\quad g)}", dashed, from=2-1, to=2-3]
                      \arrow[Rightarrow, no head, from=2-3, to=1-3]
                      \arrow["{\Delta }", from=2-3, to=3-3]
                      \arrow[Rightarrow, no head, from=3-1, to=2-1]
                      \arrow["{\binom{f\quad }{\quad g}}", from=3-1, to=3-3]
                  \end{tikzcd}.
              \end{equation}
              以算式记之, $(f+g)=\nabla (f\oplus g)\Delta$. 此处, $(f\oplus g):=\binom{f\quad }{\quad g}$.
        \item 显然 $0$ 是零元, $f$ 的逆元是 $-f$.
        \item 来源的变换 (拉回) $\beta^\ast : \mathrm{Hom}(B,A)\to \mathrm{Hom}(B',A),\quad f\mapsto f\circ \beta$ 是良定义的群同态, 去向的变换 (推出) 亦然.
    \end{enumerate}
\end{example}

\begin{remark}
    对 $\mathrm{Ext}^0=\mathrm{Hom}$ 而言, ``推出''与``拉回''是前向复合与后项复合; 对 $\mathrm{Ext}^{\geq 1}$ 而言, ``推出''与``拉回''由熟悉的 $2\times 2$ 方块构造. 这两个名词可以统一解释作 $\mathrm{Ext}$-群关于来源与去向的变换.
\end{remark}

\begin{definition}[短正合列的等价关系]
    考虑拆解短正合列的同态为如下四行正合列:
    \begin{equation}
        % https://q.uiver.app/#q=WzAsMjIsWzIsMCwiQSJdLFszLDAsIkIiXSxbNCwwLCJDIl0sWzIsMywiQSciXSxbMywzLCJCJyJdLFs0LDMsIkMnIl0sWzIsMiwiQSciXSxbMywyLCJFJyJdLFs0LDIsIkMiXSxbMiwxLCJBJyJdLFs0LDEsIkMiXSxbMywxLCJFIl0sWzEsMCwiMCJdLFsxLDEsIjAiXSxbMSwyLCIwIl0sWzEsMywiMCJdLFs1LDAsIjAiXSxbNSwxLCIwIl0sWzUsMiwiMCJdLFs1LDMsIjAiXSxbMCwwLCJbRV0iXSxbMCwzLCJbRSddIl0sWzgsNSwiXFxnYW1tYSJdLFs3LDRdLFs3LDhdLFs0LDVdLFswLDksIlxcYWxwaGEiLDJdLFswLDFdLFsxLDJdLFsyLDEwLCIiLDEseyJsZXZlbCI6Miwic3R5bGUiOnsiaGVhZCI6eyJuYW1lIjoibm9uZSJ9fX1dLFs5LDExXSxbMTEsMTBdLFsxLDExXSxbNiwzLCIiLDEseyJsZXZlbCI6Miwic3R5bGUiOnsiaGVhZCI6eyJuYW1lIjoibm9uZSJ9fX1dLFs5LDYsIiIsMSx7ImxldmVsIjoyLCJzdHlsZSI6eyJoZWFkIjp7Im5hbWUiOiJub25lIn19fV0sWzEwLDgsIiIsMSx7ImxldmVsIjoyLCJzdHlsZSI6eyJoZWFkIjp7Im5hbWUiOiJub25lIn19fV0sWzExLDddLFs2LDddLFszLDRdLFsxMiwwXSxbMTMsOV0sWzE0LDZdLFsxNSwzXSxbMiwxNl0sWzEwLDE3XSxbOCwxOF0sWzUsMTldLFsyMCwyMSwiKFxcYWxwaGEsXFxiZXRhLFxcZ2FtbWEpIiwxXSxbMjMsMjIsIlxcdGV4dHtQQn0iLDEseyJzaG9ydGVuIjp7InNvdXJjZSI6MjAsInRhcmdldCI6MjB9LCJzdHlsZSI6eyJib2R5Ijp7Im5hbWUiOiJub25lIn0sImhlYWQiOnsibmFtZSI6Im5vbmUifX19XSxbMjYsMzIsIlxcdGV4dHtQT30iLDEseyJzaG9ydGVuIjp7InNvdXJjZSI6MjAsInRhcmdldCI6MjB9LCJzdHlsZSI6eyJib2R5Ijp7Im5hbWUiOiJub25lIn0sImhlYWQiOnsibmFtZSI6Im5vbmUifX19XV0=
        \begin{tikzcd}[ampersand replacement=\&]
            {[E]} \& 0 \& A \& B \& C \& 0 \\
            \& 0 \& {A'} \& E \& C \& 0 \\
            \& 0 \& {A'} \& {E'} \& C \& 0 \\
            {[E']} \& 0 \& {A'} \& {B'} \& {C'} \& 0
            \arrow["{(\alpha,\beta,\gamma)}"{description}, from=1-1, to=4-1]
            \arrow[from=1-2, to=1-3]
            \arrow[from=1-3, to=1-4]
            \arrow[""{name=0, anchor=center, inner sep=0}, "\alpha"', from=1-3, to=2-3]
            \arrow[from=1-4, to=1-5]
            \arrow[""{name=1, anchor=center, inner sep=0}, from=1-4, to=2-4]
            \arrow[from=1-5, to=1-6]
            \arrow[Rightarrow, no head, from=1-5, to=2-5]
            \arrow[from=2-2, to=2-3]
            \arrow[from=2-3, to=2-4]
            \arrow[Rightarrow, no head, from=2-3, to=3-3]
            \arrow[from=2-4, to=2-5]
            \arrow[from=2-4, to=3-4]
            \arrow[from=2-5, to=2-6]
            \arrow[Rightarrow, no head, from=2-5, to=3-5]
            \arrow[from=3-2, to=3-3]
            \arrow[from=3-3, to=3-4]
            \arrow[Rightarrow, no head, from=3-3, to=4-3]
            \arrow[from=3-4, to=3-5]
            \arrow[""{name=2, anchor=center, inner sep=0}, from=3-4, to=4-4]
            \arrow[from=3-5, to=3-6]
            \arrow[""{name=3, anchor=center, inner sep=0}, "\gamma", from=3-5, to=4-5]
            \arrow[from=4-2, to=4-3]
            \arrow[from=4-3, to=4-4]
            \arrow[from=4-4, to=4-5]
            \arrow[from=4-5, to=4-6]
            \arrow["{\text{PO}}"{description}, draw=none, from=0, to=1]
            \arrow["{\text{PB}}"{description}, draw=none, from=2, to=3]
        \end{tikzcd}
    \end{equation}
    依照五引理, 中间两行正合列同构. 这表明 $(\alpha,\beta,\gamma)$ 实际由 $(\alpha,\gamma)$ 决定 (同构的意义下). 从短正合列的同构类的等价关系下, 定义 $\alpha_\ast [E]=\gamma^\ast[E']$. 简单地, 记作 $\alpha E=E'\gamma$.
\end{definition}

\begin{proposition}[$\mathrm{Ext}^1$ 作为双函子]
    需要依次定义 $\mathrm{Ext}^1(B,A)$ 的加法结构, 并验证加法结构与来源, 去向的变换相容.
    \begin{enumerate}
        \item 给定短正合列 $E_i:[0\to B\to X_i\to A\to 0]$, 定义加法 $E_1+E_2=\nabla (E_1\oplus E_2)\Delta$ 如下:
              \begin{equation}
                  % https://q.uiver.app/#q=WzAsMTcsWzIsMCwiQlxcb3BsdXMgQiJdLFs2LDAsIkFcXG9wbHVzIEEiXSxbNCwwLCJYXzFcXG9wbHVzIFhfMiJdLFsxLDAsIjAiXSxbNywwLCIwIl0sWzIsMSwiQiJdLFs0LDEsIlkiXSxbNiwxLCJBXFxvcGx1cyBBIl0sWzcsMSwiMCJdLFsxLDEsIjAiXSxbNiwyLCJBIl0sWzcsMiwiMCJdLFsyLDIsIkIiXSxbMSwyLCIwIl0sWzQsMiwiWiJdLFswLDAsIkVfMVxcb3BsdXMgRV8yIl0sWzAsMiwiRV8xK0VfMiJdLFswLDJdLFsyLDFdLFszLDBdLFsxLDRdLFswLDUsIlxcbmFibGEiLDJdLFs1LDZdLFs2LDddLFs3LDhdLFs5LDVdLFsxLDcsIiIsMSx7ImxldmVsIjoyLCJzdHlsZSI6eyJoZWFkIjp7Im5hbWUiOiJub25lIn19fV0sWzEwLDcsIlxcRGVsdGEgIiwyXSxbNSwxMiwiIiwyLHsibGV2ZWwiOjIsInN0eWxlIjp7ImhlYWQiOnsibmFtZSI6Im5vbmUifX19XSxbMTMsMTJdLFsxMiwxNF0sWzE0LDEwXSxbMTAsMTFdLFsyLDZdLFs2LDE0XSxbMjEsMzMsIlxcdGV4dHvmjqjlh7p9IiwxLHsic2hvcnRlbiI6eyJzb3VyY2UiOjIwLCJ0YXJnZXQiOjIwfSwic3R5bGUiOnsiYm9keSI6eyJuYW1lIjoibm9uZSJ9LCJoZWFkIjp7Im5hbWUiOiJub25lIn19fV0sWzM0LDI3LCJcXHRleHR75ouJ5ZuefSIsMSx7InNob3J0ZW4iOnsic291cmNlIjoyMCwidGFyZ2V0IjoyMH0sInN0eWxlIjp7ImJvZHkiOnsibmFtZSI6Im5vbmUifSwiaGVhZCI6eyJuYW1lIjoibm9uZSJ9fX1dXQ==
                  \begin{tikzcd}[ampersand replacement=\&]
                      {E_1\oplus E_2} \& 0 \& {B\oplus B} \&\& {X_1\oplus X_2} \&\& {A\oplus A} \& 0 \\
                      \& 0 \& B \&\& Y \&\& {A\oplus A} \& 0 \\
                      {E_1+E_2} \& 0 \& B \&\& Z \&\& A \& 0
                      \arrow[from=1-2, to=1-3]
                      \arrow[from=1-3, to=1-5]
                      \arrow[""{name=0, anchor=center, inner sep=0}, "\nabla"', from=1-3, to=2-3]
                      \arrow[from=1-5, to=1-7]
                      \arrow[""{name=1, anchor=center, inner sep=0}, from=1-5, to=2-5]
                      \arrow[from=1-7, to=1-8]
                      \arrow[Rightarrow, no head, from=1-7, to=2-7]
                      \arrow[from=2-2, to=2-3]
                      \arrow[from=2-3, to=2-5]
                      \arrow[Rightarrow, no head, from=2-3, to=3-3]
                      \arrow[from=2-5, to=2-7]
                      \arrow[""{name=2, anchor=center, inner sep=0}, from=2-5, to=3-5]
                      \arrow[from=2-7, to=2-8]
                      \arrow[from=3-2, to=3-3]
                      \arrow[from=3-3, to=3-5]
                      \arrow[from=3-5, to=3-7]
                      \arrow[""{name=3, anchor=center, inner sep=0}, "{\Delta }"', from=3-7, to=2-7]
                      \arrow[from=3-7, to=3-8]
                      \arrow["{\text{推出}}"{description}, draw=none, from=0, to=1]
                      \arrow["{\text{拉回}}"{description}, draw=none, from=2, to=3]
                  \end{tikzcd}
              \end{equation}
        \item 零元即可裂短正合列, 置上图 $E_1:[0\to B\xrightarrow i C\xrightarrow c A\to 0]$ 与 $E_2:[0\to B\to B\oplus A\to A \to 0]$, 其和为
              \begin{equation}
                  % https://q.uiver.app/#q=WzAsMTcsWzIsMCwiQlxcb3BsdXMgQiJdLFs2LDAsIkFcXG9wbHVzIEEiXSxbNCwwLCJDXFxvcGx1cyBCXFxvcGx1cyBBIl0sWzEsMCwiMCJdLFs3LDAsIjAiXSxbMiwxLCJCIl0sWzQsMSwiQ1xcb3BsdXMgQSJdLFs2LDEsIkFcXG9wbHVzIEEiXSxbNywxLCIwIl0sWzEsMSwiMCJdLFs2LDIsIkEiXSxbNywyLCIwIl0sWzIsMiwiQiJdLFsxLDIsIjAiXSxbNCwyLCJDIl0sWzAsMCwiRV8xXFxvcGx1cyBFXzIiXSxbMCwyLCJFXzErRV8yIl0sWzAsMiwiXFxsZWZ0KFxcc3Vic3RhY2t7aVxcLFxcLDBcXFxcMFxcLFxcLDFcXFxcMFxcLFxcLDB9XFxyaWdodCkiXSxbMiwxLCJcXGJpbm9te2NcXCxcXCwwXFwsXFwsMH17MFxcLFxcLDBcXCxcXCwxfSJdLFszLDBdLFsxLDRdLFswLDUsIigxXFwsXFwsMSkiLDJdLFs1LDYsIlxcYmlub20gaTAiLDJdLFs2LDcsIlxcYmlub217Y1xcLFxcLDB9ezBcXCxcXCwxfSJdLFs3LDhdLFs5LDVdLFsxLDcsIiIsMSx7ImxldmVsIjoyLCJzdHlsZSI6eyJoZWFkIjp7Im5hbWUiOiJub25lIn19fV0sWzEwLDcsIlxcYmlub217MX17MX0iLDJdLFs1LDEyLCIiLDIseyJsZXZlbCI6Miwic3R5bGUiOnsiaGVhZCI6eyJuYW1lIjoibm9uZSJ9fX1dLFsxMywxMl0sWzEyLDE0LCJpIl0sWzE0LDEwLCJjIiwyXSxbMTAsMTFdLFsyLDYsIlxcYmlub217MVxcLFxcLGlcXCxcXCwwfXswXFwsXFwsMFxcLFxcLDF9Il0sWzYsMTQsIlxcYmlub20gMWMiLDJdLFsyMywzMSwiXFx0ZXh0e+aLieWbnn0iLDEseyJzaG9ydGVuIjp7InNvdXJjZSI6MjAsInRhcmdldCI6MjB9LCJzdHlsZSI6eyJib2R5Ijp7Im5hbWUiOiJub25lIn0sImhlYWQiOnsibmFtZSI6Im5vbmUifX19XSxbMTcsMjIsIlxcdGV4dHvmjqjlh7p9IiwxLHsic2hvcnRlbiI6eyJzb3VyY2UiOjIwLCJ0YXJnZXQiOjIwfSwic3R5bGUiOnsiYm9keSI6eyJuYW1lIjoibm9uZSJ9LCJoZWFkIjp7Im5hbWUiOiJub25lIn19fV1d
                  \begin{tikzcd}[ampersand replacement=\&]
                      {E_1\oplus E_2} \& 0 \& {B\oplus B} \&\& {C\oplus B\oplus A} \&\& {A\oplus A} \& 0 \\
                      \& 0 \& B \&\& {C\oplus A} \&\& {A\oplus A} \& 0 \\
                      {E_1+E_2} \& 0 \& B \&\& C \&\& A \& 0
                      \arrow[from=1-2, to=1-3]
                      \arrow[""{name=0, anchor=center, inner sep=0}, "\begin{array}{c} \left(\substack{i\,\,0\\0\,\,1\\0\,\,0}\right) \end{array}", from=1-3, to=1-5]
                      \arrow["{(1\,\,1)}"', from=1-3, to=2-3]
                      \arrow["{\binom{c\,\,0\,\,0}{0\,\,0\,\,1}}", from=1-5, to=1-7]
                      \arrow["{\binom{1\,\,i\,\,0}{0\,\,0\,\,1}}", from=1-5, to=2-5]
                      \arrow[from=1-7, to=1-8]
                      \arrow[Rightarrow, no head, from=1-7, to=2-7]
                      \arrow[from=2-2, to=2-3]
                      \arrow[""{name=1, anchor=center, inner sep=0}, "{\binom i0}"', from=2-3, to=2-5]
                      \arrow[Rightarrow, no head, from=2-3, to=3-3]
                      \arrow[""{name=2, anchor=center, inner sep=0}, "{\binom{c\,\,0}{0\,\,1}}", from=2-5, to=2-7]
                      \arrow["{\binom 1c}"', from=2-5, to=3-5]
                      \arrow[from=2-7, to=2-8]
                      \arrow[from=3-2, to=3-3]
                      \arrow["i", from=3-3, to=3-5]
                      \arrow[""{name=3, anchor=center, inner sep=0}, "c"', from=3-5, to=3-7]
                      \arrow["{\binom{1}{1}}"', from=3-7, to=2-7]
                      \arrow[from=3-7, to=3-8]
                      \arrow["{\text{推出}}"{description}, draw=none, from=0, to=1]
                      \arrow["{\text{拉回}}"{description}, draw=none, from=2, to=3]
                  \end{tikzcd}.
              \end{equation}
        \item (加法的函子性) 只验证一侧, 另一侧同理.
              \begin{enumerate}
                  \item $(\alpha_1\oplus \alpha_2)(E_1\oplus E_2)=\alpha_1 E_1\oplus \alpha_2 E_2$. 这由矩阵乘法直接给出.
                  \item $(\alpha_1+\alpha_2)E=\nabla (\alpha_1\oplus \alpha_2)\Delta E=\nabla (\alpha_1\oplus \alpha_2)(E\oplus E)\Delta =\nabla (\alpha_1 E\oplus \alpha_2 E)\Delta =\alpha_1E+\alpha_2E$.
                  \item $\alpha (E_1+E_2)=\alpha \nabla (E_1\oplus E_2)\Delta = \nabla (\alpha\oplus \alpha)(E_1\oplus E_2)\Delta =  \nabla (\alpha E_1\oplus \alpha E_2)\Delta = \alpha E_1+\alpha E_2$.
              \end{enumerate}
              另需验证 $(\alpha E)\beta = \alpha(E\beta)$, 即, 推出与拉回交换. 实际上, 视 $(E\to E_\beta)\to \alpha (E\to E_\beta)$ 为态射范畴的推出即可. 图略.
        \item (加法结合律) 依照定义,
              \begin{equation}
                  (E_1+E_2)+E_3 = \nabla (E_1\oplus E_2)\Delta + E_3 = \nabla (\nabla (E_1\oplus E_2)\Delta \oplus E_3 ) \Delta = (\nabla (\nabla\oplus 1))(E_1\oplus E_2\oplus E_3) ((\Delta \oplus 1)\Delta).
              \end{equation}
              此处 $\nabla (\nabla\oplus 1)=\nabla (1\oplus \nabla)$, $\Delta$ 同理.
              \item (加法交换律) 依照 $E_1\oplus E_2\sim E_2\oplus E_1$ 即可. 记对换 $\tau=\binom{0\,\,1}{1\,\,0}$, 上式即 
              \begin{equation}
                \nabla (E_1\oplus E_2)\Delta = \nabla (\tau (E_1\oplus E_2))\Delta = \nabla ((E_2\oplus E_1)\tau )\Delta = \nabla (E_2\oplus E_1)\Delta. 
              \end{equation}
    \end{enumerate}
\end{proposition}

\begin{remark}
    容易验证, $0\to A\xrightarrow i B\xrightarrow c X\to 0$ 的加法逆元是 $0\to A\xrightarrow i B\xrightarrow {-c} X\to 0$. 改变任意一处符号即可. 
\end{remark}

\begin{definition}[长正合列的连接态射]
    给定 $0\to A\to B\to C\to 0$, 定义连接态射 $\delta:(X,C)\to \mathrm{Ext}^1(X,A),\quad f\mapsto Ef$ 为拉回
    \begin{equation}
        % https://q.uiver.app/#q=WzAsMTIsWzEsMSwiMCJdLFsyLDEsIkEiXSxbMywxLCJCIl0sWzQsMSwiQyJdLFs1LDEsIjAiXSxbNCwwLCJYIl0sWzMsMCwiXFxidWxsZXQiXSxbNSwwLCIwIl0sWzIsMCwiQSJdLFsxLDAsIjAiXSxbMCwwLCJFZiJdLFswLDEsIkUiXSxbMCwxXSxbMSwyXSxbMiwzXSxbMyw0XSxbNSwzLCJmIl0sWzksOF0sWzgsNiwiIiwxLHsic3R5bGUiOnsiYm9keSI6eyJuYW1lIjoiZGFzaGVkIn19fV0sWzYsNSwiIiwxLHsic3R5bGUiOnsiYm9keSI6eyJuYW1lIjoiZGFzaGVkIn19fV0sWzUsN10sWzgsMSwiIiwxLHsibGV2ZWwiOjIsInN0eWxlIjp7ImhlYWQiOnsibmFtZSI6Im5vbmUifX19XSxbNiwyLCIiLDEseyJzdHlsZSI6eyJib2R5Ijp7Im5hbWUiOiJkYXNoZWQifX19XSxbMTEsMTAsIiIsMCx7InN0eWxlIjp7ImJvZHkiOnsibmFtZSI6ImRhc2hlZCJ9fX1dLFsxOSwxNCwiXFx0ZXh0e+aLieWbnn0iLDEseyJzaG9ydGVuIjp7InNvdXJjZSI6MjAsInRhcmdldCI6MjB9LCJzdHlsZSI6eyJib2R5Ijp7Im5hbWUiOiJub25lIn0sImhlYWQiOnsibmFtZSI6Im5vbmUifX19XV0=
        \begin{tikzcd}[ampersand replacement=\&]
            Ef \& 0 \& A \& \bullet \& X \& 0 \\
            E \& 0 \& A \& B \& C \& 0
            \arrow[from=1-2, to=1-3]
            \arrow[dashed, from=1-3, to=1-4]
            \arrow[Rightarrow, no head, from=1-3, to=2-3]
            \arrow[""{name=0, anchor=center, inner sep=0}, dashed, from=1-4, to=1-5]
            \arrow[dashed, from=1-4, to=2-4]
            \arrow[from=1-5, to=1-6]
            \arrow["f", from=1-5, to=2-5]
            \arrow[dashed, from=2-1, to=1-1]
            \arrow[from=2-2, to=2-3]
            \arrow[from=2-3, to=2-4]
            \arrow[""{name=1, anchor=center, inner sep=0}, from=2-4, to=2-5]
            \arrow[from=2-5, to=2-6]
            \arrow["{\text{拉回}}"{description}, draw=none, from=0, to=1]
        \end{tikzcd}.
    \end{equation}
\end{definition}

\begin{proposition}[$\mathrm{Ext}$-群的长正合列]
    给定短正合列 $0\to A\xrightarrow{i} B \xrightarrow{c} C\to 0$, 则对任意 $X$ 均有长正合列
    \begin{equation}
        0\to (X,A)\xrightarrow{(X,i)} (X,B)\xrightarrow{(X,c)} (X,C)\xrightarrow \delta \mathrm{Ext}^1(X,A) \xrightarrow{\mathrm{Ext}^1(X,i)}\mathrm{Ext}^1(X,B) \xrightarrow{\mathrm{Ext}^1(X,c)}\mathrm{Ext}^1(X,C)\to\cdots
    \end{equation}
    \begin{proof}
        依次验证正合性即可.
        \begin{enumerate}
            \item ($(X,i)$ 是单射) 这是单射的泛性质.
            \item ($\mathrm{im}(X,i)=\ker(X,c)$) 这是核的泛性质.
            \item ($\mathrm{im}(X,c)=\mathrm{ker}(\delta)$) 任取 $(f\circ c)\in \mathrm{im}(X,c)$, 则有交换图
                  \begin{equation}
                      % https://q.uiver.app/#q=WzAsMTAsWzAsMSwiMCJdLFsxLDEsIkEiXSxbMiwxLCJCIl0sWzMsMSwiQyJdLFs0LDEsIjAiXSxbMywwLCJYIl0sWzIsMCwiWiJdLFs0LDAsIjAiXSxbMSwwLCJBIl0sWzAsMCwiMCJdLFswLDFdLFsxLDIsImkiLDJdLFsyLDMsImMiLDJdLFszLDRdLFs1LDMsImZjIl0sWzksOF0sWzgsNiwiIiwxLHsic3R5bGUiOnsiYm9keSI6eyJuYW1lIjoiZGFzaGVkIn19fV0sWzYsNSwiIiwxLHsic3R5bGUiOnsiYm9keSI6eyJuYW1lIjoiZGFzaGVkIn19fV0sWzUsN10sWzgsMSwiIiwxLHsibGV2ZWwiOjIsInN0eWxlIjp7ImhlYWQiOnsibmFtZSI6Im5vbmUifX19XSxbNiwyLCIiLDEseyJzdHlsZSI6eyJib2R5Ijp7Im5hbWUiOiJkYXNoZWQifX19XSxbNSwyLCJmIiwyLHsic3R5bGUiOnsiYm9keSI6eyJuYW1lIjoiZGFzaGVkIn19fV1d
                      \begin{tikzcd}[ampersand replacement=\&]
                          0 \& A \& Z \& X \& 0 \\
                          0 \& A \& B \& C \& 0
                          \arrow[from=1-1, to=1-2]
                          \arrow[dashed, from=1-2, to=1-3]
                          \arrow[Rightarrow, no head, from=1-2, to=2-2]
                          \arrow[dashed, from=1-3, to=1-4]
                          \arrow[dashed, from=1-3, to=2-3]
                          \arrow[from=1-4, to=1-5]
                          \arrow["f"', dashed, from=1-4, to=2-3]
                          \arrow["fc", from=1-4, to=2-4]
                          \arrow[from=2-1, to=2-2]
                          \arrow["i"', from=2-2, to=2-3]
                          \arrow["c"', from=2-3, to=2-4]
                          \arrow[from=2-4, to=2-5]
                      \end{tikzcd}.
                  \end{equation}
                  对 $X=X$ 与 $f:X\to B$ 使用拉回的泛性质, 此时得 $Z\twoheadrightarrow X$ 的右逆. 反之, 若正合列沿 $g:X\to C$ 的拉回是可裂短正合列, 则记 $X\to Z\to B$ 合成为 $f$. 此时 $cf=g$.
            \item ($\mathrm{im}(\delta)=\mathrm{ker}(\mathrm{Ext}^1(X,i))$) 任取正合列 $P\in \mathrm{ker}(\mathrm{Ext}^1(X,i))$, 考虑 $sl:Z\to B$ 在余核处的态射, 即得 $P=\delta(\widetilde{sl})$:
                  \begin{equation}
                      % https://q.uiver.app/#q=WzAsMTYsWzEsMiwiMCJdLFsyLDIsIkEiXSxbMywyLCJCIl0sWzQsMiwiQyJdLFs1LDIsIjAiXSxbNCwxLCJYIl0sWzMsMSwiWiJdLFs1LDEsIjAiXSxbMiwxLCJBIl0sWzEsMSwiMCJdLFsyLDAsIkIiXSxbMSwwLCIwIl0sWzMsMCwiQlxcb3BsdXMgWCJdLFs0LDAsIlgiXSxbNSwwLCIwIl0sWzAsMSwiUCJdLFswLDFdLFsxLDIsImkiLDJdLFsyLDMsImMiLDJdLFszLDRdLFs5LDhdLFs4LDZdLFs2LDVdLFs1LDddLFs4LDEsIiIsMSx7ImxldmVsIjoyLCJzdHlsZSI6eyJoZWFkIjp7Im5hbWUiOiJub25lIn19fV0sWzYsMiwic2wiLDAseyJzdHlsZSI6eyJib2R5Ijp7Im5hbWUiOiJkYXNoZWQifX19XSxbOCwxMCwiaSJdLFsxMSwxMF0sWzUsMTMsIiIsMix7ImxldmVsIjoyLCJzdHlsZSI6eyJoZWFkIjp7Im5hbWUiOiJub25lIn19fV0sWzEwLDEyXSxbMTIsMTNdLFsxMywxNF0sWzEyLDEwLCJzIiwwLHsiY3VydmUiOi0xfV0sWzYsMTIsImwiLDJdLFs1LDMsIlxcd2lkZXRpbGRlIHtzbH0iLDAseyJzdHlsZSI6eyJib2R5Ijp7Im5hbWUiOiJkYXNoZWQifX19XV0=
                      \begin{tikzcd}[ampersand replacement=\&]
                          \& 0 \& B \& {B\oplus X} \& X \& 0 \\
                          P \& 0 \& A \& Z \& X \& 0 \\
                          \& 0 \& A \& B \& C \& 0
                          \arrow[from=1-2, to=1-3]
                          \arrow[from=1-3, to=1-4]
                          \arrow["s", curve={height=-6pt}, from=1-4, to=1-3]
                          \arrow[from=1-4, to=1-5]
                          \arrow[from=1-5, to=1-6]
                          \arrow[from=2-2, to=2-3]
                          \arrow["i", from=2-3, to=1-3]
                          \arrow[from=2-3, to=2-4]
                          \arrow[Rightarrow, no head, from=2-3, to=3-3]
                          \arrow["l"', from=2-4, to=1-4]
                          \arrow[from=2-4, to=2-5]
                          \arrow["sl", dashed, from=2-4, to=3-4]
                          \arrow[Rightarrow, no head, from=2-5, to=1-5]
                          \arrow[from=2-5, to=2-6]
                          \arrow["{\widetilde {sl}}", dashed, from=2-5, to=3-5]
                          \arrow[from=3-2, to=3-3]
                          \arrow["i"', from=3-3, to=3-4]
                          \arrow["c"', from=3-4, to=3-5]
                          \arrow[from=3-5, to=3-6]
                      \end{tikzcd}.
                  \end{equation}
                  反之, 则可在左上构造 $Z\to B$, 因此推出是可裂单.
            \item ($\mathrm{im}(\mathrm{Ext}^1(X,i))=\mathrm{ker}(\mathrm{Ext}^1(X,c))$) 依照推出方块的复合律, $\subset$ 方向显然. 反之, 只需构造虚线处正合列
                  \begin{equation}
                      % https://q.uiver.app/#q=WzAsMTcsWzEsMSwiQSJdLFsyLDEsIkIiXSxbMywxLCJDIl0sWzQsMSwiMCJdLFsxLDMsIlgiXSxbMSwwLCIwIl0sWzEsMiwiWiJdLFsxLDQsIjAiXSxbMiwwLCIwIl0sWzIsMywiWCJdLFsyLDQsIjAiXSxbMiwyLCJXIl0sWzAsMSwiMCJdLFszLDIsIkNcXG9wbHVzIFgiXSxbMywzLCJYIl0sWzMsNCwiMCJdLFszLDAsIjAiXSxbMCwxLCJpIl0sWzEsMiwiYyJdLFsyLDNdLFs1LDAsIiIsMCx7InN0eWxlIjp7ImJvZHkiOnsibmFtZSI6ImRhc2hlZCJ9fX1dLFswLDYsIiIsMCx7InN0eWxlIjp7ImJvZHkiOnsibmFtZSI6ImRhc2hlZCJ9fX1dLFs2LDQsIiIsMCx7InN0eWxlIjp7ImJvZHkiOnsibmFtZSI6ImRhc2hlZCJ9fX1dLFs0LDcsIiIsMCx7InN0eWxlIjp7ImJvZHkiOnsibmFtZSI6ImRhc2hlZCJ9fX1dLFs4LDFdLFs0LDksIiIsMCx7ImxldmVsIjoyLCJzdHlsZSI6eyJib2R5Ijp7Im5hbWUiOiJkYXNoZWQifSwiaGVhZCI6eyJuYW1lIjoibm9uZSJ9fX1dLFsxMiwwXSxbNiwxMSwiIiwwLHsic3R5bGUiOnsiYm9keSI6eyJuYW1lIjoiZGFzaGVkIn19fV0sWzExLDldLFs5LDEwXSxbMTEsMTNdLFs5LDE0LCIiLDAseyJsZXZlbCI6Miwic3R5bGUiOnsiaGVhZCI6eyJuYW1lIjoibm9uZSJ9fX1dLFsxMywxNF0sWzE0LDE1XSxbMTYsMl0sWzEsMTFdLFsyLDEzXV0=
                      \begin{tikzcd}[ampersand replacement=\&]
                          \& 0 \& 0 \& 0 \\
                          0 \& A \& B \& C \& 0 \\
                          \& Z \& W \& {C\oplus X} \\
                          \& X \& X \& X \\
                          \& 0 \& 0 \& 0
                          \arrow[dashed, from=1-2, to=2-2]
                          \arrow[from=1-3, to=2-3]
                          \arrow[from=1-4, to=2-4]
                          \arrow[from=2-1, to=2-2]
                          \arrow["i", from=2-2, to=2-3]
                          \arrow[dashed, from=2-2, to=3-2]
                          \arrow["c", from=2-3, to=2-4]
                          \arrow[from=2-3, to=3-3]
                          \arrow[from=2-4, to=2-5]
                          \arrow[from=2-4, to=3-4]
                          \arrow[dashed, from=3-2, to=3-3]
                          \arrow[dashed, from=3-2, to=4-2]
                          \arrow[from=3-3, to=3-4]
                          \arrow[from=3-3, to=4-3]
                          \arrow[from=3-4, to=4-4]
                          \arrow[Rightarrow, dashed, no head, from=4-2, to=4-3]
                          \arrow[dashed, from=4-2, to=5-2]
                          \arrow[Rightarrow, no head, from=4-3, to=4-4]
                          \arrow[from=4-3, to=5-3]
                          \arrow[from=4-4, to=5-4]
                      \end{tikzcd}.
                  \end{equation}
                  依照 $3\times 3$ 引理, 下图三行与右两列是短正合列, 从而左列是短正合列:
                  \begin{equation}
                      % https://q.uiver.app/#q=WzAsMjEsWzEsMSwiQSJdLFsyLDEsIkIiXSxbMywxLCJDIl0sWzQsMSwiMCJdLFsxLDMsIlgiXSxbMSwwLCIwIl0sWzEsMiwiSyJdLFsxLDQsIjAiXSxbMiwwLCIwIl0sWzIsMywiWCJdLFsyLDQsIjAiXSxbMiwyLCJXIl0sWzAsMSwiMCJdLFszLDIsIkMiXSxbMywwLCIwIl0sWzMsMywiMCJdLFs0LDIsIjAiXSxbMCwzLCIwIl0sWzAsMiwiMCJdLFs0LDMsIjAiXSxbMyw0LCIwIl0sWzAsMSwiaSJdLFsxLDIsImMiXSxbMiwzXSxbNSwwLCIiLDAseyJzdHlsZSI6eyJib2R5Ijp7Im5hbWUiOiJkYXNoZWQifX19XSxbMCw2LCIiLDAseyJzdHlsZSI6eyJib2R5Ijp7Im5hbWUiOiJkYXNoZWQifX19XSxbNiw0LCIiLDAseyJzdHlsZSI6eyJib2R5Ijp7Im5hbWUiOiJkYXNoZWQifX19XSxbNCw3LCIiLDAseyJzdHlsZSI6eyJib2R5Ijp7Im5hbWUiOiJkYXNoZWQifX19XSxbOCwxXSxbNCw5LCIiLDAseyJsZXZlbCI6Miwic3R5bGUiOnsiaGVhZCI6eyJuYW1lIjoibm9uZSJ9fX1dLFsxMiwwXSxbNiwxMV0sWzExLDldLFs5LDEwXSxbMTEsMTNdLFsxNCwyXSxbMSwxMV0sWzIsMTMsIiIsMCx7ImxldmVsIjoyLCJzdHlsZSI6eyJoZWFkIjp7Im5hbWUiOiJub25lIn19fV0sWzksMTVdLFsxMywxNV0sWzEzLDE2XSxbMTcsNF0sWzE4LDZdLFsxNSwyMF0sWzE1LDE5XV0=
                      \begin{tikzcd}[ampersand replacement=\&]
                          \& 0 \& 0 \& 0 \\
                          0 \& A \& B \& C \& 0 \\
                          0 \& K \& W \& C \& 0 \\
                          0 \& X \& X \& 0 \& 0 \\
                          \& 0 \& 0 \& 0
                          \arrow[dashed, from=1-2, to=2-2]
                          \arrow[from=1-3, to=2-3]
                          \arrow[from=1-4, to=2-4]
                          \arrow[from=2-1, to=2-2]
                          \arrow["i", from=2-2, to=2-3]
                          \arrow[dashed, from=2-2, to=3-2]
                          \arrow["c", from=2-3, to=2-4]
                          \arrow[from=2-3, to=3-3]
                          \arrow[from=2-4, to=2-5]
                          \arrow[Rightarrow, no head, from=2-4, to=3-4]
                          \arrow[from=3-1, to=3-2]
                          \arrow[from=3-2, to=3-3]
                          \arrow[dashed, from=3-2, to=4-2]
                          \arrow[from=3-3, to=3-4]
                          \arrow[from=3-3, to=4-3]
                          \arrow[from=3-4, to=3-5]
                          \arrow[from=3-4, to=4-4]
                          \arrow[from=4-1, to=4-2]
                          \arrow[Rightarrow, no head, from=4-2, to=4-3]
                          \arrow[dashed, from=4-2, to=5-2]
                          \arrow[from=4-3, to=4-4]
                          \arrow[from=4-3, to=5-3]
                          \arrow[from=4-4, to=4-5]
                          \arrow[from=4-4, to=5-4]
                      \end{tikzcd}.
                  \end{equation}
        \end{enumerate}
    \end{proof}
\end{proposition}

\begin{example}[$3\times 3$ 表格, $\mathrm{Ext}^2$-群的又一刻画]
    以下 $3\times 3$ 蕴含三项 $\mathrm{Ext}^2(C_3,A_1)$ 中相同的元素: 
\begin{equation}
    % https://q.uiver.app/#q=WzAsMzMsWzEsMSwiQV8xIl0sWzIsMSwiQV8yIl0sWzMsMSwiQV8zIl0sWzEsMiwiQl8xIl0sWzIsMiwiQl8yIl0sWzMsMiwiQl8zIl0sWzEsMywiQ18xIl0sWzIsMywiQ18yIl0sWzMsMywiQ18zIl0sWzAsMSwiMCJdLFswLDIsIjAiXSxbMCwzLCIwIl0sWzEsNCwiMCJdLFsyLDQsIjAiXSxbMyw0LCIwIl0sWzQsMywiMCJdLFs0LDIsIjAiXSxbNCwxLCIwIl0sWzMsMCwiMCJdLFsyLDAsIjAiXSxbMSwwLCIwIl0sWzYsMSwiQV8xIl0sWzcsMSwiQV8yIl0sWzgsMSwiQl8zIl0sWzksMSwiQ18zIl0sWzksMiwiQ18zIl0sWzYsMiwiQV8xIl0sWzcsMiwiQl8yIl0sWzgsMiwiQ18yXFxvcGx1cyBCXzMiXSxbOCwzLCJDXzIiXSxbOSwzLCJDXzMiXSxbNiwzLCJBXzEiXSxbNywzLCJCXzEiXSxbMjAsMF0sWzAsM10sWzMsNl0sWzYsMTJdLFsxMSw2XSxbNiw3XSxbNyw4XSxbOCwxNV0sWzE4LDJdLFsyLDVdLFs1LDhdLFs4LDE0XSxbMTAsM10sWzMsNF0sWzQsNV0sWzUsMTZdLFs5LDBdLFsxLDJdLFsyLDE3XSxbMTksMV0sWzEsNF0sWzQsN10sWzcsMTNdLFswLDFdLFsxLDVdLFszLDddLFsyNSwyNCwiIiwxLHsibGV2ZWwiOjIsInN0eWxlIjp7ImhlYWQiOnsibmFtZSI6Im5vbmUifX19XSxbMjEsMjIsIiIsMSx7InN0eWxlIjp7InRhaWwiOnsibmFtZSI6Imhvb2siLCJzaWRlIjoiYm90dG9tIn19fV0sWzIyLDIzXSxbMjMsMjQsIiIsMSx7InN0eWxlIjp7ImhlYWQiOnsibmFtZSI6ImVwaSJ9fX1dLFsyNiwyMSwiIiwxLHsibGV2ZWwiOjIsInN0eWxlIjp7ImhlYWQiOnsibmFtZSI6Im5vbmUifX19XSxbMjYsMjcsIiIsMSx7InN0eWxlIjp7InRhaWwiOnsibmFtZSI6Imhvb2siLCJzaWRlIjoiYm90dG9tIn19fV0sWzI3LDI4LCJcXGJpbm9teyt9eyt9Il0sWzI4LDI1LCIoLSwrKSIsMCx7InN0eWxlIjp7ImhlYWQiOnsibmFtZSI6ImVwaSJ9fX1dLFsyMiwyNywiIiwxLHsic3R5bGUiOnsidGFpbCI6eyJuYW1lIjoiaG9vayIsInNpZGUiOiJib3R0b20ifX19XSxbMjMsMjgsIiIsMSx7InN0eWxlIjp7InRhaWwiOnsibmFtZSI6Imhvb2siLCJzaWRlIjoiYm90dG9tIn19fV0sWzMxLDMyLCIiLDEseyJzdHlsZSI6eyJ0YWlsIjp7Im5hbWUiOiJob29rIiwic2lkZSI6ImJvdHRvbSJ9fX1dLFszMiwyOV0sWzI5LDMwLCJcXGJveGVkIC0iLDIseyJzdHlsZSI6eyJoZWFkIjp7Im5hbWUiOiJlcGkifX19XSxbMzIsMjcsIiIsMSx7InN0eWxlIjp7InRhaWwiOnsibmFtZSI6Imhvb2siLCJzaWRlIjoiYm90dG9tIn19fV0sWzI5LDI4LCIiLDEseyJzdHlsZSI6eyJ0YWlsIjp7Im5hbWUiOiJob29rIiwic2lkZSI6ImJvdHRvbSJ9fX1dLFsyNiwzMSwiIiwxLHsibGV2ZWwiOjIsInN0eWxlIjp7ImhlYWQiOnsibmFtZSI6Im5vbmUifX19XSxbMjUsMzAsIiIsMSx7ImxldmVsIjoyLCJzdHlsZSI6eyJoZWFkIjp7Im5hbWUiOiJub25lIn19fV1d
\begin{tikzcd}[ampersand replacement=\&]
	\& 0 \& 0 \& 0 \\
	0 \& {A_1} \& {A_2} \& {A_3} \& 0 \&\& {A_1} \& {A_2} \& {B_3} \& {C_3} \\
	0 \& {B_1} \& {B_2} \& {B_3} \& 0 \&\& {A_1} \& {B_2} \& {C_2\oplus B_3} \& {C_3} \\
	0 \& {C_1} \& {C_2} \& {C_3} \& 0 \&\& {A_1} \& {B_1} \& {C_2} \& {C_3} \\
	\& 0 \& 0 \& 0
	\arrow[from=1-2, to=2-2]
	\arrow[from=1-3, to=2-3]
	\arrow[from=1-4, to=2-4]
	\arrow[from=2-1, to=2-2]
	\arrow[from=2-2, to=2-3]
	\arrow[from=2-2, to=3-2]
	\arrow[from=2-3, to=2-4]
	\arrow[from=2-3, to=3-3]
	\arrow[from=2-3, to=3-4]
	\arrow[from=2-4, to=2-5]
	\arrow[from=2-4, to=3-4]
	\arrow[hook', from=2-7, to=2-8]
	\arrow[from=2-8, to=2-9]
	\arrow[hook', from=2-8, to=3-8]
	\arrow[two heads, from=2-9, to=2-10]
	\arrow[hook', from=2-9, to=3-9]
	\arrow[from=3-1, to=3-2]
	\arrow[from=3-2, to=3-3]
	\arrow[from=3-2, to=4-2]
	\arrow[from=3-2, to=4-3]
	\arrow[from=3-3, to=3-4]
	\arrow[from=3-3, to=4-3]
	\arrow[from=3-4, to=3-5]
	\arrow[from=3-4, to=4-4]
	\arrow[Rightarrow, no head, from=3-7, to=2-7]
	\arrow[hook', from=3-7, to=3-8]
	\arrow[Rightarrow, no head, from=3-7, to=4-7]
	\arrow["{\binom{+}{+}}", from=3-8, to=3-9]
	\arrow["{(-,+)}", two heads, from=3-9, to=3-10]
	\arrow[Rightarrow, no head, from=3-10, to=2-10]
	\arrow[Rightarrow, no head, from=3-10, to=4-10]
	\arrow[from=4-1, to=4-2]
	\arrow[from=4-2, to=4-3]
	\arrow[from=4-2, to=5-2]
	\arrow[from=4-3, to=4-4]
	\arrow[from=4-3, to=5-3]
	\arrow[from=4-4, to=4-5]
	\arrow[from=4-4, to=5-4]
	\arrow[hook', from=4-7, to=4-8]
	\arrow[hook', from=4-8, to=3-8]
	\arrow[from=4-8, to=4-9]
	\arrow[hook', from=4-9, to=3-9]
	\arrow["{\boxed -}"', two heads, from=4-9, to=4-10]
\end{tikzcd}
\end{equation}
    因此 $A_1\to A_2\to B_3\to C_3$ 与 $A_1\to B_1\to C_2\to C_3$ 在 $\mathrm{Ext}^2$ 中是相反的元素. 特别地, $C_1=0$ 时以上四项正合列是 $\mathrm{Ext}^2$ 中的零元. 
\end{example}

\begin{definition}[$\mathrm{Ext}^n$ 群]
    归纳地进行 $\mathrm{Ext}^{m}(Y,Z)\times \mathrm{Ext}^n(X,Y)\to \mathrm{Ext}^{m+n}(X,Z)$. 定义
    \begin{align}
                      & [0\to Z\to P_1\to \cdots \to P_m\to Y\to 0] \quad \& \quad [0\to Y\to Q_1\to \cdots \to Q_n\to X\to 0] \\[6pt]
        \mapsto \quad & [0\to Z\to P_1\to \cdots \to P_m\to Q_1\to \cdots \to Q_n\to X\to 0].
    \end{align}
\end{definition}

\begin{remark}
    在可复合的意义下, 今后记 $\mathrm{Ext}^n$ 中长正合列为 $E^nE^{n-1}\cdots E^2E^1$, 此处每一 $E_i\in \mathrm{Ext}^1$.
\end{remark}

\begin{definition}[$\mathrm{Ext}^n$-群的运算]
    对 $\mathrm{Ext}^n$ 中长正合列 $E=E^nE^{n-1}\cdots E^2E^1$, 定义 $\alpha E\beta := (\alpha E^n)E^{n-1}\cdots E^2 (E^1\beta)$. 
\end{definition}

\begin{definition}[$\mathrm{Ext}^n$ 的等价关系]
    $\mathrm{Ext}^n(B,A)$ 中的``长度为 $1$ 的等价关系''描述作交换图
    \begin{equation}
        % https://q.uiver.app/#q=WzAsMTQsWzEsMCwiQSJdLFsyLDAsIlhfMSJdLFswLDAsIjAiXSxbMCwxLCIwIl0sWzEsMSwiQSJdLFsyLDEsIllfMSJdLFszLDAsIlxcY2RvdHMgIl0sWzQsMCwiWF9uIl0sWzQsMSwiWV9uIl0sWzUsMCwiQiJdLFszLDEsIlxcY2RvdHMgIl0sWzUsMSwiQiJdLFs2LDAsIjAiXSxbNiwxLCIwIl0sWzIsMF0sWzAsMV0sWzEsNl0sWzYsN10sWzcsOV0sWzksMTJdLFszLDRdLFs0LDVdLFs1LDEwXSxbMTAsOF0sWzgsMTFdLFsxMSwxM10sWzksMTEsIiIsMSx7ImxldmVsIjoyLCJzdHlsZSI6eyJoZWFkIjp7Im5hbWUiOiJub25lIn19fV0sWzcsOF0sWzEsNV0sWzAsNCwiIiwxLHsibGV2ZWwiOjIsInN0eWxlIjp7ImhlYWQiOnsibmFtZSI6Im5vbmUifX19XV0=
\begin{tikzcd}[ampersand replacement=\&]
	0 \& A \& {X_n} \& {\cdots } \& {X_1} \& B \& 0 \\
	0 \& A \& {Y_n} \& {\cdots } \& {Y_1} \& B \& 0
	\arrow[from=1-1, to=1-2]
	\arrow[from=1-2, to=1-3]
	\arrow[Rightarrow, no head, from=1-2, to=2-2]
	\arrow[from=1-3, to=1-4]
	\arrow[from=1-3, to=2-3]
	\arrow[from=1-4, to=1-5]
	\arrow[from=1-5, to=1-6]
	\arrow[from=1-5, to=2-5]
	\arrow[from=1-6, to=1-7]
	\arrow[Rightarrow, no head, from=1-6, to=2-6]
	\arrow[from=2-1, to=2-2]
	\arrow[from=2-2, to=2-3]
	\arrow[from=2-3, to=2-4]
	\arrow[from=2-4, to=2-5]
	\arrow[from=2-5, to=2-6]
	\arrow[from=2-6, to=2-7]
\end{tikzcd}.
    \end{equation}
    记以上第一条正合列为 $E_n\cdots E_1$, 其中 $E_k:[0\to \Omega_{k}\to X_k\to \Omega_{k-1}\to 0]$ 是短正合列, $\Omega_0=B$, $\Omega_n=A$. 记以上第二条正合列 $E_n'\cdots E_1'$, $\alpha_k:\Omega_k\to \Omega_k'$. 此时 $E_k'\alpha_{k-1}=\alpha_kE_k$: 
    \begin{equation}
        % https://q.uiver.app/#q=WzAsMTIsWzMsMCwiWF9rIl0sWzMsMSwiWV9rIl0sWzAsMCwiRV9rIl0sWzAsMSwiRV9rJyJdLFs0LDAsIlxcT21lZ2Ffe2stMX0iXSxbNCwxLCJcXE9tZWdhX3trLTF9JyJdLFsyLDAsIlxcT21lZ2Ffe2t9Il0sWzIsMSwiXFxPbWVnYV97a30nIl0sWzUsMCwiMCJdLFs1LDEsIjAiXSxbMSwwLCIwIl0sWzEsMSwiMCJdLFswLDFdLFs2LDBdLFswLDRdLFs0LDhdLFs3LDFdLFsxLDVdLFs1LDldLFsxMSw3XSxbMTAsNl0sWzYsNywiXFxhbHBoYV9rIl0sWzQsNSwiXFxhbHBoYV97ay0xfSJdXQ==
\begin{tikzcd}[ampersand replacement=\&]
	{E_k} \& 0 \& {\Omega_{k}} \& {X_k} \& {\Omega_{k-1}} \& 0 \\
	{E_k'} \& 0 \& {\Omega_{k}'} \& {Y_k} \& {\Omega_{k-1}'} \& 0
	\arrow[from=1-2, to=1-3]
	\arrow[from=1-3, to=1-4]
	\arrow["{\alpha_k}", from=1-3, to=2-3]
	\arrow[from=1-4, to=1-5]
	\arrow[from=1-4, to=2-4]
	\arrow[from=1-5, to=1-6]
	\arrow["{\alpha_{k-1}}", from=1-5, to=2-5]
	\arrow[from=2-2, to=2-3]
	\arrow[from=2-3, to=2-4]
	\arrow[from=2-4, to=2-5]
	\arrow[from=2-5, to=2-6]
\end{tikzcd}.
    \end{equation}
    以上关系生成 $\mathrm{Ext}^n$ 的等价关系 (锯齿图). 对 $\mathrm{Ext}^1$ 而言, 锯齿图可以拉直 (五引理). 
\end{definition}

\begin{remark}
    由定义, 若 $E\sim E'$, 则 $FEG\sim FE'G$, 从而米田积保持该等价关系.
\end{remark}

\begin{proposition}
    下图表明 $F(\varphi E')\sim (F\varphi )E'$: 
    \begin{equation}
        % https://q.uiver.app/#q=WzAsMTgsWzEsMCwiMCJdLFsyLDAsIlkiXSxbMywwLCJDIl0sWzQsMSwiQSJdLFsyLDEsIlkiXSxbMSwxLCIwIl0sWzMsMSwiQyciXSxbNCwyLCJBJyJdLFs1LDAsIkIiXSxbNSwxLCJCJyJdLFs2LDAsIlgnIl0sWzYsMSwiWCciXSxbNywwLCIwIl0sWzcsMSwiMCJdLFswLDAsIlxcbWF0aHJte0V4dH1eMShBLFkpXFxuaSBFIl0sWzAsMSwiXFxtYXRocm17RXh0fV4xKEEnLFkpXFxuaSBFJyJdLFs4LDAsIlxcbWF0aHJte0V4dH1eMShYJyxBKVxcbmkgRiJdLFs4LDEsIlxcbWF0aHJte0V4dH1eMShYJyxBJylcXG5pIEYnIl0sWzAsMV0sWzIsM10sWzUsNF0sWzQsNl0sWzYsN10sWzEsNF0sWzIsNl0sWzMsNywiXFx2YXJwaGkgIl0sWzcsOV0sWzMsOF0sWzgsMTBdLFs5LDExXSxbMTEsMTNdLFsxMCwxMl0sWzEwLDExLCIiLDEseyJsZXZlbCI6Miwic3R5bGUiOnsiaGVhZCI6eyJuYW1lIjoibm9uZSJ9fX1dLFsyLDgsIiIsMSx7ImN1cnZlIjoyLCJzdHlsZSI6eyJib2R5Ijp7Im5hbWUiOiJkYXNoZWQifX19XSxbNiw5LCIiLDEseyJjdXJ2ZSI6Miwic3R5bGUiOnsiYm9keSI6eyJuYW1lIjoiZGFzaGVkIn19fV0sWzgsOV0sWzEsMl1d
\begin{tikzcd}[ampersand replacement=\&]
	{\mathrm{Ext}^1(A,Y)\ni E} \& 0 \& Y \& C \&\& B \& {X'} \& 0 \& {\mathrm{Ext}^1(X',A)\ni F} \\
	{\mathrm{Ext}^1(A',Y)\ni E'} \& 0 \& Y \& {C'} \& A \& {B'} \& {X'} \& 0 \& {\mathrm{Ext}^1(X',A')\ni F'} \\
	\&\&\&\& {A'}
	\arrow[from=1-2, to=1-3]
	\arrow[from=1-3, to=1-4]
	\arrow[from=1-3, to=2-3]
	\arrow[curve={height=12pt}, dashed, from=1-4, to=1-6]
	\arrow[from=1-4, to=2-4]
	\arrow[from=1-4, to=2-5]
	\arrow[from=1-6, to=1-7]
	\arrow[from=1-6, to=2-6]
	\arrow[from=1-7, to=1-8]
	\arrow[Rightarrow, no head, from=1-7, to=2-7]
	\arrow[from=2-2, to=2-3]
	\arrow[from=2-3, to=2-4]
	\arrow[curve={height=12pt}, dashed, from=2-4, to=2-6]
	\arrow[from=2-4, to=3-5]
	\arrow[from=2-5, to=1-6]
	\arrow["{\varphi }", from=2-5, to=3-5]
	\arrow[from=2-6, to=2-7]
	\arrow[from=2-7, to=2-8]
	\arrow[from=3-5, to=2-6]
\end{tikzcd}.
    \end{equation}
    换言之, 态射乘法与正合列的米田积相容. 
\end{proposition}

\begin{proposition}[$\mathrm{Ext}^n$ 作为双函子]
    需要依次定义 $\mathrm{Ext}^1(B,A)$ 的加法结构, 并验证加法结构与来源, 去向的变换相容.
    \begin{enumerate}
        \item 给定 $E,E'\in \mathrm{Ext}^n(B,A)$, 定义 $E+E'=\nabla(E\oplus E')\Delta$. 
        \item 零对象即 $0\to A=A\to \cdots \to B=B\to 0$ 所在的等价类. 注: 这并不一定是可裂短正合列, 例如
        \begin{equation}
            0\to \mathbb R\to \mathbb R[x]/(x^2)\xrightarrow {\times x}\mathbb R[x]/(x^2)\to \mathbb R\to 0\quad (\mathrm{Mod}_{\mathbb R[x]})
        \end{equation}
        并非可裂短正合列. 
        \item (加法的函子性) 只验证一侧, 另一侧同理. (特别地, 长度为 $0$ 的正合列即态射.)
        \begin{enumerate}
            \item ($(F\oplus F')(E\oplus E')=FE\oplus F' E'$). 这由矩阵的乘法直接给出. 
            \item ($(E+E')F \sim (EF+E'F)$) 在等价意义下, $\Delta$ 与结合律相容. 依照 $\Delta F_k= (F_k\oplus F_k)\Delta$ 逐级计算即可. 
        \end{enumerate}
        \item (加法结合律) 在 $\sim$ 意义下, 将 $k$ 元和统一化作 $(1\,\cdots \,1)(E_1\oplus \cdots \oplus E_k)(1\,\cdots \,1)^T$ 即可. 
        \item (加法交换律) 在 $\sim$ 意义下, 依照 $\tau$ 与结合律相容, 逐级计算即可. 
    \end{enumerate}
\end{proposition}

\begin{remark}
    $E_n\cdots E_1$ 的加法逆元是 $E_n\cdots (-E_k)\cdots E_1$. 简单而言, 改变长正合列任意奇数次符号即可. 
\end{remark}

\begin{proposition}[$\mathrm{Ext}^2=0$ 的充要条件]
    给定正合列 $E:[0\to A\xrightarrow{i_X}X\xrightarrow{\pi_X} B\to 0]$, $F:[0\to B\xrightarrow{i_Y}Y\xrightarrow{\pi_Y}C\to 0]$, 以下条件等价. 
    \begin{enumerate}
        \item $EF$ 是 $\mathrm{Ext}^2(C,A)$ 中的零对象 ($EF\sim 0$); 
        \item 存在 $E\sim E'\alpha$ 使得 $\alpha F\sim 0$; 
        \item 存在 $E\sim E''\iota_Y$;
        \item 存在 $F\sim \beta F'$ 使得 $E\beta\sim0$; 
        \item 存在 $F\sim \pi_X F''$;
    \end{enumerate}
    \begin{proof}
        证明顺序: $\left[% https://q.uiver.app/#q=WzAsOCxbMSwwLCIyIl0sWzEsMSwiMyJdLFswLDAsIjQiXSxbMCwxLCI1Il0sWzMsMCwiMSJdLFsyLDAsIjJcXHdlZGdlIDMiXSxbMiwxLCIyXFx2ZWUgMyJdLFszLDEsIjEiXSxbMiwzLCIiLDEseyJzdHlsZSI6eyJ0YWlsIjp7Im5hbWUiOiJhcnJvd2hlYWQifX19XSxbMCwyLCIiLDEseyJzdHlsZSI6eyJ0YWlsIjp7Im5hbWUiOiJhcnJvd2hlYWQifX19XSxbMSwzLCIiLDEseyJzdHlsZSI6eyJ0YWlsIjp7Im5hbWUiOiJhcnJvd2hlYWQifX19XSxbNiw3XSxbNCw1XSxbNSw2LCIiLDEseyJzdHlsZSI6eyJ0YWlsIjp7Im5hbWUiOiJhcnJvd2hlYWQifSwiYm9keSI6eyJuYW1lIjoiZGFzaGVkIn19fV0sWzAsMSwiIiwxLHsic3R5bGUiOnsidGFpbCI6eyJuYW1lIjoiYXJyb3doZWFkIn0sImJvZHkiOnsibmFtZSI6ImRhc2hlZCJ9fX1dLFsxNCwxMywiIiwxLHsic2hvcnRlbiI6eyJzb3VyY2UiOjIwLCJ0YXJnZXQiOjIwfX1dXQ==
        \begin{tikzcd}[ampersand replacement=\&]
            4 \& 2 \& {2\wedge 3} \& 1 \\
            5 \& 3 \& {2\vee 3} \& 1
            \arrow[tail reversed, from=1-1, to=2-1]
            \arrow[tail reversed, from=1-2, to=1-1]
            \arrow[""{name=0, anchor=center, inner sep=0}, dashed, tail reversed, from=1-2, to=2-2]
            \arrow[""{name=1, anchor=center, inner sep=0}, dashed, tail reversed, from=1-3, to=2-3]
            \arrow[from=1-4, to=1-3]
            \arrow[tail reversed, from=2-2, to=2-1]
            \arrow[from=2-3, to=2-4]
            \arrow[shorten <=6pt, shorten >=6pt, Rightarrow, from=0, to=1]
        \end{tikzcd}\right]$. 此处 $\wedge$ 是逻辑``与'', $\vee$ 是逻辑``或''. 
        \begin{enumerate}
            \item ($2\to 1$. 同理地, $3\to 1$)
            \begin{equation}
                % https://q.uiver.app/#q=WzAsMTgsWzEsMCwiQSJdLFszLDAsIkIiXSxbNSwwLCJDIl0sWzIsMCwiWCJdLFs0LDAsIlkiXSxbMywxLCJCJyJdLFsxLDEsIkEiXSxbMiwxLCJYJyJdLFs0LDEsIkInXFxvcGx1cyBDIl0sWzUsMSwiQyJdLFswLDAsIkU9RSdcXGFscGhhICJdLFswLDEsIkUnIl0sWzYsMCwiRiJdLFs2LDEsIkYnPVxcYWxwaGEgRiJdLFsxLDIsIkEiXSxbMiwyLCJBIl0sWzQsMiwiQyJdLFs1LDIsIkMiXSxbMCwzLCIiLDEseyJzdHlsZSI6eyJ0YWlsIjp7Im5hbWUiOiJob29rIiwic2lkZSI6ImJvdHRvbSJ9fX1dLFszLDEsIiIsMSx7InN0eWxlIjp7ImhlYWQiOnsibmFtZSI6ImVwaSJ9fX1dLFsxLDQsIiIsMSx7InN0eWxlIjp7InRhaWwiOnsibmFtZSI6Imhvb2siLCJzaWRlIjoiYm90dG9tIn19fV0sWzQsMiwiIiwxLHsic3R5bGUiOnsiaGVhZCI6eyJuYW1lIjoiZXBpIn19fV0sWzEsNSwiXFxhbHBoYSIsMl0sWzYsNywiIiwwLHsic3R5bGUiOnsidGFpbCI6eyJuYW1lIjoiaG9vayIsInNpZGUiOiJib3R0b20ifX19XSxbNyw1LCIiLDAseyJzdHlsZSI6eyJoZWFkIjp7Im5hbWUiOiJlcGkifX19XSxbMyw3XSxbNiwwLCIiLDEseyJsZXZlbCI6Miwic3R5bGUiOnsiaGVhZCI6eyJuYW1lIjoibm9uZSJ9fX1dLFs1LDgsIiIsMSx7InN0eWxlIjp7InRhaWwiOnsibmFtZSI6Imhvb2siLCJzaWRlIjoiYm90dG9tIn19fV0sWzQsOF0sWzgsOSwiIiwxLHsic3R5bGUiOnsiaGVhZCI6eyJuYW1lIjoiZXBpIn19fV0sWzIsOSwiIiwxLHsibGV2ZWwiOjIsInN0eWxlIjp7ImhlYWQiOnsibmFtZSI6Im5vbmUifX19XSxbMTYsMTcsIiIsMSx7ImxldmVsIjoyLCJzdHlsZSI6eyJoZWFkIjp7Im5hbWUiOiJub25lIn19fV0sWzE0LDE1LCIiLDEseyJsZXZlbCI6Miwic3R5bGUiOnsiaGVhZCI6eyJuYW1lIjoibm9uZSJ9fX1dLFsxNSwxNiwiMCJdLFsxNyw5LCIiLDEseyJsZXZlbCI6Miwic3R5bGUiOnsiaGVhZCI6eyJuYW1lIjoibm9uZSJ9fX1dLFsxNiw4XSxbMTUsN10sWzE0LDYsIiIsMSx7ImxldmVsIjoyLCJzdHlsZSI6eyJoZWFkIjp7Im5hbWUiOiJub25lIn19fV1d
    \begin{tikzcd}[ampersand replacement=\&]
        {E=E'\alpha } \& A \& X \& B \& Y \& C \& F \\
        {E'} \& A \& {X'} \& {B'} \& {B'\oplus C} \& C \& {F'=\alpha F} \\
        \& A \& A \&\& C \& C
        \arrow[hook', from=1-2, to=1-3]
        \arrow[two heads, from=1-3, to=1-4]
        \arrow[from=1-3, to=2-3]
        \arrow[hook', from=1-4, to=1-5]
        \arrow["\alpha"', from=1-4, to=2-4]
        \arrow[two heads, from=1-5, to=1-6]
        \arrow[from=1-5, to=2-5]
        \arrow[Rightarrow, no head, from=1-6, to=2-6]
        \arrow[Rightarrow, no head, from=2-2, to=1-2]
        \arrow[hook', from=2-2, to=2-3]
        \arrow[two heads, from=2-3, to=2-4]
        \arrow[hook', from=2-4, to=2-5]
        \arrow[two heads, from=2-5, to=2-6]
        \arrow[Rightarrow, no head, from=3-2, to=2-2]
        \arrow[Rightarrow, no head, from=3-2, to=3-3]
        \arrow[from=3-3, to=2-3]
        \arrow["0", from=3-3, to=3-5]
        \arrow[from=3-5, to=2-5]
        \arrow[Rightarrow, no head, from=3-5, to=3-6]
        \arrow[Rightarrow, no head, from=3-6, to=2-6]
    \end{tikzcd}.
            \end{equation}
            \item ($2\leftrightarrow 3$. 同理地, $4\leftrightarrow 5$) 先看 $2\to 3$. 若 $E=E'\alpha$ 使得 $\alpha F=0$, 则存在分解 $\alpha = \varphi i_Y$. 取 $E'':=(E'\varphi)$. 再看 $3\to 2$. 只需证明 $\iota_Y F=0$. 依照``拉回是可裂单''等价于``被拉回的态射存在分解'', 因此 $\iota_YF$ 是可裂短正合列.
            \item ($3\leftrightarrow 5$) 这都等价于 ``$\star$ 是推出与拉回方块'': 
\begin{equation}
    % https://q.uiver.app/#q=WzAsMTIsWzEsMSwiQSJdLFsyLDEsIlgiXSxbMywxLCJCIl0sWzMsMiwiWSJdLFszLDMsIkMiXSxbMywwLCJGIl0sWzAsMSwiRSJdLFsyLDIsIlciXSxbMCwyLCJFJyciXSxbMiwwLCJGJyciXSxbMSwyLCJBIl0sWzIsMywiQyJdLFswLDEsImlfWCIsMCx7InN0eWxlIjp7InRhaWwiOnsibmFtZSI6Imhvb2siLCJzaWRlIjoiYm90dG9tIn19fV0sWzEsMiwiXFxwaV9YIiwwLHsic3R5bGUiOnsiaGVhZCI6eyJuYW1lIjoiZXBpIn19fV0sWzIsMywiaV9ZIiwwLHsic3R5bGUiOnsidGFpbCI6eyJuYW1lIjoiaG9vayIsInNpZGUiOiJib3R0b20ifX19XSxbMyw0LCJcXHBpX1kiLDAseyJzdHlsZSI6eyJoZWFkIjp7Im5hbWUiOiJlcGkifX19XSxbMTEsNCwiIiwxLHsibGV2ZWwiOjIsInN0eWxlIjp7ImhlYWQiOnsibmFtZSI6Im5vbmUifX19XSxbMCwxMCwiIiwxLHsibGV2ZWwiOjIsInN0eWxlIjp7ImhlYWQiOnsibmFtZSI6Im5vbmUifX19XSxbMTAsNywiIiwxLHsic3R5bGUiOnsidGFpbCI6eyJuYW1lIjoiaG9vayIsInNpZGUiOiJib3R0b20ifX19XSxbNywzLCIiLDEseyJzdHlsZSI6eyJoZWFkIjp7Im5hbWUiOiJlcGkifX19XSxbMSw3LCIiLDEseyJzdHlsZSI6eyJ0YWlsIjp7Im5hbWUiOiJob29rIiwic2lkZSI6ImJvdHRvbSJ9fX1dLFs3LDExLCIiLDEseyJzdHlsZSI6eyJoZWFkIjp7Im5hbWUiOiJlcGkifX19XSxbMjAsMTQsIlxcc3RhciAiLDEseyJzaG9ydGVuIjp7InNvdXJjZSI6MjAsInRhcmdldCI6MjB9LCJzdHlsZSI6eyJib2R5Ijp7Im5hbWUiOiJub25lIn0sImhlYWQiOnsibmFtZSI6Im5vbmUifX19XV0=
\begin{tikzcd}[ampersand replacement=\&]
	\&\& {F''} \& F \\
	E \& A \& X \& B \\
	{E''} \& A \& W \& Y \\
	\&\& C \& C
	\arrow["{i_X}", hook', from=2-2, to=2-3]
	\arrow[Rightarrow, no head, from=2-2, to=3-2]
	\arrow["{\pi_X}", two heads, from=2-3, to=2-4]
	\arrow[""{name=0, anchor=center, inner sep=0}, hook', from=2-3, to=3-3]
	\arrow[""{name=1, anchor=center, inner sep=0}, "{i_Y}", hook', from=2-4, to=3-4]
	\arrow[hook', from=3-2, to=3-3]
	\arrow[two heads, from=3-3, to=3-4]
	\arrow[two heads, from=3-3, to=4-3]
	\arrow["{\pi_Y}", two heads, from=3-4, to=4-4]
	\arrow[Rightarrow, no head, from=4-3, to=4-4]
	\arrow["{\star }"{description}, draw=none, from=0, to=1]
\end{tikzcd}
\end{equation}
        \item ($1\to (2\vee 4)=(2\wedge 4)$) 若 $EF\sim 0$, 则存在最小的 $k$ 使得
        \begin{align}
            EF &\overset \square= (E_1\alpha_1)F \overset \S \sim  E_1 (\alpha_1 F) \overset\dagger = E_1 (\beta_1 F_1) \overset \star \sim  (E_1\beta_1) F_1 = (E_2\alpha_2) F_1 \sim \\[6pt]
            \cdots &\sim E_{k}(\alpha_{k}F_{k-1}) \sim(E_{k}\beta_{k}) F_{k} \sim E_{k+1} (\alpha_{k+1} F_{k}) = E_{k+1} 0. 
        \end{align}
        依照数学归纳法, $1\to (2\vee 4)=(2\wedge 4)$ 对 $k=0$ 成立. 归纳步骤即命题: 若 $\square\S\dagger\star$ 中的等号后式满足 $(2\wedge 4)$, 则前式亦然. 
        \begin{itemize}
            \item ($\star$) 若 $(E_1\beta_1)F_1$ 满足``存在 $F_1\sim \gamma F'$ 使得 $(E_1\beta_1)\gamma\sim 0$'', 则 $E_1(\beta_1 F_1)$ 满足`` $\beta_1 F_1\sim \gamma \beta_1 F_1$, 使得 $E_1\gamma \beta_1 \sim 0$''. 
            \item ($\dagger$) 此处 $\alpha_1 F= \beta_1 F_1$ 是相同的对象, 故满足相同的命题. 
            \item ($\S$) 若 $E_1(\alpha_1 F)$ 满足``存在 $E_1\sim E'\delta$ 使得 $\delta\alpha_1 F\sim 0$'', 则 $(E_1\alpha)F$ 满足``$E_1\alpha_1\sim E_1\alpha_1\delta$, 使得 $\alpha_1\delta F\sim 0$''. 
            \item ($\square$) 此处 $E\alpha_1 = E$ 是相同的对象, 故满足相同的命题. 
        \end{itemize}
        \end{enumerate}
    \end{proof}
\end{proposition}

\begin{remark}
    对 $\mathrm{Ext}^n$ 的论证是同样的: 记 $E\in \mathrm{Ext}^p(B,A)$ 与 $F\in \mathrm{Ext}^q(C,B)$, 则 $EF\sim 0$ 的充要条件为: 
    \begin{enumerate}
        \item 存在 $E\sim E'\alpha$ 使得 $\alpha F\sim 0$; 或对偶地, 存在 $F\sim \beta F'$ 使得 $E\beta\sim 0$; 
        \item $E\sim E''\iota_Y$ ($\iota_Y$ 是正合列 $F$ 的首个单态射); 或对偶地, 存在 $F\sim \pi_X F''$ 使得 $E\pi_X\sim 0$ ($\pi_X$ 是正合列 $E$ 的末个满态射). 
    \end{enumerate} 
\end{remark}

\begin{proposition}[$\mathrm{Ext}^n$ 的长正合列]
    给定短正合列 $S:=0\to A\xrightarrow i B\xrightarrow c C\to 0$, 则有长正合列
    \begin{equation}
        \cdots \xrightarrow{\delta^{n-1}} \mathrm{Ext}^n(X,A)\xrightarrow{\mathrm{Ext}^n(X,i)}\mathrm{Ext}^n(X,B)\xrightarrow{\mathrm{Ext}^n(X,c)}\mathrm{Ext}^n(X,C)\xrightarrow{\delta^n} \mathrm{Ext}^{n+1}(X,A)\to \cdots .
    \end{equation}
    \begin{proof}
        依次证明 $\mathrm{Ext}^n(X,A)$, $\mathrm{Ext}^n(X,B)$ 与 $\mathrm{Ext}^n(X,C)$ 处正合性. 以上一切 $\delta^k$ 无非左复合 $S$. 
        \begin{enumerate}
            \item ($\mathrm{Ext}^n(X,A)$ 处正合) 任取 $SE\in \mathrm{im}(\delta^{n-1})$, 则自然有 $(iS)E=0E\sim 0$. 反之, 若 $iE\sim 0$, 则写 $E$ 作 $E_{n}E_{(n-1)\to 1}$. 此时存在 $E_{(n-1)\to 1}\sim \alpha F$ 使得 $iE_n\alpha\sim 0$. 依照前文对 $\delta^0$ 正合性的说明, 存在 $E_n\alpha\sim S\beta$. 因此 $E\sim S(\beta F)\in \mathrm{im}(\delta^{n-1})$. 
            \item ($\mathrm{Ext}^n(X,B)$ 处正合) 将第一问中的所有 $(i,S)$ 换作 $(c,i)$ 即可. 
            \item ($\mathrm{Ext}^n(X,C)$ 处正合) 将第一问中的所有 $(i,S)$ 换作 $(S,c)$ 即可. 
        \end{enumerate}
    \end{proof}
\end{proposition}

\begin{proposition}[拉直零对象的锯齿]
    若 $E\sim 0$, 则存在 $E\xrightarrow\sim F\xleftarrow\sim \mathcal O$ (等价地, $E\xleftarrow\sim F\xrightarrow\sim \mathcal O$). 此处 $\xrightarrow\sim$ 与 $\xleftarrow\sim $ 指``长度为 $1$ 的等价'',
    \begin{equation}
        \mathcal O:\quad 0\to X=X\to 0\to\cdots \to 0\quad \oplus \quad 0\to \cdots\to 0\to Y=Y\to 0. 
    \end{equation}
    \begin{proof}
        只看 $E\xrightarrow\sim F\xleftarrow\sim \mathcal O$. 存在 $F\xrightarrow\sim \mathcal O$ 的充要条件是, $F$ 最右端的满态射可裂. 此时取 $\alpha:E\to F$ 如下: 
        \begin{itemize}
            \item 记 $E=E_n\cdots E_1$, 则存在 $\alpha_n$ 使得 $E_n=F_n\alpha_n$, 且 $\alpha_n E_{(n-1)\to 1}\sim 0$; 
            \item 对 $\alpha_k E_{k-1}\cdots E_1\sim 0$, 存在 $\alpha_k E_{k-1}=F_{k-1}\alpha_{k-1}$ 使得 $\alpha_{k-1} E_{k-2}\cdots E_1\sim 0$; 
            \item 最后 $\alpha_2 E_1=:F_1$ 是可裂短正合列. 
        \end{itemize}
    \end{proof}
\end{proposition}

\begin{remark}
    下给出一例``长度不为 $1$ 的等价''. 在 $\mathbb R[x]$-模范畴中取长为 $2$ 的短正合列: 
    \begin{equation}
        0\to \mathbb R\to \mathbb R[x]/(x^2)\xrightarrow {\times x}\mathbb R[x]/(x^2)\to \mathbb R\to 0\quad (\mathrm{Mod}_{\mathbb R[x]}).
    \end{equation}
    显然以上正合列不可裂. 依照 $\mathrm{Ext}^2=0$ 的充要条件, 以上 $\times x$ 处态射的满-单分解是推出与拉回方块 
    \begin{equation}
        % https://q.uiver.app/#q=WzAsNCxbMCwwLCJcXG1hdGhiYiBSW3hdLyh4XjIpIl0sWzEsMSwiXFxtYXRoYmIgUlt4XS8oeF4yKSJdLFsxLDAsIlxcbWF0aGJiIFIiXSxbMCwxLCJcXG1hdGhiYiBSW3hdLyh4XjMpIl0sWzAsMiwiIiwxLHsic3R5bGUiOnsiaGVhZCI6eyJuYW1lIjoiZXBpIn19fV0sWzIsMSwiIiwxLHsic3R5bGUiOnsidGFpbCI6eyJuYW1lIjoiaG9vayIsInNpZGUiOiJib3R0b20ifX19XSxbMCwzLCIiLDEseyJzdHlsZSI6eyJ0YWlsIjp7Im5hbWUiOiJob29rIiwic2lkZSI6ImJvdHRvbSJ9fX1dLFszLDEsIiIsMSx7InN0eWxlIjp7ImhlYWQiOnsibmFtZSI6ImVwaSJ9fX1dLFswLDEsIlxcdGltZXMgeCIsMSx7InN0eWxlIjp7ImJvZHkiOnsibmFtZSI6ImRhc2hlZCJ9fX1dXQ==
\begin{tikzcd}[ampersand replacement=\&]
	{\mathbb R[x]/(x^2)} \& {\mathbb R} \\
	{\mathbb R[x]/(x^3)} \& {\mathbb R[x]/(x^2)}
	\arrow[two heads, from=1-1, to=1-2]
	\arrow[hook', from=1-1, to=2-1]
	\arrow["{\times x}"{description}, dashed, from=1-1, to=2-2]
	\arrow[hook', from=1-2, to=2-2]
	\arrow[two heads, from=2-1, to=2-2]
\end{tikzcd}.
    \end{equation}
    实际上, 由 $\mathbb R[x]$ 的整体维数是 $1$ 知 $\mathrm{Ext}^2=0$. 
\end{remark}

\begin{proposition}
    若 $E\sim F$, 则存在 $E\xrightarrow{\sim} \bullet\xleftarrow{\sim} \bullet\xrightarrow{\sim} F$ (对称表述略). 
\begin{proof}
    由于 $E+(-F)\sim 0$, 故存在 $0\xrightarrow\sim K\xleftarrow\sim E\oplus (-F)$. 在右侧直和 $\oplus F$, 依照 $(E\oplus F)\oplus G=E\oplus (F\oplus G)$, 取 $(-F)\oplus F\xrightarrow\sim 0$. 
    \begin{equation}
        F\xrightarrow\sim K\oplus F\xleftarrow\sim E\oplus (-F)\oplus F\xrightarrow \sim E
    \end{equation}
    即为所求. 
\end{proof}
\end{proposition}

\begin{remark}
    假若范畴有足够投射对象 (内射对象), 维数移位给出更简短的刻画: $E\xleftarrow\sim \bullet \xrightarrow \sim F$ ($E\xrightarrow\sim \bullet \xleftarrow \sim F$). 
\end{remark}

    