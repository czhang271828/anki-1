\begin{abstract}
    介紹基本 Quiver 表示, 並從一些代數幾何視角 (將``一個表示''對應至代數群作用下的``一個軌道''), 並證明 Gabriel 定理. 
    
    La preuve du théorème de Gabriel nécessite trois ingrédients: 
    \begin{enumerate}
        \item Classical geometric representation theory (representation varieties)
        \item Noncommutative, homological algebra (Ringel Lemma, ``brick'')
        \item Classification of graphs (due to Tits)
    \end{enumerate}
    主要參考 \cite{Brion2009RepresentationsOQ} 與 \cite{BerestMehrle2017} 等筆記. 
\end{abstract}

\subsubsection{箭圖表示的基本定理, Euler 二次型等}

\begin{definition}[箭圖 (quiver)]\label{quiver}
    要件 $(Q_0, Q_1, s, t)$, 記路代數 $A := kQ$, \textbf{默認左模}. 特別地, 有兩種``有限'': 
    \begin{enumerate}
        \item (有限型). $|Q_0| + |Q_1| < ∞$; \parnote{圖性質}
        \item (有限維). $\dim < ∞$, (等價地, 有限型且沒有環). \parnote{表示性質}
    \end{enumerate}
\end{definition}

\begin{remark}
    \textbf{默認箭圖是有限型的}: 對有限型箭圖, 冪等分解 $∑ e_i$ 方有意義. \parnote{type finie}
\end{remark}

\begin{theorem}[標準消解]
    有函子的短正合列
    \begin{equation}
        0 → ∑_{α ∈ Q_1} \underbracket{A e_{t(α)}}\limits_{P(t(α))} ⊗_k e_{s(α)} X → ∑_{i ∈ Q_0} \underbracket{Ae_i}\limits_{P(i)} ⊗_k e_i X → X → 0. 
    \end{equation} \parnote{$⊗_k$ 實際是關於``某單代數''的張量}
    ``從邊到點''的態射: 對 $r ⊗ x ∈ P(t(α)) ⊗ e_{s(α)}A$, 定義
    \begin{equation}
        r ⊗ x ↦ (r ⋅ α) ⊗ x − r ⊗ (α ⋅ x).   
    \end{equation}
\end{theorem}

\begin{example}[分次代數視角]
    $kQ$ 無非單代數 $S:= kQ_0$ 與雙模 $V:= kQ_1$ 張成的分次代數. 以上 
    \begin{equation}
        0 → A ⊗_S V ⊗_S X \xrightarrow{[1∣ 2∣ 3] \ ↦ \ ([12∣ 3]-[1∣ 23])} A ⊗_S X \xrightarrow{[1∣ 2] \ ↦ \ [12]} X → 0
    \end{equation}
\end{example}

\begin{definition}[維度向量]
    維度向量衡量了 Krull-Schmidt 範疇中不可分解對象的維度, 也就是 $K_0$ 群中對應的元素. 例如 $𝐝𝐢𝐦 M = (\dim (e_a M))_{a ∈ Q_0}$. \parnote{合成列單模數}
\end{definition}

\begin{definition}[Euler 型 $(-,-)_Q$, 或 Tits 形式]
    受 $0 → (X, Y) → (A⊗ _S X, Y) → (A⊗ _S V⊗ _S X, Y) → \mathrm{Ext}^1(X,Y) → 0$ 啟發, 定義 $𝐝𝐢𝐦 X$ 與 $𝐝𝐢𝐦 Y$ 的雙線性運算 ($C$ 是邊的鄰接矩陣): 
    \begin{equation}
        \dim \mathrm{Hom}(X,Y) - \dim \mathrm{Ext}^1(X,Y) = \underbracket{∑ _{i ∈ Q_0} (e_iX, e_i Y)}\limits_{𝐝𝐢𝐦 X⋅𝐝𝐢𝐦 Y} - \underbracket{∑ _{α ∈ Q_1}(e_{t(α)}X, e_{s(α)}Y)}\limits_{𝐝𝐢𝐦 X ⋅ C ⋅ 𝐝𝐢𝐦 Y}. 
    \end{equation}
    提煉出 $(-,-)_Q$ 定義式
    \begin{equation}
        (𝐝𝐢𝐦 X, 𝐝𝐢𝐦 Y) := (𝐝𝐢𝐦 X)^T ⋅ (I - C) ⋅ 𝐝𝐢𝐦 Y
    \end{equation}
    從不可分解模的角度, $\mathrm{Hom}(-,-)$ 貢獻點; $\mathrm{Ext}^1(-,-)$ 貢獻邊. 
\end{definition}

\begin{remark}
    此處\textbf{箭圖不帶關係}, Euler 型與 Tits 型姑且可以混同. 
\end{remark}

\begin{definition}[Euler 二次型]
    定義 $⟨ -,-⟩$ 是 $(-,-)_Q$ 的對稱化, 形如``無向圖的 Cartan 矩陣''. 記 $q(X) = ⟨ X, X⟩$
\end{definition}

\subsubsection{幾何視角}

\begin{example}[表示空間: 群作用, 軌道]
    表示論的目標: 描述所有的不可約表示. 今給定路代數 $A:= kQ$, 描述等價類的方式是
    \begin{itemize}
        \item 先確定維數向量, 再確定模結構; 等價地, \parnote{用 $𝐝𝐢𝐦$ 確定更粗的等價類}
        \item 先給定 $S:= kQ_0$ 模, 再擴張作 $A=kQ$ 模. 
    \end{itemize}
    對維數向量 $v$, 表示的等價類形如 $∏_{α ∈ Q_1} (k^{v_{t(α)}},k^{t_{s(α)}}) / ∼$, 即, ``一堆矩陣組''的等價類. 
    
    等價類即群 $∏_i \mathrm{Gl}_{v_i}(k)$-共軛作用的軌道, 例如 $(m)_{i → j}$ 被作用爲 $(g)_{v_i × v_i} ⋅ (m)_{i → j} ⋅ (g)⁻¹_{v_j × v_j}$. \parnote{廣群``共軛''}

    爲將``線性空間維度''和``概型維度''搭配, 另需引入 Krull 維度, 其精髓在於軌道公式. 
\end{example}

\begin{definition}[Krull 維度]
    選用 $𝔸 ^d := 𝔸_k ^d$ 上的 Zariski 拓撲. 稱 $X ⊆ 𝔸^d$ 局部閉, 當且僅當 $X$ 是 $\overline X$ 的開子集, 或等價地, 一個開集與一個閉集的交. \parnote{locally closed}

稱 $X$ 不可約, 當且僅當非空開子集必稠密; 等價地, 不存在 $\boxed{∘ \ ∘ }$. \parnote{irreducible}

最後定義局部閉集 $X ⊆ 𝔸^d$ 的 Krull 維度: 內部不可約集組成的閉鏈的最大長度. 
\end{definition}

\begin{example}
    例如 $𝔸^2 = \mathrm{Spec}(k[x,y])$ 中直線 $𝔸^1 ≃ \mathrm{Spec}(k[x])$ 的 Krull 維度是 $1$: $k ⊊ k[x]$ 是長度爲 $1$ 的極大鏈. 
\end{example}

\begin{proposition}
    依照 Proposition I.7.1, \cite{hartshorne1999algebraic} 等結論, 
    \begin{enumerate}
        \item $\dim 𝔸^d = d$, (約定) $\dim ∅ = -∞$; 
        \item $\dim (U ∪ V) = \max(\dim U , \dim V)$; 
        \item $\dim (X ∩ Y) ≥ \dim X+ \dim Y - d$; 
        \item $\dim U = \dim \overline U$. \parnote{$\dim \mathrm{Aut}$ = $\dim \mathrm{End}$}
    \end{enumerate}
    也可以先從拓撲空間的 Krull 維度入手. 概型的 Krull 維度就是拓撲空間的 Krull 維度. 
\end{proposition}

\begin{theorem}[軌道公式]
    假定 $G$ 是連通的代數群概型, $X$ 是 $G$-代數簇, 則 
    \begin{enumerate}
        \item 所有 $G$-軌道 $O_x$ 都是不可約的局部閉集; \parnote{軌道性質}
        \item 對任意 $x$, 穩定子群 $\mathrm{Stab}_G(x)$ 是閉子群; \parnote{穩定子群}
        \item 對任意 $x$, 有 $\dim G = \dim O_x + \dim \mathrm{Stab}_G(x)$; \parnote{維數公式}
        \item 對任意 $x$, $(\overline{O_x} \backslash O_x) = ⋃ _{\dim O_y < \dim O_x} O_y$. \parnote{主元}
    \end{enumerate}
\end{theorem}

\begin{proposition}
    假定 $\dim$ 是 Krull 維度, $\dim _k$ 是線性空間維度, 則有連接維數的關鍵公式: 對任意 $𝐝𝐢𝐦 X =v$, 總有 
    \begin{equation}
        \underbracket{\dim \mathrm{Rep}(v)}\limits_{v ^T ⋅ Q_1 ⋅ v} = \underbracket{\dim O_X}\limits_{v ^T ⋅ Q_1 ⋅ v - \dim \mathrm{End}(X)} + \dim _k\mathrm{Ext}^1 (X, X)
    \end{equation}
\end{proposition}

\begin{example}[自垂直對象]
    對給定的 $v = 𝐝𝐢𝐦 X$, 是否存在特殊的對象: $\mathrm{Ext}^1 (X , X) = 0$? \parnote{結合 Kac 定理}
    \begin{enumerate}
        \item $\mathrm{Ext}^1 (X, X) = 0$ 當且僅當 $\overline{O_X} = \mathrm{Rep}(v)$. \parnote{極大軌道}
        \item 對給定的 $v$, 作用 $\mathrm{Gl}(v) → \mathrm{Rep}(v)$ 至多有一條極大軌道; 
        \item 對不可裂短正合列 $0 → L → M → N → 0$, 總有 $O_{L ⊕ N} ⊆ O_M$; 
        \item 若極大軌道的對象 $X ≃ L ⊕ N$, 則 $\mathrm{Ext}^1(L, N) =0$. 
        \item 當且僅當 $X$ 半單, $O_X = \overline{O_X}$. 
    \end{enumerate}
\end{example}

\subsubsection{圖的結論: 一些線性代數}

\begin{definition}[圖的二次型]
    給定有限無向圖 (允許自環, 重邊) $Γ$, 記 $q_Γ$ 是鄰接矩陣對應的二次型. 此時 $q_Q = q_{|Q|}$ 與 Euler 二次型吻合. \parnote{與 Quiver 定向無關} 照常定義 $q_Γ$ 正定, 半正定, 以及零空間 $N(q_Γ)$.  
\end{definition}

\begin{theorem}[有限圖分類定理]
    給定連通圖 $Γ$, 以下分類 $q_Γ$ (有限型, 仿射型, 其他). 
    \begin{enumerate}
        \item (有限型, Dynkin 型, 即 $q_Γ$ 正定) 也就是熟知的 $A_{≥ 1}$, $D_{≥ 3}$, $E_{6,7,8}$, $n$ 是頂點數. 
        \item (仿射型, Euclidean 型, 即 $q_Γ$ 半定但 $\dim \ker N(q_Γ) =1$) 也就是熟知的 $\widetilde A_{≥ 0}$, $\widetilde D_{≥ 4}$, $\widetilde E_{6,7,8}$, 頂點數 $n +1$. \parnote{extending vertex}將張成 $\ker N(q_Γ)$ 的所有格點悉列諸下圖: 
        \begin{equation}
            % https://q.uiver.app/#q=WzAsNDQsWzAsMiwiMSJdLFsyLDIsIjEiXSxbMCwxLCJcXHZkb3RzICJdLFsyLDEsIlxcdmRvdHMgIl0sWzAsMCwiMSJdLFsyLDAsIjEiXSxbMSwxLCJcXGJveGVke1xcd2lkZXRpbGRlIEFfbn0iXSxbMywwLCIxIl0sWzMsMiwiMSJdLFszLDEsIjIiXSxbNywwLCIxIl0sWzcsMSwiMiJdLFs3LDIsIjEiXSxbNSwyLCJcXGJveGVke1xcd2lkZXRpbGRlIERfbiB9Il0sWzExLDEsIlxcYm94ZWR7XFx3aWRldGlsZGUgRV82fSJdLFs0LDEsIjIiXSxbNiwxLCIyIl0sWzUsMSwiXFxjZG90cyAiXSxbMTIsMCwiMSJdLFsxMSwwLCIyIl0sWzEwLDAsIjMiXSxbOSwwLCIyIl0sWzgsMCwiMSJdLFsxMCwxLCIyIl0sWzEwLDIsIjEiXSxbNCw1LCJcXGJveGVke1xcd2lkZXRpbGRlIEVfN30iXSxbMyw0LCI0Il0sWzAsNCwiMSJdLFsxLDQsIjIiXSxbMiw0LCIzIl0sWzQsNCwiMyJdLFs1LDQsIjIiXSxbNiw0LCIxIl0sWzMsNSwiMiJdLFs4LDUsIlxcYm94ZWR7XFx3aWRldGlsZGUgRV84fSJdLFs3LDQsIjYiXSxbNyw2LCIyIl0sWzcsNSwiNCJdLFs3LDMsIjMiXSxbOCw0LCI1Il0sWzksNCwiNCJdLFsxMCw0LCIyIl0sWzExLDQsIjIiXSxbMTIsNCwiMSJdLFswLDIsIiIsMCx7InN0eWxlIjp7ImhlYWQiOnsibmFtZSI6Im5vbmUifX19XSxbMiw0LCIiLDAseyJzdHlsZSI6eyJoZWFkIjp7Im5hbWUiOiJub25lIn19fV0sWzQsNSwiIiwwLHsic3R5bGUiOnsiaGVhZCI6eyJuYW1lIjoibm9uZSJ9fX1dLFsxLDAsIiIsMCx7InN0eWxlIjp7ImhlYWQiOnsibmFtZSI6Im5vbmUifX19XSxbNyw5LCIiLDAseyJzdHlsZSI6eyJoZWFkIjp7Im5hbWUiOiJub25lIn19fV0sWzksOCwiIiwwLHsic3R5bGUiOnsiaGVhZCI6eyJuYW1lIjoibm9uZSJ9fX1dLFsxMSwxMiwiIiwwLHsic3R5bGUiOnsiaGVhZCI6eyJuYW1lIjoibm9uZSJ9fX1dLFsxMCwxMSwiIiwwLHsic3R5bGUiOnsiaGVhZCI6eyJuYW1lIjoibm9uZSJ9fX1dLFs5LDE1LCIiLDAseyJzdHlsZSI6eyJoZWFkIjp7Im5hbWUiOiJub25lIn19fV0sWzE1LDE3LCIiLDAseyJzdHlsZSI6eyJoZWFkIjp7Im5hbWUiOiJub25lIn19fV0sWzE3LDE2LCIiLDAseyJzdHlsZSI6eyJoZWFkIjp7Im5hbWUiOiJub25lIn19fV0sWzE2LDExLCIiLDAseyJzdHlsZSI6eyJoZWFkIjp7Im5hbWUiOiJub25lIn19fV0sWzIwLDE5LCIiLDAseyJzdHlsZSI6eyJoZWFkIjp7Im5hbWUiOiJub25lIn19fV0sWzE5LDE4LCIiLDAseyJzdHlsZSI6eyJoZWFkIjp7Im5hbWUiOiJub25lIn19fV0sWzIwLDIzLCIiLDIseyJzdHlsZSI6eyJoZWFkIjp7Im5hbWUiOiJub25lIn19fV0sWzIzLDI0LCIiLDIseyJzdHlsZSI6eyJoZWFkIjp7Im5hbWUiOiJub25lIn19fV0sWzIwLDIxLCIiLDIseyJzdHlsZSI6eyJoZWFkIjp7Im5hbWUiOiJub25lIn19fV0sWzIxLDIyLCIiLDIseyJzdHlsZSI6eyJoZWFkIjp7Im5hbWUiOiJub25lIn19fV0sWzI2LDI5LCIiLDIseyJzdHlsZSI6eyJoZWFkIjp7Im5hbWUiOiJub25lIn19fV0sWzI5LDI4LCIiLDIseyJzdHlsZSI6eyJoZWFkIjp7Im5hbWUiOiJub25lIn19fV0sWzI4LDI3LCIiLDIseyJzdHlsZSI6eyJoZWFkIjp7Im5hbWUiOiJub25lIn19fV0sWzI2LDMwLCIiLDAseyJzdHlsZSI6eyJoZWFkIjp7Im5hbWUiOiJub25lIn19fV0sWzMwLDMxLCIiLDAseyJzdHlsZSI6eyJoZWFkIjp7Im5hbWUiOiJub25lIn19fV0sWzMxLDMyLCIiLDAseyJzdHlsZSI6eyJoZWFkIjp7Im5hbWUiOiJub25lIn19fV0sWzMzLDI2LCIiLDAseyJzdHlsZSI6eyJoZWFkIjp7Im5hbWUiOiJub25lIn19fV0sWzM2LDM3LCIiLDAseyJzdHlsZSI6eyJoZWFkIjp7Im5hbWUiOiJub25lIn19fV0sWzM3LDM1LCIiLDEseyJzdHlsZSI6eyJoZWFkIjp7Im5hbWUiOiJub25lIn19fV0sWzM1LDM5LCIiLDEseyJzdHlsZSI6eyJoZWFkIjp7Im5hbWUiOiJub25lIn19fV0sWzM5LDQwLCIiLDEseyJzdHlsZSI6eyJoZWFkIjp7Im5hbWUiOiJub25lIn19fV0sWzQwLDQxLCIiLDEseyJzdHlsZSI6eyJoZWFkIjp7Im5hbWUiOiJub25lIn19fV0sWzQyLDQzLCIiLDEseyJzdHlsZSI6eyJoZWFkIjp7Im5hbWUiOiJub25lIn19fV0sWzUsMywiIiwwLHsic3R5bGUiOnsiaGVhZCI6eyJuYW1lIjoibm9uZSJ9fX1dLFszLDEsIiIsMCx7InN0eWxlIjp7ImhlYWQiOnsibmFtZSI6Im5vbmUifX19XSxbMzgsMzUsIiIsMCx7InN0eWxlIjp7ImhlYWQiOnsibmFtZSI6Im5vbmUifX19XSxbNDEsNDIsIiIsMSx7InN0eWxlIjp7ImhlYWQiOnsibmFtZSI6Im5vbmUifX19XV0=
\begin{tikzcd}[ampersand replacement=\&,sep=tiny]
	1 \&\& 1 \& 1 \&\&\&\& 1 \& 1 \& 2 \& 3 \& 2 \& 1 \\
	{\vdots } \& {\boxed{\widetilde A_n}} \& {\vdots } \& 2 \& 2 \& {\cdots } \& 2 \& 2 \&\&\& 2 \& {\boxed{\widetilde E_6}} \\
	1 \&\& 1 \& 1 \&\& {\boxed{\widetilde D_n }} \&\& 1 \&\&\& 1 \\
	\&\&\&\&\&\&\& 3 \\
	1 \& 2 \& 3 \& 4 \& 3 \& 2 \& 1 \& 6 \& 5 \& 4 \& 3 \& 2 \& 1 \\
	\&\&\& 2 \& {\boxed{\widetilde E_7}} \&\&\& 4 \& {\boxed{\widetilde E_8}} \\
	\&\&\&\&\&\&\& 2
	\arrow[no head, from=1-1, to=1-3]
	\arrow[no head, from=1-3, to=2-3]
	\arrow[no head, from=1-4, to=2-4]
	\arrow[no head, from=1-8, to=2-8]
	\arrow[no head, from=1-10, to=1-9]
	\arrow[no head, from=1-11, to=1-10]
	\arrow[no head, from=1-11, to=1-12]
	\arrow[no head, from=1-11, to=2-11]
	\arrow[no head, from=1-12, to=1-13]
	\arrow[no head, from=2-1, to=1-1]
	\arrow[no head, from=2-3, to=3-3]
	\arrow[no head, from=2-4, to=2-5]
	\arrow[no head, from=2-4, to=3-4]
	\arrow[no head, from=2-5, to=2-6]
	\arrow[no head, from=2-6, to=2-7]
	\arrow[no head, from=2-7, to=2-8]
	\arrow[no head, from=2-8, to=3-8]
	\arrow[no head, from=2-11, to=3-11]
	\arrow[no head, from=3-1, to=2-1]
	\arrow[no head, from=3-3, to=3-1]
	\arrow[no head, from=4-8, to=5-8]
	\arrow[no head, from=5-2, to=5-1]
	\arrow[no head, from=5-3, to=5-2]
	\arrow[no head, from=5-4, to=5-3]
	\arrow[no head, from=5-4, to=5-5]
	\arrow[no head, from=5-5, to=5-6]
	\arrow[no head, from=5-6, to=5-7]
	\arrow[no head, from=5-8, to=5-9]
	\arrow[no head, from=5-9, to=5-10]
	\arrow[no head, from=5-10, to=5-11]
	\arrow[no head, from=5-11, to=5-12]
	\arrow[no head, from=5-12, to=5-13]
	\arrow[no head, from=6-4, to=5-4]
	\arrow[no head, from=6-8, to=5-8]
	\arrow[no head, from=7-8, to=6-8]
\end{tikzcd}.
        \end{equation}
        \item (無限型) 其他類型. 
    \end{enumerate}
\end{theorem}

\begin{definition}[根系]
    給定有限型或仿射型 $Γ$, 定義根系 $Δ := \{v ∈ ℤ ^n ∣ q_Γ (v) ≤ 1\}$. 約定 $0 ∉ Δ$. 定義
    \begin{enumerate}
        \item (實根) 使得 $q_Γ(v) = 1$ 的根, 記作 $v ∈ Δ^r$; 
        \item (虛根) 使得 $q_Γ(v) = 1$ 的根, 記作 $v ∈ Δ^i$; 
        \item (正根) 各項爲正整數的根, 記作 $v ∈ Δ_+$. 
    \end{enumerate}
\end{definition}

\begin{proposition}[根系結構定理]
    仍選用有限型或仿射型 $Γ$, 
    \begin{enumerate}
        \item $Δ$ 關於對稱, Weyl 反射封閉; 
        \item $Δ = Δ_+ ⊔ Δ _+$, 換言之, 沒有既正又負的根; 
        \item $Γ$ 是有限型 $⟺$ $Δ$ 是有限集 $⟺$ $Δ^i = ∅$; 
        \item $Γ$ 是仿射型 $⟺$ $Δ$ 是無限集 $⟺$ $Δ^i ≃ ℤ$. 
    \end{enumerate}
\end{proposition}

\begin{theorem}[Gabriel]
    給定有限型 $Γ$, 有限維不可約表示一一對應 $Δ_+$. 
    \begin{enumerate}
        \item 從不可約表示到 $Δ_+$, 對應方式 $X ↦ 𝐝𝐢𝐦 X$; 
        \item 從 $Δ_+$ 到不可約表示, 對應方式 $v ↦ \mathrm{Rep}(v)$ 中極大軌道. 
    \end{enumerate} 
    $kQ$ 的不可約表示有限, 當且僅當 $Γ$ 是有限型圖 (及其有限無交並). 
\end{theorem}

\begin{remark}
    對一般的有限箭圖, 可以類似地定義根系. Kac 定理解答了如下問題: 以 $v$ 爲維度向量的不可約表示有多少? 
\end{remark}

\begin{theorem}[Kac]
    當且僅當 $v ∈ Δ _+$, 存在以 $v$ 爲維度向量的不可約表示. 
    \begin{enumerate}
        \item $v ∈ Δ _+$ $⟺$ $\mathrm{Rep}(v)$ 存在極大軌道 $⟺$ 存在 $𝐝𝐢𝐦 X = v$ 使得 $\mathrm{Ext}^1(X,X) = 0$. 
        \item 當且僅當 $v ∈ Δ _+^r$, 以上不可約表示唯一; 
        \item 當且僅當 $v ∈ Δ _+^i$, 以上不可約表示無窮. 
    \end{enumerate}
\end{theorem}

\begin{remark}
    Ringel Hall 代數等? 
\end{remark}
