\begin{abstract}
    原始的斜置模 $T$ 滿足
    \begin{enumerate}
        \item $T$ 的 $𝐦𝐨𝐝(A)$-投射維度 $≤ 1$; 
        \item $A$ 的 $𝐦𝐨𝐝(T)$-內射維度 $≤ 1$; 
        \item $\mathrm{Ext}^1(T,T)=0$. 
    \end{enumerate}
    有限維度的斜置模將以上 $1$ 換做了 $< ∞$. \parnote{有什麼用?}

    以上推廣由宮下洋一 (Yōichi Miyashita) 在 \cite{Miyashita1986} 中首次提及. 
    \begin{itemize}
    \item 宮下在寫完此篇引用 400+ 的文章後杳然無蹤, \href{https://www.mathgenealogy.org/index.php}{數學族譜網}找不到宮下與其老師 \cite{Nagahara1995} 的任何消息, 大抵是功成名就後淡出了學術舞台. 
    \end{itemize}
    本節先概括宮下的系列工作, 之後使用 B-B 的譜序列算法 (第四章, \cite{Angeleri_Hügel_Happel_Krause_2007}) 將結論串通一遍. 

    \begin{enumerate}
        \item 定義投射維度有限的斜置模; 
        \item 相對投射 (內射) 消解的對偶, 恰好是左模的相對投射 (內射) 消解; 
        \item 給出四個相同的``消解維度'', 以及相應的斜置函子; 
        \item (重點) 原始斜置理論的``四函子''很簡單, 復合求導公式的所有二階導數均為 $0$ (因爲 $p.\dim T ≤ 1$), 將投射維數提升後, 更好的精細來自譜序列. 
    \end{enumerate}
    重要問題: 傳統的斜置模蘊含扭對, 投射維數有限的斜置模如何變化? 
\end{abstract}

\subsubsection{投射維度有限的斜置模 (斜置模的推廣)}

\begin{definition}[(餘) 消解]\label{resol}
    稱上鏈複形 $X^∙$ 是對象 $A$ 的消解, 若 $X = X^{≤ 0}$, 且 $H^0(X) = A$. \parnote{投射分解的推廣} 餘消解類似. 
\end{definition}

\begin{definition}[投射維度有限的斜置模]
    稱 $T ∈ 𝐦𝐨𝐝 _A$ 是投射維度有限的斜置模, 當且僅當以下三點成立. 
    \begin{enumerate}
        \item $T$ 具有有限長度的 $𝐚𝐝𝐝 (A)$-消解; \parnote{$𝐩𝐫𝐨𝐣 (A)$}
        \item $A$ 具有有限長度的 $𝐚𝐝𝐝 (T)$-餘消解; 
        \item 對任意 $n ≥ 1$, 總有 $\mathrm{Ext}^n (T,T)=0$. 
    \end{enumerate}
\end{definition}

\begin{remark}
    若 $T$ 有二項 $𝐚𝐝𝐝 (A)$-消解, 且 $A$ 有二項 $𝐚𝐝𝐝 (T)$-餘消解, 則 $T$ 是通常的斜置模. 
\end{remark}

\begin{example}
    \text{bf}{往後使用斜置模簡稱``投射維度有限的斜置模''}. 仿照斜置模的一般研究方式, 我們希望 $T$ 的左 $\mathrm{End}(T)$ ($B$)-模也是斜置的. 更進一步地, 能否直接從 $𝐦𝐨𝐝 _A$ 的消解直接給出 $𝐦𝐨𝐝 _{B}$ 的消解? 
\end{example}

\begin{definition}[$T$-對偶]
    記 $h_T: 𝐦𝐨𝐝_A → {_B}𝐦𝐨𝐝$ 與 $h_T: {_B}𝐦𝐨𝐝 → 𝐦𝐨𝐝 _A$ 是 $T$-對偶函子. 必要時強調左右模: 
    \begin{enumerate}
        \item $((-)_A,{_B}T_A)_A : 𝐦𝐨𝐝 _A → 𝐦𝐨𝐝_{B^{\mathrm{op}}}$; 
        \item $({_B}(-), {_B}T_A)_{B^{\mathrm{op}}} : 𝐦𝐨𝐝_{B^{\mathrm{op}}} → 𝐦𝐨𝐝 _A$. 
    \end{enumerate}
\end{definition}

\begin{theorem}[$T$-對偶模消解定理]
    給定斜置模 $T$. 假定
    \begin{itemize}
        \item  $𝐦𝐨𝐝 _A$-範疇中, $P^∙$ 是 $T$ 的有限 $𝐚𝐝𝐝 (A)$-消解, $T^∙$ 是 $A$ 的有限 $𝐚𝐝𝐝 (T)$-餘消解. 
    \end{itemize}
    此時
    \begin{itemize}
        \item ${_B}𝐦𝐨𝐝$-範疇中 $h_T(T^∙)$ 是 $T$ 的有限 $𝐚𝐝𝐝 (B)$-消解, $h_T(P^∙)$ 是 $B$ 的有限 $𝐚𝐝𝐝 (T)$-余消解.  
    \end{itemize}
    同時, ${_B}T$ 是斜置左模. 也可以從 ${_B}T$ 推到 $T_A$: 只需發現 $h_T ∘ h_T$ 建立了 $𝐚𝐝𝐝 (A ⊕ T)$ 的同構. 
    \begin{proof}
        依次證明對象類的對應, 對偶消解成立, ${_B}T$ 是斜置左模, 以及二次對偶同構. 
        \begin{enumerate}
            \item 直接地, $h_T: 𝐚𝐝𝐝 (A) → 𝐚𝐝𝐝 (T)$, $h_T: 𝐚𝐝𝐝 (T) → 𝐚𝐝𝐝 (B)$, 以及 $h_T: 𝐚𝐝𝐝 $. \parnote{檢驗對象類}
            \item 先說明 $(T^∙,T) ⇒ T$ 是有限 $𝐚𝐝𝐝 (B)$ 消解. 對象已說明, 只需檢驗正合性. 
            \begin{equation}
                % https://q.uiver.app/#q=WzAsMTgsWzYsMiwiKFReMCwgVCkiXSxbNCwyLCIoVF4xLCBUKSJdLFsyLDIsIihUXjIsIFQpIl0sWzAsMiwiKFReMywgVCkiXSxbNywyLCJUIl0sWzYsMSwiVF4wICJdLFs3LDEsIkEiXSxbNCwxLCJUXjEiXSxbMiwxLCJUXjIiXSxbMCwxLCJUXjMiXSxbNSwwLCJcXE9tZWdhXjAgIl0sWzMsMCwiXFxPbWVnYSBeMSJdLFsxLDAsIlxcT21lZ2FeMiJdLFs1LDMsIihcXE9tZWdhXjAgLFQpIl0sWzMsMywiKFxcT21lZ2EgXjEsIFQpIl0sWzEsMywiKFxcT21lZ2EgXjIsIFQpIl0sWzgsMSwiXFxtYXRoYmZ7YWRkfShBKSJdLFs4LDIsIlxcbWF0aGJme2FkZH0oVCkiXSxbMywyXSxbMiwxXSxbMCw0LCIiLDAseyJsZXZlbCI6Miwic3R5bGUiOnsiaGVhZCI6eyJuYW1lIjoibm9uZSJ9fX1dLFs2LDUsIiIsMCx7ImxldmVsIjoyLCJzdHlsZSI6eyJoZWFkIjp7Im5hbWUiOiJub25lIn19fV0sWzUsN10sWzcsOF0sWzgsOV0sWzcsMTEsIiIsMCx7InN0eWxlIjp7ImhlYWQiOnsibmFtZSI6ImVwaSJ9fX1dLFs4LDEyLCIiLDAseyJzdHlsZSI6eyJoZWFkIjp7Im5hbWUiOiJlcGkifX19XSxbMTIsOSwiIiwxLHsic3R5bGUiOnsidGFpbCI6eyJuYW1lIjoiaG9vayIsInNpZGUiOiJib3R0b20ifX19XSxbMTEsOCwiIiwxLHsic3R5bGUiOnsidGFpbCI6eyJuYW1lIjoiaG9vayIsInNpZGUiOiJib3R0b20ifX19XSxbMTAsNywiIiwxLHsic3R5bGUiOnsidGFpbCI6eyJuYW1lIjoiaG9vayIsInNpZGUiOiJib3R0b20ifX19XSxbMSwxM10sWzMsMTVdLFsyLDE0XSxbMTUsMl0sWzE0LDFdLFsxNiwxNywiaF9UIiwyXSxbNSwxMCwiIiwwLHsibGV2ZWwiOjIsInN0eWxlIjp7ImhlYWQiOnsibmFtZSI6Im5vbmUifX19XSxbMSwwXSxbMTMsMCwiIiwwLHsibGV2ZWwiOjIsInN0eWxlIjp7ImhlYWQiOnsibmFtZSI6Im5vbmUifX19XV0=
\begin{tikzcd}[ampersand replacement=\&, sep = tiny]
	\& {\Omega^2} \&\& {\Omega ^1} \&\& {\Omega^0 } \\
	{T^3} \&\& {T^2} \&\& {T^1} \&\& {T^0 } \& A \& {\mathbf{add}(A)} \\
	{(T^3, T)} \&\& {(T^2, T)} \&\& {(T^1, T)} \&\& {(T^0, T)} \& T \& {\mathbf{add}(T)} \\
	\& {(\Omega ^2, T)} \&\& {(\Omega ^1, T)} \&\& {(\Omega^0 ,T)}
	\arrow[hook', from=1-2, to=2-1]
	\arrow[hook', from=1-4, to=2-3]
	\arrow[hook', from=1-6, to=2-5]
	\arrow[two heads, from=2-3, to=1-2]
	\arrow[from=2-3, to=2-1]
	\arrow[two heads, from=2-5, to=1-4]
	\arrow[from=2-5, to=2-3]
	\arrow[equals, from=2-7, to=1-6]
	\arrow[from=2-7, to=2-5]
	\arrow[equals, from=2-8, to=2-7]
	\arrow["{h_T}"', from=2-9, to=3-9]
	\arrow[from=3-1, to=3-3]
	\arrow[from=3-1, to=4-2]
	\arrow[from=3-3, to=3-5]
	\arrow[from=3-3, to=4-4]
	\arrow[from=3-5, to=3-7]
	\arrow[from=3-5, to=4-6]
	\arrow[equals, from=3-7, to=3-8]
	\arrow[from=4-2, to=3-3]
	\arrow[from=4-4, to=3-5]
	\arrow[equals, from=4-6, to=3-7]
\end{tikzcd}.
            \end{equation}
            爲了由上正合列推得下者, 只需證明 $[0 → (Ω^{k+1}, T) → (T^{k+1}, T) → (Ω ^k, T) → 0]$ 是正合列, 也就是 $\mathrm{Ext}^1(Ω^{k+1}, T) = 0$. 依照 $\mathrm{Ext}^{≥ 1} (T,T) = 0$, 必然有 
            \begin{equation}
                \mathrm{Ext}^1(Ω^{k+1}, T) = \mathrm{Ext}^2(Ω^{k+2}, T) = \cdots = 0. 
            \end{equation}
            \begin{pinked}
                對有限長度正合列 $C^∙$, 若 $\mathrm{Ext}^{≥ 1} (C^∙ , M) =0$, 則 $\mathrm{Hom}(C^ ∙ , M)$ 正合. 
            \end{pinked}\parnote{臨時關鍵引理}
            \item 再說明 $B ⇒ (P^∙ , T)$ 是有限 $𝐚𝐝𝐝 (T)$-餘消解. 顯然 $\mathrm{Ext}^{≥ 1} (A, T)=0$ 恆成立. 由臨時關鍵引理, $(P^∙ , T)$ 正合. 
            \item 先說明 ${_B}T$ 的 $(≥ 1)$-自垂直性. 由上, $h_T(T^∙)$ 是 ${_B}T$ 的有限投射分解. 相應地, 導出群 $\mathrm{Ext}^{≥ 1}({_B}T, {_B}T)$ 由複形 $({_B}(h_T(T^∙)), {_B}T)$ 決定. 由於 $T^{≥ 1}$ 以 $𝐚𝐝𝐝 (T)$ 爲分量, 故 
            \begin{equation}
                ({_B}(h_T(T^∙)), {_B}T) ≃ [\underbracket{\cdots → T^2 → T^1 }\limits_{\text{正合}}→ B = \mathrm{End}_A(T) → 0].
            \end{equation}
            從而 $\mathrm{Ext}^{≥ 1} ({_BT}, {_BT}) = 0 $. 
            \item 二次對偶建立了 $A ≃ \mathrm{End}_A(T) = B$, 因爲 $T^∙$ 是兩者共同的有限消解. 此時 $h_T ∘ h_T$ 是限制在 $𝐚𝐝𝐝 (X ⊕ T)$ 上的同構. \parnote{$A ≃ B$}
        \end{enumerate}
    \end{proof}
\end{theorem}

\begin{remark}
    若以四要件 $(A, T, P^∙ , T^∙)$ 描述斜置模, 則 $(B^{\mathrm{op}}, T, h_T(T^∙), h_T(P^∙))$ 也是斜置模. 
\end{remark}

\begin{proposition}
    來自章節 2.2,\cite{Happel_1988}. $T_A$ 與 ${_B}T$ 投射維度相同. 
    \begin{proof}
        思路是說明, 所有極小 (餘) 消解 $ℓ (P^∙) ≥ ℓ (T^∙) = ℓ (h_T(T^∙)) ≥ ℓ (h_T(P^∙)) = ℓ (P^∙)$, 從而不等號取等. 故而只需證明如下問題:
        \begin{itemize}
            \item 若有足夠投射對象的 Abel 範疇存在 $p.\dim T =: d < ∞$, 且 $\mathrm{Ext}^{≥ 1}(T,T)=0$, 若另有 $X$ 存在有限 $𝐚𝐝𝐝 (T)$-餘消解, 則 $X$ 的餘消解維度 $≤ d$. 
        \end{itemize}
        取 $X$ 的 $𝐚𝐝𝐝 (T)$-餘消解, 由 $\mathrm{Ext}^{≥ 1}(T,T)=0$ 得維數位移 $\mathrm{Ext}^{i+1}(T, Ω^k) = \mathrm{Ext}^{i}(T, Ω^{k+1})$. 
        \begin{equation}
            % https://q.uiver.app/#q=WzAsOSxbMCwxLCIwIl0sWzEsMSwiTSJdLFszLDEsIlReMSJdLFs1LDEsIlReMiJdLFs3LDEsIlReMyJdLFsyLDAsIlxcT21lZ2EgXjAiXSxbNCwwLCJcXE9tZWdhIF4xIl0sWzYsMCwiXFxPbWVnYSBeMiJdLFs4LDEsIlxcY2RvdHMgIl0sWzAsMV0sWzEsMl0sWzIsM10sWzMsNF0sWzEsNV0sWzUsMl0sWzIsNl0sWzYsM10sWzMsN10sWzcsNF0sWzQsOF1d
\begin{tikzcd}[ampersand replacement=\&,sep=small]
	\&\& {\Omega ^0} \&\& {\Omega ^1} \&\& {\Omega ^2} \\
	0 \& M \&\& {T^1} \&\& {T^2} \&\& {T^3} \& {\cdots }
	\arrow[from=1-3, to=2-4]
	\arrow[from=1-5, to=2-6]
	\arrow[from=1-7, to=2-8]
	\arrow[from=2-1, to=2-2]
	\arrow[from=2-2, to=1-3]
	\arrow[from=2-2, to=2-4]
	\arrow[from=2-4, to=1-5]
	\arrow[from=2-4, to=2-6]
	\arrow[from=2-6, to=1-7]
	\arrow[from=2-6, to=2-8]
	\arrow[from=2-8, to=2-9]
\end{tikzcd}.
        \end{equation}
        如果極小餘消解維度 $l > d$ (至多 $T^l ≠ 0$), 則 $\mathrm{Ext}^1(T, Ω^{l-1}) = \mathrm{Ext}^{l-1}(T,Ω^1)=0$. 此時 $Ω^{l-1}↪ T^l$ 可裂, 與極小餘消解矛盾. 
    \end{proof}
\end{proposition}

\begin{theorem}
    取以上四种斜置模的極小 (餘) 消解, 記作 $(A, T, P^∙ , T^∙)$ 與 $(B^{\mathrm{op}}, T, h_T(T^∙), h_T(P^∙))$. 此時, 四條鏈長度相同. \parnote{四鏈等長}
\end{theorem}

\subsubsection{譜序列的應用}

\begin{notation}
此節記號: $T$ 是斜置右 $A$-模, 从而也是斜置左 $B = \mathrm{End}_A(T)$-模. 定义函子
\begin{enumerate}
    \item (右正合, 左伴随, 左导出) $G(-) := - ⊗_B T : 𝐦𝐨𝐝 _B → 𝐦𝐨𝐝 _A$; \parnote{$L_{-n} G$}
    \item (左正合, 右伴随, 右导出) $F(-) := (T, -)_A : 𝐦𝐨𝐝 _A → 𝐦𝐨𝐝 _B$. \parnote{$R^n F$}
\end{enumerate}
特别地, 取上述四种消解, 则 $L_{< -n}G$ 与 $R^{> n}F$ 消失. 
\end{notation}

\begin{remark}
    为了统一双复形朝向, 记左导出 $L_p =: L_{-p}$.
\end{remark}

% \begin{theorem}[Grothendieck 谱序列定理]
%     由于 $F$ 将投射对象映至 $G$-导出消失对象 (回顾 $\mathrm{Ext}^{≥ 1}(T,T)=0$). 此时存在函子的谱序列使得对任意 $M ∈ 𝐦𝐨𝐝 _A$, 
%     \begin{equation}
%         E_2 = L_{-p}G ∘ R^qF(M) ⇒ R^{p+q}(G ∘ F) = H^{p+q}(M). 
%     \end{equation}
%     考虑支撑, 则 $E_{n} = E_∞$. 特别地, 
% \end{theorem}

\begin{theorem}[$LR$-型 Grothendieck 谱序列]
    存在函子的谱序列使得对任意 $M ∈ 𝐦𝐨𝐝 _A$, 
    \begin{equation}
        E_2 = L_{-p}G ∘ R^qF(M) ⇒ H^{p+q}(M). 
    \end{equation}
    \begin{proof}
        取 $M$ 的内射分解 $M → I$, 並將 $G(-) ≃ (-) ⊗_B T$ 的分解選作 $(-) ⊗ _B (T^∙, T)$. \parnote{$T^∙$ 是 $A$ 的 $T$-餘消解} 依照 
		\begin{equation}
			(T′, X)_A ⊗ _B (Y, T′)_A ≃ (X, Y)_A \quad T′ ∈ 𝐚𝐝𝐝 T. 
		\end{equation}
		此時有兩個同構的雙複形
\begin{equation}
	% https://q.uiver.app/#q=WzAsMjcsWzAsMCwiR0ZNIixbMzAsNjAsNjAsMV1dLFsxLDAsIkdGSV4wICIsWzE4MCw2MCw2MCwxXV0sWzIsMCwiR0ZJXjEiLFsxODAsNjAsNjAsMV1dLFszLDAsIkdGSV4yIixbMTgwLDYwLDYwLDFdXSxbMCwxLCJGTVxcb3RpbWVzIChUXjAsIFQpIixbMTgwLDYwLDYwLDFdXSxbMCwyLCJGTVxcb3RpbWVzIChUXjEsIFQpIixbMTgwLDYwLDYwLDFdXSxbMCwzLCJGTVxcb3RpbWVzIChUXjIsIFQpIixbMTgwLDYwLDYwLDFdXSxbMSwxLCJGSV4wXFxvdGltZXMgKFReMCwgVCkiXSxbMSwyLCJGSV4wXFxvdGltZXMgKFReMSwgVCkiXSxbMSwzLCJGSV4wXFxvdGltZXMgKFReMiwgVCkiXSxbMiwxLCJGSV4xXFxvdGltZXMgKFReMCwgVCkiXSxbMiwyLCJGSV4xXFxvdGltZXMgKFReMSwgVCkiXSxbMiwzLCJGSV4xXFxvdGltZXMgKFReMiwgVCkiXSxbMywxLCJGSV4yXFxvdGltZXMgKFReMCwgVCkiXSxbMywyLCJGSV4yXFxvdGltZXMgKFReMSwgVCkiXSxbMywzLCJGSV4yXFxvdGltZXMgKFReMiwgVCkiXSxbMSw0LCIoVF4wLEleMCkiXSxbMSw1LCIoVF4xLEleMCkiXSxbMiw0LCIoVF4wLEleMSkiXSxbMyw0LCIoVF4wLEleMikiXSxbMiw1LCIoVF4xLEleMSkiXSxbMyw1LCIoVF4xLEleMikiXSxbMSw2LCIoVF4yLEleMCkiXSxbMiw2LCIoVF4yLEleMSkiXSxbMyw2LCIoVF4yLEleMikiXSxbNCwzLCJcXGJveGVke0VfMH0iXSxbNCw2LCJcXGJveGVke0UnXzB9Il0sWzAsMSwiIiwyLHsiY29sb3VyIjpbMTgwLDYwLDYwXX1dLFsxLDIsIiIsMix7ImNvbG91ciI6WzE4MCw2MCw2MF19XSxbMiwzLCIiLDIseyJjb2xvdXIiOlsxODAsNjAsNjBdfV0sWzUsNCwiIiwyLHsiY29sb3VyIjpbMTgwLDYwLDYwXX1dLFs0LDAsIiIsMix7ImNvbG91ciI6WzE4MCw2MCw2MF19XSxbNywxLCIiLDIseyJzdHlsZSI6eyJib2R5Ijp7Im5hbWUiOiJkb3R0ZWQifX19XSxbOCw3XSxbOSw4XSxbMTAsMiwiIiwyLHsic3R5bGUiOnsiYm9keSI6eyJuYW1lIjoiZG90dGVkIn19fV0sWzExLDEwXSxbMTIsMTFdLFsxMywzLCIiLDAseyJzdHlsZSI6eyJib2R5Ijp7Im5hbWUiOiJkb3R0ZWQifX19XSxbMTQsMTNdLFsxNSwxNF0sWzQsNywiIiwxLHsic3R5bGUiOnsiYm9keSI6eyJuYW1lIjoiZG90dGVkIn19fV0sWzcsMTBdLFsxMCwxM10sWzUsOCwiIiwxLHsic3R5bGUiOnsiYm9keSI6eyJuYW1lIjoiZG90dGVkIn19fV0sWzgsMTFdLFsxMSwxNF0sWzYsOSwiIiwxLHsic3R5bGUiOnsiYm9keSI6eyJuYW1lIjoiZG90dGVkIn19fV0sWzksMTJdLFsxMiwxNV0sWzE2LDE4XSxbMTgsMTldLFsyMSwxOV0sWzI0LDIxXSxbMTcsMTZdLFsyMiwxN10sWzIyLDIzXSxbMjMsMjRdLFsxNywyMF0sWzIwLDIxXSxbMjAsMThdLFsyMywyMF0sWzI1LDI2LCJcXHNpbWVxICIsMCx7ImxldmVsIjoyfV0sWzYsNSwiIiwyLHsiY29sb3VyIjpbMTgwLDYwLDYwXX1dXQ==
\begin{tikzcd}[ampersand replacement=\&,sep=small]
	\textcolor{rgb,255:red,214;green,153;blue,92}{GFM} \& \textcolor{rgb,255:red,92;green,214;blue,214}{{GFI^0 }} \& \textcolor{rgb,255:red,92;green,214;blue,214}{{GFI^1}} \& \textcolor{rgb,255:red,92;green,214;blue,214}{{GFI^2}} \\
	\textcolor{rgb,255:red,92;green,214;blue,214}{{FM\otimes (T^0, T)}} \& {FI^0\otimes (T^0, T)} \& {FI^1\otimes (T^0, T)} \& {FI^2\otimes (T^0, T)} \\
	\textcolor{rgb,255:red,92;green,214;blue,214}{{FM\otimes (T^1, T)}} \& {FI^0\otimes (T^1, T)} \& {FI^1\otimes (T^1, T)} \& {FI^2\otimes (T^1, T)} \\
	\textcolor{rgb,255:red,92;green,214;blue,214}{{FM\otimes (T^2, T)}} \& {FI^0\otimes (T^2, T)} \& {FI^1\otimes (T^2, T)} \& {FI^2\otimes (T^2, T)} \& {\boxed{E_0}} \\
	\& {(T^0,I^0)} \& {(T^0,I^1)} \& {(T^0,I^2)} \\
	\& {(T^1,I^0)} \& {(T^1,I^1)} \& {(T^1,I^2)} \\
	\& {(T^2,I^0)} \& {(T^2,I^1)} \& {(T^2,I^2)} \& {\boxed{E'_0}}
	\arrow[color={rgb,255:red,92;green,214;blue,214}, from=1-1, to=1-2]
	\arrow[color={rgb,255:red,92;green,214;blue,214}, from=1-2, to=1-3]
	\arrow[color={rgb,255:red,92;green,214;blue,214}, from=1-3, to=1-4]
	\arrow[color={rgb,255:red,92;green,214;blue,214}, from=2-1, to=1-1]
	\arrow[dotted, from=2-1, to=2-2]
	\arrow[dotted, from=2-2, to=1-2]
	\arrow[from=2-2, to=2-3]
	\arrow[dotted, from=2-3, to=1-3]
	\arrow[from=2-3, to=2-4]
	\arrow[dotted, from=2-4, to=1-4]
	\arrow[color={rgb,255:red,92;green,214;blue,214}, from=3-1, to=2-1]
	\arrow[dotted, from=3-1, to=3-2]
	\arrow[from=3-2, to=2-2]
	\arrow[from=3-2, to=3-3]
	\arrow[from=3-3, to=2-3]
	\arrow[from=3-3, to=3-4]
	\arrow[from=3-4, to=2-4]
	\arrow[color={rgb,255:red,92;green,214;blue,214}, from=4-1, to=3-1]
	\arrow[dotted, from=4-1, to=4-2]
	\arrow[from=4-2, to=3-2]
	\arrow[from=4-2, to=4-3]
	\arrow[from=4-3, to=3-3]
	\arrow[from=4-3, to=4-4]
	\arrow[from=4-4, to=3-4]
	\arrow["{\simeq }", Rightarrow, from=4-5, to=7-5]
	\arrow[from=5-2, to=5-3]
	\arrow[from=5-3, to=5-4]
	\arrow[from=6-2, to=5-2]
	\arrow[from=6-2, to=6-3]
	\arrow[from=6-3, to=5-3]
	\arrow[from=6-3, to=6-4]
	\arrow[from=6-4, to=5-4]
	\arrow[from=7-2, to=6-2]
	\arrow[from=7-2, to=7-3]
	\arrow[from=7-3, to=6-3]
	\arrow[from=7-3, to=7-4]
	\arrow[from=7-4, to=6-4]
\end{tikzcd}.
\end{equation}
繼而計算雙向的譜序列. 
\begin{enumerate}
	\item 先保留 $↑$. 由 $(-, I^p)$ 是正合函子, 得譜序列
	\begin{equation}
		% https://q.uiver.app/#q=WzAsMjksWzAsMCwiR0ZNIixbMzAsNjAsNjAsMV1dLFsxLDAsIkdGSV4wICIsWzE4MCw2MCw2MCwxXV0sWzIsMCwiR0ZJXjEiLFsxODAsNjAsNjAsMV1dLFszLDAsIkdGSV4yIixbMTgwLDYwLDYwLDFdXSxbMCwxLCJGTVxcb3RpbWVzIChUXjAsIFQpIixbMTgwLDYwLDYwLDFdXSxbMCwyLCJGTVxcb3RpbWVzIChUXjEsIFQpIixbMTgwLDYwLDYwLDFdXSxbMCwzLCJGTVxcb3RpbWVzIChUXjIsIFQpIixbMTgwLDYwLDYwLDFdXSxbMSwxLCIoVF4wLEleMCkiXSxbMSwyLCIoVF4xLEleMCkiXSxbMiwxLCIoVF4wLEleMSkiXSxbMywxLCIoVF4wLEleMikiXSxbMiwyLCIoVF4xLEleMSkiXSxbMywyLCIoVF4xLEleMikiXSxbMSwzLCIoVF4yLEleMCkiXSxbMiwzLCIoVF4yLEleMSkiXSxbMywzLCIoVF4yLEleMikiXSxbNCwzLCJcXGJveGVke0VfMH0iXSxbMSw0LCIoSF4wKFQpLCBJXjAgKSJdLFsxLDUsIihIXjEoVCksIEleMCkiLFswLDAsNzUsMV1dLFsyLDQsIihIXjAoVCksIEleMSkiXSxbMyw0LCIoSF4wKFQpLCBJXjIpIl0sWzIsNSwiKEheMShUKSwgSV4xKSIsWzAsMCw3NSwxXV0sWzMsNSwiKEheMShUKSwgSV4yKSIsWzAsMCw3NSwxXV0sWzQsNSwiXFxib3hlZHtFXzF9Il0sWzAsNSwiXFx0ZXh0e+a2iOWksSF9IixbMCwwLDc1LDFdXSxbMSw2LCIoQSxNKSJdLFsyLDYsIjAiLFswLDAsNzUsMV1dLFszLDYsIjAiLFswLDAsNzUsMV1dLFs0LDYsIlxcYm94ZWR7RV8yfSJdLFswLDEsIiIsMix7ImNvbG91ciI6WzE4MCw2MCw2MF19XSxbMSwyLCIiLDIseyJjb2xvdXIiOlsxODAsNjAsNjBdfV0sWzIsMywiIiwyLHsiY29sb3VyIjpbMTgwLDYwLDYwXX1dLFs1LDQsIiIsMix7ImNvbG91ciI6WzE4MCw2MCw2MF19XSxbNCwwLCIiLDIseyJjb2xvdXIiOlsxODAsNjAsNjBdfV0sWzEyLDEwXSxbMTUsMTJdLFs4LDddLFsxMyw4XSxbMTEsOV0sWzYsNSwiIiwyLHsiY29sb3VyIjpbMTgwLDYwLDYwXX1dLFsxNCwxMV0sWzE3LDE5XSxbMTksMjBdLFsxOCwyMSwiIiwyLHsiY29sb3VyIjpbMCwwLDc1XX1dLFsyMSwyMiwiIiwyLHsiY29sb3VyIjpbMCwwLDc1XX1dXQ==
\begin{tikzcd}[ampersand replacement=\&,sep=small]
	\textcolor{rgb,255:red,214;green,153;blue,92}{GFM} \& \textcolor{rgb,255:red,92;green,214;blue,214}{{GFI^0 }} \& \textcolor{rgb,255:red,92;green,214;blue,214}{{GFI^1}} \& \textcolor{rgb,255:red,92;green,214;blue,214}{{GFI^2}} \\
	\textcolor{rgb,255:red,92;green,214;blue,214}{{FM\otimes (T^0, T)}} \& {(T^0,I^0)} \& {(T^0,I^1)} \& {(T^0,I^2)} \\
	\textcolor{rgb,255:red,92;green,214;blue,214}{{FM\otimes (T^1, T)}} \& {(T^1,I^0)} \& {(T^1,I^1)} \& {(T^1,I^2)} \\
	\textcolor{rgb,255:red,92;green,214;blue,214}{{FM\otimes (T^2, T)}} \& {(T^2,I^0)} \& {(T^2,I^1)} \& {(T^2,I^2)} \& {\boxed{E_0}} \\
	\& {(H^0(T), I^0 )} \& {(H^0(T), I^1)} \& {(H^0(T), I^2)} \\
	\textcolor{rgb,255:red,191;green,191;blue,191}{{\text{消失!}}} \& \textcolor{rgb,255:red,191;green,191;blue,191}{{(H^1(T), I^0)}} \& \textcolor{rgb,255:red,191;green,191;blue,191}{{(H^1(T), I^1)}} \& \textcolor{rgb,255:red,191;green,191;blue,191}{{(H^1(T), I^2)}} \& {\boxed{E_1}} \\
	\& {(A,M)} \& \textcolor{rgb,255:red,191;green,191;blue,191}{0} \& \textcolor{rgb,255:red,191;green,191;blue,191}{0} \& {\boxed{E_2}}
	\arrow[color={rgb,255:red,92;green,214;blue,214}, from=1-1, to=1-2]
	\arrow[color={rgb,255:red,92;green,214;blue,214}, from=1-2, to=1-3]
	\arrow[color={rgb,255:red,92;green,214;blue,214}, from=1-3, to=1-4]
	\arrow[color={rgb,255:red,92;green,214;blue,214}, from=2-1, to=1-1]
	\arrow[color={rgb,255:red,92;green,214;blue,214}, from=3-1, to=2-1]
	\arrow[from=3-2, to=2-2]
	\arrow[from=3-3, to=2-3]
	\arrow[from=3-4, to=2-4]
	\arrow[color={rgb,255:red,92;green,214;blue,214}, from=4-1, to=3-1]
	\arrow[from=4-2, to=3-2]
	\arrow[from=4-3, to=3-3]
	\arrow[from=4-4, to=3-4]
	\arrow[from=5-2, to=5-3]
	\arrow[from=5-3, to=5-4]
	\arrow[color={rgb,255:red,191;green,191;blue,191}, from=6-2, to=6-3]
	\arrow[color={rgb,255:red,191;green,191;blue,191}, from=6-3, to=6-4]
\end{tikzcd}.
	\end{equation}
	從 $E_0$ 至 $E_1$: 內射模給出的 $(-,I)$ 是正合函子, 從而與同調群交換. 而 $H^∙ (T) = A$, $H^∙ (I)=M$, 故 $\boxed{E_2}$ 只留下 $M$ 一項. 從而全復形的濾過上同調是 $H^0 = M$ 與 $H^≠ 0 = 0$. 
	\item 繼而計算 $→$. 由 $(T^p, T)$ 是投射 $B$-模, 得 $⊗ (T^p, T)$ 是正合函子. 此時有 
	\begin{equation}
		% https://q.uiver.app/#q=WzAsNDAsWzAsMCwiR0ZNIixbMzAsNjAsNjAsMV1dLFsxLDAsIkdGSV4wICIsWzE4MCw2MCw2MCwxXV0sWzIsMCwiR0ZJXjEiLFsxODAsNjAsNjAsMV1dLFszLDAsIkdGSV4yIixbMTgwLDYwLDYwLDFdXSxbMCwxLCJGTVxcb3RpbWVzIChUXjAsIFQpIixbMTgwLDYwLDYwLDFdXSxbMCwyLCJGTVxcb3RpbWVzIChUXjEsIFQpIixbMTgwLDYwLDYwLDFdXSxbMCwzLCJGTVxcb3RpbWVzIChUXjIsIFQpIixbMTgwLDYwLDYwLDFdXSxbMSwxLCJGSV4wXFxvdGltZXMgKFReMCwgVCkiXSxbMSwyLCJGSV4wXFxvdGltZXMgKFReMSwgVCkiXSxbMSwzLCJGSV4wXFxvdGltZXMgKFReMiwgVCkiXSxbMiwxLCJGSV4xXFxvdGltZXMgKFReMCwgVCkiXSxbMiwyLCJGSV4xXFxvdGltZXMgKFReMSwgVCkiXSxbMiwzLCJGSV4xXFxvdGltZXMgKFReMiwgVCkiXSxbMywxLCJGSV4yXFxvdGltZXMgKFReMCwgVCkiXSxbMywyLCJGSV4yXFxvdGltZXMgKFReMSwgVCkiXSxbMywzLCJGSV4yXFxvdGltZXMgKFReMiwgVCkiXSxbNCwzLCJcXGJveGVke0VfMH0iXSxbMiw1LCIoUl4xIEZNKSBcXG90aW1lcyAoVF4wLCBUKSJdLFsyLDYsIihSXjEgRk0pIFxcb3RpbWVzIChUXjEsIFQpIl0sWzIsNywiKFJeMSBGTSkgXFxvdGltZXMgKFReMiwgVCkiXSxbMSw1LCJGTVxcb3RpbWVzIChUXjAsIFQpIl0sWzEsNiwiRk1cXG90aW1lcyAoVF4xLCBUKSJdLFsxLDcsIkZNXFxvdGltZXMgKFReMiwgVCkiXSxbMyw1LCIoUl4yIEZNKSBcXG90aW1lcyAoVF4wLCBUKSJdLFszLDYsIihSXjIgRk0pIFxcb3RpbWVzIChUXjEsIFQpIl0sWzMsNywiKFJeMiBGTSkgXFxvdGltZXMgKFReMiwgVCkiXSxbNCw3LCJcXGJveGVke0VfMX0iXSxbMSw4LCJHRk0iXSxbMSw0LCJGTSBcXG90aW1lcyBUIixbMTgwLDYwLDYwLDFdXSxbMiw0LCJSXjFGTSBcXG90aW1lcyBUIixbMTgwLDYwLDYwLDFdXSxbMyw0LCJSXjJGTSBcXG90aW1lcyBUIixbMTgwLDYwLDYwLDFdXSxbMiw4LCJHKFJeMUYpTSJdLFszLDgsIkcoUl4yRilNIl0sWzIsOSwiKExfey0xfUcpKFJeMUYpTSJdLFsxLDksIihMX3stMX1HKUZNIl0sWzMsOSwiKExfey0xfSlHKFJeMkYpTSJdLFsyLDEwLCIoTF97LTJ9RykoUl4xRilNIl0sWzEsMTAsIihMX3stMn1HKUZNIl0sWzMsMTAsIihMX3stMn1HKShSXjJGKU0iXSxbNCwxMCwiXFxib3hlZHtFXzJ9Il0sWzAsMSwiIiwyLHsiY29sb3VyIjpbMTgwLDYwLDYwXX1dLFsxLDIsIiIsMix7ImNvbG91ciI6WzE4MCw2MCw2MF19XSxbMiwzLCIiLDIseyJjb2xvdXIiOlsxODAsNjAsNjBdfV0sWzQsNywiIiwxLHsic3R5bGUiOnsiYm9keSI6eyJuYW1lIjoiZG90dGVkIn19fV0sWzcsMTBdLFsxMCwxM10sWzUsOCwiIiwxLHsic3R5bGUiOnsiYm9keSI6eyJuYW1lIjoiZG90dGVkIn19fV0sWzgsMTFdLFsxMSwxNF0sWzYsOSwiIiwxLHsic3R5bGUiOnsiYm9keSI6eyJuYW1lIjoiZG90dGVkIn19fV0sWzksMTJdLFsxMiwxNV0sWzI1LDI0XSxbMjQsMjNdLFsxOSwxOF0sWzE4LDE3XSxbMjIsMjFdLFsyMSwyMF0sWzM2LDI3XSxbMzgsMzFdXQ==
\begin{tikzcd}[ampersand replacement=\&, sep = small]
	\textcolor{rgb,255:red,214;green,153;blue,92}{GFM} \& \textcolor{rgb,255:red,92;green,214;blue,214}{{GFI^0 }} \& \textcolor{rgb,255:red,92;green,214;blue,214}{{GFI^1}} \& \textcolor{rgb,255:red,92;green,214;blue,214}{{GFI^2}} \\
	\textcolor{rgb,255:red,92;green,214;blue,214}{{FM\otimes (T^0, T)}} \& {FI^0\otimes (T^0, T)} \& {FI^1\otimes (T^0, T)} \& {FI^2\otimes (T^0, T)} \\
	\textcolor{rgb,255:red,92;green,214;blue,214}{{FM\otimes (T^1, T)}} \& {FI^0\otimes (T^1, T)} \& {FI^1\otimes (T^1, T)} \& {FI^2\otimes (T^1, T)} \\
	\textcolor{rgb,255:red,92;green,214;blue,214}{{FM\otimes (T^2, T)}} \& {FI^0\otimes (T^2, T)} \& {FI^1\otimes (T^2, T)} \& {FI^2\otimes (T^2, T)} \& {\boxed{E_0}} \\
	\& \textcolor{rgb,255:red,92;green,214;blue,214}{{FM \otimes T}} \& \textcolor{rgb,255:red,92;green,214;blue,214}{{R^1FM \otimes T}} \& \textcolor{rgb,255:red,92;green,214;blue,214}{{R^2FM \otimes T}} \\
	\& {FM\otimes (T^0, T)} \& {(R^1 FM) \otimes (T^0, T)} \& {(R^2 FM) \otimes (T^0, T)} \\
	\& {FM\otimes (T^1, T)} \& {(R^1 FM) \otimes (T^1, T)} \& {(R^2 FM) \otimes (T^1, T)} \\
	\& {FM\otimes (T^2, T)} \& {(R^1 FM) \otimes (T^2, T)} \& {(R^2 FM) \otimes (T^2, T)} \& {\boxed{E_1}} \\
	\& GFM \& {G(R^1F)M} \& {G(R^2F)M} \\
	\& {(L_{-1}G)FM} \& {(L_{-1}G)(R^1F)M} \& {(L_{-1})G(R^2F)M} \\
	\& {(L_{-2}G)FM} \& {(L_{-2}G)(R^1F)M} \& {(L_{-2}G)(R^2F)M} \& {\boxed{E_2}}
	\arrow[draw={rgb,255:red,92;green,214;blue,214}, from=1-1, to=1-2]
	\arrow[draw={rgb,255:red,92;green,214;blue,214}, from=1-2, to=1-3]
	\arrow[draw={rgb,255:red,92;green,214;blue,214}, from=1-3, to=1-4]
	\arrow[dotted, from=2-1, to=2-2]
	\arrow[from=2-2, to=2-3]
	\arrow[from=2-3, to=2-4]
	\arrow[dotted, from=3-1, to=3-2]
	\arrow[from=3-2, to=3-3]
	\arrow[from=3-3, to=3-4]
	\arrow[dotted, from=4-1, to=4-2]
	\arrow[from=4-2, to=4-3]
	\arrow[from=4-3, to=4-4]
	\arrow[from=7-2, to=6-2]
	\arrow[from=7-3, to=6-3]
	\arrow[from=7-4, to=6-4]
	\arrow[from=8-2, to=7-2]
	\arrow[from=8-3, to=7-3]
	\arrow[from=8-4, to=7-4]
	\arrow[from=11-3, to=9-2]
	\arrow[from=11-4, to=9-3]
\end{tikzcd}.
	\end{equation}
\end{enumerate}
以上證明了存在譜序列 $\mathrm{Tor}^B_{-q}(T, \mathrm{Ext}^p_A(T, -)) ⇒ δ_{p+q, 0}⋅ \mathrm{id}$. 
    \end{proof}
\end{theorem}

\begin{remark}
	一些粗淺的視角: 右上角處 
\begin{equation}
	% https://q.uiver.app/#q=WzAsOSxbMCwxLCJHKFJee24tMX1GKU0iLFszMCw2MCw2MCwxXV0sWzEsMSwiRyhSXm5GKU0iLFsxODAsNjAsNjAsMV1dLFswLDIsIihMX3stMX1HKShSXntuLTF9RilNIixbMCw2MCw2MCwxXV0sWzEsMiwiKExfey0xfSlHKFJebkYpTSIsWzE4MCw2MCw2MCwxXV0sWzAsMywiKExfey0yfUcpKFJee24tMX1GKU0iXSxbMSwzLCIoTF97LTJ9RykoUl5uRilNIixbMzAsNjAsNjAsMV1dLFsxLDQsIihMX3stM31HKShSXm5GKU0iLFswLDYwLDYwLDFdXSxbMCw0LCIoTF97LTN9RykoUl57bi0xfUYpTSJdLFswLDAsIjAiXSxbNSwwLCJcXHNpbWVxIiwxLHsiY29sb3VyIjpbMzAsNjAsNjBdfSxbMzAsNjAsNjAsMV1dLFs2LDIsIlxcdmFyZXBzaWxvbiAiLDEseyJjb2xvdXIiOlswLDYwLDYwXX0sWzAsNjAsNjAsMV1dLFswLDMsIjAiLDEseyJzdHlsZSI6eyJib2R5Ijp7Im5hbWUiOiJkb3R0ZWQifX19XSxbMiw1LCIwIiwxLHsic3R5bGUiOnsiYm9keSI6eyJuYW1lIjoiZG90dGVkIn19fV0sWzMsOCwiIiwxLHsiY29sb3VyIjpbMTgwLDYwLDYwXX1dLFs4LDEsIjAiLDEseyJjb2xvdXIiOlsxODAsNjAsNjBdLCJzdHlsZSI6eyJib2R5Ijp7Im5hbWUiOiJkb3R0ZWQifX19LFsxODAsNjAsNjAsMV1dXQ==
\begin{tikzcd}[ampersand replacement=\&]
	0 \\
	\textcolor{rgb,255:red,214;green,153;blue,92}{{G(R^{n-1}F)M}} \& \textcolor{rgb,255:red,92;green,214;blue,214}{{G(R^nF)M}} \\
	\textcolor{rgb,255:red,214;green,92;blue,92}{{(L_{-1}G)(R^{n-1}F)M}} \& \textcolor{rgb,255:red,92;green,214;blue,214}{{(L_{-1})G(R^nF)M}} \\
	{(L_{-2}G)(R^{n-1}F)M} \& \textcolor{rgb,255:red,214;green,153;blue,92}{{(L_{-2}G)(R^nF)M}} \\
	{(L_{-3}G)(R^{n-1}F)M} \& \textcolor{rgb,255:red,214;green,92;blue,92}{{(L_{-3}G)(R^nF)M}}
	\arrow["0"{description}, color={rgb,255:red,92;green,214;blue,214}, dotted, from=1-1, to=2-2]
	\arrow["0"{description}, dotted, from=2-1, to=3-2]
	\arrow["0"{description}, dotted, from=3-1, to=4-2]
	\arrow[color={rgb,255:red,92;green,214;blue,214}, from=3-2, to=1-1]
	\arrow["\simeq"{description}, color={rgb,255:red,214;green,153;blue,92}, from=4-2, to=2-1]
	\arrow["{\varepsilon }"{description}, color={rgb,255:red,214;green,92;blue,92}, from=5-2, to=3-1]
\end{tikzcd}.
\end{equation}
	得到以下兩則結果 (假定 $n > 3$): 
	\begin{enumerate}
		\item $\mathrm{Ext}_A^n(T, M) ⊗_B T = 0 = \mathrm{Tor}_1^B(\mathrm{Ext}_A^n(T, M), T)$; 
		\item $\mathrm{Tor}_2^B(\mathrm{Ext}_A^n(T, M), T) ≃ \mathrm{Ext}^{n-1}(T, M) ⊗ T$; 
		\item $(L_{-3}G)(R^nF)M \overset ε ↠ (L_{-1}G)(R^{n-1}F)M → H^{n-2} = 0$ 給出滿射 $ε$.
	\end{enumerate}
\end{remark}

\begin{proposition}[$RL$-型 Grothendieck 譜序列]
	對 $𝐦𝐨𝐝_B → 𝐦𝐨𝐝_A → 𝐦𝐨𝐝_B$ 型的函子, 有收斂 
	\begin{equation}
		\mathrm{Ext}_A^p(T, \mathrm{Tor}^B_{-q}(-, T)) ⇒ δ_{p+q, 0} ⋅ \mathrm{id}. 
	\end{equation}
	\begin{proof}
		證明略. 對 $X ⊗_B (P,T)_A ≃ (P, X ⊗_B T)_A$ 計算譜序列即可. 
	\end{proof}
\end{proposition}

\begin{example}[特例: $n=1$, 通常的斜置理論]
	混合係數公式來自譜序列: 
	\begin{equation}
		% https://q.uiver.app/#q=WzAsMTYsWzAsMiwiR0ZNIl0sWzEsMiwiRyhSXjFGKU0iXSxbMSwzLCIoTF97LTF9RykoUl4xRilNIl0sWzAsMywiKExfey0xfUcpRk0iXSxbMCwxLCIwIl0sWzEsNCwiMCJdLFswLDAsIjAiXSxbMSw1LCIwIl0sWzMsMiwiRkdOIl0sWzMsMywiKFJeMUYpR04iXSxbNCwzLCIoUl4xRikoTF97LTF9RylOIl0sWzQsMiwiRihMX3stMX1HKU4iXSxbNCw0LCIwIl0sWzMsMSwiMCJdLFszLDAsIjAiXSxbNCw1LCIwIl0sWzIsNF0sWzUsMF0sWzEsNl0sWzcsM10sWzQsMSwiMCIsMSx7InN0eWxlIjp7ImJvZHkiOnsibmFtZSI6ImRvdHRlZCJ9fX1dLFswLDIsIk0iLDEseyJzdHlsZSI6eyJib2R5Ijp7Im5hbWUiOiJkb3R0ZWQifX19XSxbMyw1LCIwIiwxLHsic3R5bGUiOnsiYm9keSI6eyJuYW1lIjoiZG90dGVkIn19fV0sWzEwLDgsIk4iLDEseyJzdHlsZSI6eyJib2R5Ijp7Im5hbWUiOiJkb3R0ZWQifX19XSxbOCwxMl0sWzEzLDEwXSxbOSwxNV0sWzE0LDExXSxbMTEsMTMsIiIsMSx7InN0eWxlIjp7ImJvZHkiOnsibmFtZSI6ImRvdHRlZCJ9fX1dLFsxMiw5LCIiLDEseyJzdHlsZSI6eyJib2R5Ijp7Im5hbWUiOiJkb3R0ZWQifX19XV0=
\begin{tikzcd}[ampersand replacement=\&, sep = small]
	0 \&\&\& 0 \\
	0 \&\&\& 0 \\
	GFM \& {G(R^1F)M} \&\& FGN \& {F(L_{-1}G)N} \\
	{(L_{-1}G)FM} \& {(L_{-1}G)(R^1F)M} \&\& {(R^1F)GN} \& {(R^1F)(L_{-1}G)N} \\
	\& 0 \&\&\& 0 \\
	\& 0 \&\&\& 0
	\arrow[from=1-4, to=3-5]
	\arrow["0"{description}, dotted, from=2-1, to=3-2]
	\arrow[from=2-4, to=4-5]
	\arrow["M"{description}, dotted, from=3-1, to=4-2]
	\arrow[from=3-2, to=1-1]
	\arrow[from=3-4, to=5-5]
	\arrow[dotted, from=3-5, to=2-4]
	\arrow["0"{description}, dotted, from=4-1, to=5-2]
	\arrow[from=4-2, to=2-1]
	\arrow[from=4-4, to=6-5]
	\arrow["N"{description}, dotted, from=4-5, to=3-4]
	\arrow[from=5-2, to=3-1]
	\arrow[dotted, from=5-5, to=4-4]
	\arrow[from=6-2, to=4-1]
\end{tikzcd}.
	\end{equation}
\end{example}

\begin{example}[特例: $n=2$]
	此時 $E_3 = E_∞$. 特別地, 計算譜序列 
\begin{equation}
	% https://q.uiver.app/#q=WzAsMzEsWzEsMCwiR0ZNIl0sWzIsMCwiRyhSXjFGKU0iXSxbMiwxLCJcXGJveGVkeyhMX3stMX1HKShSXjFGKU19Il0sWzEsMSwiXFxib3hlZHsoTF97LTF9RylGTX0iXSxbMywyLCIoTF97LTJ9RykoUl4yRilNIl0sWzMsMSwiXFxib3hlZHsoTF97LTF9KUcoUl4yRilNfSJdLFszLDAsIlxcYm94ZWR7RyhSXjJGKU19Il0sWzIsMiwiKExfey0yfUcpKFJeMUYpTSJdLFsxLDIsIlxcYm94ZWR7KExfey0yfUcpRk19Il0sWzAsMiwiXFxib3hlZHtFXzJ9Il0sWzAsNiwiXFxib3hlZHtFX1xcaW5mdHl9Il0sWzEsNSwiXFxib3hlZHsoTF97LTJ9RylGTX0iLFswLDAsNzUsMV1dLFsxLDQsIlxcYm94ZWR7KExfey0xfUcpRk19IixbMCwwLDc1LDFdXSxbMiw0LCJcXGJveGVkeyhMX3stMX1HKShSXjFGKU19Il0sWzMsNCwiXFxib3hlZHsoTF97LTF9KUcoUl4yRilNfSIsWzAsMCw3NSwxXV0sWzMsMywiXFxib3hlZHtHKFJeMkYpTX0iLFswLDAsNzUsMV1dLFsyLDMsIlxcbWF0aHJte2Nva31cXCBhIixbMCwwLDc1LDFdXSxbMSwzLCJcXG1hdGhybXtjb2t9XFwgYiJdLFszLDUsIlxca2VyIGEiXSxbMiw1LCJcXGtlciBiIixbMCwwLDc1LDFdXSxbMSw2LCJcXG1hdGhybXtjb2t9XFwgYiJdLFsyLDYsIj8iXSxbMyw2LCJcXGJveGVkeyhMX3stMX1HKShSXjFGKU19Il0sWzIsOCwiXFxrZXIgYSJdLFsyLDcsIk0iXSxbMyw4LCIoTF97LTJ9RykoUl4yRilNIl0sWzQsOCwiRyhSXjFGKU0iXSxbMSw3LCIoTF97LTJ9RykoUl4xRilNIl0sWzEsOCwiR0ZNIl0sWzAsNywiRV9cXGluZnR5XnswfSJdLFswLDUsIlxcYm94ZWR7RV8zfSJdLFs0LDEsImEiLDEseyJzdHlsZSI6eyJoZWFkIjp7Im5hbWUiOiJlcGkifX19XSxbNywwLCJiIiwxLHsic3R5bGUiOnsidGFpbCI6eyJuYW1lIjoiaG9vayIsInNpZGUiOiJ0b3AifX19XSxbMjAsMjEsIiIsMSx7InN0eWxlIjp7InRhaWwiOnsibmFtZSI6Imhvb2siLCJzaWRlIjoidG9wIn19fV0sWzIxLDIyLCIiLDEseyJzdHlsZSI6eyJoZWFkIjp7Im5hbWUiOiJlcGkifX19XSxbMjEsMjQsIiIsMSx7InN0eWxlIjp7InRhaWwiOnsibmFtZSI6Imhvb2siLCJzaWRlIjoiYm90dG9tIn19fV0sWzIzLDI1LCIiLDEseyJzdHlsZSI6eyJ0YWlsIjp7Im5hbWUiOiJob29rIiwic2lkZSI6InRvcCJ9fX1dLFsyNSwyNiwiIiwxLHsic3R5bGUiOnsiaGVhZCI6eyJuYW1lIjoiZXBpIn19fV0sWzI4LDI3LCIiLDEseyJzdHlsZSI6eyJ0YWlsIjp7Im5hbWUiOiJob29rIiwic2lkZSI6ImJvdHRvbSJ9fX1dLFsyNCwyMywiIiwxLHsic3R5bGUiOnsiaGVhZCI6eyJuYW1lIjoiZXBpIn19fV0sWzI3LDIwLCIiLDEseyJzdHlsZSI6eyJoZWFkIjp7Im5hbWUiOiJlcGkifX19XSxbMzAsMTAsIiIsMSx7ImxldmVsIjoyLCJzdHlsZSI6eyJoZWFkIjp7Im5hbWUiOiJub25lIn19fV1d
\begin{tikzcd}[ampersand replacement=\&, sep = small]
	\& GFM \& {G(R^1F)M} \& {\boxed{G(R^2F)M}} \\
	\& {\boxed{(L_{-1}G)FM}} \& {\boxed{(L_{-1}G)(R^1F)M}} \& {\boxed{(L_{-1})G(R^2F)M}} \\
	{\boxed{E_2}} \& {\boxed{(L_{-2}G)FM}} \& {(L_{-2}G)(R^1F)M} \& {(L_{-2}G)(R^2F)M} \\
	\& {\mathrm{cok}\ b} \& \textcolor{rgb,255:red,191;green,191;blue,191}{{\mathrm{cok}\ a}} \& \textcolor{rgb,255:red,191;green,191;blue,191}{{\boxed{G(R^2F)M}}} \\
	\& \textcolor{rgb,255:red,191;green,191;blue,191}{{\boxed{(L_{-1}G)FM}}} \& {\boxed{(L_{-1}G)(R^1F)M}} \& \textcolor{rgb,255:red,191;green,191;blue,191}{{\boxed{(L_{-1})G(R^2F)M}}} \\
	{\boxed{E_3}} \& \textcolor{rgb,255:red,191;green,191;blue,191}{{\boxed{(L_{-2}G)FM}}} \& \textcolor{rgb,255:red,191;green,191;blue,191}{{\ker b}} \& {\ker a} \\
	{\boxed{E_\infty}} \& {\mathrm{cok}\ b} \& {?} \& {\boxed{(L_{-1}G)(R^1F)M}} \\
	{E_\infty^{0}} \& {(L_{-2}G)(R^1F)M} \& M \\
	\& GFM \& {\ker a} \& {(L_{-2}G)(R^2F)M} \& {G(R^1F)M}
	\arrow["b"{description}, hook, from=3-3, to=1-2]
	\arrow["a"{description}, two heads, from=3-4, to=1-3]
	\arrow[equals, from=6-1, to=7-1]
	\arrow[hook, from=7-2, to=7-3]
	\arrow[two heads, from=7-3, to=7-4]
	\arrow[hook', from=7-3, to=8-3]
	\arrow[two heads, from=8-2, to=7-2]
	\arrow[two heads, from=8-3, to=9-3]
	\arrow[hook', from=9-2, to=8-2]
	\arrow[hook, from=9-3, to=9-4]
	\arrow[two heads, from=9-4, to=9-5]
\end{tikzcd}.
\end{equation}
特別地, 以下三者等價: 
\begin{enumerate}
	\item $GF M → (L_{-2}G)(R^1F)M$ 是滿射; 
	\begin{itemize}
		\item $(T, M) ⊗ T → \mathrm{Tor}_2(\mathrm{Ext}^1(T, M), T)$ 
	\end{itemize}
	\item $GF M → (L_{-2}G)(R^1F)M$ 是同構; 
	\item $0 → (L_{-1}G)(R^1F) M → M → (L_{-2}G)(R^2F) → G(R^1F)M → 0$ 是四項正合列. 
\end{enumerate}
對偶命題略. \parnote{映射構造?}
\end{example}

\begin{definition}[導出垂直]
	原始版本的斜置理論中, $(ℱ,𝒯,𝒳,𝒴)$ 通過四個函子的 $\ker$ 定義. 今推廣 
	\begin{enumerate}
		\item $K^p(A) := ⋂_{0 ≤ k ≤ n}^{k ≠ p}\mathrm{Ext}_A^k(T, -)$; 
		\item $K_p(B) := ⋂_{0 ≤ k ≤ n}^{k ≠ p}\mathrm{Tor}^B_k(-, T)$. 
	\end{enumerate}
\end{definition}

\begin{theorem}
	存在全子範疇間的互逆函子 $R^tF : K^t (A) ≃ K_t (B) : L_{-t}G$. 
	\begin{proof}
		對 $M ∈ K^p(A)$, 由譜序列的濾過知 $E_2^{p,q} = (L_{-p}G)(R^qF)M$ 僅在 $t$-列非零, 這也蘊含 $E_2 = E_∞$. 此時
\begin{equation}
	(L_{-∙}G)(R^tF)(M) = [0 ∣ \cdots ∣ 0 ∣ M ∣ 0 ∣ \cdots ∣ 0 ]. 
\end{equation}
	這說明 $(R^tF)(M) ∈ K_p (B)$. 逆函子等顯然. 
	\end{proof}
\end{theorem}

\begin{example}[王憲鍾序列]
	王憲鍾序列是一類特殊的譜序列: $E_2$ 中僅有兩行非零. 這說明, 可以對某一 $E_2 = E_r$ 使用``小技巧'', 從而導出長正合列. 

	記 $K^{i,j}(A) = ⋂ _{0 ≤ k ≤ n}^{k ≠ i,j}\mathrm{Ext}^k_A(T, -)$, \parnote{$i < j$} 則 $E_2$ 中僅有兩縱列非零. 計算得 
	\begin{enumerate}
		\item 存在五項正合列 
		\begin{align}
			0 → (L_{-j+1}G)(R^j F)M → (L_{-i}G)(R^i F)M → M \qquad \\ 
			\qquad → (L_{-j}G)(R^j F)M → (L_{-i-1}G)(R^i F)M → 0
		\end{align}
		\item $(L_{p-j+1}G)(R^j F)M ≃ (L_{p-i}G)(R^i F)M$ 對 $p ≠ 0, -1$ 成立. 
		\item $(L_{-([0,n-(j-i)])}G)(R^iF)M$ 與 $(L_{-([(j-i),n])}G)(R^jF)M$ 或非零; 其餘 $(L_{-p}G)(R^qF)$ 必爲 $0$.  
	\end{enumerate}
\end{example}

\begin{remark}
	王憲鍾的生平參考 \cite{严志达}.
\end{remark}

\begin{example}[$(i,j) = (0,n)$]
	此時有函子圖 
	\begin{equation}
		% https://q.uiver.app/#q=WzAsNixbMCwxLCJLXnswLG59KEEpIl0sWzAsMCwiS18wKEIpIl0sWzAsMiwiS19uIChCKSJdLFszLDAsIkteMCAoQSkiXSxbMywyLCJLXm4oQSkiXSxbMywxLCJLX3swLG59KEIpIl0sWzAsMSwiKFQsLSkiXSxbMCwyLCJcXG1hdGhybXtFeHR9Xm4oVCwtKSIsMl0sWzEsMywiLSBcXG90aW1lcyBUIiwyLHsib2Zmc2V0IjozfV0sWzIsNCwiXFxtYXRocm17VG9yfV9uKC0sVCkiLDIseyJvZmZzZXQiOjN9XSxbNSwzLCItIFxcb3RpbWVzIFQiLDJdLFs1LDQsIlxcbWF0aHJte1Rvcn1fbigtLFQpIl0sWzMsMSwiKFQsLSkiLDIseyJvZmZzZXQiOjN9XSxbNCwyLCJcXG1hdGhybXtFeHR9Xm4oVCwtKSIsMix7Im9mZnNldCI6M31dLFswLDUsIj8/PyIsMSx7InN0eWxlIjp7InRhaWwiOnsibmFtZSI6ImFycm93aGVhZCJ9fX1dLFsxMiw4LCJcXGNvbmcgIiwxLHsic2hvcnRlbiI6eyJzb3VyY2UiOjIwLCJ0YXJnZXQiOjIwfSwic3R5bGUiOnsiYm9keSI6eyJuYW1lIjoibm9uZSJ9LCJoZWFkIjp7Im5hbWUiOiJub25lIn19fV0sWzEzLDksIlxcY29uZyAiLDEseyJzaG9ydGVuIjp7InNvdXJjZSI6MjAsInRhcmdldCI6MjB9LCJzdHlsZSI6eyJib2R5Ijp7Im5hbWUiOiJub25lIn0sImhlYWQiOnsibmFtZSI6Im5vbmUifX19XV0=
\begin{tikzcd}[ampersand replacement=\&]
	{K_0(B)} \&\&\& {K^0 (A)} \\
	{K^{0,n}(A)} \&\&\& {K_{0,n}(B)} \\
	{K_n (B)} \&\&\& {K^n(A)}
	\arrow[""{name=0, anchor=center, inner sep=0}, "{- \otimes T}"', shift right=3, from=1-1, to=1-4]
	\arrow[""{name=1, anchor=center, inner sep=0}, "{(T,-)}"', shift right=3, from=1-4, to=1-1]
	\arrow["{(T,-)}", from=2-1, to=1-1]
	\arrow["{???}"{description}, tail reversed, from=2-1, to=2-4]
	\arrow["{\mathrm{Ext}^n(T,-)}"', from=2-1, to=3-1]
	\arrow["{- \otimes T}"', from=2-4, to=1-4]
	\arrow["{\mathrm{Tor}_n(-,T)}", from=2-4, to=3-4]
	\arrow[""{name=2, anchor=center, inner sep=0}, "{\mathrm{Tor}_n(-,T)}"', shift right=3, from=3-1, to=3-4]
	\arrow[""{name=3, anchor=center, inner sep=0}, "{\mathrm{Ext}^n(T,-)}"', shift right=3, from=3-4, to=3-1]
	\arrow["{\cong }"{description}, draw=none, from=1, to=0]
	\arrow["{\cong }"{description}, draw=none, from=3, to=2]
\end{tikzcd}.
	\end{equation}
	\begin{pinked}
		$K^{0,n}(A)$ 與 $K_{0,n}(B)$ 有何聯繫? 
	\end{pinked}
\end{example}

\begin{example}[$j - i = Δ$]
	此時有函子圖
	\begin{equation}
		% https://q.uiver.app/#q=WzAsNyxbMywyLCJLXntpLGp9KEEpIl0sWzYsMiwiS197WzAsIG4gLSBcXERlbHRhIF19KEIpIl0sWzAsMiwiS197W1xcRGVsdGEgLG5dfShCKSJdLFszLDEsIktee2ktMSxqLTF9KEEpIl0sWzMsMywiS157aSsxLGorMX0oQSkiXSxbMywwLCJcXHZkb3RzICJdLFszLDQsIlxcdmRvdHMgIl0sWzAsMiwiXFxtYXRocm17RXh0fV5qKFQsLSkiLDFdLFswLDEsIlxcbWF0aHJte0V4dH1eaShULC0pIiwxXSxbMywyLCJcXG1hdGhybXtFeHR9XntqLTF9KFQsLSkiLDEseyJjdXJ2ZSI6NX1dLFszLDEsIlxcbWF0aHJte0V4dH1ee2ktMX0oVCwtKSIsMSx7ImN1cnZlIjotNX1dLFs0LDIsIlxcbWF0aHJte0V4dH1ee2orMX0oVCwtKSIsMSx7ImN1cnZlIjotNX1dLFs0LDEsIlxcbWF0aHJte0V4dH1ee2krMX0oVCwtKSIsMSx7ImN1cnZlIjo1fV0sWzMsMCwiPz8/IiwxLHsic3R5bGUiOnsidGFpbCI6eyJuYW1lIjoiYXJyb3doZWFkIn19fV0sWzAsNCwiPz8/IiwxLHsic3R5bGUiOnsidGFpbCI6eyJuYW1lIjoiYXJyb3doZWFkIn19fV0sWzUsMywiPz8/IiwxLHsic3R5bGUiOnsidGFpbCI6eyJuYW1lIjoiYXJyb3doZWFkIn19fV0sWzQsNiwiPz8/IiwxLHsic3R5bGUiOnsidGFpbCI6eyJuYW1lIjoiYXJyb3doZWFkIn19fV1d
\begin{tikzcd}[ampersand replacement=\&]
	\&\&\& {\vdots } \\
	\&\&\& {K^{i-1,j-1}(A)} \\
	{K_{[\Delta ,n]}(B)} \&\&\& {K^{i,j}(A)} \&\&\& {K_{[0, n - \Delta ]}(B)} \\
	\&\&\& {K^{i+1,j+1}(A)} \\
	\&\&\& {\vdots }
	\arrow["{???}"{description}, tail reversed, from=1-4, to=2-4]
	\arrow["{\mathrm{Ext}^{j-1}(T,-)}"{description}, curve={height=30pt}, from=2-4, to=3-1]
	\arrow["{???}"{description}, tail reversed, from=2-4, to=3-4]
	\arrow["{\mathrm{Ext}^{i-1}(T,-)}"{description}, curve={height=-30pt}, from=2-4, to=3-7]
	\arrow["{\mathrm{Ext}^j(T,-)}"{description}, from=3-4, to=3-1]
	\arrow["{\mathrm{Ext}^i(T,-)}"{description}, from=3-4, to=3-7]
	\arrow["{???}"{description}, tail reversed, from=3-4, to=4-4]
	\arrow["{\mathrm{Ext}^{j+1}(T,-)}"{description}, curve={height=-30pt}, from=4-4, to=3-1]
	\arrow["{\mathrm{Ext}^{i+1}(T,-)}"{description}, curve={height=30pt}, from=4-4, to=3-7]
	\arrow["{???}"{description}, tail reversed, from=4-4, to=5-4]
\end{tikzcd}.
	\end{equation}
	\begin{pinked}
		此時諸 $K^{i-s, j-s}$ 有何聯繫? 
	\end{pinked}
\end{example}


\begin{example}[Happel 定理]
	对双模 $_BT_A$, 若
	\begin{enumerate}
		\item $B ≃ \mathrm{End}_A(T,T)$, \parnote{此處的題設}
		\item $T$ 有有限的 $𝐚𝐝𝐝(A)$-消解, 
		\item $A$ 有有限的 $𝐚𝐝𝐝(T)$-餘消解, 
		\item $\mathrm{Ext}_A^{≥ 1}(T,T) = 0$. 
	\end{enumerate}
	回憶兩則經典定理. 
	\begin{enumerate}
		\item (習題 5.10, \cite{2015三角范畴与导出范畴}) 給定 Abel 範疇 $𝒜$. 若對象 $X ∈ 𝒜$ 滿足 $\mathrm{Ext}^{≥ 1}(X,X)=0$, 則典範函子 $K^b(𝐚𝐝𝐝(M)) → D^b𝒜$ 是全忠實的. \parnote{見筆記, 歸納長度即可}
		\item (習題 5.4.1, \cite{2015三角范畴与导出范畴}) 假定 Abel 範疇有無限餘積和足夠投射對象, 則 $D^b𝒜 = D^b(𝒫(𝒜))$ 當且僅當 $𝒜$ 的整體維數有限, 亦當且僅當 $𝒜$ 中任意對象的投射維數有限. 
	\end{enumerate}
此時, $K^b (𝐚𝐝𝐝 (T)) → D^b(𝐦𝐨𝐝_A) = D^b(A)$ 是範疇等價. $(T,-)$ 與 $-⊗ T$ 給出了導出等價
\begin{equation}
	% https://q.uiver.app/#q=WzAsNixbMiwwLCJLXmIgKFxcbWF0aGJme2FkZH0oVCkpIl0sWzMsMCwiRF5iIChBKSJdLFsyLDEsIkteYiAoXFxtYXRoYmZ7YWRkfShCKSkiXSxbMywxLCJEXmIgKEIpIl0sWzAsMCwiVCJdLFswLDEsIihULCBUKSJdLFswLDEsIlxcc2ltZXEiLDAseyJzdHlsZSI6eyJ0YWlsIjp7Im5hbWUiOiJhcnJvd2hlYWQifX19XSxbMiwzLCJcXHNpbWVxICIsMix7InN0eWxlIjp7InRhaWwiOnsibmFtZSI6ImFycm93aGVhZCJ9fX1dLFs0LDUsIihULC0pX0EiLDIseyJvZmZzZXQiOjV9XSxbNSw0LCItXFxvdGltZXNfQiBUIiwyLHsib2Zmc2V0Ijo1fV0sWzAsMiwiXFxzaW1lcSAiLDAseyJzdHlsZSI6eyJ0YWlsIjp7Im5hbWUiOiJhcnJvd2hlYWQifX19XSxbMSwzLCJcXHNpbWVxICIsMCx7InN0eWxlIjp7InRhaWwiOnsibmFtZSI6ImFycm93aGVhZCJ9LCJib2R5Ijp7Im5hbWUiOiJkYXNoZWQifX19XSxbOCw5LCJcXGNvbmcgIiwxLHsic2hvcnRlbiI6eyJzb3VyY2UiOjIwLCJ0YXJnZXQiOjIwfSwic3R5bGUiOnsiYm9keSI6eyJuYW1lIjoibm9uZSJ9LCJoZWFkIjp7Im5hbWUiOiJub25lIn19fV1d
\begin{tikzcd}[ampersand replacement=\&]
	T \&\& {K^b (\mathbf{add}(T))} \& {D^b (A)} \\
	{(T, T)} \&\& {K^b (\mathbf{add}(B))} \& {D^b (B)}
	\arrow[""{name=0, anchor=center, inner sep=0}, "{(T,-)_A}"', shift right=5, from=1-1, to=2-1]
	\arrow["\simeq", tail reversed, from=1-3, to=1-4]
	\arrow["{\simeq }", tail reversed, from=1-3, to=2-3]
	\arrow["{\simeq }", dashed, tail reversed, from=1-4, to=2-4]
	\arrow[""{name=1, anchor=center, inner sep=0}, "{-\otimes_B T}"', shift right=5, from=2-1, to=1-1]
	\arrow["{\simeq }"', tail reversed, from=2-3, to=2-4]
	\arrow["{\cong }"{description}, draw=none, from=0, to=1]
\end{tikzcd}.
\end{equation}
\end{example}








